\part{Combray}

% text based on https://ebooks.adelaide.edu.au/p/proust/marcel/p96d
% with corrections from Gallimard (Bibliotheque de la Pleiade) 1987 edition

\chapter*{I}

Longtemps, je me suis couché de bonne heure. Parfois, à peine ma bougie éteinte,
mes yeux se fermaient si vite que je n’avais pas le temps de me dire: «Je
m’endors.» Et, une demi-heure après, la pensée qu’il était temps de chercher le
sommeil m’éveillait; je voulais poser le volume que je croyais avoir encore dans
les mains et souffler ma lumière; je n’avais pas cessé en dormant de faire des
réflexions sur ce que je venais de lire, mais ces réflexions avaient pris un
tour un peu particulier; il me semblait que j’étais moi-même ce dont parlait
l’ouvrage: une église, un quatuor, la rivalité de François Ier et de Charles
Quint. Cette croyance survivait pendant quelques secondes à mon réveil; elle ne
choquait pas ma raison mais pesait comme des écailles sur mes yeux et les
empêchait de se rendre compte que le bougeoir n’était plus allumé. Puis elle
commençait à me devenir inintelligible, comme après la métempsycose les pensées
d’une existence antérieure; le sujet du livre se détachait de moi, j’étais libre
de m’y appliquer ou non; aussitôt je recouvrais la vue et j’étais bien étonné de
trouver autour de moi une obscurité, douce et reposante pour mes yeux, mais
peut-être plus encore pour mon esprit, à qui elle apparaissait comme une chose
sans cause, incompréhensible, comme une chose vraiment obscure. Je me demandais
quelle heure il pouvait être; j’entendais le sifflement des trains qui, plus ou
moins éloigné, comme le chant d’un oiseau dans une forêt, relevant les
distances, me décrivait l’étendue de la campagne déserte où le voyageur se hâte
vers la station prochaine; et le petit chemin qu’il suit va être gravé dans son
souvenir par l’excitation qu’il doit à des lieux nouveaux, à des actes
inaccoutumés, à la causerie récente et aux adieux sous la lampe étrangère qui le
suivent encore dans le silence de la nuit, à la douceur prochaine du retour.

J’appuyais tendrement mes joues contre les belles joues de l’oreiller qui,
pleines et fraîches, sont comme les joues de notre enfance. Je frottais une
allumette pour regarder ma montre. Bientôt minuit. C’est l’instant où le malade,
qui a été obligé de partir en voyage et a dû coucher dans un hôtel inconnu,
réveillé par une crise, se réjouit en apercevant sous la porte une raie de jour.
Quel bonheur c’est déjà le matin! Dans un moment les domestiques seront levés,
il pourra sonner, on viendra lui porter secours. L’espérance d’être soulagé lui
donne du courage pour souffrir. Justement il a cru entendre des pas; les pas se
rapprochent, puis s’éloignent. Et la raie de jour qui était sous sa porte a
disparu. C’est minuit; on vient d’éteindre le gaz; le dernier domestique est
parti et il faudra rester toute la nuit à souffrir sans remède.

Je me rendormais, et parfois je n’avais plus que de courts réveils d’un instant,
le temps d’entendre les craquements organiques des boiseries, d’ouvrir les yeux
pour fixer le kaléidoscope de l’obscurité, de goûter grâce à une lueur
momentanée de conscience le sommeil où étaient plongés les meubles, la chambre,
le tout dont je n’étais qu’une petite partie et à l’insensibilité duquel je
retournais vite m’unir. Ou bien en dormant j’avais rejoint sans effort un âge à
jamais révolu de ma vie primitive, retrouvé telle de mes terreurs enfantines
comme celle que mon grand-oncle me tirât par mes boucles et qu’avait dissipée le
jour — date pour moi d’une ère nouvelle — où on les avait coupées. J’avais
oublié cet événement pendant mon sommeil, j’en retrouvais le souvenir aussitôt
que j’avais réussi à m’éveiller pour échapper aux mains de mon grand-oncle, mais
par mesure de précaution j’entourais complètement ma tête de mon oreiller avant
de retourner dans le monde des rêves.

Quelquefois, comme Eve naquit d’une côte d’Adam, une femme naissait pendant mon
sommeil d’une fausse position de ma cuisse. Formée du plaisir que j’étais sur le
point de goûter, je m’imaginais que c’était elle qui me l’offrait. Mon corps qui
sentait dans le sien ma propre chaleur voulait s’y rejoindre, je m’éveillais. Le
reste des humains m’apparaissait comme bien lointain auprès de cette femme que
j’avais quittée il y avait quelques moments à peine; ma joue était chaude encore
de son baiser, mon corps courbaturé par le poids de sa taille. Si, comme il
arrivait quelquefois, elle avait les traits d’une femme que j’avais connue dans
la vie, j’allais me donner tout entier à ce but: la retrouver, comme ceux qui
partent en voyage pour voir de leurs yeux une cité désirée et s’imaginent qu’on
peut goûter dans une réalité le charme du songe. Peu à peu son souvenir
s’évanouissait, j’avais oublié la fille de mon rêve.

Un homme qui dort, tient en cercle autour de lui le fil des heures, l’ordre des
années et des mondes. Il les consulte d’instinct en s’éveillant et y lit en une
seconde le point de la terre qu’il occupe, le temps qui s’est écoulé jusqu’à son
réveil; mais leurs rangs peuvent se mêler, se rompre. Que vers le matin après
quelque insomnie, le sommeil le prenne en train de lire, dans une posture trop
différente de celle où il dort habituellement, il suffit de son bras soulevé
pour arrêter et faire reculer le soleil, et à la première minute de son réveil,
il ne saura plus l’heure, il estimera qu’il vient à peine de se coucher. Que
s’il s’assoupit dans une position encore plus déplacée et divergente, par
exemple après dîner assis dans un fauteuil, alors le bouleversement sera complet
dans les mondes désorbités, le fauteuil magique le fera voyager à toute vitesse
dans le temps et dans l’espace, et au moment d’ouvrir les paupières, il se
croira couché quelques mois plus tôt dans une autre contrée. Mais il suffisait
que, dans mon lit même, mon sommeil fût profond et détendît entièrement mon
esprit; alors celui-ci lâchait le plan du lieu où je m’étais endormi, et quand
je m’éveillais au milieu de la nuit, comme j’ignorais où je me trouvais, je ne
savais même pas au premier instant qui j’étais; j’avais seulement dans sa
simplicité première, le sentiment de l’existence comme il peut frémir au fond
d’un animal: j’étais plus dénué que l’homme des cavernes; mais alors le souvenir
— non encore du lieu où j’étais, mais de quelques-uns de ceux que j’avais
habités et où j’aurais pu être — venait à moi comme un secours d’en haut pour me
tirer du néant d’où je n’aurais pu sortir tout seul; je passais en une seconde
par-dessus des siècles de civilisation, et l’image confusément entrevue de
lampes à pétrole, puis de chemises à col rabattu, recomposaient peu à peu les
traits originaux de mon moi.

Peut-être l’immobilité des choses autour de nous leur est-elle imposée par
notre certitude que ce sont elles et non pas d’autres, par l’immobilité de
notre pensée en face d’elles. Toujours est-il que, quand je me réveillais
ainsi, mon esprit s’agitant pour chercher, sans y réussir, à savoir où
j’étais, tout tournait autour de moi dans l’obscurité, les choses, les
pays, les années. Mon corps, trop engourdi pour remuer, cherchait, d’après
la forme de sa fatigue, à repérer la position de ses membres pour en induire
la direction du mur, la place des meubles, pour reconstruire et pour nommer la
demeure où il se trouvait. Sa mémoire, la mémoire de ses côtes, de ses
genoux, de ses épaules, lui présentait successivement plusieurs des chambres
où il avait dormi, tandis qu’autour de lui les murs invisibles, changeant de
place selon la forme de la pièce imaginée, tourbillonnaient dans les
ténèbres. Et avant même que ma pensée, qui hésitait au seuil des temps et
des formes, eût identifié le logis en rapprochant les circonstances, lui
— mon corps — se rappelait pour chacun le genre du lit, la place des portes,
la prise de jour des fenêtres, l’existence d’un couloir, avec la pensée
que j’avais en m’y endormant et que je retrouvais au réveil. Mon côté
ankylosé, cherchant à deviner son orientation, s’imaginait, par exemple,
allongé face au mur dans un grand lit à baldaquin et aussitôt je me disais:
«Tiens, j’ai fini par m’endormir quoique maman ne soit pas venue me dire
bonsoir», j’étais à la campagne chez mon grand-père, mort depuis bien des
années; et mon corps, le côté sur lequel je reposais, gardiens fidèles
d’un passé que mon esprit n’aurait jamais dû oublier, me rappelaient la
flamme de la veilleuse de verre de Bohême, en forme d’urne, suspendue au
plafond par des chaînettes, la cheminée en marbre de Sienne, dans ma chambre à
coucher de Combray, chez mes grands-parents, en des jours lointains qu’en ce
moment je me figurais actuels sans me les représenter exactement et que je
reverrais mieux tout à l’heure quand je serais tout à fait éveillé.

Puis renaissait le souvenir d’une nouvelle attitude; le mur filait dans une
autre direction: j’étais dans ma chambre chez Mme de Saint-Loup, à la campagne;
mon Dieu! Il est au moins dix heures, on doit avoir fini de dîner! J’aurai trop
prolongé la sieste que je fais tous les soirs en rentrant de ma promenade avec
Mme de Saint-Loup, avant d’endosser mon habit. Car bien des années ont passé
depuis Combray, où, dans nos retours les plus tardifs, c’était les reflets
rouges du couchant que je voyais sur le vitrage de ma fenêtre. C’est un autre
genre de vie qu’on mène à Tansonville, chez Mme de Saint-Loup, un autre genre de
plaisir que je trouve à ne sortir qu’à la nuit, à suivre au clair de lune ces
chemins où je jouais jadis au soleil; et la chambre où je me serai endormi au
lieu de m’habiller pour le dîner, de loin je l’aperçois, quand nous rentrons,
traversée par les feux de la lampe, seul phare dans la nuit.

Ces évocations tournoyantes et confuses ne duraient jamais que quelques
secondes; souvent, ma brève incertitude du lieu où je me trouvais ne distinguait
pas mieux les unes des autres les diverses suppositions dont elle était faite,
que nous n’isolons, en voyant un cheval courir, les positions successives que
nous montre le kinétoscope. Mais j’avais revu tantôt l’une, tantôt l’autre, des
chambres que j’avais habitées dans ma vie, et je finissais par me les rappeler
toutes dans les longues rêveries qui suivaient mon réveil; chambres d’hiver où
quand on est couché, on se blottit la tête dans un nid qu’on se tresse avec les
choses les plus disparates: un coin de l’oreiller, le haut des couvertures, un
bout de châle, le bord du lit, et un numéro des \textit{Débats roses}, qu’on finit par
cimenter ensemble selon la technique des oiseaux en s’y appuyant indéfiniment;
où, par un temps glacial le plaisir qu’on goûte est de se sentir séparé du
dehors (comme l’hirondelle de mer qui a son nid au fond d’un souterrain dans la
chaleur de la terre), et où, le feu étant entretenu toute la nuit dans la
cheminée, on dort dans un grand manteau d’air chaud et fumeux, traversé des
lueurs des tisons qui se rallument, sorte d’impalpable alcôve, de chaude caverne
creusée au sein de la chambre même, zone ardente et mobile en ses contours
thermiques, aérée de souffles qui nous rafraîchissent la figure et viennent des
angles, des parties voisines de la fenêtre ou éloignées du foyer et qui se sont
refroidies; — chambres d’été où l’on aime être uni à la nuit tiède, où le clair
de lune appuyé aux volets entr’ouverts, jette jusqu’au pied du lit son échelle
enchantée, où on dort presque en plein air, comme la mésange balancée par la
brise à la pointe d’un rayon ; — parfois la chambre Louis XVI, si gaie que même
le premier soir je n’y avais pas été trop malheureux et où les colonnettes qui
soutenaient légèrement le plafond s’écartaient avec tant de grâce pour montrer
et réserver la place du lit; parfois au contraire celle, petite et si élevée de
plafond, creusée en forme de pyramide dans la hauteur de deux étages et
partiellement revêtue d’acajou, où dès la première seconde j’avais été intoxiqué
moralement par l’odeur inconnue du vétiver, convaincu de l’hostilité des rideaux
violets et de l’insolente indifférence de la pendule que jacassait tout haut
comme si je n’eusse pas été là; — où une étrange et impitoyable glace à pieds
quadrangulaires, barrant obliquement un des angles de la pièce, se creusait à
vif dans la douce plénitude de mon champ visuel accoutumé un emplacement qui n’y
était pas prévu; — où ma pensée, s’efforçant pendant des heures de se disloquer,
de s’étirer en hauteur pour prendre exactement la forme de la chambre et arriver
à remplir jusqu’en haut son gigantesque entonnoir, avait souffert bien de dures
nuits, tandis que j’étais étendu dans mon lit, les yeux levés, l’oreille
anxieuse, la narine rétive, le cœur battant: jusqu’à ce que l’habitude eût
changé la couleur des rideaux, fait taire la pendule, enseigné la pitié à la
glace oblique et cruelle, dissimulé, sinon chassé complètement, l’odeur du
vétiver et notablement diminué la hauteur apparente du plafond. L’habitude!
aménageuse habile mais bien lente et qui commence par laisser souffrir notre
esprit pendant des semaines dans une installation provisoire; mais que malgré
tout il est bien heureux de trouver, car sans l’habitude et réduit à ses seuls
moyens il serait impuissant à nous rendre un logis habitable.

Certes, j’étais bien éveillé maintenant, mon corps avait viré une dernière fois
et le bon ange de la certitude avait tout arrêté autour de moi, m’avait couché
sous mes couvertures, dans ma chambre, et avait mis approximativement à leur
place dans l’obscurité ma commode, mon bureau, ma cheminée, la fenêtre sur la
rue et les deux portes. Mais j’avais beau savoir que je n’étais pas dans les
demeures dont l’ignorance du réveil m’avait en un instant sinon présenté l’image
distincte, du moins fait croire la présence possible, le branle était donné à ma
mémoire; généralement je ne cherchais pas à me rendormir tout de suite; je
passais la plus grande partie de la nuit à me rappeler notre vie d’autrefois, à
Combray chez ma grand’tante, à Balbec, à Paris, à Doncières, à Venise, ailleurs
encore, à me rappeler les lieux, les personnes que j’y avais connues, ce que
j’avais vu d’elles, ce qu’on m’en avait raconté.

%%%%  checked  till here  %%%%%%%

À Combray, tous les jours dès la fin de l’après-midi, longtemps avant le moment
où il faudrait me mettre au lit et rester, sans dormir, loin de ma mère et de ma
grand’mère, ma chambre à coucher redevenait le point fixe et douloureux de mes
préoccupations. On avait bien inventé, pour me distraire les soirs où on me
trouvait l’air trop malheureux, de me donner une lanterne magique, dont, en
attendant l’heure du dîner, on coiffait ma lampe; et, à l’instar des premiers
architectes et maîtres verriers de l’âge gothique, elle substituait à l’opacité
des murs d’impalpables irisations, de surnaturelles apparitions multicolores, où
des légendes étaient dépeintes comme dans un vitrail vacillant et momentané.
Mais ma tristesse n’en était qu’accrue, parce que rien que le changement
d’éclairage détruisait l’habitude que j’avais de ma chambre et grâce à quoi,
sauf le supplice du coucher, elle m’était devenue supportable. Maintenant je ne
la reconnaissais plus et j’y étais inquiet, comme dans une chambre d’hôtel ou de
«chalet», où je fusse arrivé pour la première fois en descendant de chemin de
fer.

Au pas saccadé de son cheval, Golo, plein d’un affreux dessein, sortait de la
petite forêt triangulaire qui veloutait d’un vert sombre la pente d’une colline,
et s’avançait en tressautant vers le château de la pauvre Geneviève de Brabant.
Ce château était coupé selon une ligne courbe qui n’était autre que la limite
d’un des ovales de verre ménagés dans le châssis qu’on glissait entre les
coulisses de la lanterne. Ce n’était qu’un pan de château et il avait devant lui
une lande où rêvait Geneviève qui portait une ceinture bleue. Le château et la
lande étaient jaunes et je n’avais pas attendu de les voir pour connaître leur
couleur car, avant les verres du châssis, la sonorité mordorée du nom de Brabant
me l’avait montrée avec évidence. Golo s’arrêtait un instant pour écouter avec
tristesse le boniment lu à haute voix par ma grand’tante et qu’il avait l’air de
comprendre parfaitement, conformant son attitude avec une docilité qui
n’excluait pas une certaine majesté, aux indications du texte; puis il
s’éloignait du même pas saccadé. Et rien ne pouvait arrêter sa lente chevauchée.
Si on bougeait la lanterne, je distinguais le cheval de Golo qui continuait à
s’avancer sur les rideaux de la fenêtre, se bombant de leurs plis, descendant
dans leurs fentes. Le corps de Golo lui-même, d’une essence aussi surnaturelle
que celui de sa monture, s’arrangeait de tout obstacle matériel, de tout objet
gênant qu’il rencontrait en le prenant comme ossature et en se le rendant
intérieur, fût-ce le bouton de la porte sur lequel s’adaptait aussitôt et
surnageait invinciblement sa robe rouge ou sa figure pâle toujours aussi noble
et aussi mélancolique, mais qui ne laissait paraître aucun trouble de cette
transvertébration.

Certes je leur trouvais du charme à ces brillantes projections qui semblaient
émaner d’un passé mérovingien et promenaient autour de moi des reflets
d’histoire si anciens. Mais je ne peux dire quel malaise me causait pourtant
cette intrusion du mystère et de la beauté dans une chambre que j’avais fini par
remplir de mon moi au point de ne pas faire plus attention à elle qu’à lui-même.
L’influence anesthésiante de l’habitude ayant cessé, je me mettais à penser, à
sentir, choses si tristes. Ce bouton de la porte de ma chambre, qui différait
pour moi de tous les autres boutons de porte du monde en ceci qu’il semblait
ouvrir tout seul, sans que j’eusse besoin de le tourner, tant le maniement m’en
était devenu inconscient, le voilà qui servait maintenant de corps astral à
Golo. Et dès qu’on sonnait le dîner, j’avais hâte de courir à la salle à manger,
où la grosse lampe de la suspension, ignorante de Golo et de Barbe-Bleue, et qui
connaissait mes parents et le bœuf à la casserole, donnait sa lumière de tous
les soirs; et de tomber dans les bras de maman que les malheurs de Geneviève de
Brabant me rendaient plus chère, tandis que les crimes de Golo me faisaient
examiner ma propre conscience avec plus de scrupules.

Après le dîner, hélas, j’étais bientôt obligé de quitter maman qui restait à
causer avec les autres, au jardin s’il faisait beau, dans le petit salon où tout
le monde se retirait s’il faisait mauvais. Tout le monde, sauf ma grand’mère qui
trouvait que «c’est une pitié de rester enfermé à la campagne» et qui avait
d’incessantes discussions avec mon père, les jours de trop grande pluie, parce
qu’il m’envoyait lire dans ma chambre au lieu de rester dehors. «Ce n’est pas
comme cela que vous le rendrez robuste et énergique, disait-elle tristement,
surtout ce petit qui a tant besoin de prendre des forces et de la volonté.» Mon
père haussait les épaules et il examinait le baromètre, car il aimait la
météorologie, pendant que ma mère, évitant de faire du bruit pour ne pas le
troubler, le regardait avec un respect attendri, mais pas trop fixement pour ne
pas chercher à percer le mystère de ses supériorités. Mais ma grand’mère, elle,
par tous les temps, même quand la pluie faisait rage et que Françoise avait
précipitamment rentré les précieux fauteuils d’osier de peur qu’ils ne fussent
mouillés, on la voyait dans le jardin vide et fouetté par l’averse, relevant ses
mèches désordonnées et grises pour que son front s’imbibât mieux de la salubrité
du vent et de la pluie. Elle disait: «Enfin, on respire!» et parcourait les
allées détrempées — trop symétriquement alignées à son gré par le nouveau
jardinier dépourvu du sentiment de la nature et auquel mon père avait demandé
depuis le matin si le temps s’arrangerait — de son petit pas enthousiaste et
saccadé, réglé sur les mouvements divers qu’excitaient dans son âme l’ivresse de
l’orage, la puissance de l’hygiène, la stupidité de mon éducation et la symétrie
des jardins, plutôt que sur le désir inconnu d’elle d’éviter à sa jupe prune les
taches de boue sous lesquelles elle disparaissait jusqu’à une hauteur qui était
toujours pour sa femme de chambre un désespoir et un problème.

Quand ces tours de jardin de ma grand’mère avaient lieu après dîner, une chose
avait le pouvoir de la faire rentrer: c’était, à un des moments où la révolution
de sa promenade la ramenait périodiquement, comme un insecte, en face des
lumières du petit salon où les liqueurs étaient servies sur la table à jeu — si
ma grand’tante lui criait: «Bathilde! viens donc empêcher ton mari de boire du
cognac!» Pour la taquiner, en effet (elle avait apporté dans la famille de mon
père un esprit si différent que tout le monde la plaisantait et la tourmentait),
comme les liqueurs étaient défendues à mon grand-père, ma grand’tante lui en
faisait boire quelques gouttes. Ma pauvre grand’mère entrait, priait ardemment
son mari de ne pas goûter au cognac; il se fâchait, buvait tout de même sa
gorgée, et ma grand’mère repartait, triste, découragée, souriante pourtant, car
elle était si humble de cœur et si douce que sa tendresse pour les autres et le
peu de cas qu’elle faisait de sa propre personne et de ses souffrances, se
conciliaient dans son regard en un sourire où, contrairement à ce qu’on voit
dans le visage de beaucoup d’humains, il n’y avait d’ironie que pour
elle-même, et pour nous tous comme un baiser de ses yeux qui ne pouvaient voir
ceux qu’elle chérissait sans les caresser passionnément du regard. Ce supplice
que lui infligeait ma grand’tante, le spectacle des vaines prières de ma
grand’mère et de sa faiblesse, vaincue d’avance, essayant inutilement d’ôter à
mon grand-père le verre à liqueur, c’était de ces choses à la vue desquelles on
s’habitue plus tard jusqu’à les considérer en riant et à prendre le parti du
persécuteur assez résolument et gaiement pour se persuader à soi-même qu’il ne
s’agit pas de persécution; elles me causaient alors une telle horreur, que
j’aurais aimé battre ma grand’tante. Mais dès que j’entendais: «Bathilde, viens
donc empêcher ton mari de boire du cognac!» déjà homme par la lâcheté, je
faisais ce que nous faisons tous, une fois que nous sommes grands, quand il y a
devant nous des souffrances et des injustices: je ne voulais pas les voir; je
montais sangloter tout en haut de la maison à côté de la salle d’études, sous
les toits, dans une petite pièce sentant l’iris, et que parfumait aussi un
cassis sauvage poussé au dehors entre les pierres de la muraille et qui passait
une branche de fleurs par la fenêtre entr’ouverte. Destinée à un usage plus
spécial et plus vulgaire, cette pièce, d’où l’on voyait pendant le jour jusqu’au
donjon de Roussainville-le-Pin, servit longtemps de refuge pour moi, sans doute
parce qu’elle était la seule qu’il me fût permis de fermer à clef, à toutes
celles de mes occupations qui réclamaient une inviolable solitude: la lecture,
la rêverie, les larmes et la volupté. Hélas! je ne savais pas que, bien plus
tristement que les petits écarts de régime de son mari, mon manque de volonté,
ma santé délicate, l’incertitude qu’ils projetaient sur mon avenir,
préoccupaient ma grand’mère, au cours de ces déambulations incessantes, de
l’après-midi et du soir, où on voyait passer et repasser, obliquement levé vers
le ciel, son beau visage aux joues brunes et sillonnées, devenues au retour de
l’âge presque mauves comme les labours à l’automne, barrées, si elle sortait,
par une voilette à demi relevée, et sur lesquelles, amené là par le froid ou
quelque triste pensée, était toujours en train de sécher un pleur involontaire.

Ma seule consolation, quand je montais me coucher, était que maman viendrait
m’embrasser quand je serais dans mon lit. Mais ce bonsoir durait si peu de
temps, elle redescendait si vite, que le moment où je l’entendais monter, puis
où passait dans le couloir à double porte le bruit léger de sa robe de jardin en
mousseline bleue, à laquelle pendaient de petits cordons de paille tressée,
était pour moi un moment douloureux. Il annonçait celui qui allait le suivre, où
elle m’aurait quitté, où elle serait redescendue. De sorte que ce bonsoir que
j’aimais tant, j’en arrivais à souhaiter qu’il vînt le plus tard possible, à ce
que se prolongeât le temps de répit où maman n’était pas encore venue.
Quelquefois quand, après m’avoir embrassé, elle ouvrait la porte pour partir, je
voulais la rappeler, lui dire «embrasse-moi une fois encore», mais je savais
qu’aussitôt elle aurait son visage fâché, car la concession qu’elle faisait à ma
tristesse et à mon agitation en montant m’embrasser, en m’apportant ce baiser de
paix, agaçait mon père qui trouvait ces rites absurdes, et elle eût voulu tâcher
de m’en faire perdre le besoin, l’habitude, bien loin de me laisser prendre
celle de lui demander, quand elle était déjà sur le pas de la porte, un baiser
de plus. Or la voir fâchée détruisait tout le calme qu’elle m’avait apporté un
instant avant, quand elle avait penché vers mon lit sa figure aimante, et me
l’avait tendue comme une hostie pour une communion de paix où mes lèvres
puiseraient sa présence réelle et le pouvoir de m’endormir. Mais ces soirs-là,
où maman en somme restait si peu de temps dans ma chambre, étaient doux encore
en comparaison de ceux où il y avait du monde à dîner et où, à cause de cela,
elle ne montait pas me dire bonsoir. Le monde se bornait habituellement à M.
Swann, qui, en dehors de quelques étrangers de passage, était à peu près la
seule personne qui vînt chez nous à Combray, quelquefois pour dîner en voisin
(plus rarement depuis qu’il avait fait ce mauvais mariage, parce que mes parents
ne voulaient pas recevoir sa femme), quelquefois après le dîner, à l’improviste.
Les soirs où, assis devant la maison sous le grand marronnier, autour de la
table de fer, nous entendions au bout du jardin, non pas le grelot profus et
criard qui arrosait, qui étourdissait au passage de son bruit ferrugineux,
intarissable et glacé, toute personne de la maison qui le déclenchait en entrant
«sans sonner», mais le double tintement timide, ovale et doré de la clochette
pour les étrangers, tout le monde aussitôt se demandait: «Une visite, qui cela
peut-il être?» mais on savait bien que cela ne pouvait être que M. Swann; ma
grand’tante parlant à haute voix, pour prêcher d’exemple, sur un ton qu’elle
s’efforçait de rendre naturel, disait de ne pas chuchoter ainsi; que rien n’est
plus désobligeant pour une personne qui arrive et à qui cela fait croire qu’on
est en train de dire des choses qu’elle ne doit pas entendre; et on envoyait en
éclaireur ma grand’mère, toujours heureuse d’avoir un prétexte pour faire un
tour de jardin de plus, et qui en profitait pour arracher subrepticement au
passage quelques tuteurs de rosiers afin de rendre aux roses un peu de naturel,
comme une mère qui, pour les faire bouffer, passe la main dans les cheveux de
son fils que le coiffeur a trop aplatis.

Nous restions tous suspendus aux nouvelles que ma grand’mère allait nous
apporter de l’ennemi, comme si on eût pu hésiter entre un grand nombre possible
d’assaillants, et bientôt après mon grand-père disait: «Je reconnais la voix de
Swann.» On ne le reconnaissait en effet qu’à la voix, on distinguait mal son
visage au nez busqué, aux yeux verts, sous un haut front entouré de cheveux
blonds presque roux, coiffés à la Bressant, parce que nous gardions le moins de
lumière possible au jardin pour ne pas attirer les moustiques et j’allais, sans
en avoir l’air, dire qu’on apportât les sirops; ma grand’mère attachait beaucoup
d’importance, trouvant cela plus aimable, à ce qu’ils n’eussent pas l’air de
figurer d’une façon exceptionnelle, et pour les visites seulement. M. Swann,
quoique beaucoup plus jeune que lui, était très lié avec mon grand-père qui
avait été un des meilleurs amis de son père, homme excellent mais singulier,
chez qui, paraît-il, un rien suffisait parfois pour interrompre les élans du
cœur, changer le cours de la pensée. J’entendais plusieurs fois par an mon
grand-père raconter à table des anecdotes toujours les mêmes sur l’attitude
qu’avait eue M. Swann le père, à la mort de sa femme qu’il avait veillée jour et
nuit. Mon grand-père qui ne l’avait pas vu depuis longtemps était accouru auprès
de lui dans la propriété que les Swann possédaient aux environs de Combray, et
avait réussi, pour qu’il n’assistât pas à la mise en bière, à lui faire quitter
un moment, tout en pleurs, la chambre mortuaire. Ils firent quelques pas dans le
parc où il y avait un peu de soleil. Tout d’un coup, M. Swann prenant mon
grand-père par le bras, s’était écrié: «Ah! mon vieil ami, quel bonheur de se
promener ensemble par ce beau temps. Vous ne trouvez pas ça joli tous ces
arbres, ces aubépines et mon étang dont vous ne m’avez jamais félicité? Vous
avez l’air comme un bonnet de nuit. Sentez-vous ce petit vent? Ah! on a beau
dire, la vie a du bon tout de même, mon cher Amédée!» Brusquement le souvenir de
sa femme morte lui revint, et trouvant sans doute trop compliqué de chercher
comment il avait pu à un pareil moment se laisser aller à un mouvement de joie,
il se contenta, par un geste qui lui était familier chaque fois qu’une question
ardue se présentait à son esprit, de passer la main sur son front, d’essuyer ses
yeux et les verres de son lorgnon. Il ne put pourtant pas se consoler de la mort
de sa femme, mais pendant les deux années qu’il lui survécut, il disait à mon
grand-père: «C’est drôle, je pense très souvent à ma pauvre femme, mais je ne
peux y penser beaucoup à la fois.» «Souvent, mais peu à la fois, comme le pauvre
père Swann», était devenu une des phrases favorites de mon grand-père qui la
prononçait à propos des choses les plus différentes. Il m’aurait paru que ce
père de Swann était un monstre, si mon grand-père que je considérais comme
meilleur juge et dont la sentence faisant jurisprudence pour moi, m’a souvent
servi dans la suite à absoudre des fautes que j’aurais été enclin à condamner,
ne s’était récrié: «Mais comment? c’était un cœur d’or!»

Pendant bien des années, où pourtant, surtout avant son mariage, M. Swann, le
fils, vint souvent les voir à Combray, ma grand’tante et mes grands-parents ne
soupçonnèrent pas qu’il ne vivait plus du tout dans la société qu’avait
fréquentée sa famille et que sous l’espèce d’incognito que lui faisait chez nous
ce nom de Swann, ils hébergeaient — avec la parfaite innocence d’honnêtes
hôteliers qui ont chez eux, sans le savoir, un célèbre brigand — un des membres
les plus élégants du Jockey-Club, ami préféré du comte de Paris et du prince de
Galles, un des hommes les plus choyés de la haute société du faubourg
Saint-Germain.

L’ignorance où nous étions de cette brillante vie mondaine que menait Swann
tenait évidemment en partie à la réserve et à la discrétion de son caractère,
mais aussi à ce que les bourgeois d’alors se faisaient de la société une idée un
peu hindoue et la considéraient comme composée de castes fermées où chacun, dès
sa naissance, se trouvait placé dans le rang qu’occupaient ses parents, et d’où
rien, à moins des hasards d’une carrière exceptionnelle ou d’un mariage
inespéré, ne pouvait vous tirer pour vous faire pénétrer dans une caste
supérieure. M. Swann, le père, était agent de change; le «fils Swann» se
trouvait faire partie pour toute sa vie d’une caste où les fortunes, comme dans
une catégorie de contribuables, variaient entre tel et tel revenu. On savait
quelles avaient été les fréquentations de son père, on savait donc quelles
étaient les siennes, avec quelles personnes il était «en situation» de frayer.
S’il en connaissait d’autres, c’étaient relations de jeune homme sur lesquelles
des amis anciens de sa famille, comme étaient mes parents, fermaient d’autant
plus bienveillamment les yeux qu’il continuait, depuis qu’il était orphelin, à
venir très fidèlement nous voir; mais il y avait fort à parier que ces gens
inconnus de nous qu’il voyait, étaient de ceux qu’il n’aurait pas osé saluer si,
étant avec nous, il les avait rencontrés. Si l’on avait voulu à toute force
appliquer à Swann un coefficient social qui lui fût personnel, entre les autres
fils d’agents de situation égale à celle de ses parents, ce coefficient eût été
pour lui un peu inférieur parce que, très simple de façon et ayant toujours eu
une «toquade» d’objets anciens et de peinture, il demeurait maintenant dans un
vieil hôtel où il entassait ses collections et que ma grand’mère rêvait de
visiter, mais qui était situé quai d’Orléans, quartier que ma grand’tante
trouvait infamant d’habiter. «Êtes-vous seulement connaisseur? je vous demande
cela dans votre intérêt, parce que vous devez vous faire repasser des croûtes
par les marchands», lui disait ma grand’tante; elle ne lui supposait en effet
aucune compétence et n’avait pas haute idée même au point de vue intellectuel
d’un homme qui dans la conversation évitait les sujets sérieux et montrait une
précision fort prosaïque non seulement quand il nous donnait, en entrant dans
les moindres détails, des recettes de cuisine, mais même quand les sœurs de ma
grand’mère parlaient de sujets artistiques. Provoqué par elles à donner son
avis, à exprimer son admiration pour un tableau, il gardait un silence presque
désobligeant et se rattrapait en revanche s’il pouvait fournir sur le musée où
il se trouvait, sur la date où il avait été peint, un renseignement matériel.
Mais d’habitude il se contentait de chercher à nous amuser en racontant chaque
fois une histoire nouvelle qui venait de lui arriver avec des gens choisis parmi
ceux que nous connaissions, avec le pharmacien de Combray, avec notre
cuisinière, avec notre cocher. Certes ces récits faisaient rire ma grand’tante,
mais sans qu’elle distinguât bien si c’était à cause du rôle ridicule que s’y
donnait toujours Swann ou de l’esprit qu’il mettait à les conter: «On peut dire
que vous êtes un vrai type, monsieur Swann!» Comme elle était la seule personne
un peu vulgaire de notre famille, elle avait soin de faire remarquer aux
étrangers, quand on parlait de Swann, qu’il aurait pu, s’il avait voulu, habiter
boulevard Haussmann ou avenue de l’Opéra, qu’il était le fils de M. Swann qui
avait dû lui laisser quatre ou cinq millions, mais que c’était sa fantaisie.
Fantaisie qu’elle jugeait du reste devoir être si divertissante pour les autres,
qu’à Paris, quand M. Swann venait le 1er janvier lui apporter son sac de marrons
glacés, elle ne manquait pas, s’il y avait du monde, de lui dire: «Eh bien! M.
Swann, vous habitez toujours près de l’Entrepôt des vins, pour être sûr de ne
pas manquer le train quand vous prenez le chemin de Lyon?» Et elle regardait du
coin de l’œil, par-dessus son lorgnon, les autres visiteurs.

Mais si l’on avait dit à ma grand’mère que ce Swann qui, en tant que fils Swann
était parfaitement «qualifié» pour être reçu par toute la «belle bourgeoisie»,
par les notaires ou les avoués les plus estimés de Paris (privilège qu’il
semblait laisser tomber en peu en quenouille), avait, comme en cachette, une vie
toute différente; qu’en sortant de chez nous, à Paris, après nous avoir dit
qu’il rentrait se coucher, il rebroussait chemin à peine la rue tournée et se
rendait dans tel salon que jamais l’œil d’aucun agent ou associé d’agent ne
contempla, cela eût paru aussi extraordinaire à ma tante qu’aurait pu l’être
pour une dame plus lettrée la pensée d’être personnellement liée avec Aristée
dont elle aurait compris qu’il allait, après avoir causé avec elle, plonger au
sein des royaumes de Thétis, dans un empire soustrait aux yeux des mortels et où
Virgile nous le montre reçu à bras ouverts; ou, pour s’en tenir à une image qui
avait plus de chance de lui venir à l’esprit, car elle l’avait vue peinte sur
nos assiettes à petits fours de Combray — d’avoir eu à dîner Ali-Baba, lequel
quand il se saura seul, pénétrera dans la caverne, éblouissante de trésors
insoupçonnés.

Un jour qu’il était venu nous voir à Paris après dîner en s’excusant d’être en
habit, Françoise ayant, après son départ, dit tenir du cocher qu’il avait dîné
«chez une princesse» — «Oui, chez une princesse du demi-monde!» avait répondu ma
tante en haussant les épaules sans lever les yeux de sur son tricot, avec une
ironie sereine.

Aussi, ma grand’tante en usait-elle cavalièrement avec lui. Comme elle croyait
qu’il devait être flatté par nos invitations, elle trouvait tout naturel qu’il
ne vînt pas nous voir l’été sans avoir à la main un panier de pêches ou de
framboises de son jardin et que de chacun de ses voyages d’Italie il m’eût
rapporté des photographies de chefs-d’œuvre.

On ne se gênait guère pour l’envoyer quérir dès qu’on avait besoin d’une recette
de sauce gribiche ou de salade à l’ananas pour des grands dîners où on ne
l’invitait pas, ne lui trouvant pas un prestige suffisant pour qu’on pût le
servir à des étrangers qui venaient pour la première fois. Si la conversation
tombait sur les princes de la Maison de France: «des gens que nous ne
connaîtrons jamais ni vous ni moi et nous nous en passons, n’est-ce pas», disait
ma grand’tante à Swann qui avait peut-être dans sa poche une lettre de
Twickenham; elle lui faisait pousser le piano et tourner les pages les soirs où
la sœur de ma grand’mère chantait, ayant pour manier cet être ailleurs si
recherché, la naïve brusquerie d’un enfant qui joue avec un bibelot de
collection sans plus de précautions qu’avec un objet bon marché. Sans doute le
Swann que connurent à la même époque tant de clubmen était bien différent de
celui que créait ma grand’tante, quand le soir, dans le petit jardin de Combray,
après qu’avaient retenti les deux coups hésitants de la clochette, elle
injectait et vivifiait de tout ce qu’elle savait sur la famille Swann, l’obscur
et incertain personnage qui se détachait, suivi de ma grand’mère, sur un fond de
ténèbres, et qu’on reconnaissait à la voix. Mais même au point de vue des plus
insignifiantes choses de la vie, nous ne sommes pas un tout matériellement
constitué, identique pour tout le monde et dont chacun n’a qu’à aller prendre
connaissance comme d’un cahier des charges ou d’un testament; notre personnalité
sociale est une création de la pensée des autres. Même l’acte si simple que nous
appelons «voir une personne que nous connaissons» est en partie un acte
intellectuel. Nous remplissons l’apparence physique de l’être que nous voyons,
de toutes les notions que nous avons sur lui et dans l’aspect total que nous
nous représentons, ces notions ont certainement la plus grande part. Elles
finissent par gonfler si parfaitement les joues, par suivre en une adhérence si
exacte la ligne du nez, elles se mêlent si bien de nuancer la sonorité de la
voix comme si celle-ci n’était qu’une transparente enveloppe, que chaque fois
que nous voyons ce visage et que nous entendons cette voix, ce sont ces notions
que nous retrouvons, que nous écoutons. Sans doute, dans le Swann qu’ils
s’étaient constitué, mes parents avaient omis par ignorance de faire entrer une
foule de particularités de sa vie mondaine que étaient cause que d’autres
personnes, quand elles étaient en sa présence, voyaient les élégances régner
dans son visage et s’arrêter à son nez busqué comme à leur frontière naturelle;
mais aussi ils avaient pu entasser dans ce visage désaffecté de son prestige,
vacant et spacieux, au fond de ces yeux dépréciés, le vague et doux résidu —
mi-mémoire, mi-oubli — des heures oisives passées ensemble après nos dîners
hebdomadaires, autour de la table de jeu ou au jardin, durant notre vie de bon
voisinage campagnard. L’enveloppe corporelle de notre ami en avait été si bien
bourrée, ainsi que de quelques souvenirs relatifs à ses parents, que ce Swann-là
était devenu un être complet et vivant, et que j’ai l’impression de quitter une
personne pour aller vers une autre qui en est distincte, quand, dans ma mémoire,
du Swann que j’ai connu plus tard avec exactitude je passe à ce premier Swann —
à ce premier Swann dans lequel je retrouve les erreurs charmantes de ma
jeunesse, et qui d’ailleurs ressemble moins à l’autre qu’aux personnes que j’ai
connues à la même époque, comme s’il en était de notre vie ainsi que d’un musée
où tous les portraits d’un même temps ont un air de famille, une même tonalité—à
ce premier Swann rempli de loisir, parfumé par l’odeur du grand marronnier, des
paniers de framboises et d’un brin d’estragon.

Pourtant un jour que ma grand’mère était allée demander un service à une dame
qu’elle avait connue au Sacré-Cœur (et avec laquelle, à cause de notre
conception des castes elle n’avait pas voulu rester en relations malgré une
sympathie réciproque), la marquise de Villeparisis, de la célèbre famille de
Bouillon, celle-ci lui avait dit: «Je crois que vous connaissez beaucoup M.
Swann qui est un grand ami de mes neveux des Laumes». Ma grand’mère était
revenue de sa visite enthousiasmée par la maison qui donnait sur des jardins et
où Mme de Villeparisis lui conseillait de louer, et aussi par un giletier et sa
fille, qui avaient leur boutique dans la cour et chez qui elle était entrée
demander qu’on fît un point à sa jupe qu’elle avait déchirée dans l’escalier. Ma
grand’mère avait trouvé ces gens parfaits, elle déclarait que la petite était
une perle et que le giletier était l’homme le plus distingué, le mieux qu’elle
eût jamais vu. Car pour elle, la distinction était quelque chose d’absolument
indépendant du rang social. Elle s’extasiait sur une réponse que le giletier lui
avait faite, disant à maman: «Sévigné n’aurait pas mieux dit!» et en revanche,
d’un neveu de Mme de Villeparisis qu’elle avait rencontré chez elle: «Ah! ma
fille, comme il est commun!»

Or le propos relatif à Swann avait eu pour effet non pas de relever celui-ci
dans l’esprit de ma grand’tante, mais d’y abaisser Mme de Villeparisis. Il
semblait que la considération que, sur la foi de ma grand’mère, nous accordions
à Mme de Villeparisis, lui créât un devoir de ne rien faire qui l’en rendît
moins digne et auquel elle avait manqué en apprenant l’existence de Swann, en
permettant à des parents à elle de le fréquenter. «Comment elle connaît Swann?
Pour une personne que tu prétendais parente du maréchal de Mac-Mahon!» Cette
opinion de mes parents sur les relations de Swann leur parut ensuite confirmée
par son mariage avec une femme de la pire société, presque une cocotte que,
d’ailleurs, il ne chercha jamais à présenter, continuant à venir seul chez nous,
quoique de moins en moins, mais d’après laquelle ils crurent pouvoir juger —
supposant que c’était là qu’il l’avait prise — le milieu, inconnu d’eux, qu’il
fréquentait habituellement.

Mais une fois, mon grand-père lut dans un journal que M. Swann était un des plus
fidèles habitués des déjeuners du dimanche chez le duc de X . . ., dont le père
et l’oncle avaient été les hommes d’État les plus en vue du règne de
Louis-Philippe. Or mon grand-père était curieux de tous les petits faits qui
pouvaient l’aider à entrer par la pensée dans la vie privée d’hommes comme Molé,
comme le duc Pasquier, comme le duc de Broglie. Il fut enchanté d’apprendre que
Swann fréquentait des gens qui les avaient connus. Ma grand’tante au contraire
interpréta cette nouvelle dans un sens défavorable à Swann: quelqu’un qui
choisissait ses fréquentations en dehors de la caste où il était né, en dehors
de sa «classe» sociale, subissait à ses yeux un fâcheux déclassement. Il lui
semblait qu’on renonçât d’un coup au fruit de toutes les belles relations avec
des gens bien posés, qu’avaient honorablement entretenues et engrangées pour
leurs enfants les familles prévoyantes; (ma grand’tante avait même cessé de voir
le fils d’un notaire de nos amis parce qu’il avait épousé une altesse et était
par là descendu pour elle du rang respecté de fils de notaire à celui d’un de
ces aventuriers, anciens valets de chambre ou garçons d’écurie, pour qui on
raconte que les reines eurent parfois des bontés). Elle blâma le projet qu’avait
mon grand-père d’interroger Swann, le soir prochain où il devait venir dîner,
sur ces amis que nous lui découvrions. D’autre part les deux sœurs de ma
grand’mère, vieilles filles qui avaient sa noble nature mais non son esprit,
déclarèrent ne pas comprendre le plaisir que leur beau-frère pouvait trouver à
parler de niaiseries pareilles. C’étaient des personnes d’aspirations élevées et
qui à cause de cela même étaient incapables de s’intéresser à ce qu’on appelle
un potin, eût-il même un intérêt historique, et d’une façon générale à tout ce
qui ne se rattachait pas directement à un objet esthétique ou vertueux. Le
désintéressement de leur pensée était tel, à l’égard de tout ce qui, de près ou
de loin semblait se rattacher à la vie mondaine, que leur sens auditif — ayant
fini par comprendre son inutilité momentanée dès qu’à dîner la conversation
prenait un ton frivole ou seulement terre à terre sans que ces deux vieilles
demoiselles aient pu la ramener aux sujets qui leur étaient chers — mettait
alors au repos ses organes récepteurs et leur laissait subir un véritable
commencement d’atrophie. Si alors mon grand-père avait besoin d’attirer
l’attention des deux sœurs, il fallait qu’il eût recours à ces avertissements
physiques dont usent les médecins aliénistes à l’égard de certains maniaques de
la distraction: coups frappés à plusieurs reprises sur un verre avec la lame
d’un couteau, coïncidant avec une brusque interpellation de la voix et du
regard, moyens violents que ces psychiatres transportent souvent dans les
rapports courants avec des gens bien portants, soit par habitude
professionnelle, soit qu’ils croient tout le monde un peu fou.

Elles furent plus intéressées quand la veille du jour où Swann devait venir
dîner, et leur avait personnellement envoyé une caisse de vin d’Asti, ma tante,
tenant un numéro du Figaro où à côté du nom d’un tableau qui était à une
Exposition de Corot, il y avait ces mots: «de la collection de M. Charles
Swann», nous dit: «Vous avez vu que Swann a «les honneurs» du Figaro?»—«Mais je
vous ai toujours dit qu’il avait beaucoup de goût», dit ma grand’mère.
«Naturellement toi, du moment qu’il s’agit d’être d’un autre avis que nous»,
répondit ma grand’tante qui, sachant que ma grand’mère n’était jamais du même
avis qu’elle, et n’étant bien sûre que ce fût à elle-même que nous donnions
toujours raison, voulait nous arracher une condamnation en bloc des opinions de
ma grand’mère contre lesquelles elle tâchait de nous solidariser de force avec
les siennes. Mais nous restâmes silencieux. Les sœurs de ma grand’mère ayant
manifesté l’intention de parler à Swann de ce mot du Figaro, ma grand’tante le
leur déconseilla. Chaque fois qu’elle voyait aux autres un avantage si petit
fût-il qu’elle n’avait pas, elle se persuadait que c’était non un avantage mais
un mal et elle les plaignait pour ne pas avoir à les envier. «Je crois que vous
ne lui feriez pas plaisir; moi je sais bien que cela me serait très désagréable
de voir mon nom imprimé tout vif comme cela dans le journal, et je ne serais pas
flattée du tout qu’on m’en parlât.» Elle ne s’entêta pas d’ailleurs à persuader
les sœurs de ma grand’mère; car celles-ci par horreur de la vulgarité poussaient
si loin l’art de dissimuler sous des périphrases ingénieuses une allusion
personnelle qu’elle passait souvent inaperçue de celui même à qui elle
s’adressait. Quant à ma mère elle ne pensait qu’à tâcher d’obtenir de mon père
qu’il consentît à parler à Swann non de sa femme mais de sa fille qu’il adorait
et à cause de laquelle disait-on il avait fini par faire ce mariage. «Tu
pourrais ne lui dire qu’un mot, lui demander comment elle va. Cela doit être si
cruel pour lui.» Mais mon père se fâchait: «Mais non! tu as des idées absurdes.
Ce serait ridicule.»

Mais le seul d’entre nous pour qui la venue de Swann devint l’objet d’une
préoccupation douloureuse, ce fut moi. C’est que les soirs où des étrangers, ou
seulement M. Swann, étaient là, maman ne montait pas dans ma chambre. 
Je dînais avant tout le monde et je venais ensuite
m’asseoir à table, jusqu’à huit heures où il était convenu que je devais monter;
ce baiser précieux et fragile que maman me confiait d’habitude dans mon lit au
moment de m’endormir il me fallait le transporter de la salle à manger dans ma
chambre et le garder pendant tout le temps que je me déshabillais, sans que se
brisât sa douceur, sans que se répandît et s’évaporât sa vertu volatile et,
justement ces soirs-là où j’aurais eu besoin de le recevoir avec plus de
précaution, il fallait que je le prisse, que je le dérobasse brusquement,
publiquement, sans même avoir le temps et la liberté d’esprit nécessaires pour
porter à ce que je faisais cette attention des maniaques qui s’efforcent de ne
pas penser à autre chose pendant qu’ils ferment une porte, pour pouvoir, quand
l’incertitude maladive leur revient, lui opposer victorieusement le souvenir du
moment où ils l’ont fermée. Nous étions tous au jardin quand retentirent les
deux coups hésitants de la clochette. On savait que c’était Swann; néanmoins
tout le monde se regarda d’un air interrogateur et on envoya ma grand’mère en
reconnaissance. «Pensez à le remercier intelligiblement de son vin, vous savez
qu’il est délicieux et la caisse est énorme», recommanda mon grand-père à ses
deux belles-sœurs. «Ne commencez pas à chuchoter, dit ma grand’tante. Comme
c’est confortable d’arriver dans une maison où tout le monde parle bas! -- Ah!
voilà M. Swann. Nous allons lui demander s’il croit qu’il fera beau demain», dit
mon père. Ma mère pensait qu’un mot d’elle effacerait toute la peine que dans
notre famille on avait pu faire à Swann depuis son mariage. Elle trouva le moyen
de l’emmener un peu à l’écart. Mais je la suivis; je ne pouvais me décider à la
quitter d’un pas en pensant que tout à l’heure il faudrait que je la laisse dans
la salle à manger et que je remonte dans ma chambre sans avoir comme les autres
soirs la consolation qu’elle vînt m’embrasser. «Voyons, monsieur Swann, lui
dit-elle, parlez-moi un peu de votre fille; je suis sûre qu’elle a déjà le goût
des belles œuvres comme son papa. -- Mais venez donc vous asseoir avec nous tous
sous la véranda», dit mon grand-père en s’approchant. Ma mère fut obligée de
s’interrompre, mais elle tira de cette contrainte même une pensée délicate de
plus, comme les bons poètes que la tyrannie de la rime force à trouver leurs
plus grandes beautés: «Nous reparlerons d’elle quand nous serons tous les deux,
dit-elle à mi-voix à Swann. Il n’y a qu’une maman qui soit digne de vous
comprendre. Je suis sûre que la sienne serait de mon avis.» Nous nous assîmes
tous autour de la table de fer. J’aurais voulu ne pas penser aux heures
d’angoisse que je passerais ce soir seul dans ma chambre sans pouvoir
m’endormir; je tâchais de me persuader qu’elles n’avaient aucune importance,
puisque je les aurais oubliées demain matin, de m’attacher à des idées d’avenir
qui auraient dû me conduire comme sur un pont au delà de l’abîme prochain qui
m’effrayait. Mais mon esprit tendu par ma préoccupation, rendu convexe comme le
regard que je dardais sur ma mère, ne se laissait pénétrer par aucune impression
étrangère. Les pensées entraient bien en lui, mais à condition de laisser dehors
tout élément de beauté ou simplement de drôlerie qui m’eût touché ou distrait.
Comme un malade, grâce à un anesthésique, assiste avec une pleine lucidité à
l’opération qu’on pratique sur lui, mais sans rien sentir, je pouvais me réciter
des vers que j’aimais ou observer les efforts que mon grand-père faisait pour
parler à Swann du duc d’Audiffret-Pasquier, sans que les premiers me fissent
éprouver aucune émotion, les seconds aucune gaîté. Ces efforts furent
infructueux. À peine mon grand-père eut-il posé à Swann une question relative à
cet orateur qu’une des sœurs de ma grand’mère aux oreilles de qui cette question
résonna comme un silence profond mais intempestif et qu’il était poli de rompre,
interpella l’autre: «Imagine-toi, Céline, que j’ai fait la connaissance d’une
jeune institutrice suédoise qui m’a donné sur les coopératives dans les pays
scandinaves des détails tout ce qu’il y a de plus intéressants. Il faudra
qu’elle vienne dîner ici un soir. -- Je crois bien! répondit sa sœur Flora, mais
je n’ai pas perdu mon temps non plus. J’ai rencontré chez M. Vinteuil un vieux
savant qui connaît beaucoup Maubant, et à qui Maubant a expliqué dans le plus
grand détail comment il s’y prend pour composer un rôle. C’est tout ce qu’il y a
de plus intéressant. C’est un voisin de M. Vinteuil, je n’en savais rien; et il
est très aimable. -- Il n’y a pas que M. Vinteuil qui ait des voisins aimables»,
s’écria ma tante Céline d’une voix que la timidité rendait forte et la
préméditation, factice, tout en jetant sur Swann ce qu’elle appelait un regard
significatif. En même temps ma tante Flora qui avait compris que cette phrase
était le remerciement de Céline pour le vin d’Asti, regardait également Swann
avec un air mêlé de congratulation et d’ironie, soit simplement pour souligner
le trait d’esprit da sa sœur, soit qu’elle enviât Swann de l’avoir inspiré, soit
qu’elle ne pût s’empêcher de se moquer de lui parce qu’elle le croyait sur la
sellette. «Je crois qu’on pourra réussir à avoir ce monsieur à dîner, continua
Flora; quand on le met sur Maubant ou sur Mme Materna, il parle des heures sans
s’arrêter. -- Ce doit être délicieux», soupira mon grand-père dans l’esprit de
qui la nature avait malheureusement aussi complètement omis d’inclure la
possibilité de s’intéresser passionnément aux coopératives suédoises ou à la
composition des rôles de Maubant, qu’elle avait oublié de fournir celui des
sœurs de ma grand’mère du petit grain de sel qu’il faut ajouter soi-même pour y
trouver quelque saveur, à un récit sur la vie intime de Molé ou du comte de
Paris. «Tenez, dit Swann à mon grand-père, ce que je vais vous dire a plus de
rapports que cela n’en a l’air avec ce que vous me demandiez, car sur certains
points les choses n’ont pas énormément changé. Je relisais ce matin dans
Saint-Simon quelque chose qui vous aurait amusé. C’est dans le volume sur son
ambassade d’Espagne; ce n’est pas un des meilleurs, ce n’est guère qu’un
journal, mais du moins un journal merveilleusement écrit, ce qui fait déjà une
première différence avec les assommants journaux que nous nous croyons obligés
de lire matin et soir. -- Je ne suis pas de votre avis, il y a des jours où la
lecture des journaux me semble fort agréable . . . », interrompit ma tante
Flora, pour montrer qu’elle avait lu la phrase sur le Corot de Swann dans le
Figaro. «Quand ils parlent de choses ou de gens qui nous intéressent!» enchérit
ma tante Céline. «Je ne dis pas non, répondit Swann étonné. Ce que je reproche
aux journaux c’est de nous faire faire attention tous les jours à des choses
insignifiantes tandis que nous lisons trois ou quatre fois dans notre vie les
livres où il y a des choses essentielles. Du moment que nous déchirons
fiévreusement chaque matin la bande du journal, alors on devrait changer les
choses et mettre dans le journal, moi je ne sais pas, les . . . Pensées de
Pascal! (il détacha ce mot d’un ton d’emphase ironique pour ne pas avoir l’air
pédant). Et c’est dans le volume doré sur tranches que nous n’ouvrons qu’une
fois tous les dix ans»,  ajouta-t-il en témoignant pour les choses mondaines ce
dédain qu’affectent certains hommes du monde, «que nous lirions que la reine de
Grèce est allée à Cannes ou que la princesse de Léon a donné un bal costumé.
Comme cela la juste proportion serait rétablie.» Mais regrettant de s’être
laissé aller à parler même légèrement de choses sérieuses: «Nous avons une bien
belle conversation, dit-il ironiquement, je ne sais pas pourquoi nous abordons
ces ``sommets''», et se tournant vers mon grand-père: «Donc Saint-Simon raconte que
Maulevrier avait eu l’audace de tendre la main à ses fils. Vous savez, c’est ce
Maulevrier dont il dit: ``Jamais je ne vis dans cette épaisse bouteille que de
l’humeur, de la grossièreté et des sottises.'' -- Épaisses ou non, je connais des
bouteilles où il y a tout autre chose», dit vivement Flora, qui tenait à avoir
remercié Swann elle aussi, car le présent de vin d’Asti s’adressait aux deux.
Céline se mit à rire. Swann interloqué reprit: «''Je ne sais si ce fut ignorance
ou panneau'', écrit Saint-Simon, ``il voulut donner la main à mes enfants. Je m’en
aperçus assez tôt pour l’en empêcher.''» Mon grand-père s’extasiait déjà sur
«ignorance ou panneau», mais Mlle Céline, chez qui le nom de Saint-Simon — un
littérateur — avait empêché l’anesthésie complète des facultés auditives,
s’indignait déjà: «Comment? vous admirez cela? Eh bien! c’est du joli! Mais
qu’est-ce que cela peut vouloir dire; est-ce qu’un homme n’est pas autant qu’un
autre? Qu’est-ce que cela peut faire qu’il soit duc ou cocher s’il a de
l’intelligence et du cœur? Il avait une belle manière d’élever ses enfants,
votre Saint-Simon, s’il ne leur disait pas de donner la main à tous les honnêtes
gens. Mais c’est abominable, tout simplement. Et vous osez citer cela?» Et mon
grand-père navré, sentant l’impossibilité, devant cette obstruction, de chercher
à faire raconter à Swann, les histoires qui l’eussent amusé disait à voix basse
à maman: «Rappelle-moi donc le vers que tu m’as appris et qui me soulage tant
dans ces moments-là. Ah! oui: ``Seigneur, que de vertus vous nous faites haïr!”
Ah! comme c’est bien!»

Je ne quittais pas ma mère des yeux, je savais que quand on serait à table, on
ne me permettrait pas de rester pendant toute la durée du dîner et que pour ne
pas contrarier mon père, maman ne me laisserait pas l’embrasser à plusieurs
reprises devant le monde, comme si ç’avait été dans ma chambre. Aussi je me
promettais, dans la salle à manger, pendant qu’on commencerait à dîner et que je
sentirais approcher l’heure, de faire d’avance de ce baiser qui serait si court
et furtif, tout ce que j’en pouvais faire seul, de choisir avec mon regard la
place de la joue que j’embrasserais, de préparer ma pensée pour pouvoir grâce à
ce commencement mental de baiser consacrer toute la minute que m’accorderait
maman à sentir sa joue contre mes lèvres, comme un peintre qui ne peut obtenir
que de courtes séances de pose, prépare sa palette, et a fait d’avance de
souvenir, d’après ses notes, tout ce pour quoi il pouvait à la rigueur se passer
de la présence du modèle. Mais voici qu’avant que le dîner fût sonné mon
grand-père eut la férocité inconsciente de dire: «Le petit a l’air fatigué, il
devrait monter se coucher. On dîne tard du reste ce soir.» Et mon père, qui ne
gardait pas aussi scrupuleusement que ma grand’mère et que ma mère la foi des
traités, dit: «Oui, allons, vas te coucher.» Je voulus embrasser maman, à cet
instant on entendit la cloche du dîner. «Mais non, voyons, laisse ta mère, vous
vous êtes assez dit bonsoir comme cela, ces manifestations sont ridicules.
Allons, monte!» Et il me fallut partir sans viatique; il me fallut monter chaque
marche de l’escalier, comme dit l’expression populaire, à «contre-cœur», montant
contre mon cœur qui voulait retourner près de ma mère parce qu’elle ne lui avait
pas, en m’embrassant, donné licence de me suivre. Cet escalier détesté où je
m’engageais toujours si tristement, exhalait une odeur de vernis qui avait en
quelque sorte absorbé, fixé, cette sorte particulière de chagrin que je
ressentais chaque soir et la rendait peut-être plus cruelle encore pour ma
sensibilité parce que sous cette forme olfactive mon intelligence n’en pouvait
plus prendre sa part. Quand nous dormons et qu’une rage de dents n’est encore
perçue par nous que comme une jeune fille que nous nous efforçons deux cents
fois de suite de tirer de l’eau ou que comme un vers de Molière que nous nous
répétons sans arrêter, c’est un grand soulagement de nous réveiller et que notre
intelligence puisse débarrasser l’idée de rage de dents, de tout déguisement
héroïque ou cadencé. C’est l’inverse de ce soulagement que j’éprouvais quand mon
chagrin de monter dans ma chambre entrait en moi d’une façon infiniment plus
rapide, presque instantanée, à la fois insidieuse et brusque, par l’inhalation —
beaucoup plus toxique que la pénétration morale — de l’odeur de vernis
particulière à cet escalier. Une fois dans ma chambre, il fallut boucher toutes
les issues, fermer les volets, creuser mon propre tombeau, en défaisant mes
couvertures, revêtir le suaire de ma chemise de nuit. Mais avant de m’ensevelir
dans le lit de fer qu’on avait ajouté dans la chambre parce que j’avais trop
chaud l’été sous les courtines de reps du grand lit, j’eus un mouvement de
révolte, je voulus essayer d’une ruse de condamné. J’écrivis à ma mère en la
suppliant de monter pour une chose grave que je ne pouvais lui dire dans ma
lettre. Mon effroi était que Françoise, la cuisinière de ma tante qui était
chargée de s’occuper de moi quand j’étais à Combray, refusât de porter mon mot.
Je me doutais que pour elle, faire une commission à ma mère quand il y avait du
monde lui paraîtrait aussi impossible que pour le portier d’un théâtre de
remettre une lettre à un acteur pendant qu’il est en scène. Elle possédait à
l’égard des choses qui peuvent ou ne peuvent pas se faire un code impérieux,
abondant, subtil et intransigeant sur des distinctions insaisissables ou
oiseuses (ce qui lui donnait l’apparence de ces lois antiques qui, à côté de
prescriptions féroces comme de massacrer les enfants à la mamelle, défendent
avec une délicatesse exagérée de faire bouillir le chevreau dans le lait de sa
mère, ou de manger dans un animal le nerf de la cuisse). Ce code, si l’on en
jugeait par l’entêtement soudain qu’elle mettait à ne pas vouloir faire
certaines commissions que nous lui donnions, semblait avoir prévu des
complexités sociales et des raffinements mondains tels que rien dans l’entourage
de Françoise et dans sa vie de domestique de village n’avait pu les lui
suggérer; et l’on était obligé de se dire qu’il y avait en elle un passé
français très ancien, noble et mal compris, comme dans ces cités manufacturières
où de vieux hôtels témoignent qu’il y eut jadis une vie de cour, et où les
ouvriers d’une usine de produits chimiques travaillent au milieu de délicates
sculptures qui représentent le miracle de saint Théophile ou les quatre fils
Aymon. Dans le cas particulier, l’article du code à cause duquel il était peu
probable que sauf le cas d’incendie Françoise allât déranger maman en présence
de M. Swann pour un aussi petit personnage que moi, exprimait simplement le
respect qu’elle professait non seulement pour les parents — comme pour les
morts, les prêtres et les rois — mais encore pour l’étranger à qui on donne
l’hospitalité, respect qui m’aurait peut-être touché dans un livre mais qui
m’irritait toujours dans sa bouche, à cause du ton grave et attendri qu’elle
prenait pour en parler, et davantage ce soir où le caractère sacré qu’elle
conférait au dîner avait pour effet qu’elle refuserait d’en troubler la
cérémonie. Mais pour mettre une chance de mon côté, je n’hésitai pas à mentir et
à lui dire que ce n’était pas du tout moi qui avais voulu écrire à maman, mais
que c’était maman qui, en me quittant, m’avait recommandé de ne pas oublier de
lui envoyer une réponse relativement à un objet qu’elle m’avait prié de
chercher; et elle serait certainement très fâchée si on ne lui remettait pas ce
mot. Je pense que Françoise ne me crut pas, car, comme les hommes primitifs dont
les sens étaient plus puissants que les nôtres, elle discernait immédiatement, à
des signes insaisissables pour nous, toute vérité que nous voulions lui cacher;
elle regarda pendant cinq minutes l’enveloppe comme si l’examen du papier et
l’aspect de l’écriture allaient la renseigner sur la nature du contenu ou lui
apprendre à quel article de son code elle devait se référer. Puis elle sortit
d’un air résigné qui semblait signifier: «C’est-il pas malheureux pour des
parents d’avoir un enfant pareil!» Elle revint au bout d’un moment me dire qu’on
n’en était encore qu’à la glace, qu’il était impossible au maître d’hôtel de
remettre la lettre en ce moment devant tout le monde, mais que, quand on serait
aux rince-bouche, on trouverait le moyen de la faire passer à maman. Aussitôt
mon anxiété tomba; maintenant ce n’était plus comme tout à l’heure pour jusqu’à
demain que j’avais quitté ma mère, puisque mon petit mot allait, la fâchant sans
doute (et doublement parce que ce manège me rendrait ridicule aux yeux de
Swann), me faire du moins entrer invisible et ravi dans la même pièce qu’elle,
allait lui parler de moi à l’oreille; puisque cette salle à manger interdite,
hostile, où, il y avait un instant encore, la glace elle-même — le «granité»— et
les rince-bouche me semblaient recéler des plaisirs malfaisants et mortellement
tristes parce que maman les goûtait loin de moi, s’ouvrait à moi et, comme un
fruit devenu doux qui brise son enveloppe, allait faire jaillir, projeter
jusqu’à mon cœur enivré l’attention de maman tandis qu’elle lirait mes lignes.
Maintenant je n’étais plus séparé d’elle; les barrières étaient tombées, un fil
délicieux nous réunissait. Et puis, ce n’était pas tout: maman allait sans doute
venir!

L’angoisse que je venais d’éprouver, je pensais que Swann s’en serait bien moqué
s’il avait lu ma lettre et en avait deviné le but; or, au contraire, comme je
l’ai appris plus tard, une angoisse semblable fut le tourment de longues années
de sa vie et personne, aussi bien que lui peut-être, n’aurait pu me comprendre;
lui, cette angoisse qu’il y a à sentir l’être qu’on aime dans un lieu de plaisir
où l’on n’est pas, où l’on ne peut pas le rejoindre, c’est l’amour qui la lui a
fait connaître, l’amour auquel elle est en quelque sorte prédestinée, par lequel
elle sera accaparée, spécialisée; mais quand, comme pour moi, elle est entrée en
nous avant qu’il ait encore fait son apparition dans notre vie, elle flotte en
l’attendant, vague et libre, sans affectation déterminée, au service un jour
d’un sentiment, le lendemain d’un autre, tantôt de la tendresse filiale ou de
l’amitié pour un camarade. Et la joie avec laquelle je fis mon premier
apprentissage quand Françoise revint me dire que ma lettre serait remise, Swann
l’avait bien connue aussi cette joie trompeuse que nous donne quelque ami,
quelque parent de la femme que nous aimons, quand arrivant à l’hôtel ou au
théâtre où elle se trouve, pour quelque bal, redoute, ou première où il va la
retrouver, cet ami nous aperçoit errant dehors, attendant désespérément quelque
occasion de communiquer avec elle. Il nous reconnaît, nous aborde familièrement,
nous demande ce que nous faisons là. Et comme nous inventons que nous avons
quelque chose d’urgent à dire à sa parente ou amie, il nous assure que rien
n’est plus simple, nous fait entrer dans le vestibule et nous promet de nous
l’envoyer avant cinq minutes. Que nous l’aimons — comme en ce moment j’aimais
Françoise — l’intermédiaire bien intentionné qui d’un mot vient de nous rendre
supportable, humaine et presque propice la fête inconcevable, infernale, au sein
de laquelle nous croyions que des tourbillons ennemis, pervers et délicieux
entraînaient loin de nous, la faisant rire de nous, celle que nous aimons. Si
nous en jugeons par lui, le parent qui nous a accosté et qui est lui aussi un
des initiés des cruels mystères, les autres invités de la fête ne doivent rien
avoir de bien démoniaque. Ces heures inaccessibles et suppliciantes où elle
allait goûter des plaisirs inconnus, voici que par une brèche inespérée nous y
pénétrons; voici qu’un des moments dont la succession les aurait composées, un
moment aussi réel que les autres, même peut-être plus important pour nous, parce
que notre maîtresse y est plus mêlée, nous nous le représentons, nous le
possédons, nous y intervenons, nous l’avons créé presque: le moment où on va lui
dire que nous sommes là, en bas. Et sans doute les autres moments de la fête ne
devaient pas être d’une essence bien différente de celui-là, ne devaient rien
avoir de plus délicieux et qui dût tant nous faire souffrir puisque l’ami
bienveillant nous a dit: «Mais elle sera ravie de descendre! Cela lui fera
beaucoup plus de plaisir de causer avec vous que de s’ennuyer là-haut.» Hélas!
Swann en avait fait l’expérience, les bonnes intentions d’un tiers sont sans
pouvoir sur une femme qui s’irrite de se sentir poursuivie jusque dans une fête
par quelqu’un qu’elle n’aime pas. Souvent, l’ami redescend seul.

Ma mère ne vint pas, et sans ménagements pour mon amour-propre (engagé à ce que
la fable de la recherche dont elle était censée m’avoir prié de lui dire le
résultat ne fût pas démentie) me fit dire par Françoise ces mots: «Il n’y a pas
de réponse» que depuis j’ai si souvent entendu des concierges de «palaces» ou
des valets de pied de tripots, rapporter à quelque pauvre fille qui s’étonne:
«Comment, il n’a rien dit, mais c’est impossible! Vous avez pourtant bien remis
ma lettre. C’est bien, je vais attendre encore.» Et — de même qu’elle assure
invariablement n’avoir pas besoin du bec supplémentaire que le concierge veut
allumer pour elle, et reste là, n’entendant plus que les rares propos sur le
temps qu’il fait échanger entre le concierge et un chasseur qu’il envoie tout
d’un coup en s’apercevant de l’heure, faire rafraîchir dans la glace la boisson
d’un client — ayant décliné l’offre de Françoise de me faire de la tisane ou de
rester auprès de moi, je la laissai retourner à l’office, je me couchai et je
fermai les yeux en tâchant de ne pas entendre la voix de mes parents qui
prenaient le café au jardin. Mais au bout de quelques secondes, je sentis qu’en
écrivant ce mot à maman, en m’approchant, au risque de la fâcher, si près d’elle
que j’avais cru toucher le moment de la revoir, je m’étais barré la possibilité
de m’endormir sans l’avoir revue, et les battements de mon cœur, de minute en
minute devenaient plus douloureux parce que j’augmentais mon agitation en me
prêchant un calme qui était l’acceptation de mon infortune. Tout à coup mon
anxiété tomba, une félicité m’envahit comme quand un médicament puissant
commence à agir et nous enlève une douleur: je venais de prendre la résolution
de ne plus essayer de m’endormir sans avoir revu maman, de l’embrasser coûte que
coûte, bien que ce fût avec la certitude d’être ensuite fâché pour longtemps
avec elle, quand elle remonterait se coucher. Le calme qui résultait de mes
angoisses finies me mettait dans un allégresse extraordinaire, non moins que
l’attente, la soif et la peur du danger. J’ouvris la fenêtre sans bruit et
m’assis au pied de mon lit; je ne faisais presque aucun mouvement afin qu’on ne
m’entendît pas d’en bas. Dehors, les choses semblaient, elles aussi, figées en
une muette attention à ne pas troubler le clair de lune, qui doublant et
reculant chaque chose par l’extension devant elle de son reflet, plus dense et
concret qu’elle-même, avait à la fois aminci et agrandi le paysage comme un plan
replié jusque-là, qu’on développe. Ce qui avait besoin de bouger, quelque
feuillage de marronnier, bougeait. Mais son frissonnement minutieux, total,
exécuté jusque dans ses moindres nuances et ses dernières délicatesses, ne
bavait pas sur le reste, ne se fondait pas avec lui, restait circonscrit.
Exposés sur ce silence qui n’en absorbait rien, les bruits les plus éloignés,
ceux qui devaient venir de jardins situés à l’autre bout de la ville, se
percevaient détaillés avec un tel «fini» qu’ils semblaient ne devoir cet effet
de lointain qu’à leur pianissimo, comme ces motifs en sourdine si bien exécutés
par l’orchestre du Conservatoire que quoiqu’on n’en perde pas une note on croit
les entendre cependant loin de la salle du concert et que tous les vieux abonnés
— les sœurs de ma grand’mère aussi quand Swann leur avait donné ses places —
tendaient l’oreille comme s’ils avaient écouté les progrès lointains d’une armée
en marche qui n’aurait pas encore tourné la rue de Trévise.

Je savais que le cas dans lequel je me mettais était de tous celui qui pouvait
avoir pour moi, de la part de mes parents, les conséquences les plus graves,
bien plus graves en vérité qu’un étranger n’aurait pu le supposer, de celles
qu’il aurait cru que pouvaient produire seules des fautes vraiment honteuses.
Mais dans l’éducation qu’on me donnait, l’ordre des fautes n’était pas le même
que dans l’éducation des autres enfants et on m’avait habitué à placer avant
toutes les autres (parce que sans doute il n’y en avait pas contre lesquelles
j’eusse besoin d’être plus soigneusement gardé) celles dont je comprends
maintenant que leur caractère commun est qu’on y tombe en cédant à une impulsion
nerveuse. Mais alors on ne prononçait pas ce mot, on ne déclarait pas cette
origine qui aurait pu me faire croire que j’étais excusable d’y succomber ou
même peut-être incapable d’y résister. Mais je les reconnaissais bien à
l’angoisse qui les précédait comme à la rigueur du châtiment qui les suivait; et
je savais que celle que je venais de commettre était de la même famille que
d’autres pour lesquelles j’avais été sévèrement puni, quoique infiniment plus
grave. Quand j’irais me mettre sur le chemin de ma mère au moment où elle
monterait se coucher, et qu’elle verrait que j’étais resté levé pour lui redire
bonsoir dans le couloir, on ne me laisserait plus rester à la maison, on me
mettrait au collège le lendemain, c’était certain. Eh bien! dussé-je me jeter
par la fenêtre cinq minutes après, j’aimais encore mieux cela. Ce que je voulais
maintenant c’était maman, c’était lui dire bonsoir, j’étais allé trop loin dans
la voie qui menait à la réalisation de ce désir pour pouvoir rebrousser chemin.

J’entendis les pas de mes parents qui accompagnaient Swann; et quand le grelot
de la porte m’eut averti qu’il venait de partir, j’allai à la fenêtre. Maman
demandait à mon père s’il avait trouvé la langouste bonne et si M. Swann avait
repris de la glace au café et à la pistache. «Je l’ai trouvée bien quelconque,
dit ma mère; je crois que la prochaine fois il faudra essayer d’un autre
parfum.» «Je ne peux pas dire comme je trouve que Swann change, dit ma
grand’tante, il est d’un vieux!» Ma grand’tante avait tellement l’habitude de
voir toujours en Swann un même adolescent, qu’elle s’étonnait de le trouver tout
à coup moins jeune que l’âge qu’elle continuait à lui donner. Et mes parents du
reste commençaient à lui trouver cette vieillesse anormale, excessive, honteuse
et méritée des célibataires, de tous ceux pour qui il semble que le grand jour
qui n’a pas de lendemain soit plus long que pour les autres, parce que pour eux
il est vide et que les moments s’y additionnent depuis le matin sans se diviser
ensuite entre des enfants. «Je crois qu’il a beaucoup de soucis avec sa coquine
de femme qui vit au su de tout Combray avec un certain monsieur de Charlus.
C’est la fable de la ville.» Ma mère fit remarquer qu’il avait pourtant l’air
bien moins triste depuis quelque temps. «Il fait aussi moins souvent ce geste
qu’il a tout à fait comme son père de s’essuyer les yeux et de se passer la main
sur le front. Moi je crois qu’au fond il n’aime plus cette femme. -- Mais
naturellement il ne l’aime plus, répondit mon grand-père. J’ai reçu de lui il y
a déjà longtemps une lettre à ce sujet, à laquelle je me suis empressé de ne pas
me conformer, et qui ne laisse aucun doute sur ses sentiments au moins d’amour,
pour sa femme. Hé bien! vous voyez, vous ne l’avez pas remercié pour l’Asti»,
ajouta mon grand-père en se tournant vers ses deux belles-sœurs. «Comment, nous
ne l’avons pas remercié? je crois, entre nous, que je lui ai même tourné cela
assez délicatement, répondit ma tante Flora. -- Oui, tu as très bien arrangé
cela: je t’ai admirée, dit ma tante Céline. -- Mais toi tu as été très bien
aussi. -- Oui j’étais assez fière de ma phrase sur les voisins aimables. -- Comment, 
c’est cela que vous appelez remercier! s’écria mon grand-père. J’ai
bien entendu cela, mais du diable si j’ai cru que c’était pour Swann. Vous
pouvez être sûres qu’il n’a rien compris. -- Mais voyons, Swann n’est pas bête,
je suis certaine qu’il a apprécié. Je ne pouvais cependant pas lui dire le
nombre de bouteilles et le prix du vin!» Mon père et ma mère restèrent seuls, et
s’assirent un instant; puis mon père dit: «Hé bien! si tu veux, nous allons
monter nous coucher. -- Si tu veux, mon ami, bien que je n’aie pas l’ombre de
sommeil; ce n’est pas cette glace au café si anodine qui a pu pourtant me tenir
si éveillée; mais j’aperçois de la lumière dans l’office et puisque la pauvre
Françoise m’a attendue, je vais lui demander de dégrafer mon corsage pendant que
tu vas te déshabiller.» Et ma mère ouvrit la porte treillagée du vestibule qui
donnait sur l’escalier. Bientôt, je l’entendis qui montait fermer sa fenêtre.
J’allai sans bruit dans le couloir; mon cœur battait si fort que j’avais de la
peine à avancer, mais du moins il ne battait plus d’anxiété, mais d’épouvante et
de joie. Je vis dans la cage de l’escalier la lumière projetée par la bougie de
maman. Puis je la vis elle-même; je m’élançai. A la première seconde, elle me
regarda avec étonnement, ne comprenant pas ce qui était arrivé. Puis sa figure
prit une expression de colère, elle ne me disait même pas un mot, et en effet
pour bien moins que cela on ne m’adressait plus la parole pendant plusieurs
jours. Si maman m’avait dit un mot, ç’aurait été admettre qu’on pouvait me
reparler et d’ailleurs cela peut-être m’eût paru plus terrible encore, comme un
signe que devant la gravité du châtiment qui allait se préparer, le silence, la
brouille, eussent été puérils. Une parole c’eût été le calme avec lequel on
répond à un domestique quand on vient de décider de le renvoyer; le baiser qu’on
donne à un fils qu’on envoie s’engager alors qu’on le lui aurait refusé si on
devait se contenter d’être fâché deux jours avec lui. Mais elle entendit mon
père qui montait du cabinet de toilette où il était allé se déshabiller et pour
éviter la scène qu’il me ferait, elle me dit d’une voix entrecoupée par la
colère: «Sauve-toi, sauve-toi, qu’au moins ton père ne t’ait vu ainsi attendant
comme un fou!» Mais je lui répétais: «Viens me dire bonsoir», terrifié en voyant
que le reflet de la bougie de mon père s’élevait déjà sur le mur, mais aussi
usant de son approche comme d’un moyen de chantage et espérant que maman, pour
éviter que mon père me trouvât encore là si elle continuait à refuser, allait me
dire: «Rentre dans ta chambre, je vais venir.» Il était trop tard, mon père
était devant nous. Sans le vouloir, je murmurai ces mots que personne
n’entendit: «Je suis perdu!»

Il n’en fut pas ainsi. Mon père me refusait constamment des permissions qui
m’avaient été consenties dans les pactes plus larges octroyés par ma mére et ma
grand’mère parce qu’il ne se souciait pas des «principes» et qu’il n’y avait pas
avec lui de «Droit des gens». Pour une raison toute contingente, ou même sans
raison, il me supprimait au dernier moment telle promenade si habituelle, si
consacrée, qu’on ne pouvait m’en priver sans parjure, ou bien, comme il avait
encore fait ce soir, longtemps avant l’heure rituelle, il me disait: «Allons,
monte te coucher, pas d’explication!» Mais aussi, parce qu’il n’avait pas de
principes (dans le sens de ma grand’mère), il n’avait pas à proprement parler
d’intransigeance. Il me regarda un instant d’un air étonné et fâché, puis dès
que maman lui eut expliqué en quelques mots embarrassés ce qui était arrivé, il
lui dit: «Mais va donc avec lui, puisque tu disais justement que tu n’as pas
envie de dormir, reste un peu dans sa chambre, moi je n’ai besoin de rien. -- Mais, 
mon ami, répondit timidement ma mère, que j’aie envie ou non de dormir,
ne change rien à la chose, on ne peut pas habituer cet enfant . . .  -- Mais il
ne s’agit pas d’habituer, dit mon père en haussant les épaules, tu vois bien que
ce petit a du chagrin, il a l’air désolé, cet enfant; voyons, nous ne sommes pas
des bourreaux! Quand tu l’auras rendu malade, tu seras bien avancée! Puisqu’il y
a deux lits dans sa chambre, dis donc à Françoise de te préparer le grand lit et
couche pour cette nuit auprès de lui. Allons, bonsoir, moi qui ne suis pas si
nerveux que vous, je vais me coucher.»

On ne pouvait pas remercier mon père; on l’eût agacé par ce qu’il appelait des
sensibleries. Je restai sans oser faire un mouvement; il était encore devant
nous, grand, dans sa robe de nuit blanche sous le cachemire de l’Inde violet et
rose qu’il nouait autour de sa tête depuis qu’il avait des névralgies, avec le
geste d’Abraham dans la gravure d’après Benozzo Gozzoli que m’avait donnée M.
Swann, disant à Sarah qu’elle a à se départir du côté d’Ïsaac. Il y a bien des
années de cela. La muraille de l’escalier, où je vis monter le reflet de sa
bougie n’existe plus depuis longtemps. En moi aussi bien des choses ont été
détruites que je croyais devoir durer toujours et de nouvelles se sont édifiées
donnant naissance à des peines et à des joies nouvelles que je n’aurais pu
prévoir alors, de même que les anciennes me sont devenues difficiles à
comprendre. Il y a bien longtemps aussi que mon père a cessé de pouvoir dire à
maman: «Va avec le petit.» La possibilité de telles heures ne renaîtra jamais
pour moi. Mais depuis peu de temps, je recommence à très bien percevoir si je
prête l’oreille, les sanglots que j’eus la force de contenir devant mon père et
qui n’éclatèrent que quand je me retrouvai seul avec maman. En réalité ils n’ont
jamais cessé; et c’est seulement parce que la vie se tait maintenant davantage
autour de moi que je les entends de nouveau, comme ces cloches de couvents que
couvrent si bien les bruits de la ville pendant le jour qu’on les croirait
arrêtées mais qui se remettent à sonner dans le silence du soir.

Maman passa cette nuit-là dans ma chambre; au moment où je venais de commettre
une faute telle que je m’attendais à être obligé de quitter la maison, mes
parents m’accordaient plus que je n’eusse jamais obtenu d’eux comme récompense
d’une belle action. Même à l’heure où elle se manifestait par cette grâce, la
conduite de mon père à mon égard gardait ce quelque chose d’arbitraire et
d’immérité qui la caractérisait et qui tenait à ce que généralement elle
résultait plutôt de convenances fortuites que d’un plan prémédité. Peut-être
même que ce que j’appelais sa sévérité, quand il m’envoyait me coucher, méritait
moins ce nom que celle de ma mère ou ma grand’mère, car sa nature, plus
différente en certains points de la mienne que n’était la leur, n’avait
probablement pas deviné jusqu’ici combien j’étais malheureux tous les soirs, ce
que ma mère et ma grand’mère savaient bien; mais elles m’aimaient assez pour ne
pas consentir à m’épargner de la souffrance, elles voulaient m’apprendre à la
dominer afin de diminuer ma sensibilité nerveuse et fortifier ma volonté. Pour
mon père, dont l’affection pour moi était d’une autre sorte, je ne sais pas s’il
aurait eu ce courage: pour une fois où il venait de comprendre que j’avais du
chagrin, il avait dit à ma mère: «Va donc le consoler.» Maman resta cette
nuit-là dans ma chambre et, comme pour ne gâter d’aucun remords ces heures si
différentes de ce que j’avais eu le droit d’espérer, quand Françoise, comprenant
qu’il se passait quelque chose d’extraordinaire en voyant maman assise près de
moi, qui me tenait la main et me laissait pleurer sans me gronder, lui demanda:
«Mais Madame, qu’a donc Monsieur à pleurer ainsi?» maman lui répondit: «Mais il
ne sait pas lui-même, Françoise, il est énervé; préparez-moi vite le grand lit
et montez vous coucher.» Ainsi, pour la première fois, ma tristesse n’était plus
considérée comme une faute punissable mais comme un mal involontaire qu’on
venait de reconnaître officiellement, comme un état nerveux dont je n’étais pas
responsable; j’avais le soulagement de n’avoir plus à mêler de scrupules à
l’amertume de mes larmes, je pouvais pleurer sans péché. Je n’étais pas non plus
médiocrement fier vis-à-vis de Françoise de ce retour des choses humaines, qui,
une heure après que maman avait refusé de monter dans ma chambre et m’avait fait
dédaigneusement répondre que je devrais dormir, m’élevait à la dignité de grande
personne et m’avait fait atteindre tout d’un coup à une sorte de puberté du
chagrin, d’émancipation des larmes. J’aurais dû être heureux: je ne l’étais pas.
Il me semblait que ma mère venait de me faire une première concession qui devait
lui être douloureuse, que c’était une première abdication de sa part devant
l’idéal qu’elle avait conçu pour moi, et que pour la première fois, elle, si
courageuse, s’avouait vaincue. Il me semblait que si je venais de remporter une
victoire c’était contre elle, que j’avais réussi comme auraient pu faire la
maladie, des chagrins, ou l’âge, à détendre sa volonté, à faire fléchir sa
raison et que cette soirée commençait une ère, resterait comme une triste date.
Si j’avais osé maintenant, j’aurais dit à maman: «Non je ne veux pas, ne couche
pas ici.» Mais je connaissais la sagesse pratique, réaliste comme on dirait
aujourd’hui, qui tempérait en elle la nature ardemment idéaliste de ma
grand’mère, et je savais que, maintenant que le mal était fait, elle aimerait
mieux m’en laisser du moins goûter le plaisir calmant et ne pas déranger mon
père. Certes, le beau visage de ma mère brillait encore de jeunesse ce soir-là
où elle me tenait si doucement les mains et cherchait à arrêter mes larmes; mais
justement il me semblait que cela n’aurait pas dû être, sa colère eût moins
triste pour moi que cette douceur nouvelle que n’avait pas connue mon enfance;
il me semblait que je venais d’une main impie et secrète de tracer dans son âme
une première ride et d’y faire apparaître un premier cheveu blanc. Cette pensée
redoubla mes sanglots et alors je vis maman, qui jamais ne se laissait aller à
aucun attendrissement avec moi, être tout d’un coup gagnée par le mien et
essayer de retenir une envie de pleurer. Comme elle sentit que je m’en étais
aperçu, elle me dit en riant: «Voilà mon petit jaunet, mon petit serin, qui va
rendre sa maman aussi bêtasse que lui, pour peu que cela continue. Voyons,
puisque tu n’as pas sommeil ni ta maman non plus, ne restons pas à nous énerver,
faisons quelque chose, prenons un de tes livres.» Mais je n’en avais pas là.
«Est-ce que tu aurais moins de plaisir si je sortais déjà les livres que ta
grand’mère doit te donner pour ta fête? Pense bien: tu ne seras pas déçu de ne
rien avoir après-demain?» J’étais au contraire enchanté et maman alla chercher
un paquet de livres dont je ne pus deviner, à travers le papier qui les
enveloppait, que la taille courte et large, mais qui, sous ce premier aspect,
pourtant sommaire et voilé, éclipsaient déjà la boîte à couleurs du Jour de l’An
et les vers à soie de l’an dernier. C’était la Mare au Diable, François le
Champi, la Petite Fadette et les Maîtres Sonneurs. Ma grand’mère, ai-je su
depuis, avait d’abord choisi les poésies de Musset, un volume de Rousseau et
Indiana; car si elle jugeait les lectures futiles aussi malsaines que les
bonbons et les pâtisseries, elle ne pensait pas que les grands souffles du
génie eussent sur l’esprit même d’un enfant une influence plus dangereuse et
moins vivifiante que sur son corps le grand air et le vent du large. Mais mon
père l’ayant presque traitée de folle en apprenant les livres qu’elle voulait me
donner, elle était retournée elle-même à Jouy-le-Vicomte chez le libraire pour
que je ne risquasse pas de ne pas avoir mon cadeau (c’était un jour brûlant et
elle était rentrée si souffrante que le médecin avait averti ma mère de ne pas
la laisser se fatiguer ainsi) et elle s’était rabattue sur les quatre romans
champêtres de George Sand. «Ma fille, disait-elle à maman, je ne pourrais me
décider à donner à cet enfant quelque chose de mal écrit.»

En réalité, elle ne se résignait jamais à rien acheter dont on ne pût tirer un
profit intellectuel, et surtout celui que nous procurent les belles choses en
nous apprenant à chercher notre plaisir ailleurs que dans les satisfactions du
bien-être et de la vanité. Même quand elle avait à faire à quelqu’un un cadeau
dit utile, quand elle avait à donner un fauteuil, des couverts, une canne, elle
les cherchait «anciens», comme si leur longue désuétude ayant effacé leur
caractère d’utilité, ils paraissaient plutôt disposés pour nous raconter la vie
des hommes d’autrefois que pour servir aux besoins de la nôtre. Elle eût aimé
que j’eusse dans ma chambre des photographies des monuments ou des paysages les
plus beaux. Mais au moment d’en faire l’emplette, et bien que la chose
représentée eût une valeur esthétique, elle trouvait que la vulgarité, l’utilité
reprenaient trop vite leur place dans le mode mécanique de représentation, la
photographie. Elle essayait de ruser et sinon d’éliminer entièrement la banalité
commerciale, du moins de la réduire, d’y substituer pour la plus grande partie
de l’art encore, d’y introduire comme plusieurs «épaisseurs» d’art: au lieu de
photographies de la Cathédrale de Chartres, des Grandes Eaux de Saint-Cloud, du
Vésuve, elle se renseignait auprès de Swann si quelque grand peintre ne les
avait pas représentés, et préférait me donner des photographies de la Cathédrale
de Chartres par Corot, des Grandes Eaux de Saint-Cloud par Hubert Robert, du
Vésuve par Turner, ce qui faisait un degré d’art de plus. Mais si le photographe
avait été écarté de la représentation du chef-d’œuvre ou de la nature et
remplacé par un grand artiste, il reprenait ses droits pour reproduire cette
interprétation même. Arrivée à l’échéance de la vulgarité, ma grand’mère tâchait
de la reculer encore. Elle demandait à Swann si l’œuvre n’avait pas été gravée,
préférant, quand c’était possible, des gravures anciennes et ayant encore un
intérêt au delà d’elles-mêmes, par exemple celles qui représentent un
chef-d’œuvre dans un état où nous ne pouvons plus le voir aujourd’hui (comme la
gravure de la Cène de Léonard avant sa dégradation, par Morghen). Il faut dire
que les résultats de cette manière de comprendre l’art de faire un cadeau ne
furent pas toujours très brillants. L’idée que je pris de Venise d’après un
dessin du Titien qui est censé avoir pour fond la lagune, était certainement
beaucoup moins exacte que celle que m’eussent donnée de simples photographies.
On ne pouvait plus faire le compte à la maison, quand ma grand’tante voulait
dresser un réquisitoire contre ma grand’mère, des fauteuils offerts par elle à
de jeunes fiancés ou à de vieux époux, qui, à la première tentative qu’on avait
faite pour s’en servir, s’étaient immédiatement effondrés sous le poids d’un des
destinataires. Mais ma grand’mère aurait cru mesquin de trop s’occuper de la
solidité d’une boiserie où se distinguaient encore une fleurette, un sourire,
quelquefois une belle imagination du passé. Même ce qui dans ces meubles
répondait à un besoin, comme c’était d’une façon à laquelle nous ne sommes plus
habitués, la charmait comme les vieilles manières de dire où nous voyons une
métaphore, effacée, dans notre moderne langage, par l’usure de l’habitude. Or,
justement, les romans champêtres de George Sand qu’elle me donnait pour ma fête,
étaient pleins ainsi qu’un mobilier ancien, d’expressions tombées en désuétude
et redevenues imagées, comme on n’en trouve plus qu’à la campagne. Et ma
grand’mère les avait achetés de préférence à d’autres comme elle eût loué plus
volontiers une propriété où il y aurait eu un pigeonnier gothique ou quelqu’une
de ces vieilles choses qui exercent sur l’esprit une heureuse influence en lui
donnant la nostalgie d’impossibles voyages dans le temps.

Maman s’assit à côté de mon lit; elle avait pris François le Champi à qui sa
couverture rougeâtre et son titre incompréhensible, donnaient pour moi une
personnalité distincte et un attrait mystérieux. Je n’avais jamais lu encore de
vrais romans. J’avais entendu dire que George Sand était le type du romancier.
Cela me disposait déjà à imaginer dans François le Champi quelque chose
d’indéfinissable et de délicieux. Les procédés de narration destinés à exciter
la curiosité ou l’attendrissement, certaines façons de dire qui éveillent
l’inquiétude et la mélancolie, et qu’un lecteur un peu instruit reconnaît pour
communs à beaucoup de romans, me paraissaient simples —à moi qui considérais un
livre nouveau non comme une chose ayant beaucoup de semblables, mais comme une
personne unique, n’ayant de raison d’exister qu’en soi — une émanation
troublante de l’essence particulière à François le Champi. Sous ces événements
si journaliers, ces choses si communes, ces mots si courants, je sentais comme
une intonation, une accentuation étrange. L’action s’engagea; elle me parut
d’autant plus obscure que dans ce temps-là, quand je lisais, je rêvassais
souvent, pendant des pages entières, à tout autre chose. Et aux lacunes que
cette distraction laissait dans le récit, s’ajoutait, quand c’était maman qui me
lisait à haute voix, qu’elle passait toutes les scènes d’amour. Aussi tous les
changements bizarres qui se produisent dans l’attitude respective de la meunière
et de l’enfant et qui ne trouvent leur explication que dans les progrès d’un
amour naissant me paraissaient empreints d’un profond mystère dont je me
figurais volontiers que la source devait être dans ce nom inconnu et si doux de
«Champi» qui mettait sur l’enfant, qui le portait sans que je susse pourquoi, sa
couleur vive, empourprée et charmante. Si ma mère était une lectrice infidèle
c’était aussi, pour les ouvrages où elle trouvait l’accent d’un sentiment vrai,
une lectrice admirable par le respect et la simplicité de l’interprétation, par
la beauté et la douceur du son. Même dans la vie, quand c’étaient des êtres et
non des œuvres d’art qui excitaient ainsi son attendrissement ou son admiration,
c’était touchant de voir avec quelle déférence elle écartait de sa voix, de son
geste, de ses propos, tel éclat de gaîté qui eût pu faire mal à cette mère qui
avait autrefois perdu un enfant, tel rappel de fête, d’anniversaire, qui aurait
pu faire penser ce vieillard à son grand âge, tel propos de ménage qui aurait
paru fastidieux à ce jeune savant. De même, quand elle lisait la prose de George
Sand, qui respire toujours cette bonté, cette distinction morale que maman avait
appris de ma grand’mère à tenir pour supérieures à tout dans la vie, et que je
ne devais lui apprendre que bien plus tard à ne pas tenir également pour
supérieures à tout dans les livres, attentive à bannir de sa voix toute
petitesse, toute affectation qui eût pu empêcher le flot puissant d’y être reçu,
elle fournissait toute la tendresse naturelle, toute l’ample douceur qu’elles
réclamaient à ces phrases qui semblaient écrites pour sa voix et qui pour ainsi
dire tenaient tout entières dans le registre de sa sensibilité. Elle retrouvait
pour les attaquer dans le ton qu’il faut, l’accent cordial qui leur préexiste et
les dicta, mais que les mots n’indiquent pas; grâce à lui elle amortissait au
passage toute crudité dans les temps des verbes, donnait à l’imparfait et au
passé défini la douceur qu’il y a dans la bonté, la mélancolie qu’il y a dans la
tendresse, dirigeait la phrase qui finissait vers celle qui allait commencer,
tantôt pressant, tantôt ralentissant la marche des syllabes pour les faire
entrer, quoique leurs quantités fussent différentes, dans un rythme uniforme,
elle insufflait à cette prose si commune une sorte de vie sentimentale et
continue.

Mes remords étaient calmés, je me laissais aller à la douceur de cette nuit où
j’avais ma mère auprès de moi. Je savais qu’une telle nuit ne pourrait se
renouveler; que le plus grand désir que j’eusse au monde, garder ma mère dans ma
chambre pendant ces tristes heures nocturnes, était trop en opposition avec les
nécessités de la vie et le vœu de tous, pour que l’accomplissement qu’on lui
avait accordé ce soir pût être autre chose que factice et exceptionnel. Demain
mes angoisses reprendraient et maman ne resterait pas là. Mais quand mes
angoisses étaient calmées, je ne les comprenais plus; puis demain soir était
encore lointain; je me disais que j’aurais le temps d’aviser, bien que ce
temps-là ne pût m’apporter aucun pouvoir de plus, qu’il s’agissait de choses qui
ne dépendaient pas de ma volonté et que seul me faisait paraître plus évitables
l’intervalle qui les séparait encore de moi.

%. . .

\PRLsep

C’est ainsi que, pendant longtemps, quand, réveillé la nuit, je me ressouvenais
de Combray, je n’en revis jamais que cette sorte de pan lumineux, découpé au
milieu d’indistinctes ténèbres, pareil à ceux que l’embrasement d’un feu de
bengale ou quelque projection électrique éclairent et sectionnent dans un
édifice dont les autres parties restent plongées dans la nuit: à la base assez
large, le petit salon, la salle à manger, l’amorce de l’allée obscure par où
arriverait M. Swann, l’auteur inconscient de mes tristesses, le vestibule où je
m’acheminais vers la première marche de l’escalier, si cruel à monter, qui
constituait à lui seul le tronc fort étroit de cette pyramide irrégulière; et,
au faîte, ma chambre à coucher avec le petit couloir à porte vitrée pour
l’entrée de maman; en un mot, toujours vu à la même heure, isolé de tout ce
qu’il pouvait y avoir autour, se détachant seul sur l’obscurité, le décor
strictement nécessaire (comme celui qu’on voit indiqué en tête des vieilles
pièces pour les représentations en province), au drame de mon déshabillage;
comme si Combray n’avait consisté qu’en deux étages reliés par un mince
escalier, et comme s’il n’y avait jamais été que sept heures du soir. À vrai
dire, j’aurais pu répondre à qui m’eût interrogé que Combray comprenait encore
autre chose et existait à d’autres heures. Mais comme ce que je m’en serais
rappelé m’eût été fourni seulement par la mémoire volontaire, la mémoire de
l’intelligence, et comme les renseignements qu’elle donne sur le passé ne
conservent rien de lui, je n’aurais jamais eu envie de songer à ce reste de
Combray. Tout cela était en réalité mort pour moi.

Mort à jamais? C’était possible.

Il y a beaucoup de hasard en tout ceci, et un second hasard, celui de notre
mort, souvent ne nous permet pas d’attendre longtemps les faveurs du premier.

Je trouve très raisonnable la croyance celtique que les âmes de ceux que nous
avons perdus sont captives dans quelque être inférieur, dans une bête, un
végétal, une chose inanimée, perdues en effet pour nous jusqu’au jour, qui pour
beaucoup ne vient jamais, où nous nous trouvons passer près de l’arbre, entrer
en possession de l’objet qui est leur prison. Alors elles tressaillent, nous
appellent, et sitôt que nous les avons reconnues, l’enchantement est brisé.
Délivrées par nous, elles ont vaincu la mort et reviennent vivre avec nous.

Il en est ainsi de notre passé. C’est peine perdue que nous cherchions à
l’évoquer, tous les efforts de notre intelligence sont inutiles. Il est caché
hors de son domaine et de sa portée, en quelque objet matériel (en la sensation
que nous donnerait cet objet matériel), que nous ne soupçonnons pas. Cet objet,
il dépend du hasard que nous le rencontrions avant de mourir, ou que nous ne le
rencontrions pas.

Il y avait déjà bien des années que, de Combray, tout ce qui n’était pas le
théâtre et la drame de mon coucher, n’existait plus pour moi, quand un jour
d’hiver, comme je rentrais à la maison, ma mère, voyant que j’avais froid, me
proposa de me faire prendre, contre mon habitude, un peu de thé. Je refusai
d’abord et, je ne sais pourquoi, me ravisai. Elle envoya chercher un de ces
gâteaux courts et dodus appelés Petites Madeleines qui semblent avoir été
moulés dans la valve rainurée d’une coquille de Saint-Jacques. Et bientôt,
machinalement, accablé par la morne journée et la perspective d’un triste
lendemain, je portai à mes lèvres une cuillerée du thé où j’avais laissé
s’amollir un morceau de madeleine. Mais à l’instant même où la gorgée mêlée des
miettes du gâteau toucha mon palais, je tressaillis, attentif à ce qui se
passait d’extraordinaire en moi. Un plaisir délicieux m’avait envahi, isolé,
sans la notion de sa cause. Il m’avait aussitôt rendu les vicissitudes de la vie
indifférentes, ses désastres inoffensifs, sa brièveté illusoire, de la même
façon qu’opère l’amour, en me remplissant d’une essence précieuse: ou plutôt
cette essence n’était pas en moi, elle était moi. J’avais cessé de me sentire
médiocre, contingent, mortel. D’où avait pu me venir cette puissante joie? Je
sentais qu’elle était liée au goût du thé et du gâteau, mais qu’elle le dépassait
infiniment, ne devait pas être de même nature. D’où venait-elle? Que
signifiait-elle? Où l’appréhender? Je bois une seconde gorgée où je ne trouve
rien de plus que dans la première, une troisième qui m’apporte un peu moins que
la seconde. Il est temps que je m’arrête, la vertu du breuvage semble diminuer.
Il est clair que la vérité que je cherche n’est pas en lui, mais en moi. Il l’y
a éveillée, mais ne la connaît pas, et ne peut que répéter indéfiniment, avec de
moins en moins de force, ce même témoignage que je ne sais pas interpréter et
que je veux au moins pouvoir lui redemander et retrouver intact, à ma
disposition, tout à l’heure, pour un éclaircissement décisif. Je pose la tasse
et me tourne vers mon esprit. C’est à lui de trouver la vérité. Mais comment?
Grave incertitude, toutes les fois que l’esprit se sent dépassé par lui-même;
quand lui, le chercheur, est tout ensemble le pays obscur où il doit chercher et
où tout son bagage ne lui sera de rien. Chercher? pas seulement: créer. Il est
en face de quelque chose qui n’est pas encore et que seul il peut réaliser, puis
faire entrer dans sa lumière.

Et je recommence à me demander quel pouvait être cet état inconnu, qui
n’apportait aucune preuve logique, mais l’évidence de sa félicité, de sa réalité
devant laquelle les autres s’évanouissaient. Je veux essayer de le faire
réapparaître. Je rétrograde par la pensée au moment où je pris la première
cuillerée de thé. Je retrouve le même état, sans une clarté nouvelle. Je demande
à mon esprit un effort de plus, de ramener encore une fois la sensation qui
s’enfuit. Et pour que rien ne brise l’élan dont il va tâcher de la ressaisir,
j’écarte tout obstacle, toute idée étrangère, j’abrite mes oreilles et mon
attention contre les bruits de la chambre voisine. Mais sentant mon esprit qui
se fatigue sans réussir, je le force au contraire à prendre cette distraction
que je lui refusais, à penser à autre chose, à se refaire avant une tentative
suprême. Puis une deuxième fois, je fais le vide devant lui, je remets en face
de lui la saveur encore récente de cette première gorgée et je sens tressaillir
en moi quelque chose qui se déplace, voudrait s’élever, quelque chose qu’on
aurait désancré, à une grande profondeur; je ne sais ce que c’est, mais cela
monte lentement; j’éprouve la résistance et j’entends la rumeur des distances
traversées.

Certes, ce qui palpite ainsi au fond de moi, ce doit être l’image, le souvenir
visuel, qui, lié à cette saveur, tente de la suivre jusqu’à moi. Mais il se
débat trop loin, trop confusément; à peine si je perçois le reflet neutre où se
confond l’insaisissable tourbillon des couleurs remuées; mais je ne puis
distinguer la forme, lui demander comme au seul interprète possible, de me
traduire le témoignage de sa contemporaine, de son inséparable compagne, la
saveur, lui demander de m’apprendre de quelle circonstance particulière, de
quelle époque du passé il s’agit.

Arrivera-t-il jusqu’à la surface de ma claire conscience, ce souvenir, l’instant
ancien que l’attraction d’un instant identique est venue de si loin solliciter,
émouvoir, soulever tout au fond de moi? Je ne sais. Maintenant je ne sens plus
rien, il est arrêté, redescendu peut-être; qui sait s’il remontera jamais de sa
nuit? Dix fois il me faut recommencer, me pencher vers lui. Et chaque fois la
lâcheté qui nous détourne de toute tâche difficile, de toute œuvre importante,
m’a conseillé de laisser cela, de boire mon thé en pensant simplement à mes
ennuis d’aujourd’hui, à mes désirs de demain qui se laissent remâcher sans
peine.

Et tout d’un coup le souvenir m’est apparu. Ce goût celui du petit morceau de
madeleine que le dimanche matin à Combray (parce que ce jour-là je ne sortais
pas avant l’heure de la messe), quand j’allais lui dire bonjour dans sa chambre,
ma tante Léonie m’offrait après l’avoir trempé dans son infusion de thé ou de
tilleul. La vue de la petite madeleine ne m’avait rien rappelé avant que je n’y
eusse goûté; peut-être parce que, en ayant souvent aperçu depuis, sans en
manger, sur les tablettes des pâtissiers, leur image avait quitté ces jours de
Combray pour se lier à d’autres plus récents; peut-être parce que de ces
souvenirs abandonnés si longtemps hors de la mémoire, rien ne survivait, tout
s’était désagrégé; les formes — et celle aussi du petit coquillage de
pâtisserie, si grassement sensuel, sous son plissage sévère et dévot — s’étaient
abolies, ou, ensommeillées, avaient perdu la force d’expansion qui leur eût
permis de rejoindre la conscience. Mais, quand d’un passé ancien rien ne
subsiste, après la mort des êtres, après la destruction des choses, seules, plus
frêles mais plus vivaces, plus immatérielles, plus persistantes, plus fidèles,
l’odeur et la saveur restent encore longtemps, comme des âmes, à se rappeler, à
attendre, à espérer, sur la ruine de tout le reste, à porter sans fléchir, sur
leur gouttelette presque impalpable, l’édifice immense du souvenir.

Et dès que j’eus reconnu le goût du morceau de madeleine trempé dans le tilleul
que me donnait ma tante (quoique je ne susse pas encore et dusse remettre à bien
plus tard de découvrir pourquoi ce souvenir me rendait si heureux), aussitôt la
vieille maison grise sur la rue, où était sa chambre, vint comme un décor de
théâtre s’appliquer au petit pavillon, donnant sur le jardin, qu’on avait
construit pour mes parents sur ses derrières (ce pan tronqué que seul j’avais
revu jusque-là); et avec la maison, la ville, la Place où on m’envoyait avant
déjeuner, les rues où j’allais faire des courses depuis le matin jusqu’au soir
et par tous les temps, les chemins qu’on prenait si le temps était beau. Et
comme dans ce jeu où les Japonais s’amusent à tremper dans un bol de porcelaine
rempli d’eau, de petits morceaux de papier jusque-là indistincts qui, à peine y
sont-ils plongés s’étirent, se contournent, se colorent, se différencient,
deviennent des fleurs, des maisons, des personnages consistants et
reconnaissables, de même maintenant toutes les fleurs de notre jardin et celles
du parc de M. Swann, et les nymphéas de la Vivonne, et les bonnes gens du
village et leurs petits logis et l’église et tout Combray et ses environs, tout
cela que prend forme et solidité, est sorti, ville et jardins, de ma tasse de
thé.

%%%%%%%%%%%%%%%%%%%%
\chapter*{II}

Combray de loin, à dix lieues à la ronde, vu du chemin de fer quand nous y
arrivions la dernière semaine avant Pâques, ce n’était qu’une église résumant la
ville, la représentant, parlant d’elle et pour elle aux lointains, et, quand on
approchait, tenant serrés autour de sa haute mante sombre, en plein champ,
contre le vent, comme une pastoure ses brebis, les dos laineux et gris des
maisons rassemblées qu’un reste de remparts du moyen âge cernait çà et là d’un
trait aussi parfaitement circulaire qu’une petite ville dans un tableau de
primitif. A l’habiter, Combray était un peu triste, comme ses rues dont les
maisons construites en pierres noirâtres du pays, précédées de degrés
extérieurs, coiffées de pignons qui rabattaient l’ombre devant elles, étaient
assez obscures pour qu’il fallût dès que le jour commençait à tomber relever les
rideaux dans les «salles»; des rues aux graves noms de saints (desquels
plusieurs seigneurs de Combray): rue Saint-Hilaire, rue Saint-Jacques où était
la maison de ma tante, rue Sainte-Hildegarde, où donnait la grille, et rue du
Saint-Esprit sur laquelle s’ouvrait la petite porte latérale de son jardin; et
ces rues de Combray existent dans une partie de ma mémoire si reculée, peinte de
couleurs si différentes de celles qui maintenant revêtent pour moi le monde,
qu’en vérité elles me paraissent toutes, et l’église qui les dominait sur la
Place, plus irréelles encore que les projections de la lanterne magique; et qu’à
certains moments, il me semble que pouvoir encore traverser la rue
Saint-Hilaire, pouvoir louer une chambre rue de l’Oiseau —à la vieille
hôtellerie de l’Oiseau flesché, des soupiraux de laquelle montait une odeur de
cuisine que s’élève encore par moments en moi aussi intermittente et aussi
chaude — serait une entrée en contact avec l’Au-delà plus merveilleusement
surnaturelle que de faire la connaissance de Golo et de causer avec Geneviève de
Brabant.

La cousine de mon grand-père — ma grand’tante — chez qui nous habitions, était
la mère de cette tante Lèonie qui, depuis la mort de son mari, mon oncle Octave,
n’avait plus voulu quitter, d’abord Combray, puis à Combray sa maison, puis sa
chambre, puis son lit et ne «descendait» plus, toujours couchée dans un état
incertain de chagrin, de débilité physique, de maladie, d’idée fixe et de
dévotion. Son appartement particulier donnait sur la rue Saint-Jacques qui
aboutissait beaucoup plus loin au Grand-Pré (par opposition au Petit-Pré,
verdoyant au milieu de la ville, entre trois rues), et qui, unie, grisâtre, avec
les trois hautes marches de grès presque devant chaque porte, semblait comme un
défilé pratiqué par un tailleur d’images gothiques à même la pierre où il eût
sculpté une crèche ou un calvaire. Ma tante n’habitait plus effectivement que
deux chambres contiguës, restant l’après-midi dans l’une pendant qu’on aérait
l’autre. C’étaient de ces chambres de province qui — de même qu’en certains pays
des parties entières de l’air ou de la mer sont illuminées ou parfumées par des
myriades de protozoaires que nous ne voyons pas — nous enchantent des mille
odeurs qu’y dégagent les vertus, la sagesse, les habitudes, toute une vie
secrète, invisible, surabondante et morale que l’atmosphère y tient en suspens;
odeurs naturelles encore, certes, et couleur du temps comme celles de la
campagne voisine, me déjà casanières, humaines et renfermées, gelée exquise
industrieuse et limpide de tous les fruits de l’année qui ont quitté le verger
pour l’armoire; saisonnières, mais mobilières et domestiques, corrigeant le
piquant de la gelée blanche par la douceur du pain chaud, oisives et ponctuelles
comme une horloge de village, flâneuses et rangées, insoucieuses et prévoyantes,
lingères, matinales, dévotes, heureuses d’une paix qui n’apporte qu’un surcroît
d’anxiété et d’un prosaïsme que sert de grand réservoir de poésie à celui qui la
traverse sans y avoir vécu. L’air y était saturé de la fine fleur d’un silence
si nourricier, si succulent que je ne m’y avançais qu’avec une sorte de
gourmandise, surtout par ces premiers matins encore froids de la semaine de
Pâques où je le goûtais mieux parce que je venais seulement d’arriver à Combray:
avant que j’entrasse souhaiter le bonjour à ma tante on me faisait attendre un
instant, dans la première pièce où le soleil, d’hiver encore, était venu se
mettre au chaud devant le feu, déjà allumé entre les deux briques et qui
badigeonnait toute la chambre d’une odeur de suie, en faisait comme un de ces
grands «devants de four» de campagne, ou de ces manteaux de cheminée de
châteaux, sous lesquels on souhaite que se déclarent dehors la pluie, la neige,
même quelque catastrophe diluvienne pour ajouter au confort de la réclusion la
poésie de l’hivernage; je faisais quelques pas de prie-Dieu aux fauteuils en
velours frappé, toujours revêtus d’un appui-tête au crochet; et le feu cuisant
comme une pâte les appétissantes odeurs dont l’air de la chambre était tout
grumeleux et qu’avait déjà fait travailler et «lever» la fraîcheur humide et
ensoleillée du matin, il les feuilletait, les dorait, les godait, les
boursouflait, en faisant un invisible et palpable gâteau provincial, un immense
«chausson» où, à peine goûtés les arômes plus croustillants, plus fins, plus
réputés, mais plus secs aussi du placard, de la commode, du papier à ramages, je
revenais toujours avec une convoitise inavouée m’engluer dans l’odeur médiane,
poisseuse, fade, indigeste et fruitée de couvre-lit à fleurs.

Dans la chambre voisine, j’entendais ma tante qui causait toute seule à mi-voix.
Elle ne parlait jamais qu’assez bas parce qu’elle croyait avoir dans la tête
quelque chose de cassé et de flottant qu’elle eût déplacé en parlant trop fort,
mais elle ne restait jamais longtemps, même seule, sans dire quelque chose,
parce qu’elle croyait que c’était salutaire pour sa gorge et qu’en empêchant le
sang de s’y arrêter, cela rendrait moins fréquents les étouffements et les
angoisses dont elle souffrait; puis, dans l’inertie absolue où elle vivait, elle
prêtait à ses moindres sensations une importance extraordinaire; elle les douait
d’une motilité qui lui rendait difficile de les garder pour elle, et à défaut de
confident à qui les communiquer, elle se les annonçait à elle-même, en un
perpétuel monologue qui était sa seule forme d’activité. Malheureusement, ayant
pris l’habitude de penser tout haut, elle ne faisait pas toujours attention à ce
qu’il n’y eût personne dans la chambre voisine, et je l’entendais souvent se
dire à elle-même: «Il faut que je me rappelle bien que je n’ai pas dormi» (car
ne jamais dormir était sa grande prétention dont notre langage à tous gardait le
respect et la trace: le matin Françoise ne venait pas «l’éveiller», mais
«entrait» chez elle; quand ma tante voulait faire un somme dans la journée, on
disait qu’elle voulait «réfléchir» ou «reposer»; et quand il lui arrivait de
s’oublier en causant jusqu’à dire: «Ce qui m’a réveillée» ou «j’ai rêvé que»,
elle rougissait et se reprenait au plus vite).

Au bout d’un moment, j’entrais l’embrasser; Françoise faisait infuser son thé;
ou, si ma tante se sentait agitée, elle demandait à la place sa tisane et
c’était moi qui étais chargé de faire tomber du sac de pharmacie dans une
assiette la quantité de tilleul qu’il fallait mettre ensuite dans l’eau
bouillante. Le desséchement des tiges les avait incurvées en un capricieux
treillage dans les entrelacs duquel s’ouvraient les fleurs pâles, comme si un
peintre les eût arrangées, les eût fait poser de la façon la plus ornementale.
Les feuilles, ayant perdu ou changé leur aspect, avaient l’air des choses les
impossible disparates, d’une aile transparente de mouche, de l’envers blanc
d’une étiquette, d’un pétale de rose, mais qui eussent été empilées, concassées
ou tressées comme dans la confection d’un nid. Mille petits détails inutiles —
charmante prodigalité du pharmacien — qu’on eût supprimés dans une préparation
factice, me donnaient, comme un livre où on s’émerveille de rencontrer le nom
d’une personne de connaissance, le plaisir de comprendre que c’était bien des
tiges de vrais tilleuls, comme ceux que je voyais avenue de la Gare, modifiées,
justement parce que c’étaient non des doubles, mais elles-même et qu’elles
avaient vieilli. Et chaque caractère nouveau n’y étant que la métamorphose d’un
caractère ancien, dans de petites boules grises je reconnaissais les boutons
verts qui ne sont pas venus à terme; mais surtout l’éclat rose, lunaire et doux
qui faisait se détacher les fleurs dans la forêt fragile des tiges où elles
étaient suspendues comme de petites roses d’or — signe, comme la lueur qui
révèle encore sur une muraille la place d’une fresque effacée, de la différence
entre les parties de l’arbre qui avaient été «en couleur» et celles qui ne
l’avaient pas été— me montrait que ces pétales étaient bien ceux qui avant de
fleurir le sac de pharmacie avaient embaumé les soirs de printemps. Cette flamme
rose de cierge, c’était leur couleur encore, mais à demi éteinte et assoupie
dans cette vie diminuée qu’était la leur maintenant et qui est comme le
crépuscule des fleurs. Bientôt ma tante pouvait tremper dans l’infusion
bouillante dont elle savourait le goût de feuille morte ou de fleur fanée une
petite madeleine dont elle me tendait un morceau quand il était suffisamment
amolli.

D’un côté de son lit était une grande commode jaune en bois de citronnier et une
table qui tenait à la fois de l’officine et du maître-autel, où, au-dessus d’une
statuette de la Vierge et d’une bouteille de Vichy-Célestins, on trouvait des
livres de messe et des ordonnances de médicaments, tous ce qu’il fallait pour
suivre de son lit les offices et son régime, pour ne manquer l’heure ni de la
pepsine, ni des Vêpres. De l’autre côté, son lit longeait la fenêtre, elle avait
la rue sous les yeux et y lisait du matin au soir, pour se désennuyer, à la
façon des princes persans, la chronique quotidienne mais immémoriale de Combray,
qu’elle commentait ensuite avec Françoise.

Je n’étais pas avec ma tante depuis cinq minutes, qu’elle me renvoyait par peur
que je la fatigue. Elle tendait à mes lèvres son triste front pâle et fade sur
lequel, à cette heure matinale, elle n’avait pas encore arrangé ses faux
cheveux, et où les vertèbres transparaissaient comme les pointes d’une couronne
d’épines ou les grains d’un rosaire, et elle me disait: «Allons, mon pauvre
enfant, va-t’en, va te préparer pour la messe; et si en bas tu rencontres
Françoise, dis-lui de ne pas s’amuser trop longtemps avec vous, qu’elle monte
bientôt voir si je n’ai besoin de rien.»

Françoise, en effet, qui était depuis des années a son service et ne se doutait
pas alors qu’elle entrerait un jour tout à fait au nôtre délaissait un peu ma
tante pendant les mois où nous étions là. Il y avait eu dans mon enfance, avant
que nous allions à Combray, quand ma tante Léonie passait encore l’hiver à Paris
chez sa mère, un temps où je connaissais si peu Françoise que, le 1er janvier,
avant d’entrer chez ma grand’tante, ma mère me mettait dans la main une pièce de
cinq francs et me disait: «Surtout ne te trompe pas de personne. Attends pour
donner que tu m’entendes dire: «Bonjour Françoise»; en même temps je te
toucherai légèrement le bras.» A peine arrivions-nous dans l’obscure antichambre
de ma tante que nous apercevions dans l’ombre, sous les tuyaux d’un bonnet
éblouissant, raide et fragile comme s’il avait été de sucre filé, les remous
concentriques d’un sourire de reconnaissance anticipé. C’était Françoise,
immobile et debout dans l’encadrement de la petite porte du corridor comme une
statue de sainte dans sa niche. Quand on était un peu habitué à ces ténèbres de
chapelle, on distinguait sur son visage l’amour désintéressé de l’humanité, le
respect attendri pour les hautes classes qu’exaltait dans les meilleures régions
de son cœur l’espoir des étrennes. Maman me pinçait le bras avec violence et
disait d’une voix forte: «Bonjour Françoise.» A ce signal mes doigts s’ouvraient
et je lâchais la pièce qui trouvait pour la recevoir une main confuse, mais
tendue. Mais depuis que nous allions à Combray je ne connaissais personne mieux
que Françoise; nous étions ses préférés, elle avait pour nous, au moins pendant
les premières années, avec autant de considération que pour ma tante, un goût
plus vif, parce que nous ajoutions, au prestige de faire partie de la famille
(elle avait pour les liens invisibles que noue entre les membres d’une famille
la circulation d’un même sang, autant de respect qu’un tragique grec), le charme
de n’être pas ses maîtres habituels. Aussi, avec quelle joie elle nous recevait,
nous plaignant de n’avoir pas encore plus beau temps, le jour de notre arrivée,
la veille de Pâques, où souvent il faisait un vent glacial, quand maman lui
demandait des nouvelles de sa fille et de ses neveux, si son petit-fils était
gentil, ce qu’on comptait faire de lui, s’il ressemblerait à sa grand’mère.

Et quand il n’y avait plus de monde là, maman qui savait que Françoise pleurait
encore ses parents morts depuis des années, lui parlait d’eux avec douceur, lui
demandait mille détails sur ce qu’avait été leur vie.

Elle avait deviné que Françoise n’aimait pas son gendre et qu’il lui gâtait le
plaisir qu’elle avait à être avec sa fille, avec qui elle ne causait pas aussi
librement quand il était là. Aussi, quand Françoise allait les voir, à quelques
lieues de Combray, maman lui disait en souriant: «N’est-ce pas Françoise, si
Julien a été obligé de s’absenter et si vous avez Margeurite à vous toute seule
pour toute la journée, vous serez désolée, mais vous vous ferez une raison?» Et
Françoise disait en riant: «Madame sait tout; madame est pire que les rayons X
(elle disait x avec une difficulté affectée et un sourire pour se railler
elle-même, ignorante, d’employer ce terme savant), qu’on a fait venir pour Mme
Octave et qui voient ce que vous avez dans le cœur», et disparaissait, confuse
qu’on s’occupât d’elle, peut-être pour qu’on ne la vît pas pleurer; maman était
la première personne qui lui donnât cette douce émotion de sentir que sa vie,
ses bonheurs, ses chagrins de paysanne pouvaient présenter de l’intérêt, être un
motif de joie ou de tristesse pour une autre qu’elle-même. Ma tante se résignait
à se priver un peu d’elle pendant notre séjour, sachant combien ma mère
appréciait le service de cette bonne si intelligente et active, qui était aussi
belle dès cinq heures du matin dans sa cuisine, sous son bonnet dont le
tuyautage éclatant et fixe avait l’air d’être en biscuit, que pour aller à la
grand’messe; qui faisait tout bien, travaillant comme un cheval, qu’elle fût
bien portante ou non, mais sans bruit, sans avoir l’air de rien faire, la seule
des bonnes de ma tante qui, quand maman demandait de l’eau chaude ou du café
noir, les apportait vraiment bouillants; elle était un de ces serviteurs qui,
dans une maison, sont à la fois ceux qui déplaisent le plus au premier abord à
un étranger, peut-être parce qu’ils ne prennent pas la peine de faire sa
conquête et n’ont pas pour lui de prévenance, sachant très bien qu’ils n’ont
aucun besoin de lui, qu’on cesserait de le recevoir plutôt que de les renvoyer;
et qui sont en revanche ceux à qui tiennent le plus les maîtres qui ont éprouvé
leur capacités réelles, et ne se soucient pas de cet agrément superficiel, de ce
bavardage servile qui fait favorablement impression à un visiteur, mais qui
recouvre souvent une inéducable nullité.

Quand Françoise, après avoir veillé à ce que mes parents eussent tout ce qu’il
leur fallait, remontait une première fois chez ma tante pour lui donner sa
pepsine et lui demander ce qu’elle prendrait pour déjeuner, il était bien rare
qu’il ne fallût pas donner déjà son avis ou fournir des explications sur quelque
événement d’importance:

—«Françoise, imaginez-vous que Mme Goupil est passée plus d’un quart d’heure en
retard pour aller chercher sa sœur; pour peu qu’elle s’attarde sur son chemin
cela ne me surprendrait point qu’elle arrive après l’élévation.»

—«Hé! il n’y aurait rien d’étonnant», répondait Françoise.

—«Françoise, vous seriez venue cinq minutes plus tôt, vous auriez vu passer Mme
Imbert qui tenait des asperges deux fois grosses comme celles de la mère Callot;
tâchez donc de savoir par sa bonne où elle les a eues. Vous qui, cette année,
nous mettez des asperges à toutes les sauces, vous auriez pu en prendre de
pareilles pour nos voyageurs.»

—«Il n’y aurait rien d’étonnant qu’elles viennent de chez M. le Curé», disait
Françoise.

—«Ah! je vous crois bien, ma pauvre Françoise, répondait ma tante en haussant
les épaules, chez M. le Curé! Vous savez bien qu’il ne fait pousser que de
petites méchantes asperges de rien. Je vous dis que celles-là étaient grosses
comme le bras. Pas comme le vôtre, bien sûr, mais comme mon pauvre bras qui a
encore tant maigri cette année.»

—«Françoise, vous n’avez pas entendu ce carillon qui m’a cassé la tête?»

—«Non, madame Octave.»

—«Ah! ma pauvre fille, il faut que vous l’ayez solide votre tête, vous pouvez
remercier le Bon Dieu. C’était la Maguelone qui était venue chercher le docteur
Piperaud. Il est ressorti tout de suite avec elle et ils ont tourné par la rue
de l’Oiseau. Il faut qu’il y ait quelque enfant de malade.»

—«Eh! là, mon Dieu», soupirait Françoise, qui ne pouvait pas entendre parler
d’un malheur arrivé à un inconnu, même dans une partie du monde éloignée, sans
commencer à gémir.

—«Françoise, mais pour qui donc a-t-on sonné la cloche des morts? Ah! mon Dieu,
ce sera pour Mme Rousseau. Voilà-t-il pas que j’avais oublié qu’elle a passé
l’autre nuit. Ah! il est temps que le Bon Dieu me rappelle, je ne sais plus ce
que j’ai fait de ma tête depuis la mort de mon pauvre Octave. Mais je vous fais
perdre votre temps, ma fille.»

—«Mais non, madame Octave, mon temps n’est pas si cher; celui qui l’a fait ne
nous l’a pas vendu. Je vais seulement voir si mon feu ne s’éteint pas.»

Ainsi Françoise et ma tante appréciaient-elles ensemble au cours de cette séance
matinale, les premiers événements du jour. Mais quelquefois ces événements
revêtaient un caractère si mystérieux et si grave que ma tante sentait qu’elle
ne pourrait pas attendre le moment où Françoise monterait, et quatre coups de
sonnette formidables retentissaient dans la maison.

—«Mais, madame Octave, ce n’est pas encore l’heure de la pepsine, disait
Françoise. Est-ce que vous vous êtes senti une faiblesse?»

—«Mais non, Françoise, disait ma tante, c’est-à-dire si, vous savez bien que
maintenant les moments où je n’ai pas de faiblesse sont bien rares; un jour je
passerai comme Mme Rousseau sans avoir eu le temps de me reconnaître; mais ce
n’est pas pour cela que je sonne. Croyez-vous pas que je viens de voir comme je
vous vois Mme Goupil avec une fillette que je ne connais point. Allez donc
chercher deux sous de sel chez Camus. C’est bien rare si Théodore ne peut pas
vous dire qui c’est.»

—«Mais ça sera la fille à M. Pupin», disait Françoise qui préférait s’en tenir à
une explication immédiate, ayant été déjà deux fois depuis le matin chez Camus.

—«La fille à M. Pupin! Oh! je vous crois bien, ma pauvre Françoise! Avec cela
que je ne l’aurais pas reconnue?»

—«Mais je ne veux pas dire la grande, madame Octave, je veux dire la gamine,
celle qui est en pension à Jouy. Il me ressemble de l’avoir déjà vue ce matin.»

—«Ah! à moins de ça, disait ma tante. Il faudrait qu’elle soit venue pour les
fêtes. C’est cela! Il n’y a pas besoin de chercher, elle sera venue pour les
fêtes. Mais alors nous pourrions bien voir tout à l’heure Mme Sazerat venir
sonner chez sa sœur pour le déjeuner. Ce sera ça! J’ai vu le petit de chez
Galopin qui passait avec une tarte! Vous verrez que la tarte allait chez Mme
Goupil.»

—«Dès l’instant que Mme Goupil a de la visite, madame Octave, vous n’allez pas
tarder à voir tout son monde rentrer pour le déjeuner, car il commence à ne plus
être de bonne heure», disait Françoise qui, pressé de redescendre s’occuper du
déjeuner, n’était pas fâchée de laisser à ma tante cette distraction en
perspective.

—«Oh! pas avant midi», répondait ma tante d’un ton résigné, tout en jetant sur la
pendule un coup d’œil inquiet, mais furtif pour ne pas laisser voir qu’elle, qui
avait renoncé à tout, trouvait pourtant, à apprendre que Mme Goupil avait à
déjeuner, un plaisir aussi vif, et qui se ferait malheureusement attendre encore
un peu plus d’une heure. «Et encore cela tombera pendant mon déjeuner!»
ajouta-t-elle à mi-voix pour elle-même. Son déjeuner lui était une distraction
suffisante pour qu’elle n’en souhaitât pas une autre en même temps. «Vous
n’oublierez pas au moins de me donner mes œufs à la crème dans une assiette
plate?» C’étaient les seules qui fussent ornées de sujets, et ma tante s’amusait
à chaque repas à lire la légende de celle qu’on lui servait ce jour-là. Elle
mettait ses lunettes, déchiffrait: Alibaba et quarante voleurs, Aladin ou la
Lampe merveilleuse, et disait en souriant: «Très bien, très bien.» 

—«Je serais bien allée chez Camus . . . » disait Françoise en voyant que ma
tante ne l’y enverrait plus.

—«Mais non, ce n’est plus la peine, c’est sûrement Mlle Pupin. Ma pauvre
Françoise, je regrette de vous avoir fait monter pour rien.»

Mais ma tante savait bien que ce n’était pas pour rien qu’elle avait sonné
Françoise, car, à Combray, une personne «qu’on ne connaissait point» était un
être aussi peu croyable qu’un dieu de la mythologie, et de fait on ne se
souvenait pas que, chaque fois que s’était produite, dans la rue de Saint-Esprit
ou sur la place, une de ces apparitions stupéfiantes, des recherches bien
conduites n’eussent pas fini par réduire le personnage fabuleux aux proportions
d’une «personne qu’on connaissait», soit personnellement, soit abstraitement,
dans son état civil, en tant qu’ayant tel degré de parenté avec des gens de
Combray. C’était le fils de Mme Sauton qui rentrait du service, la nièce de
l’abbé Perdreau qui sortait de couvent, le frère du curé, percepteur à
Châteaudun qui venait de prendre sa retraite ou qui était venu passer les fêtes.
On avait eu en les apercevant l’émotion de croire qu’il y avait à Combray des
gens qu’on ne connaissait point simplement parce qu’on ne les avait pas reconnus
ou identifiés tout de suite. Et pourtant, longtemps à l’avance, Mme Sauton et le
curé avaient prévenu qu’ils attendaient leurs «voyageurs». Quand le soir, je
montais, en rentrant, raconter notre promenade à ma tante, si j’avais
l’imprudence de lui dire que nous avions rencontré près du Pont-Vieux, un homme
que mon grand-père ne connaissait pas: «Un homme que grand-père ne connaissait
point, s’écriait elle. Ah! je te crois bien!» Néanmoins un peu émue de cette
nouvelle, elle voulait en avoir le cœur net, mon grand-père était mandé. «Qui
donc est-ce que vous avez rencontré près du Pont-Vieux, mon oncle? un homme que
vous ne connaissiez point?»—«Mais si, répondait mon grand-père, c’était Prosper
le frère du jardinier de Mme Bouillebœuf.»—«Ah! bien», disait ma tante,
tranquillisée et un peu rouge; haussant les épaules avec un sourire ironique,
elle ajoutait: «Aussi il me disait que vous aviez rencontré un homme que vous ne
connaissiez point!» Et on me recommandait d’être plus circonspect une autre fois
et de ne plus agiter ainsi ma tante par des paroles irréfléchies. On connaissait
tellement bien tout le monde, à Combray, bêtes et gens, que si ma tante avait vu
par hasard passer un chien «qu’elle ne connaissait point», elle ne cessait d’y
penser et de consacrer à ce fait incompréhensible ses talents d’induction et ses
heures de liberté.

—«Ce sera le chien de Mme Sazerat», disait Françoise, sans grande conviction,
mais dans un but d’apaisement et pour que ma tante ne se «fende pas la tête.»

—«Comme si je ne connaissais pas le chien de Mme Sazerat!» répondait ma tante
donc l’esprit critique n’admettait pas si facilement un fait.

—«Ah! ce sera le nouveau chien que M. Galopin a rapporté de Lisieux.»

—«Ah! à moins de ça.»

—«Il paraît que c’est une bête bien affable», ajoutait Françoise qui tenait le
renseignement de Théodore, «spirituelle comme une personne, toujours de bonne
humeur, toujours aimable, toujours quelque chose de gracieux. C’est rare qu’une
bête qui n’a que cet âge-là soit déjà si galante. Madame Octave, il va falloir
que je vous quitte, je n’ai pas le temps de m’amuser, voilà bientôt dix heures,
mon fourneau n’est seulement pas éclairé, et j’ai encore à plumer mes asperges.»

—«Comment, Françoise, encore des asperges! mais c’est une vraie maladie
d’asperges que vous avez cette année, vous allez en fatiguer nos Parisiens!»

—«Mais non, madame Octave, ils aiment bien ça. Ils rentreront de l’église avec
de l’appétit et vous verrez qu’ils ne les mangeront pas avec le dos de la
cuiller.»

—«Mais à l’église, ils doivent y être déjà; vous ferez bien de ne pas perdre de
temps. Allez surveiller votre déjeuner.»

Pendant que ma tante devisait ainsi avec Françoise, j’accompagnais mes parents à
la messe. Que je l’aimais, que je la revois bien, notre Église! Son vieux porche
par lequel nous entrions, noir, grêlé comme une écumoire, était dévié et
profondément creusé aux angles (de même que le bénitier où il nous conduisait)
comme si le doux effleurement des mantes des paysannes entrant à l’église et de
leurs doigts timides prenant de l’eau bénite, pouvait, répété pendant des
siècles, acquérir une force destructive, infléchir la pierre et l’entailler de
sillons comme en trace la roue des carrioles dans la borne contre laquelle elle
bute tous les jours. Ses pierres tombales, sous lesquelles la noble poussière
des abbés de Combray, enterrés là, faisait au chœur comme un pavage spirituel,
n’étaient plus elles-mêmes de la matière inerte et dure, car le temps les avait
rendues douces et fait couler comme du miel hors des limites de leur propre
équarrissure qu’ici elles avaient dépassées d’un flot blond, entraînant à la
dérive une majuscule gothique en fleurs, noyant les violettes blanches du
marbre; et en deçà desquelles, ailleurs, elles s’étaient résorbées, contractant
encore l’elliptique inscription latine, introduisant un caprice de plus dans la
disposition de ces caractères abrégés, rapprochant deux lettres d’un mot dont
les autres avaient été démesurément distendues. Ses vitraux ne chatoyaient
jamais tant que les jours où le soleil se montrait peu, de sorte que fît-il gris
dehors, on était sûr qu’il ferait beau dans l’église; l’un était rempli dans
toute sa grandeur par un seul personnage pareil à un Roi de jeu de cartes, qui
vivait là-haut, sous un dais architectural, entre ciel et terre; (et dans le
reflet oblique et bleu duquel, parfois les jours de semaine, à midi, quand il
n’y a pas d’office — à l’un de ces rares moments où l’église aérée, vacante,
plus humaine, luxueuse, avec du soleil sur son riche mobilier, avait l’air
presque habitable comme le hall de pierre sculptée et de verre peint, d’un hôtel
de style moyen âge — on voyait s’agenouiller un instant Mme Sazerat, posant sur
le prie-Dieu voisin un paquet tout ficelé de petits fours qu’elle venait de
prendre chez le pâtissier d’en face et qu’elle allait rapporter pour le
déjeuner); dans un autre une montagne de neige rose, au pied de laquelle se
livrait un combat, semblait avoir givré à même la verrière qu’elle boursouflait
de son trouble grésil comme une vitre à laquelle il serait resté des flocons,
mais des flocons éclairés par quelque aurore (par la même sans doute qui
empourprait le rétable de l’autel de tons si frais qu’ils semblaient plutôt
posés là momentanément par une lueur du dehors prête à s’évanouir que par des
couleurs attachées à jamais à la pierre); et tous étaient si anciens qu’on
voyait çà et là leur vieillesse argentée étinceler de la poussière des siècles
et montrer brillante et usée jusqu’à la corde la trame de leur douce tapisserie
de verre. Il y en avait un qui était un haut compartiment divisé en une centaine
de petits vitraux rectangulaires où dominait le bleu, comme un grand jeu de
cartes pareil à ceux qui devaient distraire le roi Charles VI; mais soit qu’un
rayon eût brillé, soit que mon regard en bougeant eût promené à travers la
verrière tour à tour éteinte et rallumée, un mouvant et précieux incendie,
l’instant d’après elle avait pris l’éclat changeant d’une traîne de paon, puis
elle tremblait et ondulait en une pluie flamboyante et fantastique qui
dégouttait du haut de la voûte sombre et rocheuse, le long des parois humides,
comme si c’était dans la nef de quelque grotte irisée de sinueuses stalactites que
je suivais mes parents, qui portaient leur paroissien; un instant après les
petits vitraux en losange avaient pris la transparence profonde, l’infrangible
dureté de saphirs qui eussent été juxtaposés sur quelque immense pectoral, mais
derrière lesquels on sentait, plus aimé que toutes ces richesses, un sourire
momentané de soleil; il était aussi reconnaissable dans le flot bleu et doux
dont il baignait les pierreries que sur le pavé de la place ou la paille du
marché; et, même à nos premiers dimanches quand nous étions arrivés avant
Pâques, il me consolait que la terre fût encore nue et noire, en faisant
épanouir, comme en un printemps historique et qui datait des successeurs de
saint Louis, ce tapis éblouissant et doré de myosotis en verre.

Deux tapisseries de haute lice représentaient le couronnement d’Esther (le
tradition voulait qu’on eût donné à Assuérus les traits d’un roi de France et à
Esther ceux d’une dame de Guermantes dont il était amoureux) auxquelles leurs
couleurs, en fondant, avaient ajouté une expression, un relief, un éclairage: un
peu de rose flottait aux lèvres d’Esther au-delà du dessin de leur contour, le
jaune de sa robe s’étalait si onctueusement, si grassement, qu’elle en prenait
une sorte de consistance et s’enlevait vivement sur l’atmosphère refoulée; et la
verdure des arbres restée vive dans les parties basses du panneau de soie et de
laine, mais ayant «passé» dans le haut, faisait se détacher en plus pâle,
au-dessus des troncs foncés, les hautes branches jaunissantes, dorées et comme à
demi effacées par la brusque et oblique illumination d’un soleil invisible. Tout
cela et plus encore les objets précieux venus à l’église de personnages qui
étaient pour moi presque des personnages de légende (la croix d’or travaillée
disait-on par saint Éloi et donnée par Dagobert, le tombeau des fils de Louis le
Germanique, en porphyre et en cuivre émaillé) à cause de quoi je m’avançais dans
l’église, quand nous gagnions nos chaises, comme dans une vallée visitée des
fées, où le paysan s’émerveille de voir dans un rocher, dans un arbre, dans une
mare, la trace palpable de leur passage surnaturel, tout cela faisait d’elle
pour moi quelque chose d’entièrement différent du reste de la ville: un édifice
occupant, si l’on peut dire, un espace à quatre dimensions — la quatrième étant
celle du Temps — déployant à travers les siècles son vaisseau qui, de travée en
travée, de chapelle en chapelle, semblait vaincre et franchir non pas seulement
quelques mètres, mais des époques successives d’où il sortait victorieux;
dérobant le rude et farouche XIe siècle dans l’épaisseur de ses murs, d’où il
n’apparaissait avec ses lourds cintres bouchés et aveuglés de grossiers moellons
que par la profonde entaille que creusait près du porche l’escalier du clocher,
et, même là, dissimulé par les gracieuses arcades gothiques qui se pressaient
coquettement devant lui comme de plus grandes sœurs, pour le cacher aux
étrangers, se placent en souriant devant un jeune frère rustre, grognon et mal
vêtu; élevant dans le ciel au-dessus de la Place, sa tour qui avait contemplé
saint Louis et semblait le voir encore; et s’enfonçant avec sa crypte dans une
nuit mérovingienne où, nous guidant à tâtons sous la voûte obscure et
puissamment nervurée comme la membrane d’une immense chauve-souris de pierre,
Théodore et sa sœur nous éclairaient d’une bougie le tombeau de la petite fille
de Sigebert, sur lequel une profonde valve — comme la trace d’un fossile — avait
été creusée, disait-on, «par une lampe de cristal qui, le soir du meurtre de la
princesse franque, s’était détachée d’elle-même des chaînes d’or où elle était
suspendue à la place de l’actuelle abside, et, sans que le cristal se brisât,
sans que la flamme s’éteignît, s’était enfoncée dans la pierre et l’avait fait
mollement céder sous elle.»

L’abside de l’église de Combray, peut-on vraiment en parler? Elle était si
grossière, si dénuée de beauté artistique et même d’élan religieux. Du dehors,
comme le croisement des rues sur lequel elle donnait était en contre-bas, sa
grossière muraille s’exhaussait d’un soubassement en moellons nullement polis,
hérissés de cailloux, et qui n’avait rien de particulièrement ecclésiastique,
les verrières semblaient percées à une hauteur excessive, et le tout avait plus
l’air d’un mur de prison que d’église. Et certes, plus tard, quand je me
rappelais toutes les glorieuses absides que j’ai vues, il ne me serait jamais
venu à la pensée de rapprocher d’elles l’abside de Combray. Seulement, un jour,
au détour d’une petite rue provinciale, j’aperçus, en face du croisement de
trois ruelles, une muraille fruste et surélevée, avec des verrières percées en
haut et offrant le même aspect asymétrique que l’abside de Combray. Alors je ne
me suis pas demandé comme à Chartres ou à Reims avec quelle puissance y était
exprimé le sentiment religieux, mais je me suis involontairement écrié:
«L’Église!»

L’église! Familière; mitoyenne, rue Saint-Hilaire, où était sa porte nord, de
ses deux voisines, la pharmacie de M. Rapin et la maison de Mme Loiseau, qu’elle
touchait sans aucune séparation; simple citoyenne de Combray qui aurait pu avoir
son numéro dans la rue si les rues de Combray avaient eu des numéros, et où il
semble que le facteur aurait dû s’arrêter le matin quand il faisait sa
distribution, avant d’entrer chez Mme Loiseau et en sortant de chez M. Rapin, il
y avait pourtant entre elle et tout ce qui n’était pas elle une démarcation que
mon esprit n’a jamais pu arriver à franchir. Mme Loiseau avait beau avoir à sa
fenêtre des fuchsias, qui prenaient la mauvaise habitude de laisser leurs
branches courir toujours partout tête baissée, et dont les fleurs n’avaient rien
de plus pressé, quand elles étaient assez grandes, que d’aller rafraîchir leurs
joues violettes et congestionnées contre la sombre façade de l’église, les
fuchsias ne devenaient pas sacrés pour cela pour moi; entre les fleurs et la
pierre noircie sur laquelle elles s’appuyaient, si mes yeux ne percevaient pas
d’intervalle, mon esprit réservait un abîme.

On reconnaissait le clocher de Saint-Hilaire de bien loin, inscrivant sa figure
inoubliable à l’horizon où Combray n’apparaissait pas encore; quand du train
qui, la semaine de Pâques, nous amenait de Paris, mon père l’apercevait qui
filait tour à tour sur tous les sillons du ciel, faisant courir en tous sens son
petit coq de fer, il nous disait: «Allons, prenez les couvertures, on est
arrivé.» Et dans une des plus grandes promenades que nous faisions de Combray,
il y avait un endroit où la route resserrée débouchait tout à coup sur un
immense plateau fermé à l’horizon par des forêts déchiquetées que dépassait seul
la fine pointe du clocher de Saint-Hilaire, mais si mince, si rose, qu’elle
semblait seulement rayée sur le ciel par un ongle qui aurait voulu donner à ce
paysage, à ce tableau rien que de nature, cette petite marque d’art, cette
unique indication humaine. Quand on se rapprochait et qu’on pouvait apercevoir
le reste de la tour carrée et à demi détruite qui, moins haute, subsistait à
côté de lui, on était frappé surtout de ton rougeâtre et sombre des pierres; et,
par un matin brumeux d’automne, on aurait dit, s’élevant au-dessus du violet
orageux des vignobles, une ruine de pourpre presque de la couleur de la vigne
vierge.

Souvent sur la place, quand nous rentrions, ma grand’mère me faisait arrêter
pour le regarder. Des fenêtres de sa tour, placées deux par deux les unes
au-dessus des autres, avec cette juste et originale proportion dans les
distances qui ne donne pas de la beauté et de la dignité qu’aux visages humains,
il lâchait, laissait tomber à intervalles réguliers des volées de corbeaux qui,
pendant un moment, tournoyaient en criant, comme si les vieilles pierres qui les
laissaient s’ébattre sans paraître les voir, devenues tout d’un coup
inhabitables et dégageant un principe d’agitation infinie, les avait frappés et
repoussés. Puis, après avoir rayé en tous sens le velours violet de l’air du
soir, brusquement calmés ils revenaient s’absorber dans la tour, de néfaste
redevenue propice, quelques-uns posés çà et là, ne semblant pas bouger, mais
happant peut-être quelque insecte, sur la pointe d’un clocheton, comme une
mouette arrêtée avec l’immobilité d’un pêcheur à la crête d’une vague. Sans trop
savoir pourquoi, ma grand’mère trouvait au clocher de Saint-Hilaire cette
absence de vulgarité, de prétention, de mesquinerie, qui lui faisait aimer et
croire riches d’une influence bienfaisante, la nature, quand la main de l’homme
ne l’avait pas, comme faisait le jardinier de ma grand’tante, rapetissée, et les
œuvres de génie. Et sans doute, toute partie de l’église qu’on apercevait la
distinguait de tout autre édifice par une sorte de pensée qui lui était infuse,
mais c’était dans son clocher qu’elle semblait prendre conscience d’elle-même,
affirmer une existence individuelle et responsable. C’était lui qui parlait pour
elle. Je crois surtout que, confusément, ma grand’mère trouvait au clocher de
Combray ce qui pour elle avait le plus de prix au monde, l’air naturel et l’air
distingué. Ignorante en architecture, elle disait: «Mes enfants, moquez-vous de
moi si vous voulez, il n’est peut-être pas beau dans les règles, mais sa vieille
figure bizarre me plaît. Je suis sûre que s’il jouait du piano, il ne jouerait
pas sec.» Et en le regardant, en suivant des yeux la douce tension,
l’inclinaison fervente de ses pentes de pierre qui se rapprochaient en s’élevant
comme des mains jointes qui prient, elle s’unissait si bien à l’effusion de la
flèche, que son regard semblait s’élancer avec elle; et en même temps elle
souriait amicalement aux vieilles pierres usées dont le couchant n’éclairait
plus que le faîte et qui, à partir du moment où elles entraient dans cette zone
ensoleillée, adoucies par la lumière, paraissaient tout d’un coup montées bien
plus haut, lointaines, comme un chant repris «en voix de tête» une octave
au-dessus.

C’était le clocher de Saint-Hilaire qui donnait à toutes les occupations, à
toutes les heures, à tous les points de vue de la ville, leur figure, leur
couronnement, leur consécration. De ma chambre, je ne pouvais apercevoir que sa
base qui avait été recouverte d’ardoises; mais quand, le dimanche, je les
voyais, par une chaude matinée d’été, flamboyer comme un soleil noir, je me
disais: «Mon-Dieu! neuf heures! il faut se préparer pour aller à la grand’messe
si je veux avoir le temps d’aller embrasser tante Léonie avant», et je savais
exactement la couleur qu’avait le soleil sur la place, la chaleur et la
poussière du marché, l’ombre que faisait le store du magasin où maman entrerait
peut-être avant la messe dans une odeur de toile écrue, faire emplette de
quelque mouchoir que lui ferait montrer, en cambrant la taille, le patron qui,
tout en se préparant à fermer, venait d’aller dans l’arrière-boutique passer sa
veste du dimanche et se savonner les mains qu’il avait l’habitude, toutes les
cinq minutes, même dans les circonstances les plus mélancoliques, de frotter
l’une contre l’autre d’un air d’entreprise, de partie fine et de réussite.

Quand après la messe, on entrait dire à Théodore d’apporter une brioche plus
grosse que d’habitude parce que nos cousins avaient profité du beau temps pour
venir de Thiberzy déjeuner avec nous, on avait devant soi le clocher qui, doré
et cuit lui-même comme une plus grande brioche bénie, avec des écailles et des
égouttements gommeux de soleil, piquait sa pointe aiguë dans le ciel bleu. Et le
soir, quand je rentrais de promenade et pensais au moment où il faudrait tout à
l’heure dire bonsoir à ma mère et ne plus la voir, il était au contraire si
doux, dans la journée finissante, qu’il avait l’air d’être posé et enfoncé comme
un coussin de velours brun sur le ciel pâli qui avait cédé sous sa pression,
s’était creusé légèrement pour lui faire sa place et refluait sur ses bords; et
les cris des oiseaux qui tournaient autour de lui semblaient accroître son
silence, élancer encore sa flèche et lui donner quelque chose d’ineffable.

Même dans les courses qu’on avait à faire derrière l’église, là où on ne la
voyait pas, tout semblait ordonné par rapport au clocher surgi ici ou là entre
les maisons, peut-être plus émouvant encore quand il apparaissait ainsi sans
l’église. Et certes, il y en a bien d’autres qui sont plus beaux vus de cette
façon, et j’ai dans mon souvenir des vignettes de clochers dépassant les toits,
qui ont un autre caractère d’art que celles que composaient les tristes rues de
Combray. Je n’oublierai jamais, dans une curieuse ville de Normandie voisine de
Balbec, deux charmants hôtels du XVIIIe siècle, qui me sont à beaucoup d’égards
chers et vénérables et entre lesquels, quand on la regarde du beau jardin qui
descend des perrons vers la rivière, la flèche gothique d’une église qu’ils
cachent s’élance, ayant l’air de terminer, de surmonter leurs façades, mais
d’une matière si différente, si précieuse, si annelée, si rose, si vernie, qu’on
voit bien qu’elle n’en fait pas plus partie que de deux beaux galets unis, entre
lesquels elle est prise sur la plage, la flèche purpurine et crénelée de quelque
coquillage fuselé en tourelle et glacé d’émail. Même à Paris, dans un des
quartiers les plus laids de la ville, je sais un fenêtre où on voit après un
premier, un second et même un troisième plan fait des toits amoncelés de
plusieurs rues, une cloche violette, parfois rougeâtre, parfois aussi, dans les
plus nobles «épreuves» qu’en tire l’atmosphère, d’un noir décanté de cendres,
laquelle n’est autre que le dôme Saint-Augustin et qui donne à cette vue de
Paris le caractère de certaines vues de Rome par Piranesi. Mais comme dans
aucune de ces petites gravures, avec quelque goût que ma mémoire ait pu les
exécuter elle ne put mettre ce que j’avais perdu depuis longtemps, le sentiment
qui nous fait non pas considérer une chose comme un spectacle, mais y croire
comme en un être sans équivalent, aucune d’elles ne tient sous sa dépendance
toute une partie profonde de ma vie, comme fait le souvenir de ces aspects du
clocher de Combray dans les rues qui sont derrière l’église. Qu’on le vît à cinq
heures, quand on allait chercher les lettres à la poste, à quelques maisons de
soi, à gauche, surélevant brusquement d’une cime isolée la ligne de faîte des
toits; que si, au contraire, on voulait entrer demander des nouvelles de Mme
Sazerat, on suivît des yeux cette ligne redevenue basse après la descente de son
autre versant en sachant qu’il faudrait tourner à la deuxième rue après le
clocher; soit qu’encore, poussant plus loin, si on allait à la gare, on le vît
obliquement, montrant de profil des arêtes et des surfaces nouvelles comme un
solide surpris à un moment inconnu de sa révolution; ou que, des bords de la
Vivonne, l’abside musculeusement ramassée et remontée par la perspective semblât
jaillir de l’effort que le clocher faisait pour lancer sa flèche au cœur du
ciel: c’était toujours à lui qu’il fallait revenir, toujours lui qui dominait
tout, sommant les maisons d’un pinacle inattendu, levé avant moi comme le doigt
de Dieu dont le corps eût été caché dans la foule des humains sans que je le
confondisse pour cela avec elle. Et aujourd’hui encore si, dans une grande ville
de province ou dans un quartier de Paris que je connais mal, un passant qui m’a
«mis dans mon chemin» me montre au loin, comme un point de repère, tel beffroi
d’hôpital, tel clocher de couvent levant la pointe de son bonnet ecclésiastique
au coin d’une rue que je dois prendre, pour peu que ma mémoire puisse
obscurément lui trouver quelque trait de ressemblance avec la figure chère et
disparue, le passant, s’il se retourne pour s’assurer que je ne m’égare pas,
peut, à son étonnement, m’apercevoir qui, oublieux de la promenade entreprise ou
de la course obligée, reste là, devant le clocher, pendant des heures, immobile,
essayant de me souvenir, sentant au fond de moi des terres reconquises sur
l’oubli qui s’assèchent et se rebâtissent; et sans doute alors, et plus
anxieusement que tout à l’heure quand je lui demandais de me renseigner, je
cherche encore mon chemin, je tourne une rue . . . mais . . . c’est dans mon
cœur . . .

En rentrant de la messe, nous rencontrions souvent M. Legrandin qui, retenu à
Paris par sa profession d’ingénieur, ne pouvait, en dehors des grandes vacances,
venir à sa propriété de Combray que du samedi soir au lundi matin. C’était un de
ces hommes qui, en dehors d’une carrière scientifique où ils ont d’ailleurs
brillamment réussi, possèdent une culture toute différente, littéraire,
artistique, que leur spécialisation professionelle n’utilise pas et dont profite
leur conversation. Plus lettrés que bien des littérateurs (nous ne savions pas à
cette époque que M. Legrandin eût une certaine réputation comme écrivain et nous
fûmes très étonnés de voir qu’un musicien célèbre avait composé une mélodie sur
des vers de lui), doués de plus de «facilité» que bien des peintres, ils
s’imaginent que la vie qu’ils mènent n’est pas celle qui leur aurait convenu et
apportent à leurs occupations positives soit une insouciance mêlée de fantaisie,
soit une application soutenue et hautaine, méprisante, amère et consciencieuse.
Grand, avec une belle tournure, un visage pensif et fin aux longues moustaches
blondes, au regard bleu et désenchanté, d’une politesse raffinée, causeur comme
nous n’en avions jamais entendu, il était aux yeux de ma famille qui le citait
toujours en exemple, le type de l’homme d’élite, prenant la vie de la façon la
plus noble et la plus délicate. Ma grand’mère lui reprochait seulement de parler
un peu trop bien, un peu trop comme un livre, de ne pas avoir dans son langage
le naturel qu’il y avait dans ses cravates lavallière toujours flottantes, dans
son veston droit presque d’écolier. Elle s’étonnait aussi des tirades enflammées
qu’il entamait souvent contre l’aristocratie, la vie mondaine, le snobisme,
«certainement le péché auquel pense saint Paul quand il parle du péché pour
lequel il n’y a pas de rémission.»

L’ambition mondaine était un sentiment que ma grand’mère était si incapable de
ressentir et presque de comprendre qu’il lui paraissait bien inutile de mettre
tant d’ardeur à la flétrir. De plus elle ne trouvait pas de très bon goût que M.
Legrandin dont la sœur était mariée près de Balbec avec un gentilhomme
bas-normand se livrât à des attaques aussi violentes encore les nobles, allant
jusqu’à reprocher à la Révolution de ne les avoir pas tous guillotinés.

— Salut, amis! nous disait-il en venant à notre rencontre. Vous êtes heureux
d’habiter beaucoup ici; demain il faudra que je rentre à Paris, dans ma niche.

—«Oh! ajoutait-il, avec ce sourire doucement ironique et déçu, un peu distrait,
qui lui était particulier, certes il y a dans ma maison toutes les choses
inutiles. Il n’y manque que le nécessaire, un grand morceau de ciel comme ici.
Tâchez de garder toujours un morceau de ciel au-dessus de votre vie, petit
garçon, ajoutait-il en se tournant vers moi. Vous avez une jolie âme, d’une
qualité rare, une nature d’artiste, ne la laissez pas manquer de ce qu’il lui
faut.»

Quand, à notre retour, ma tante nous faisait demander si Mme Goupil était
arrivée en retard à la messe, nous étions incapables de la renseigner. En
revanche nous ajoutions à son trouble en lui disant qu’un peintre travaillait
dans l’église à copier le vitrail de Gilbert le Mauvais. Françoise, envoyée
aussitôt chez l’épicier, était revenue bredouille par la faute de l’absence de
Théodore à qui sa double profession de chantre ayant une part de l’entretien de
l’église, et de garçon épicier donnait, avec des relations dans tous les mondes,
un savoir universel.

—«Ah! soupirait ma tante, je voudrais que ce soit déjà l’heure d’Eulalie. Il n’y
a vraiment qu’elle qui pourra me dire cela.»

Eulalie était une fille boiteuse, active et sourde qui s’était «retirée» après
la mort de Mme de la Bretonnerie où elle avait été en place depuis son enfance
et qui avait pris à côté de l’église une chambre, d’où elle descendait tout le
temps soit aux offices, soit, en dehors des offices, dire une petite prière ou
donner un coup de main à Théodore; le reste du temps elle allait voir des
personnes malades comme ma tante Léonie à qui elle racontait ce qui s’était
passé à la messe ou aux vêpres. Elle ne dédaignait pas d’ajouter quelque casuel
à la petite rente que lui servait la famille de ses anciens maîtres en allant de
temps en temps visiter le linge du curé ou de quelque autre personnalité
marquante du monde clérical de Combray. Elle portait au-dessus d’une mante de
drap noir un petit béguin blanc, presque de religieuse, et une maladie de peau
donnait à une partie de ses joues et à son nez recourbé, les tons rose vif de la
balsamine. Ses visites étaient la grande distraction de ma tante Léonie qui ne
recevait plus guère personne d’autre, en dehors de M. le Curé. Ma tante avait
peu à peu évincé tous les autres visiteurs parce qu’ils avaient le tort à ses
yeux de rentrer tous dans l’une ou l’autre des deux catégories de gens qu’elle
détestait. Les uns, les pires et dont elle s’était débarrassée les premiers,
étaient ceux qui lui conseillaient de ne pas «s’écouter» et professaient, fût-ce
négativement et en ne la manifestant que par certains silences de désapprobation
ou par certains sourires de doute, la doctrine subversive qu’une petite
promenade au soleil et un bon bifteck saignant (quand elle gardait quatorze
heures sur l’estomac deux méchantes gorgées d’eau de Vichy!) lui feraient plus
de bien que son lit et ses médecines. L’autre catégorie se composait des
personnes qui avaient l’air de croire qu’elle était plus gravement malade
qu’elle ne pensait, était aussi gravement malade qu’elle le disait. Aussi, ceux
qu’elle avait laissé monter après quelques hésitations et sur les officieuses
instances de Françoise et qui, au cours de leur visite, avaient montré combien
ils étaient indignes de la faveur qu’on leur faisait en risquant timidement un:
«Ne croyez-vous pas que si vous vous secouiez un peu par un beau temps», ou qui,
au contraire, quand elle leur avait dit: «Je suis bien bas, bien bas, c’est la
fin, mes pauvres amis», lui avaient répondu: «Ah! quand on n’a pas la santé!
Mais vous pouvez durer encore comme ça», ceux-là, les uns comme les autres,
étaient sûrs de ne plus jamais être reçus. Et si Françoise s’amusait de l’air
épouvanté de ma tante quand de son lit elle avait aperçu dans la rue du
Saint-Esprit une de ces personnes qui avait l’air de venir chez elle ou quand
elle avait entendu un coup de sonnette, elle riait encore bien plus, et comme
d’un bon tour, des ruses toujours victorieuses de ma tante pour arriver à les
faire congédier et de leur mine déconfite en s’en retournant sans l’avoir vue,
et, au fond admirait sa maîtresse qu’elle jugeait supérieure à tous ces gens
puisque’elle ne voulait pas les recevoir. En somme, ma tante exigeait à la fois
qu’on l’approuvât dans son régime, qu’on la plaignît pour ses souffrances et
qu’on la rassurât sur son avenir.

C’est à quoi Eulalie excellait. Ma tante pouvait lui dire vingt fois en une
minute: «C’est la fin, ma pauvre Eulalie», vingt fois Eulalie répondait:
«Connaissant votre maladie comme vous la connaissez, madame Octave, vous irez à
cent ans, comme me disait hier encore Mme Sazerin.» (Une des plus fermes
croyances d’Eulalie et que le nombre imposant des démentis apportés par
l’expérience n’avait pas suffi à entamer, était que Mme Sazerat s’appelait Mme
Sazerin.)

— Je ne demande pas à aller à cent ans, répondait ma tante qui préférait ne pas
voir assigner à ses jours un terme précis.

Et comme Eulalie savait avec cela comme personne distraire ma tante sans la
fatiguer, ses visites qui avaient lieu régulièrement tous les dimanches sauf
empêchement inopiné, étaient pour ma tante un plaisir dont la perspective
l’entretenait ces jours-là dans un état agréable d’abord, mais bien vite
douloureux comme une faim excessive, pour peu qu’Eulalie fût en retard. Trop
prolongée, cette volupté d’attendre Eulalie tournait en supplice, ma tante ne
cessait de regarder l’heure, bâillait, se sentait des faiblesses. Le coup de
sonnette d’Eulalie, s’il arrivait tout à la fin de la journée, quand elle ne
l’espérait plus, la faisait presque se trouver mal. En réalité, le dimanche,
elle ne pensait qu’à cette visite et sitôt le déjeuner fini, Françoise avait
hâte que nous quittions la salle à manger pour qu’elle pût monter «occuper» ma
tante. Mais (surtout à partir du moment où les beaux jours s’installaient à
Combray) il y avait bien longtemps que l’heure altière de midi, descendue de la
tour de Saint-Hilaire qu’elle armoriait des douze fleurons momentanés de sa
couronne sonore avait retenti autour de notre table, auprès du pain bénit venu
lui aussi familièrement en sortant de l’église, quand nous étions encore assis
devant les assiettes des Mille et une Nuits, appesantis par la chaleur et
surtout par le repas. Car, au fond permanent d’œufs, de côtelettes, de pommes de
terre, de confitures, de biscuits, qu’elle ne nous annonçait même plus,
Françoise ajoutait — selon les travaux des champs et des vergers, le fruit de la
marée, les hasards du commerce, les politesses des voisins et son propre génie,
et si bien que notre menu, comme ces quatre-feuilles qu’on sculptait au XIIIe
siècle au portail des cathédrales, reflétait un peu le rythme des saisons et les
épisodes de la vie —: une barbue parce que la marchande lui en avait garanti la
fraîcheur, une dinde parce qu’elle en avait vu une belle au marché de
Roussainville-le-Pin, des cardons à la moelle parce qu’elle ne nous en avait pas
encore fait de cette manière-là, un gigot rôti parce que le grand air creuse et
qu’il avait bien le temps de descendre d’ici sept heures, des épinards pour
changer, des abricots parce que c’était encore une rareté, des groseilles parce
que dans quinze jours il n’y en aurait plus, des framboises que M. Swann avait
apportées exprès, des cerises, les premières qui vinssent du cerisier du jardin
après deux ans qu’il n’en donnait plus, du fromage à la crème que j’aimais bien
autrefois, un gâteau aux amandes parce que’elle l’avait commandé la veille, une
brioche parce que c’était notre tour de l’offrir. Quand tout cela était fini,
composée expressément pour nous, mais dédiée plus spécialement à mon père qui
était amateur, une crème au chocolat, inspiration, attention personnelle de
Françoise, nous était offerte, fugitive et légère comme une œuvre de
circonstance où elle avait mis tout son talent. Celui qui eût refusé d’en goûter
en disant: «J’ai fini, je n’ai plus faim», se serait immédiatement ravalé au
rang de ces goujats qui, même dans le présent qu’un artiste leur fait d’une de
ses œuvres, regardent au poids et à la matière alors que n’y valent que
l’intention et la signature. Même en laisser une seule goutte dans le plat eût
témoigné de la même impolitesse que se lever avant la fin du morceau au nez du
compositeur.

Enfin ma mère me disait: «Voyons, ne reste pas ici indéfiniment, monte dans ta
chambre si tu as trop chaud dehors, mais va d’abord prendre l’air un instant
pour ne pas lire en sortant de table.» J’allais m’asseoir près de la pompe et de
son auge, souvent ornée, comme un font gothique, d’une salamandre, qui sculptait
sur la pierre fruste le relief mobile de son corps allégorique et fuselé, sur le
banc sans dossier ombragé d’un lilas, dans ce petit coin du jardin qui s’ouvrait
par une porte de service sur la rue du Saint-Esprit et de la terre peu soignée
duquel s’élevait par deux degrés, en saillie de la maison, et comme une
construction indépendante, l’arrière-cuisine. On apercevait son dallage rouge et
luisant comme du porphyre. Elle avait moins l’air de l’antre de Françoise que
d’un petit temple à Vénus. Elle regorgeait des offrandes du crémier, du
fruitier, de la marchande de légumes, venus parfois de hameaux assez lointains
pour lui dédier les prémices de leurs champs. Et son faîte était toujours
couronné du roucoulement d’une colombe.

Autrefois, je ne m’attardais pas dans le bois consacré qui l’entourait, car,
avant de monter lire, j’entrais dans le petit cabinet de repos que mon oncle
Adolphe, un frère de mon grand-père, ancien militaire qui avait pris sa retraite
comme commandant, occupait au rez-de-chaussée, et qui, même quand les fenêtres
ouvertes laissaient entrer la chaleur, sinon les rayons du soleil qui
atteignaient rarement jusque-là, dégageait inépuisablement cette odeur obscure
et fraîche, à la fois forestière et ancien régime, qui fait rêver longuement les
narines, quand on pénètre dans certains pavillons de chasse abandonnés. Mais
depuis nombre d’années je n’entrais plus dans le cabinet de mon oncle Adolphe,
ce dernier ne venant plus à Combray à cause d’une brouille qui était survenue
entre lui et ma famille, par ma faute, dans les circonstances suivantes:

Une ou deux fois par mois, à Paris, on m’envoyait lui faire une visite, comme il
finissait de déjeuner, en simple vareuse, servi par son domestique en veste de
travail de coutil rayé violet et blanc. Il se plaignait en ronchonnant que je
n’étais pas venu depuis longtemps, qu’on l’abandonnait; il m’offrait un
massepain ou une mandarine, nous traversions un salon dans lequel on ne
s’arrêtait jamais, où on ne faisait jamais de feu, dont les murs étaient ornés
de moulures doreés, les plafonds peints d’un bleu qui prétendait imiter le ciel
et les meubles capitonnés en satin comme chez mes grands-parents, mais jaune;
puis nous passions dans ce qu’il appelait son cabinet de «travail» aux murs
duquel étaient accrochées de ces gravures représentant sur fond noir une déesse
charnue et rose conduisant un char, montée sur un globe, ou une étoile au front,
qu’on aimait sous le second Empire parce qu’on leur trouvait un air pompéien,
puis qu’on détesta, et qu’on recommence à aimer pour une seul et même raison,
malgré les autres qu’on donne et qui est qu’elles ont l’air second Empire. Et je
restais avec mon oncle jusqu’à ce que son valet de chambre vînt lui demander, de
la part du cocher, pour quelle heure celui-ci devait atteler. Mon oncle se
plongeait alors dans une méditation qu’aurait craint de troubler d’un seul
mouvement son valet de chambre émerveillé, et dont il attendait avec curiosité
le résultat, toujours identique. Enfin, après une hésitation suprême, mon oncle
prononçait infailliblement ces mots: «Deux heures et quart», que le valet de
chambre répétait avec étonnement, mais sans discuter: «Deux heures et quart?
bien . . . je vais le dire . . . »

À cette époque j’avais l’amour du théâtre, amour platonique, car mes parents ne
m’avaient encore jamais permis d’y aller, et je me représentais d’une façon si
peu exacte les plaisirs qu’on y goûtait que je n’étais pas éloigné de croire que
chaque spectateur regardait comme dans un stéréoscope un décor qui n’était que
pour lui, quoique semblable au millier d’autres que regardait, chacun pour soi,
le reste des spectateurs.

Tous les matins je courais jusqu’à la colonne Moriss pour voir les spectacles
qu’elle annonçait. Rien n’était plus désintéressé et plus heureux que les rêves
offerts à mon imagination par chaque pièce annoncée et qui étaient conditionnés
à la fois par les images inséparables des mots qui en composaient le titre et
aussi de la couleur des affiches encore humides et boursouflées de colle sur
lesquelles il se détachait. Si ce n’est une de ces œuvres étranges comme le
Testament de César Girodot et Œdipe-Roi lesquelles s’inscrivaient, non sur
l’affiche verte de l’Opéra-Comique, mais sur l’affiche lie de vin de la
Comédie-Française, rien ne me paraissait plus différent de l’aigrette
étincelante et blanche des Diamants de la Couronne que le satin lisse et
mystérieux du Domino Noir, et, mes parents m’ayant dit que quand j’irais pour la
première fois au théâtre j’aurais à choisir entre ces deux pièces, cherchant à
approfondir successivement le titre de l’une et le titre de l’autre, puisque
c’était tout ce que je connaissais d’elles, pour tâcher de saisir en chacun le
plaisir qu’il me promettait et de le comparer à celui que recélait l’autre,
j’arrivais à me représenter avec tant de force, d’une part une pièce
éblouissante et fière, de l’autre une pièce douce et veloutée, que j’étais aussi
incapable de décider laquelle aurait ma préférence, que si, pour le dessert, on
m’avait donné à opter encore du riz à l’Impératrice et de la crème au chocolat.

Toutes mes conversations avec mes camarades portaient sur ces acteurs dont
l’art, bien qu’il me fût encore inconnu, était la première forme, entre toutes
celles qu’il revêt, sous laquelle se laissait pressentir par moi, l’Art. Entre
la manière que l’un ou l’autre avait de débiter, de nuancer une tirade, les
différences les plus minimes me semblaient avoir une importance incalculable.
Et, d’après ce que l’on m’avait dit d’eux, je les classais par ordre de talent,
dans des listes que je me récitais toute la journée: et qui avaient fini par
durcir dans mon cerveau et par le gêner de leur inamovibilité.

Plus tard, quand je fus au collège, chaque fois que pendant les classes, je
correspondais, aussitôt que le professeur avait la tête tournée, avec un nouvel
ami, ma première question était toujours pour lui demander s’il était déjà allé
au théâtre et s’il trouvait que le plus grand acteur était bien Got, le second
Delaunay, etc. Et si, à son avis, Febvre ne venait qu’après Thiron, ou Delaunay
qu’après Coquelin, la soudaine motilité que Coquelin, perdant la rigidité de la
pierre, contractait dans mon esprit pour y passer au deuxième rang, et l’agilité
miraculeuse, la féconde animation dont se voyait doué Delaunay pour reculer au
quatrième, rendait la sensation du fleurissement et de la vie à mon cerveau
assoupli et fertilisé.

Mais si les acteurs me préoccupaient ainsi, si la vue de Maubant sortant un
après-midi du Théâtre-Français m’avait causé le saisissement et les souffrances
de l’amour, combien le nom d’une étoile flamboyant à la porte d’un théâtre,
combien, à la glace d’un coupé qui passait dans la rue avec ses chevaux fleuris
de roses au frontail, la vue du visage d’une femme que je pensais être peut-être
une actrice, laissait en moi un trouble plus prolongé, un effort impuissant et
douloureux pour me représenter sa vie! Je classais par ordre de talent les plus
illustres: Sarah Bernhardt, la Berma, Bartet, Madeleine Brohan, Jeanne Samary,
mais toutes m’intéressaient. Or mon oncle en connaissait beaucoup, et aussi des
cocottes que je ne distinguais pas nettement des actrices. Il les recevait chez
lui. Et si nous n’allions le voir qu’à certains jours c’est que, les autres
jours, venaient des femmes avec lesquelles sa famille n’aurait pas pu se
rencontrer, du moins à son avis à elle, car, pour mon oncle, au contraire, sa
trop grande facilité à faire à de jolies veuves qui n’avaient peut-être jamais
été mariées, à des comtesses de nom ronflant, qui n’était sans doute qu’un nom
de guerre, la politesse de les présenter à ma grand’mère ou même à leur donner
des bijoux de famille, l’avait déjà brouillé plus d’une fois avec mon
grand-père. Souvent, à un nom d’actrice qui venait dans la conversation,
j’entendais mon père dire à ma mère, en souriant: «Une amie de ton oncle»; et je
pensais que le stage que peut-être pendant des années des hommes importants
faisaient inutilement à la porte de telle femme qui ne répondait pas à leurs
lettres et les faisait chasser par le concierge de son hôtel, mon oncle aurait
pu en dispenser un gamin comme moi en le présentant chez lui à l’actrice,
inapprochable à tant d’autres, qui était pour lui une intime amie.

Aussi — sous le prétexte qu’une leçon qui avait été déplacée tombait maintenant
si mal qu’elle m’avait empêché plusieurs fois et m’empêcherait encore de voir
mon oncle — un jour, autre que celui qui était réservé aux visites que nous lui
faisions, profitant de ce que mes parents avaient déjeuné de bonne heure, je
sortis et au lieu d’aller regarder la colonne d’affiches, pour quoi on me
laissait aller seul, je courus jusqu’à lui. Je remarquai devant sa porte une
voiture attelée de deux chevaux qui avaient aux œillères un œillet rouge comme
avait le cocher à sa boutonnière. De l’escalier j’entendis un rire et une voix
de femme, et dès que j’eus sonné, un silence, puis le bruit de portes qu’on
fermait. Le valet de chambre vint ouvrir, et en me voyant parut embarrassé, me
dit que mon oncle était très occupé, ne pourrait sans doute pas me recevoir et
tandis qu’il allait pourtant le prévenir la même voix que j’avais entendue
disait: «Oh, si! laisse-le entrer; rien qu’une minute, cela m’amuserait tant.
Sur la photographie qui est sur ton bureau, il ressemble tant à sa maman, ta
nièce, dont la photographie est à côté de la sienne, n’est-ce pas? Je voudrais
le voir rien qu’un instant, ce gosse.»

J’entendis mon oncle grommeler, se fâcher; finalement le valet de chambre me fit
entrer.

Sur la table, il y avait la même assiette de massepains que d’habitude; mon
oncle avait sa vareuse de tous les jours, mais en face de lui, en robe de soie
rose avec un grand collier de perles au cou, était assise une jeune femme qui
achevait de manger une mandarine. L’incertitude où j’étais s’il fallait dire
madame ou mademoiselle me fit rougir et n’osant pas trop tourner les yeux de son
côté de peur d’avoir à lui parler, j’allai embrasser mon oncle. Elle me
regardait en souriant, mon oncle lui dit: «Mon neveu», sans lui dire mon nom, ni
me dire le sien, sans doute parce que, depuis les difficultés qu’il avait eues
avec mon grand-père, il tâchait autant que possible d’éviter tout trait d’union
entre sa famille et ce genre de relations.

—«Comme il ressemble à sa mère,» dit-elle.

—«Mais vous n’avez jamais vu ma nièce qu’en photographie, dit vivement mon oncle
d’un ton bourru.»

—«Je vous demande pardon, mon cher ami, je l’ai croisée dans l’escalier l’année
dernière quand vous avez été si malade. Il est vrai que je ne l’ai vue que le
temps d’un éclair et que votre escalier est bien noir, mais cela m’a suffi pour
l’admirer. Ce petit jeune homme a ses beaux yeux et aussi ça, dit-elle, en
traçant avec son doigt une ligne sur le bas de son front. Est-ce que madame
votre nièce porte le même nom que vous, ami? demanda-t-elle à mon oncle.»

—«Il ressemble surtout à son père, grogna mon oncle qui ne se souciait pas plus
de faire des présentations à distance en disant le nom de maman que d’en faire
de près. C’est tout à fait son père et aussi ma pauvre mère.»

—«Je ne connais pas son père, dit la dame en rose avec une légère inclinaison de
la tête, et je n’ai jamais connu votre pauvre mère, mon ami. Vous vous souvenez,
c’est peu après votre grand chagrin que nous nous sommes connus.»

J’éprouvais une petite déception, car cette jeune dame ne différait pas des
autres jolies femmes que j’avais vues quelquefois dans ma famille notamment de
la fille d’un de nos cousins chez lequel j’allais tous les ans le premier
janvier. Mieux habillée seulement, l’amie de mon oncle avait le même regard vif
et bon, elle avait l’air aussi franc et aimant. Je ne lui trouvais rien de
l’aspect théâtral que j’admirais dans les photographies d’actrices, ni de
l’expression diabolique qui eût été en rapport avec la vie qu’elle devait mener.
J’avais peine à croire que ce fût une cocotte et surtout je n’aurais pas cru que
ce fût une cocotte chic si je n’avais pas vu la voiture à deux chevaux, la robe
rose, le collier de perles, si je n’avais pas su que mon oncle n’en connaissait
que de la plus haute volée. Mais je me demandais comment le millionnaire qui lui
donnait sa voiture et son hôtel et ses bijoux pouvait avoir du plaisir à manger
sa fortune pour une personne qui avait l’air si simple et comme il faut. Et
pourtant en pensant à ce que devait être sa vie, l’immoralité m’en troublait
peut-être plus que si elle avait été concrétisée devant moi en une apparence
spéciale — d’être ainsi invisible comme le secret de quelque roman, de quelque
scandale qui avait fait sortir de chez ses parents bourgeois et voué à tout le
monde, qui avait fait épanouir en beauté et haussé jusqu’au demi-monde et à la
notoriété celle que ses jeux de physionomie, ses intonations de voix, pareils à
tant d’autres que je connaissais déjà, me faisaient malgré moi considérer comme
une jeune fille de bonne famille, qui n’était plus d’aucune famille.

On était passé dans le «cabinet de travail», et mon oncle, d’un air un peu gêné
par ma présence, lui offrit des cigarettes.

—«Non, dit-elle, cher, vous savez que je suis habituée à celles que le grand-duc
m’envoie. Je lui ai dit que vous en étiez jaloux.» Et elle tira d’un étui des
cigarettes couvertes d’inscriptions étrangères et dorées. «Mais si, reprit-elle
tout d’un coup, je dois avoir rencontré chez vous le père de ce jeune homme.
N’est-ce pas votre neveu? Comment ai-je pu l’oublier? Il a été tellement bon,
tellement exquis pour moi, dit-elle d’un air modeste et sensible.» Mais en
pensant à ce qu’avait pu être l’accueil rude qu’elle disait avoir trouvé exquis,
de mon père, moi qui connaissais sa réserve et sa froideur, j’étais gêné, comme
par une indélicatesse qu’il aurait commise, de cette inégalité entre la
reconnaissance excessive qui lui était accordée et son amabilité insuffisante.
Il m’a semblé plus tard que c’était un des côtés touchants du rôle de ces femmes
oisives et studieuses qu’elles consacrent leur générosité, leur talent, un rêve
disponible de beauté sentimentale — car, comme les artistes, elles ne le
réalisent pas, ne le font pas entrer dans les cadres de l’existence commune — et
un or qui leur coûte peu, à enrichir d’un sertissage précieux et fin la vie
fruste et mal dégrossie des hommes. Comme celle-ci, dans le fumoir où mon oncle
était en vareuse pour la recevoir, répandait son corps si doux, sa robe de soie
rose, ses perles, l’élégance qui émane de l’amitié d’un grand-duc, de même elle
avait pris quelque propos insignifiant de mon père, elle l’avait travaillé avec
délicatesse, lui avait donné un tour, une appellation précieuse et y enchâssant
un de ses regards d’une si belle eau, nuancé d’humilité et de gratitude, elle le
rendait changé en un bijou artiste, en quelque chose de «tout à fait exquis».

—«Allons, voyons, il est l’heure que tu t’en ailles», me dit mon oncle.

Je me levai, j’avais une envie irrésistible de baiser la main de la dame en
rose, mais il me semblait que c’eût été quelque chose d’audacieux comme un
enlèvement. Mon cœur battait tandis que je me disais: «Faut-il le faire, faut-il
ne pas le faire», puis je cessai de me demander ce qu’il fallait faire pour
pouvoir faire quelque chose. Et d’un geste aveugle et insensé, dépouillé de
toutes les raisons que je trouvais il y avait un moment en sa faveur, je portai
à mes lèvres la main qu’elle me tendait.

—«Comme il est gentil! il est déja galant, il a un petit œil pour les femmes: il
tient de son oncle. Ce sera un parfait gentleman», ajouta-t-elle en serrant les
dents pour donner à la phrase un accent légèrement britannique. «Est-ce qu’il ne
pourrait pas venir une fois prendre a cup of tea, comme disent nos voisins les
Anglais; il n’aurait qu’à m’envoyer un «bleu» le matin.

Je ne savais pas ce que c’était qu’un «bleu». Je ne comprenais pas la moitié des
mots que disait la dame, mais la crainte que n’y fut cachée quelque question à
laquelle il eût été impoli de ne pas répondre, m’empêchait de cesser de les
écouter avec attention, et j’en éprouvais une grande fatigue.

—«Mais non, c’est impossible, dit mon oncle, en haussant les épaules, il est
très tenu, il travaille beaucoup. Il a tous les prix à son cours, ajouta-t-il, à
voix basse pour que je n’entende pas ce mensonge et que je n’y contredise pas.
Qui sait, ce sera peut-être un petit Victor Hugo, une espèce de Vaulabelle, vous
savez.»

—«J’adore les artistes, répondit la dame en rose, il n’y a qu’eux qui
comprennent les femmes . . . Qu’eux et les êtres d’élite comme vous. Excusez mon
ignorance, ami. Qui est Vaulabelle? Est-ce les volumes dorés qu’il y a dans la
petite bibliothèque vitrée de votre boudoir? Vous savez que vous m’avez promis
de me les prêter, j’en aurai grand soin.»

Mon oncle qui détestait prêter ses livres ne répondit rien et me conduisit
jusqu’à l’antichambre. Éperdu d’amour pour la dame en rose, je couvris de
baisers fous les joues pleines de tabac de mon vieil oncle, et tandis qu’avec
assez d’embarras il me laissait entendre sans oser me le dire ouvertement qu’il
aimerait autant que je ne parlasse pas de cette visite à mes parents, je lui
disais, les larmes aux yeux, que le souvenir de sa bonté était en moi si fort
que je trouverais bien un jour le moyen de lui témoigner ma reconnaissance. Il
était si fort en effet que deux heures plus tard, après quelques phrases
mystérieuses et qui ne me parurent pas donner à mes parents une idée assez nette
de la nouvelle importance dont j’étais doué, je trouvai plus explicite de leur
raconter dans les moindres détails la visite que je venais de faire. Je ne
croyais pas ainsi causer d’ennuis à mon oncle. Comment l’aurais-je cru, puisque
je ne le désirais pas. Et je ne pouvais supposer que mes parents trouveraient du
mal dans une visite où je n’en trouvais pas. N’arrive-t-il pas tous les jours
qu’un ami nous demande de ne pas manquer de l’excuser auprès d’une femme à qui
il a été empêché d’écrire, et que nous négligions de le faire jugeant que cette
personne ne peut pas attacher d’importance à un silence qui n’en a pas pour
nous? Je m’imaginais, comme tout le monde, que le cerveau des autres était un
réceptacle inerte et docile, sans pouvoir de réaction spécifique sur ce qu’on y
introduisait; et je ne doutais pas qu’en déposant dans celui de mes parents la
nouvelle de la connaissance que mon oncle m’avait fait faire, je ne leur
transmisse en même temps comme je le souhaitais, le jugement bienveillant que je
portais sur cette présentation. Mes parents malheureusement s’en remirent à des
principes entièrement différents de ceux que je leur suggérais d’adopter, quand
ils voulurent apprécier l’action de mon oncle. Mon père et mon grand-père eurent
avec lui des explications violentes; j’en fus indirectement informé. Quelques
jours après, croisant dehors mon oncle qui passait en voiture découverte, je
ressentis la douleur, la reconnaissance, le remords que j’aurais voulu lui
exprimer. A côté de leur immensité, je trouvai qu’un coup de chapeau serait
mesquin et pourrait faire supposer à mon oncle que je ne me croyais pas tenu
envers lui à plus qu’à une banale politesse. Je résolus de m’abstenir de ce
geste insuffisant et je détournai la tête. Mon oncle pensa que je suivais en
cela les ordres de mes parents, il ne le leur pardonna pas, et il est mort bien
des années après sans qu’aucun de nous l’ait jamais revu.

Aussi je n’entrais plus dans le cabinet de repos maintenant fermé, de mon oncle
Adolphe, et après m’être attardé aux abords de l’arrière-cuisine, quand
Françoise, apparaissant sur le parvis, me disait: «Je vais laisser ma fille de
cuisine servir le café et monter l’eau chaude, il faut que je me sauve chez Mme
Octave», je me décidais à rentrer et montais directement lire chez moi. La fille
de cuisine était une personne morale, une institution permanente à qui des
attributions invariables assuraient une sorte de continuité et d’identité, à
travers la succession des formes passagères en lesquelles elle s’incarnait: car
nous n’eûmes jamais la même deux ans de suite. L’année où nous mangeâmes tant
d’asperges, la fille de cuisine habituellement chargée de les «plumer» était une
pauvre créature maladive, dans un état de grossesse déjà assez avancé quand nous
arrivâmes à Pâques, et on s’étonnait même que Françoise lui laissât faire tant
de courses et de besogne, car elle commençait à porter difficilement devant elle
la mystérieuse corbeille, chaque jour plus remplie, dont on devinait sous ses
amples sarraux la forme magnifique. Ceux-ci rappelaient les houppelandes qui
revêtent certaines des figures symboliques de Giotto dont M. Swann m’avait donné
des photographies. C’est lui-même qui nous l’avait fait remarquer et quand il
nous demandait des nouvelles de la fille de cuisine, il nous disait: «Comment va
la Charité de Giotto?» D’ailleurs elle-même, la pauvre fille, engraissée par sa
grossesse, jusqu’à la figure, jusqu’aux joues qui tombaient droites et carrées,
ressemblait en effet assez à ces vierges, fortes et hommasses, matrones plutôt,
dans lesquelles les vertus sont personnifiées à l’Arena. Et je me rends compte
maintenant que ces Vertus et ces Vices de Padoue lui ressemblaient encore d’une
autre manière. De même que l’image de cette fille était accrue par le symbole
ajouté qu’elle portait devant son ventre, sans avoir l’air d’en comprendre le
sens, sans que rien dans son visage en traduisît la beauté et l’esprit, comme un
simple et pesant fardeau, de même c’est sans paraître s’en douter que la
puissante ménagère qui est représentée à l’Arena au-dessous du nom «Caritas» et
dont la reproduction était accrochée au mur de ma salle d’études, à Combray,
incarne cette vertu, c’est sans qu’aucune pensée de charité semble avoir jamais
pu être exprimée par son visage énergique et vulgaire. Par une belle invention
du peintre elle foule aux pieds les trésors de la terre, mais absolument comme
si elle piétinait des raisins pour en extraire le jus ou plutôt comme elle
aurait monté sur des sacs pour se hausser; et elle tend à Dieu son cœur
enflammé, disons mieux, elle le lui «passe», comme une cuisinière passe un
tire-bouchon par le soupirail de son sous-sol à quelqu’un qui le lui demande à
la fenêtre du rez-de-chaussée. L’Envie, elle, aurait eu davantage une certaine
expression d’envie. Mais dans cette fresque-là encore, le symbole tient tant de
place et est représenté comme si réel, le serpent qui siffle aux lèvres de
l’Envie est si gros, il lui remplit si complètement sa bouche grande ouverte,
que les muscles de sa figure sont distendus pour pouvoir le contenir, comme ceux
d’un enfant qui gonfle un ballon avec son souffle, et que l’attention de l’Envie
— et la nôtre du même coup — tout entière concentrée sur l’action de ses lèvres,
n’a guère de temps à donner à d’envieuses pensées.

Malgré toute l’admiration que M. Swann professait pour ces figures de Giotto, je
n’eus longtemps aucun plaisir à considérer dans notre salle d’études, où on
avait accroché les copies qu’il m’en avait rapportées, cette Charité sans
charité, cette Envie qui avait l’air d’une planche illustrant seulement dans un
livre de médecine la compression de la glotte ou de la luette par une tumeur de
la langue ou par l’introduction de l’instrument de l’opérateur, une Justice,
dont le visage grisâtre et mesquinement régulier était celui-là même qui, à
Combray, caractérisait certaines jolies bourgeoises pieuses et sèches que je
voyais à la messe et dont plusieurs étaient enrôlées d’avance dans les milices
de réserve de l’Injustice. Mais plus tard j’ai compris que l’étrangeté
saisissante, la beauté spéciale de ces fresques tenait à la grande place que le
symbole y occupait, et que le fait qu’il fût représenté non comme un symbole
puisque la pensée symbolisée n’était pas exprimée, mais comme réel, comme
effectivement subi ou matériellement manié, donnait à la signification de
l’œuvre quelque chose de plus littéral et de plus précis, à son enseignement
quelque chose de plus concret et de plus frappant. Chez la pauvre fille de
cuisine, elle aussi, l’attention n’était-elle pas sans cesse ramenée à son
ventre par le poids qui le tirait; et de même encore, bien souvent la pensée des
agonisants est tournée vers le côté effectif, douloureux, obscur, viscéral, vers
cet envers de la mort qui est précisément le côté qu’elle leur présente, qu’elle
leur fait rudement sentir et qui ressemble beaucoup plus à un fardeau qui les
écrase, à une difficulté de respirer, à un besoin de boire, qu’à ce que nous
appelons l’idée de la mort.

Il fallait que ces Vertus et ces Vices de Padoue eussent en eux bien de la
réalité puisqu’ils m’apparaissaient comme aussi vivants que la servante
enceinte, et qu’elle-même ne me semblait pas beaucoup moins allégorique. Et
peut-être cette non-participation (du moins apparente) de l’âme d’un être à la
vertu qui agit par lui, a aussi en dehors de sa valeur esthétique une réalité
sinon psychologique, au moins, comme on dit, physiognomonique. Quand, plus tard,
j’ai eu l’occasion de rencontrer, au cours de ma vie, dans des couvents par
exemple, des incarnations vraiment saintes de la charité active, elles avaient
généralement un air allègre, positif, indifférent et brusque de chirurgien
pressé, ce visage où ne se lit aucune commisération, aucun attendrissement
devant la souffrance humaine, aucune crainte de la heurter, et qui est le visage
sans douceur, le visage antipathique et sublime de la vraie bonté.

Pendant que la fille de cuisine — faisant briller involontairement la
supériorité de Françoise, comme l’Erreur, par le contraste, rend plus éclatant
le triomphe de la Vérité— servait du café qui, selon maman n’était que de l’eau
chaude, et montait ensuite dans nos chambres de l’eau chaude qui était à peine
tiède, je m’étais étendu sur mon lit, un livre à la main, dans ma chambre qui
protégeait en tremblant sa fraîcheur transparente et fragile contre le soleil de
l’après-midi derrière ses volets presque clos où un reflet de jour avait
pourtant trouvé moyen de faire passer ses ailes jaunes, et restait immobile
entre le bois et le vitrage, dans un coin, comme un papillon posé. Il faisait à
peine assez clair pour lire, et la sensation de la splendeur de la lumière ne
m’était donnée que par les coups frappés dans la rue de la Cure par Camus
(averti par Françoise que ma tante ne «reposait pas» et qu’on pouvait faire du
bruit) contre des caisses poussiéreuses, mais qui, retentissant dans
l’atmosphère sonore, spéciale aux temps chauds, semblaient faire voler au loin
des astres écarlates; et aussi par les mouches qui exécutaient devant moi, dans
leur petit concert, comme la musique de chambre de l’été: elle ne l’évoque pas à
la façon d’un air de musique humaine, qui, entendu par hasard à la belle saison,
vous la rappelle ensuite; elle est unie à l’été par un lien plus nécessaire: née
des beaux jours, ne renaissant qu’avec eux, contenant un peu de leur essence,
elle n’en réveille pas seulement l’image dans notre mémoire, elle en certifie le
retour, la présence effective, ambiante, immédiatement accessible.

Cette obscure fraîcheur de ma chambre était au plein soleil de la rue, ce que
l’ombre est au rayon, c’est-à-dire aussi lumineuse que lui, et offrait à mon
imagination le spectacle total de l’été dont mes sens si j’avais été en
promenade, n’auraient pu jouir que par morceaux; et ainsi elle s’accordait bien
à mon repos qui (grâce aux aventures racontées par mes livres et qui venaient
l’émouvoir) supportait pareil au repos d’une main immobile au milieu d’une eau
courante, le choc et l’animation d’un torrent d’activité.

Mais ma grand’mère, même si le temps trop chaud s’était gâté, si un orage ou
seulement un grain était survenu, venait me supplier de sortir. Et ne voulant
pas renoncer à ma lecture, j’allais du moins la continuer au jardin, sous le
marronnier, dans une petite guérite en sparterie et en toile au fond de laquelle
j’étais assis et me croyais caché aux yeux des personnes qui pourraient venir
faire visite à mes parents.

Et ma pensée n’était-elle pas aussi comme une autre crèche au fond de laquelle
je sentais que je restais enfoncé, même pour regarder ce qui se passait au
dehors? Quand je voyais un objet extérieur, la conscience que je le voyais
restait entre moi et lui, le bordait d’un mince liseré spirituel qui m’empêchait
de jamais toucher directement sa matière; elle se volatilisait en quelque sorte
avant que je prisse contact avec elle, comme un corps incandescent qu’on
approche d’un objet mouillé ne touche pas son humidité parce qu’il se fait
toujours précéder d’une zone d’évaporation. Dans l’espèce d’écran diapré d’états
différents que, tandis que je lisais, déployait simultanément ma conscience, et
qui allaient des aspirations les plus profondément cachées en moi-même jusqu’à
la vision tout extérieure de l’horizon que j’avais, au bout du jardin, sous les
yeux, ce qu’il y avait d’abord en moi, de plus intime, la poignée sans cesse en
mouvement qui gouvernait le reste, c’était ma croyance en la richesse
philosophique, en la beauté du livre que je lisais, et mon désir de me les
approprier, quel que fût ce livre. Car, même si je l’avais acheté à Combray, en
l’apercevant devant l’épicerie Borange, trop distante de la maison pour que
Françoise pût s’y fournir comme chez Camus, mais mieux achalandée comme
papeterie et librairie, retenu par des ficelles dans la mosaïque des brochures
et des livraisons qui revêtaient les deux vantaux de sa porte plus mystérieuse,
plus semée de pensées qu’une porte de cathédrale, c’est que je l’avais reconnu
pour m’avoir été cité comme un ouvrage remarquable par le professeur ou le
camarade qui me paraissait à cette époque détenir le secret de la vérité et de
la beauté à demi pressenties, à demi incompréhensibles, dont la connaissance
était le but vague mais permanent de ma pensée.

Après cette croyance centrale qui, pendant ma lecture, exécutait d’incessants
mouvements du dedans au dehors, vers la découverte de la vérité, venaient les
émotions que me donnait l’action à laquelle je prenais part, car ces
après-midi-là étaient plus remplis d’événements dramatiques que ne l’est souvent
toute une vie. C’était les événements qui survenaient dans le livre que je
lisais; il est vrai que les personnages qu’ils affectaient n’étaient pas
«Réels», comme disait Françoise. Mais tous les sentiments que nous font éprouver
la joie ou l’infortune d’un personnage réel ne se produisent en nous que par
l’intermédiaire d’une image de cette joie ou de cette infortune; l’ingéniosité
du premier romancier consista à comprendre que dans l’appareil de nos émotions,
l’image étant le seul élément essentiel, la simplification qui consisterait à
supprimer purement et simplement les personnages réels serait un
perfectionnement décisif. Un être réel, si profondément que nous sympathisions
avec lui, pour une grande part est perçu par nos sens, c’est-à-dire nous reste
opaque, offre un poids mort que notre sensibilité ne peut soulever. Qu’un
malheur le frappe, ce n’est qu’en une petite partie de la notion totale que nous
avons de lui, que nous pourrons en être émus; bien plus, ce n’est qu’en une
partie de la notion totale qu’il a de soi qu’il pourra l’être lui-même. La
trouvaille du romancier a été d’avoir l’idée de remplacer ces parties
impénétrables à l’âme par une quantité égale de parties immatérielles,
c’est-à-dire que notre âme peut s’assimiler. Qu’importe dès lors que les
actions, les émotions de ces êtres d’un nouveau genre nous apparaissent comme
vraies, puisque nous les avons faites nôtres, puisque c’est en nous qu’elles se
produisent, qu’elles tiennent sous leur dépendance, tandis que nous tournons
fiévreusement les pages du livre, la rapidité de notre respiration et
l’intensité de notre regard. Et une fois que le romancier nous a mis dans cet
état, où comme dans tous les états purement intérieurs, toute émotion est
décuplée, où son livre va nous troubler à la façon d’un rêve mais d’un rêve plus
clair que ceux que nous avons en dormant et dont le souvenir durera davantage,
alors, voici qu’il déchaîne en nous pendant une heure tous les bonheurs et tous
les malheurs possibles dont nous mettrions dans la vie des années à connaître
quelques-uns, et dont les plus intenses ne nous seraient jamais révélés parce
que la lenteur avec laquelle ils se produisent nous en ôte la perception; (ainsi
notre cœur change, dans la vie, et c’est la pire douleur; mais nous ne la
connaissons que dans la lecture, en imagination: dans la réalité il change,
comme certains phénomènes de la nature se produisent, assez lentement pour que,
si nous pouvons constater successivement chacun de ses états différents, en
revanche la sensation même du changement nous soit épargnée).

Déjà moins intérieur à mon corps que cette vie des personnages, venait ensuite,
à demi projeté devant moi, le paysage où se déroulait l’action et qui exerçait
sur ma pensée une bien plus grande influence que l’autre, que celui que j’avais
sous les yeux quand je les levais du livre. C’est ainsi que pendant deux étés,
dans la chaleur du jardin de Combray, j’ai eu, à cause du livre que je lisais
alors, la nostalgie d’un pays montueux et fluviatile, où je verrais beaucoup de
scieries et où, au fond de l’eau claire, des morceaux de bois pourrissaient sous
des touffes de cresson: non loin montaient le long de murs bas, des grappes de
fleurs violettes et rougeâtres. Et comme le rêve d’une femme qui m’aurait aimé
était toujours présent à ma pensée, ces étés-là ce rêve fut imprégné de la
fraîcheur des eaux courantes; et quelle que fût la femme que j’évoquais, des
grappes de fleurs violettes et rougeâtres s’élevaient aussitôt de chaque côté
d’elle comme des couleurs complémentaires.

Ce n’était pas seulement parce qu’une image dont nous rêvons reste toujours
marquée, s’embellit et bénéficie du reflet des couleurs étrangères qui par
hasard l’entourent dans notre rêverie; car ces paysages des livres que je lisais
n’étaient pas pour moi que des paysages plus vivement représentés à mon
imagination que ceux que Combray mettait sous mes yeux, mais qui eussent été
analogues. Par le choix qu’en avait fait l’auteur, par la foi avec laquelle ma
pensée allait au-devant de sa parole comme d’une révélation, ils me semblaient
être — impression que ne me donnait guère le pays où je me trouvais, et surtout
notre jardin, produit sans prestige de la correcte fantaisie du jardinier que
méprisait ma grand’mère — une part véritable de la Nature elle-même, digne
d’être étudiée et approfondie.

Si mes parents m’avaient permis, quand je lisais un livre, d’aller visiter la
région qu’il décrivait, j’aurais cru faire un pas inestimable dans la conquête
de la vérité. Car si on a la sensation d’être toujours entouré de son âme, ce
n’est pas comme d’une prison immobile: plutôt on est comme emporté avec elle
dans un perpétuel élan pour la dépasser, pour atteindre à l’extérieur, avec une
sorte de découragement, entendant toujours autour de soi cette sonorité
identique qui n’est pas écho du dehors mais retentissement d’une vibration
interne. On cherche à retrouver dans les choses, devenues par là précieuses, le
reflet que notre âme a projeté sur elles; on est déçu en constatant qu’elles
semblent dépourvues dans la nature, du charme qu’elles devaient, dans notre
pensée, au voisinage de certaines idées; parfois on convertit toutes les forces
de cette âme en habileté, en splendeur pour agir sur des êtres dont nous sentons
bien qu’ils sont situés en dehors de nous et que nous ne les atteindrons jamais.
Aussi, si j’imaginais toujours autour de la femme que j’aimais, les lieux que je
désirais le plus alors, si j’eusse voulu que ce fût elle qui me les fît visiter,
qui m’ouvrît l’accès d’un monde inconnu, ce n’était pas par le hasard d’une
simple association de pensée; non, c’est que mes rêves de voyage et d’amour
n’étaient que des moments — que je sépare artificiellement aujourd’hui comme si
je pratiquais des sections à des hauteurs différentes d’un jet d’eau irisé et en
apparence immobile — dans un même et infléchissable jaillissement de toutes les
forces de ma vie.

Enfin, en continuant à suivre du dedans au dehors les états simultanément
juxtaposés dans ma conscience, et avant d’arriver jusqu’à l’horizon réel qui les
enveloppait, je trouve des plaisirs d’un autre genre, celui d’être bien assis,
de sentir la bonne odeur de l’air, de ne pas être dérangé par une visite; et,
quand une heure sonnait au clocher de Saint-Hilaire, de voir tomber morceau par
morceau ce qui de l’après-midi était déjà consommé, jusqu’à ce que j’entendisse
le dernier coup qui me permettait de faire le total et après lequel, le long
silence qui le suivait, semblait faire commencer, dans le ciel bleu, toute la
partie qui m’était encore concédée pour lire jusqu’au bon dîner qu’apprêtait
Françoise et qui me réconforterait des fatigues prises, pendant la lecture du
livre, à la suite de son héros. Et à chaque heure il me semblait que c’était
quelques instants seulement auparavant que la précédente avait sonné; la plus
récente venait s’inscrire tout près de l’autre dans le ciel et je ne pouvais
croire que soixante minutes eussent tenu dans ce petit arc bleu qui était
compris entre leurs deux marques d’or. Quelquefois même cette heure prématurée
sonnait deux coups de plus que la dernière; il y en avait donc une que je
n’avais pas entendue, quelque chose qui avait eu lieu n’avait pas eu lieu pour
moi; l’intérêt de la lecture, magique comme un profond sommeil, avait donné le
change à mes oreilles hallucinées et effacé la cloche d’or sur la surface azurée
du silence. Beaux après-midi du dimanche sous le marronnier du jardin de
Combray, soigneusement vidés par moi des incidents médiocres de mon existence
personnelle que j’y avais remplacés par une vie d’aventures et d’aspirations
étranges au sein d’un pays arrosé d’eaux vives, vous m’évoquez encore cette vie
quand je pense à vous et vous la contenez en effet pour l’avoir peu à peu
contournée et enclose — tandis que je progressais dans ma lecture et que tombait
la chaleur du jour — dans le cristal successif, lentement changeant et traversé
de feuillages, de vos heures silencieuses, sonores, odorantes et limpides.

Quelquefois j’étais tiré de ma lecture, dès le milieu de l’après-midi par la
fille du jardinier, qui courait comme une folle, renversant sur son passage un
oranger, se coupant un doigt, se cassant une dent et criant: «Les voilà, les
voilà!» pour que Françoise et moi nous accourions et ne manquions rien du
spectacle. C’était les jours où, pour des manœuvres de garnison, la troupe
traversait Combray, prenant généralement la rue Sainte-Hildegarde. Tandis que
nos domestiques, assis en rang sur des chaises en dehors de la grille,
regardaient les promeneurs dominicaux de Combray et se faisaient voir d’eux, la
fille du jardinier par la fente que laissaient entre elles deux maisons
lointaines de l’avenue de la Gare, avait aperçu l’éclat des casques. Les
domestiques avaient rentré précipitamment leurs chaises, car quand les
cuirassiers défilaient rue Sainte-Hildegarde, ils en remplissaient toute la
largeur, et le galop des chevaux rasait les maisons couvrant les trottoirs
submergés comme des berges qui offrent un lit trop étroit à un torrent déchaîné.

—«Pauvres enfants, disait Françoise à peine arrivée à la grille et déjà en
larmes; pauvre jeunesse qui sera fauchée comme un pré; rien que d’y penser j’en
suis choquée», ajoutait-elle en mettant la main sur son cœur, là où elle avait
reçu ce choc.

—«C’est beau, n’est-ce pas, madame Françoise, de voir des jeunes gens qui ne
tiennent pas à la vie? disait le jardinier pour la faire «monter».

Il n’avait pas parlé en vain:

—«De ne pas tenir à la vie? Mais à quoi donc qu’il faut tenir, si ce n’est pas à
la vie, le seul cadeau que le bon Dieu ne fasse jamais deux fois. Hélas! mon
Dieu! C’est pourtant vrai qu’ils n’y tiennent pas! Je les ai vus en 70; ils
n’ont plus peur de la mort, dans ces misérables guerres; c’est ni plus ni moins
des fous; et puis ils ne valent plus la corde pour les pendre, ce n’est pas des
hommes, c’est des lions.» (Pour Françoise la comparaison d’un homme à un lion,
qu’elle prononçait li-on, n’avait rien de flatteur.)

La rue Sainte-Hildegarde tournait trop court pour qu’on pût voir venir de loin,
et c’était par cette fente entre les deux maisons de l’avenue de la gare qu’on
apercevait toujours de nouveaux casques courant et brillant au soleil. Le
jardinier aurait voulu savoir s’il y en avait encore beaucoup à passer, et il
avait soif, car le soleil tapait. Alors tout d’un coup, sa fille s’élançant
comme d’une place assiégée, faisait une sortie, atteignait l’angle de la rue, et
après avoir bravé cent fois la mort, venait nous rapporter, avec une carafe de
coco, la nouvelle qu’ils étaient bien un mille qui venaient sans arrêter, du
côté de Thiberzy et de Méséglise. Françoise et le jardinier, réconciliés,
discutaient sur la conduite à tenir en cas de guerre:

—«Voyez-vous, Françoise, disait le jardinier, la révolution vaudrait mieux,
parce que quand on la déclare il n’y a que ceux qui veulent partir qui y vont.»

—«Ah! oui, au moins je comprends cela, c’est plus franc.»

Le jardinier croyait qu’à la déclaration de guerre on arrêtait tous les chemins
de fer.

—«Pardi, pour pas qu’on se sauve», disait Françoise.

Et le jardinier: «Ah! ils sont malins», car il n’admettait pas que la guerre ne
fût pas une espèce de mauvais tour que l’État essayait de jouer au peuple et
que, si on avait eu le moyen de le faire, il n’est pas une seule personne qui
n’eût filé.

Mais Françoise se hâtait de rejoindre ma tante, je retournais à mon livre, les
domestiques se réinstallaient devant la porte à regarder tomber la poussière et
l’émotion qu’avaient soulevées les soldats. Longtemps après que l’accalmie était
venue, un flot inaccoutumé de promeneurs noircissait encore les rues de Combray.
Et devant chaque maison, même celles où ce n’était pas l’habitude, les
domestiques ou même les maîtres, assis et regardant, festonnaient le seuil d’un
liséré capricieux et sombre comme celui des algues et des coquilles dont une
forte marée laisse le crêpe et la broderie au rivage, après qu’elle s’est
éloignée.

Sauf ces jours-là, je pouvais d’habitude, au contraire, lire tranquille. Mais
l’interruption et le commentaire qui furent apportés une fois par une visite de
Swann à la lecture que j’étais en train de faire du livre d’un auteur tout
nouveau pour moi, Bergotte, eut cette conséquence que, pour longtemps, ce ne fut
plus sur un mur décoré de fleurs violettes en quenouille, mais sur un fond tout
autre, devant le portail d’une cathédrale gothique, que se détacha désormais
l’image d’une des femmes dont je rêvais.

J’avais entendu parler de Bergotte pour la première fois par un de mes camarades
plus âgé que moi et pour qui j’avais une grande admiration, Bloch. En
m’entendant lui avouer mon admiration pour la Nuit d’Octobre, il avait fait
éclater un rire bruyant comme une trompette et m’avait dit: «Défie-toi de ta
dilection assez basse pour le sieur de Musset. C’est un coco des plus
malfaisants et une assez sinistre brute. Je dois confesser, d’ailleurs, que lui
et même le nommé Racine, ont fait chacun dans leur vie un vers assez bien
rythmé, et qui a pour lui, ce qui est selon moi le mérite suprême, de ne
signifier absolument rien. C’est: «La blanche Oloossone et la blanche Camire» et
«La fille de Minos et de Pasiphaë». Ils m’ont été signalés à la décharge de ces
deux malandrins par un article de mon très cher maître, le père Leconte,
agréable aux Dieux Immortels. A propos voici un livre que je n’ai pas le temps
de lire en ce moment qui est recommandé, paraît-il, par cet immense bonhomme. Il
tient, m’a-t-on dit, l’auteur, le sieur Bergotte, pour un coco des plus subtils;
et bien qu’il fasse preuve, des fois, de mansuétudes assez mal explicables, sa
parole est pour moi oracle delphique. Lis donc ces proses lyriques, et si le
gigantesque assembleur de rythmes qui a écrit Bhagavat et le Levrier de Magnus a
dit vrai, par Apollôn, tu goûteras, cher maître, les joies nectaréennes de
l’Olympos.» C’est sur un ton sarcastique qu’il m’avait demandé de l’appeler
«cher maître» et qu’il m’appelait lui-même ainsi. Mais en réalité nous prenions
un certain plaisir à ce jeu, étant encore rapprochés de l’âge où on croit qu’on
crée ce qu’on nomme.

Malheureusement, je ne pus pas apaiser en causant avec Bloch et en lui demandant
des explications, le trouble où il m’avait jeté quand il m’avait dit que les
beaux vers (à moi qui n’attendais d’eux rien moins que la révélation de la
vérité) étaient d’autant plus beaux qu’ils ne signifiaient rien du tout. Bloch
en effet ne fut pas réinvité à la maison. Il y avait d’abord été bien accueilli.
Mon grand-père, il est vrai, prétendait que chaque fois que je me liais avec un
de mes camarades plus qu’avec les autres et que je l’amenais chez nous, c’était
toujours un juif, ce qui ne lui eût pas déplu en principe — même son ami Swann
était d’origine juive — s’il n’avait trouvé que ce n’était pas d’habitude parmi
les meilleurs que je le choisissais. Aussi quand j’amenais un nouvel ami il
était bien rare qu’il ne fredonnât pas: «O Dieu de nos Pères» de la Juive ou
bien «Israël romps ta chaîne», ne chantant que l’air naturellement (Ti la lam ta
lam, talim), mais j’avais peur que mon camarade ne le connût et ne rétablît les
paroles.

Avant de les avoir vus, rien qu’en entendant leur nom qui, bien souvent, n’avait
rien de particulièrement israélite, il devinait non seulement l’origine juive de
ceux de mes amis qui l’étaient en effet, mais même ce qu’il y avait quelquefois
de fâcheux dans leur famille.

—«Et comment s’appelle-t-il ton ami qui vient ce soir?»

—«Dumont, grand-père.»

—«Dumont! Oh! je me méfie.»

Et il chantait:

\begin{center}\textit{
«Archers, faites bonne garde!}  \par   
\textit{Veillez sans trêve et sans bruit»; 
}\end{center}

Et après nous avoir posé adroitement quelques questions plus précises, il
s’écriait: «A la garde! A la garde!» ou, si c’était le patient lui-même déjà
arrivé qu’il avait forcé à son insu, par un interrogatoire dissimulé, à
confesser ses origines, alors pour nous montrer qu’il n’avait plus aucun doute,
il se contentait de nous regarder en fredonnant imperceptiblement:

\begin{center}\textit{
«De ce timide Israëlite} \par \textit{
Quoi! vous guidez ici les pas!» 
}\end{center}

\noindent ou:

\begin{center}\textit{
«Champs paternels, Hébron, douce vallée.»
}\end{center}

\noindent ou encore:

\begin{center}\textit{
«Oui, je suis de la race élue.»
}\end{center}

Ces petites manies de mon grand-père n’impliquaient aucun sentiment malveillant
à l’endroit de mes camarades. Mais Bloch avait déplu à mes parents pour d’autres
raisons. Il avait commencé par agacer mon père qui, le voyant mouillé, lui avait
dit avec intérêt:

—«Mais, monsieur Bloch, quel temps fait-il donc, est-ce qu’il a plu? Je n’y
comprends rien, le baromètre était excellent.»

Il n’en avait tiré que cette réponse:

—«Monsieur, je ne puis absolument vous dire s’il a plu. Je vis si résolument en
dehors des contingences physiques que mes sens ne prennent pas la peine de me
les notifier.»

—«Mais, mon pauvre fils, il est idiot ton ami, m’avait dit mon père quand Bloch
fut parti. Comment! il ne peut même pas me dire le temps qu’il fait! Mais il n’y
a rien de plus intéressant! C’est un imbécile.

Puis Bloch avait déplu à ma grand’mère parce que, après le déjeuner comme elle
disait qu’elle était un peu souffrante, il avait étouffé un sanglot et essuyé
des larmes.

—«Comment veux-tu que ça soit sincère, me dit-elle, puisqu’il ne me connaît pas;
ou bien alors il est fou.»

Et enfin il avait mécontenté tout le monde parce que, étant venu déjeuner une
heure et demie en retard et couvert de boue, au lieu de s’excuser, il avait dit:

—«Je ne me laisse jamais influencer par les perturbations de l’atmosphère ni par
les divisions conventionnelles du temps. Je réhabiliterais volontiers l’usage de
la pipe d’opium et du kriss malais, mais j’ignore celui de ces instruments
infiniment plus pernicieux et d’ailleurs platement bourgeois, la montre et le
parapluie.»

Il serait malgré tout revenu à Combray. Il n’était pas pourtant l’ami que mes
parents eussent souhaité pour moi; ils avaient fini par penser que les larmes
que lui avait fait verser l’indisposition de ma grand’mère n’étaient pas
feintes; mais ils savaient d’instinct ou par expérience que les élans de notre
sensibilité ont peu d’empire sur la suite de nos actes et la conduite de notre
vie, et que le respect des obligations morales, la fidélité aux amis,
l’exécution d’une œuvre, l’observance d’un régime, ont un fondement plus sûr
dans des habitudes aveugles que dans ces transports momentanés, ardents et
stériles. Ils auraient préféré pour moi à Bloch des compagnons qui ne me
donneraient pas plus qu’il n’est convenu d’accorder à ses amis, selon les règles
de la morale bourgeoise; qui ne m’enverraient pas inopinément une corbeille de
fruits parce qu’ils auraient ce jour-là pensé à moi avec tendresse, mais qui,
n’étant pas capables de faire pencher en ma faveur la juste balance des devoirs
et des exigences de l’amitié sur un simple mouvement de leur imagination et de
leur sensibilité, ne la fausseraient pas davantage à mon préjudice. Nos torts
même font difficilement départir de ce qu’elles nous doivent ces natures dont ma
grand’tante était le modèle, elle qui brouillée depuis des années avec une nièce
à qui elle ne parlait jamais, ne modifia pas pour cela le testament où elle lui
laissait toute sa fortune, parce que c’était sa plus proche parente et que cela
«se devait».

Mais j’aimais Bloch, mes parents voulaient me faire plaisir, les problèmes
insolubles que je me posais à propos de la beauté dénuée de signification de la
fille de Minos et de Pasiphaé me fatiguaient davantage et me rendaient plus
souffrant que n’auraient fait de nouvelles conversations avec lui, bien que ma
mère les jugeât pernicieuses. Et on l’aurait encore reçu à Combray si, après ce
dîner, comme il venait de m’apprendre — nouvelle qui plus tard eut beaucoup
d’influence sur ma vie, et la rendit plus heureuse, puis plus malheureuse — que
toutes les femmes ne pensaient qu’à l’amour et qu’il n’y en a pas dont on ne pût
vaincre les résistances, il ne m’avait assuré avoir entendu dire de la façon la
plus certaine que ma grand’tante avait eu une jeunesse orageuse et avait été
publiquement entretenue. Je ne pus me tenir de répéter ces propos à mes parents,
on le mit à la porte quand il revint, et quand je l’abordai ensuite dans la rue,
il fut extrêmement froid pour moi.

Mais au sujet de Bergotte il avait dit vrai.

Les premiers jours, comme un air de musique dont on raffolera, mais qu’on ne
distingue pas encore, ce que je devais tant aimer dans son style ne m’apparut
pas. Je ne pouvais pas quitter le roman que je lisais de lui, mais me croyais
seulement intéressé par le sujet, comme dans ces premiers moments de l’amour où
on va tous les jours retrouver une femme à quelque réunion, à quelque
divertissement par les agréments desquels on se croit attiré. Puis je remarquai
les expressions rares, presque archaïques qu’il aimait employer à certains
moments où un flot caché d’harmonie, un prélude intérieur, soulevait son style;
et c’était aussi à ces moments-là qu’il se mettait à parler du «vain songe de la
vie», de «l’inépuisable torrent des belles apparences», du «tourment stérile et
délicieux de comprendre et d’aimer», des «émouvantes effigies qui anoblissent à
jamais la façade vénérable et charmante des cathédrales», qu’il exprimait toute
une philosophie nouvelle pour moi par de merveilleuses images dont on aurait dit
que c’était elles qui avaient éveillé ce chant de harpes qui s’élevait alors et
à l’accompagnement duquel elles donnaient quelque chose de sublime. Un de ces
passages de Bergotte, le troisième ou le quatrième que j’eusse isolé du reste,
me donna une joie incomparable à celle que j’avais trouvée au premier, une joie
que je me sentis éprouver en une région plus profonde de moi-même, plus unie,
plus vaste, d’où les obstacles et les séparations semblaient avoir été enlevés.
C’est que, reconnaissant alors ce même goût pour les expressions rares, cette
même effusion musicale, cette même philosophie idéaliste qui avait déjà été les
autres fois, sans que je m’en rendisse compte, la cause de mon plaisir, je n’eus
plus l’impression d’être en présence d’un morceau particulier d’un certain livre
de Bergotte, traçant à la surface de ma pensée une figure purement linéaire,
mais plutôt du «morceau idéal» de Bergotte, commun à tous ses livres et auquel
tous les passages analogues qui venaient se confondre avec lui, auraient donné
une sorte d’épaisseur, de volume, dont mon esprit semblait agrandi.

Je n’étais pas tout à fait le seul admirateur de Bergotte; il était aussi
l’écrivain préféré d’une amie de ma mère qui était très lettrée; enfin pour lire
son dernier livre paru, le docteur du Boulbon faisait attendre ses malades; et
ce fut de son cabinet de consultation, et d’un parc voisin de Combray, que
s’envolèrent quelques-unes des premières graines de cette prédilection pour
Bergotte, espèce si rare alors, aujourd’hui universellement répandue, et dont on
trouve partout en Europe, en Amérique, jusque dans le moindre village, la fleur
idéale et commune. Ce que l’amie de ma mère et, paraît-il, le docteur du Boulbon
aimaient surtout dans les livres de Bergotte c’était comme moi, ce même flux
mélodique, ces expressions anciennes, quelques autres très simples et connues,
mais pour lesquelles la place où il les mettait en lumière semblait révéler de
sa part un goût particulier; enfin, dans les passages tristes, une certaine
brusquerie, un accent presque rauque. Et sans doute lui-même devait sentir que
là étaient ses plus grands charmes. Car dans les livres qui suivirent, s’il
avait rencontré quelque grande vérité, ou le nom d’une célèbre cathédrale, il
interrompait son récit et dans une invocation, une apostrophe, une longue
prière, il donnait un libre cours à ces effluves qui dans ses premiers ouvrages
restaient intérieurs à sa prose, décelés seulement alors par les ondulations de
la surface, plus douces peut-être encore, plus harmonieuses quand elles étaient
ainsi voilées et qu’on n’aurait pu indiquer d’une manière précise où naissait,
où expirait leur murmure. Ces morceaux auxquels il se complaisait étaient nos
morceaux préférés. Pour moi, je les savais par cœur. J’étais déçu quand il
reprenait le fil de son récit. Chaque fois qu’il parlait de quelque chose dont
la beauté m’était restée jusque-là cachée, des forêts de pins, de la grêle, de
Notre-Dame de Paris, d’Athalie ou de Phèdre, il faisait dans une image exploser
cette beauté jusqu’à moi. Aussi sentant combien il y avait de parties de
l’univers que ma perception infirme ne distinguerait pas s’il ne les rapprochait
de moi, j’aurais voulu posséder une opinion de lui, une métaphore de lui, sur
toutes choses, surtout sur celles que j’aurais l’occasion de voir moi-même, et
entre celles-là, particulièrement sur d’anciens monuments français et certains
paysages maritimes, parce que l’insistance avec laquelle il les citait dans ses
livres prouvait qu’il les tenait pour riches de signification et de beauté.
Malheureusement sur presque toutes choses j’ignorais son opinion. Je ne doutais
pas qu’elle ne fût entièrement différente des miennes, puisqu’elle descendait
d’un monde inconnu vers lequel je cherchais à m’élever: persuadé que mes pensées
eussent paru pure ineptie à cet esprit parfait, j’avais tellement fait table
rase de toutes, que quand par hasard il m’arriva d’en rencontrer, dans tel de
ses livres, une que j’avais déjà eue moi-même, mon cœur se gonflait comme si un
Dieu dans sa bonté me l’avait rendue, l’avait déclarée légitime et belle. Il
arrivait parfois qu’une page de lui disait les mêmes choses que j’écrivais
souvent la nuit à ma grand’mère et à ma mère quand je ne pouvais pas dormir, si
bien que cette page de Bergotte avait l’air d’un recueil d’épigraphes pour être
placées en tête de mes lettres. Même plus tard, quand je commençai de composer
un livre, certaines phrases dont la qualité ne suffit pas pour me décider à le
continuer, j’en retrouvai l’équivalent dans Bergotte. Mais ce n’était qu’alors,
quand je les lisais dans son œuvre, que je pouvais en jouir; quand c’était moi
qui les composais, préoccupé qu’elles reflétassent exactement ce que
j’apercevais dans ma pensée, craignant de ne pas «faire ressemblant», j’avais
bien le temps de me demander si ce que j’écrivais était agréable! Mais en
réalité il n’y avait que ce genre de phrases, ce genre d’idées que j’aimais
vraiment. Mes efforts inquiets et mécontents étaient eux-mêmes une marque
d’amour, d’amour sans plaisir mais profond. Aussi quand tout d’un coup je
trouvais de telles phrases dans l’œuvre d’un autre, c’est-à-dire sans plus avoir
de scrupules, de sévérité, sans avoir à me tourmenter, je me laissais enfin
aller avec délices au goût que j’avais pour elles, comme un cuisinier qui pour
une fois où il n’a pas à faire la cuisine trouve enfin le temps d’être gourmand.
Un jour, ayant rencontré dans un livre de Bergotte, à propos d’une vieille
servante, une plaisanterie que le magnifique et solennel langage de l’écrivain
rendait encore plus ironique mais qui était la même que j’avais souvent faite à
ma grand’mère en parlant de Françoise, une autre fois où je vis qu’il ne jugeait
pas indigne de figurer dans un de ces miroirs de la vérité qu’étaient ses
ouvrages, une remarque analogue à celle que j’avais eu l’occasion de faire sur
notre ami M. Legrandin (remarques sur Françoise et M. Legrandin qui étaient
certes de celles que j’eusse le plus délibérément sacrifiées à Bergotte,
persuadé qu’il les trouverait sans intérêt), il me sembla soudain que mon humble
vie et les royaumes du vrai n’étaient pas aussi séparés que j’avais cru, qu’ils
coïncidaient même sur certains points, et de confiance et de joie je pleurai sur
les pages de l’écrivain comme dans les bras d’un père retrouvé.

D’après ses livres j’imaginais Bergotte comme un vieillard faible et déçu qui
avait perdu des enfants et ne s’était jamais consolé. Aussi je lisais, je
chantais intérieurement sa prose, plus «dolce», plus «lento» peut-être qu’elle
n’était écrite, et la phrase la plus simple s’adressait à moi avec une
intonation attendrie. Plus que tout j’aimais sa philosophie, je m’étais donné à
elle pour toujours. Elle me rendait impatient d’arriver à l’âge où j’entrerais
au collège, dans la classe appelée Philosophie. Mais je ne voulais pas qu’on y
fît autre chose que vivre uniquement par la pensée de Bergotte, et si l’on
m’avait dit que les métaphysiciens auxquels je m’attacherais alors ne lui
ressembleraient en rien, j’aurais ressenti le désespoir d’un amoureux qui veut
aimer pour la vie et à qui on parle des autres maîtresses qu’il aura plus tard.

Un dimanche, pendant ma lecture au jardin, je fus dérangé par Swann qui venait
voir mes parents.

—«Qu’est-ce que vous lisez, on peut regarder? Tiens, du Bergotte? Qui donc vous
a indiqué ses ouvrages?» Je lui dis que c’était Bloch.

—«Ah! oui, ce garçon que j’ai vu une fois ici, qui ressemble tellement au
portrait de Mahomet II par Bellini. Oh! c’est frappant, il a les mêmes sourcils
circonflexes, le même nez recourbé, les mêmes pommettes saillantes. Quand il
aura une barbiche ce sera la même personne. En tout cas il a du goût, car
Bergotte est un charmant esprit.» Et voyant combien j’avais l’air d’admirer
Bergotte, Swann qui ne parlait jamais des gens qu’il connaissait fit, par bonté,
une exception et me dit:

—«Je le connais beaucoup, si cela pouvait vous faire plaisir qu’il écrive un mot
en tête de votre volume, je pourrais le lui demander.» Je n’osai pas accepter
mais posai à Swann des questions sur Bergotte. «Est-ce que vous pourriez me dire
quel est l’acteur qu’il préfère?»

—«L’acteur, je ne sais pas. Mais je sais qu’il n’égale aucun artiste homme à la
Berma qu’il met au-dessus de tout. L’avez-vous entendue?»

—«Non monsieur, mes parents ne me permettent pas d’aller au théâtre.»

—«C’est malheureux. Vous devriez leur demander. La Berma dans Phèdre, dans le
Cid, ce n’est qu’une actrice si vous voulez, mais vous savez je ne crois pas
beaucoup à la «hiérarchie!» des arts; (et je remarquai, comme cela m’avait
souvent frappé dans ses conversations avec les sœurs de ma grand’mère que quand
il parlait de choses sérieuses, quand il employait une expression qui semblait
impliquer une opinion sur un sujet important, il avait soin de l’isoler dans une
intonation spéciale, machinale et ironique, comme s’il l’avait mise entre
guillemets, semblant ne pas vouloir la prendre à son compte, et dire: «la
hiérarchie, vous savez, comme disent les gens ridicules»? Mais alors, si c’était
ridicule, pourquoi disait-il la hiérarchie?). Un instant après il ajouta: «Cela
vous donnera une vision aussi noble que n’importe quel chef-d’œuvre, je ne sais
pas moi . . . que»— et il se mit à rire —«les Reines de Chartres!» Jusque-là
cette horreur d’exprimer sérieusement son opinion m’avait paru quelque chose qui
devait être élégant et parisien et qui s’opposait au dogmatisme provincial des
sœurs de ma grand’mère; et je soupçonnais aussi que c’était une des formes de
l’esprit dans la coterie où vivait Swann et où par réaction sur le lyrisme des
générations antérieures on réhabilitait à l’excès les petits faits précis,
réputés vulgaires autrefois, et on proscrivait les «phrases». Mais maintenant je
trouvais quelque chose de choquant dans cette attitude de Swann en face des
choses. Il avait l’air de ne pas oser avoir une opinion et de n’être tranquille
que quand il pouvait donner méticuleusement des renseignements précis. Mais il
ne se rendait donc pas compte que c’était professer l’opinion, postuler, que
l’exactitude de ces détails avait de l’importance. Je repensai alors à ce dîner
où j’étais si triste parce que maman ne devait pas monter dans ma chambre et où
il avait dit que les bals chez la princesse de Léon n’avaient aucune importance.
Mais c’était pourtant à ce genre de plaisirs qu’il employait sa vie. Je trouvais
tout cela contradictoire. Pour quelle autre vie réservait-il de dire enfin
sérieusement ce qu’il pensait des choses, de formuler des jugements qu’il pût ne
pas mettre entre guillemets, et de ne plus se livrer avec une politesse
pointilleuse à des occupations dont il professait en même temps qu’elles sont
ridicules? Je remarquai aussi dans la façon dont Swann me parla de Bergotte
quelque chose qui en revanche ne lui était pas particulier mais au contraire
était dans ce temps-là commun à tous les admirateurs de l’écrivain, à l’amie de
ma mère, au docteur du Boulbon. Comme Swann, ils disaient de Bergotte: «C’est un
charmant esprit, si particulier, il a une façon à lui de dire les choses un peu
cherchée, mais si agréable. On n’a pas besoin de voir la signature, on reconnaît
tout de suite que c’est de lui.» Mais aucun n’aurait été jusqu’à dire: «C’est un
grand écrivain, il a un grand talent.» Ils ne disaient même pas qu’il avait du
talent. Ils ne le disaient pas parce qu’ils ne le savaient pas. Nous sommes très
longs à reconnaître dans la physionomie particulière d’un nouvel écrivain le
modèle qui porte le nom de «grand talent» dans notre musée des idées générales.
Justement parce que cette physionomie est nouvelle nous ne la trouvons pas tout
à fait ressemblante à ce que nous appelons talent. Nous disons plutôt
originalité, charme, délicatesse, force; et puis un jour nous nous rendons
compte que c’est justement tout cela le talent.

—«Est-ce qu’il y a des ouvrages de Bergotte où il ait parlé de la Berma?»
demandai-je à M. Swann.

— Je crois dans sa petite plaquette sur Racine, mais elle doit être épuisée. Il
y a peut-être eu cependant une réimpression. Je m’informerai. Je peux d’ailleurs
demander à Bergotte tout ce que vous voulez, il n’y a pas de semaine dans
l’année où il ne dîne à la maison. C’est le grand ami de ma fille. Ils vont
ensemble visiter les vieilles villes, les cathédrales, les châteaux.

Comme je n’avais aucune notion sur la hiérarchie sociale, depuis longtemps
l’impossibilité que mon père trouvait à ce que nous fréquentions Mme et Mlle
Swann avait eu plutôt pour effet, en me faisant imaginer entre elles et nous de
grandes distances, de leur donner à mes yeux du prestige. Je regrettais que ma
mère ne se teignît pas les cheveux et ne se mît pas de rouge aux lèvres comme
j’avais entendu dire par notre voisine Mme Sazerat que Mme Swann le faisait pour
plaire, non à son mari, mais à M. de Charlus, et je pensais que nous devions
être pour elle un objet de mépris, ce qui me peinait surtout à cause de Mlle
Swann qu’on m’avait dit être une si jolie petite fille et à laquelle je rêvais
souvent en lui prêtant chaque fois un même visage arbitraire et charmant. Mais
quand j’eus appris ce jour-là que Mlle Swann était un être d’une condition si
rare, baignant comme dans son élément naturel au milieu de tant de privilèges,
que quand elle demandait à ses parents s’il y avait quelqu’un à dîner, on lui
répondait par ces syllabes remplies de lumière, par le nom de ce convive d’or
qui n’était pour elle qu’un vieil ami de sa famille: Bergotte; que, pour elle,
la causerie intime à table, ce qui correspondait à ce qu’était pour moi la
conversation de ma grand’tante, c’étaient des paroles de Bergotte sur tous ces
sujets qu’il n’avait pu aborder dans ses livres, et sur lesquels j’aurais voulu
l’écouter rendre ses oracles, et qu’enfin, quand elle allait visiter des villes,
il cheminait à côté d’elle, inconnu et glorieux, comme les Dieux qui
descendaient au milieu des mortels, alors je sentis en même temps que le prix
d’un être comme Mlle Swann, combien je lui paraîtrais grossier et ignorant, et
j’éprouvai si vivement la douceur et l’impossibilité qu’il y aurait pour moi à
être son ami, que je fus rempli à la fois de désir et de désespoir. Le plus
souvent maintenant quand je pensais à elle, je la voyais devant le porche d’une
cathédrale, m’expliquant la signification des statues, et, avec un sourire qui
disait du bien de moi, me présentant comme son ami, à Bergotte. Et toujours le
charme de toutes les idées que faisaient naître en moi les cathédrales, le
charme des coteaux de l’Ile-de-France et des plaines de la Normandie faisait
refluer ses reflets sur l’image que je me formais de Mlle Swann: c’était être
tout prêt à l’aimer. Que nous croyions qu’un être participe à une vie inconnue
où son amour nous ferait pénétrer, c’est, de tout ce qu’exige l’amour pour
naître, ce à quoi il tient le plus, et qui lui fait faire bon marché du reste.
Même les femmes qui prétendent ne juger un homme que sur son physique, voient en
ce physique l’émanation d’une vie spéciale. C’est pourquoi elles aiment les
militaires, les pompiers; l’uniforme les rend moins difficiles pour le visage;
elles croient baiser sous la cuirasse un cœur différent, aventureux et doux; et
un jeune souverain, un prince héritier, pour faire les plus flatteuses
conquêtes, dans les pays étrangers qu’il visite, n’a pas besoin du profil
régulier qui serait peut-être indispensable à un coulissier.

Tandis que je lisais au jardin, ce que ma grand’tante n’aurait pas compris que
je fisse en dehors du dimanche, jour où il est défendu de s’occuper à rien de
sérieux et où elle ne cousait pas (un jour de semaine, elle m’aurait dit
«Comment tu t’amuses encore à lire, ce n’est pourtant pas dimanche» en donnant
au mot amusement le sens d’enfantillage et de perte de temps), ma tante Léonie
devisait avec Françoise en attendant l’heure d’Eulalie. Elle lui annonçait
qu’elle venait de voir passer Mme Goupil «sans parapluie, avec la robe de soie
qu’elle s’est fait faire à Châteaudun. Si elle a loin à aller avant vêpres elle
pourrait bien la faire saucer».

—«Peut-être, peut-être (ce qui signifiait peut-être non)» disait Françoise pour
ne pas écarter définitivement la possibilité d’une alternative plus favorable.

—«Tiens, disait ma tante en se frappant le front, cela me fait penser que je
n’ai point su si elle était arrivée à l’église après l’élévation. Il faudra que
je pense à le demander à Eulalie . . . Françoise, regardez-moi ce nuage noir
derrière le clocher et ce mauvais soleil sur les ardoises, bien sûr que la
journée ne se passera pas sans pluie. Ce n’était pas possible que ça reste comme
ça, il faisait trop chaud. Et le plus tôt sera le mieux, car tant que l’orage
n’aura pas éclaté, mon eau de Vichy ne descendra pas, ajoutait ma tante dans
l’esprit de qui le désir de hâter la descente de l’eau de Vichy l’emportait
infiniment sur la crainte de voir Mme Goupil gâter sa robe.»

—«Peut-être, peut-être.»

—«Et c’est que, quand il pleut sur la place, il n’y a pas grand abri.»

—«Comment, trois heures? s’écriait tout à coup ma tante en pâlissant, mais alors
les vêpres sont commencées, j’ai oublié ma pepsine! Je comprends maintenant
pourquoi mon eau de Vichy me restait sur l’estomac.»

Et se précipitant sur un livre de messe relié en velours violet, monté d’or, et
d’où, dans sa hâte, elle laissait s’échapper de ces images, bordées d’un bandeau
de dentelle de papier jaunissante, qui marquent les pages des fêtes, ma tante,
tout en avalant ses gouttes commençait à lire au plus vite les textes sacrés
dont l’intelligence lui était légèrement obscurcie par l’incertitude de savoir
si, prise aussi longtemps après l’eau de Vichy, la pepsine serait encore capable
de la rattraper et de la faire descendre. «Trois heures, c’est incroyable ce que
le temps passe!»

Un petit coup au carreau, comme si quelque chose l’avait heurté, suivi d’une
ample chute légère comme de grains de sable qu’on eût laissé tomber d’une
fenêtre au-dessus, puis la chute s’étendant, se réglant, adoptant un rythme,
devenant fluide, sonore, musicale, innombrable, universelle: c’était la pluie.

—«Eh bien! Françoise, qu’est-ce que je disais? Ce que cela tombe! Mais je crois
que j’ai entendu le grelot de la porte du jardin, allez donc voir qui est-ce qui
peut être dehors par un temps pareil.»

Françoise revenait:

—«C’est Mme Amédée (ma grand’mère) qui a dit qu’elle allait faire un tour. Ça
pleut pourtant fort.»

— Cela ne me surprend point, disait ma tante en levant les yeux au ciel. J’ai
toujours dit qu’elle n’avait point l’esprit fait comme tout le monde. J’aime
mieux que ce soit elle que moi qui soit dehors en ce moment.

— Mme Amédée, c’est toujours tout l’extrême des autres, disait Françoise avec
douceur, réservant pour le moment où elle serait seule avec les autres
domestiques, de dire qu’elle croyait ma grand’mère un peu «piquée».

— Voilà le salut passé! Eulalie ne viendra plus, soupirait ma tante; ce sera le
temps qui lui aura fait peur.»

—«Mais il n’est pas cinq heures, madame Octave, il n’est que quatre heures et
demie.»

— Que quatre heures et demie? et j’ai été obligée de relever les petits rideaux
pour avoir un méchant rayon de jour. A quatre heures et demie! Huit jours avant
les Rogations! Ah! ma pauvre Françoise, il faut que le bon Dieu soit bien en
colère après nous. Aussi, le monde d’aujourd’hui en fait trop! Comme disait mon
pauvre Octave, on a trop oublié le bon Dieu et il se venge.

Une vive rougeur animait les joues de ma tante, c’était Eulalie.
Malheureusement, à peine venait-elle d’être introduite que Françoise rentrait et
avec un sourire qui avait pour but de se mettre elle-même à l’unisson de la joie
qu’elle ne doutait pas que ses paroles allaient causer à ma tante, articulant
les syllabes pour montrer que, malgré l’emploi du style indirect, elle
rapportait, en bonne domestique, les paroles mêmes dont avait daigné se servir
le visiteur:

—«M. le Curé serait enchanté, ravi, si Madame Octave ne repose pas et pouvait le
recevoir. M. le Curé ne veut pas déranger. M. le Curé est en bas, j’y ai dit
d’entrer dans la salle.»

En réalité, les visites du curé ne faisaient pas à ma tante un aussi grand
plaisir que le supposait Françoise et l’air de jubilation dont celle-ci croyait
devoir pavoiser son visage chaque fois qu’elle avait à l’annoncer ne répondait
pas entièrement au sentiment de la malade. Le curé (excellent homme avec qui je
regrette de ne pas avoir causé davantage, car s’il n’entendait rien aux arts, il
connaissait beaucoup d’étymologies), habitué à donner aux visiteurs de marque
des renseignements sur l’église (il avait même l’intention d’écrire un livre sur
la paroisse de Combray), la fatiguait par des explications infinies et
d’ailleurs toujours les mêmes. Mais quand elle arrivait ainsi juste en même
temps que celle d’Eulalie, sa visite devenait franchement désagréable à ma
tante. Elle eût mieux aimé bien profiter d’Eulalie et ne pas avoir tout le monde
à la fois. Mais elle n’osait pas ne pas recevoir le curé et faisait seulement
signe à Eulalie de ne pas s’en aller en même temps que lui, qu’elle la garderait
un peu seule quand il serait parti.

—«Monsieur le Curé, qu’est-ce que l’on me disait, qu’il y a un artiste qui a
installé son chevalet dans votre église pour copier un vitrail. Je peux dire que
je suis arrivée à mon âge sans avoir jamais entendu parler d’une chose pareille!
Qu’est-ce que le monde aujourd’hui va donc chercher! Et ce qu’il y a de plus
vilain dans l’église!»

—«Je n’irai pas jusqu’à dire que c’est ce qu’il y a de plus vilain, car s’il y a
à Saint-Hilaire des parties qui méritent d’être visitées, il y en a d’autres qui
sont bien vieilles, dans ma pauvre basilique, la seule de tout le diocèse qu’on
n’ait même pas restaurée! Mon dieu, le porche est sale et antique, mais enfin
d’un caractère majestueux; passe même pour les tapisseries d’Esther dont
personnellement je ne donnerais pas deux sous, mais qui sont placées par les
connaisseurs tout de suite après celles de Sens. Je reconnais d’ailleurs, qu’à
côté de certains détails un peu réalistes, elles en présentent d’autres qui
témoignent d’un véritable esprit d’observation. Mais qu’on ne vienne pas me
parler des vitraux. Cela a-t-il du bon sens de laisser des fenêtres qui ne
donnent pas de jour et trompent même la vue par ces reflets d’une couleur que je
ne saurais définir, dans une église où il n’y a pas deux dalles qui soient au
même niveau et qu’on se refuse à me remplacer sous prétexte que ce sont les
tombes des abbés de Combray et des seigneurs de Guermantes, les anciens comtes
de Brabant. Les ancêtres directs du duc de Guermantes d’aujourd’hui et aussi de
la Duchesse puisqu’elle est une demoiselle de Guermantes qui a épousé son
cousin.» (Ma grand’mère qui à force de se désintéresser des personnes finissait
par confondre tous les noms, chaque fois qu’on prononçait celui de la Duchesse
de Guermantes prétendait que ce devait être une parente de Mme de Villeparisis.
Tout le monde éclatait de rire; elle tâchait de se défendre en alléguant une
certaine lettre de faire part: «Il me semblait me rappeler qu’il y avait du
Guermantes là-dedans.» Et pour une fois j’étais avec les autres contre elle, ne
pouvant admettre qu’il y eût un lien entre son amie de pension et la descendante
de Geneviève de Brabant.)—«Voyez Roussainville, ce n’est plus aujourd’hui qu’une
paroisse de fermiers, quoique dans l’antiquité cette localité ait dû un grand
essor au commerce de chapeaux de feutre et des pendules. (Je ne suis pas certain
de l’étymologie de Roussainville. Je croirais volontiers que le nom primitif
était Rouville (Radulfi villa) comme Châteauroux (Castrum Radulfi) mais je vous
parlerai de cela une autre fois. Hé bien! l’église a des vitraux superbes,
presque tous modernes, et cette imposante Entrée de Louis-Philippe à Combray qui
serait mieux à sa place à Combray même, et qui vaut, dit-on, la fameuse verrière
de Chartres. Je voyais même hier le frère du docteur Percepied qui est amateur
et qui la regarde comme d’un plus beau travail.

«Mais, comme je le lui disais, à cet artiste qui semble du reste très poli, qui
est paraît-il, un véritable virtuose du pinceau, que lui trouvez-vous donc
d’extraordinaire à ce vitrail, qui est encore un peu plus sombre que les
autres?»

—«Je suis sûre que si vous le demandiez à Monseigneur, disait mollement ma tante
qui commençait à penser qu’elle allait être fatiguée, il ne vous refuserait pas
un vitrail neuf.»

—«Comptez-y, madame Octave, répondait le curé. Mais c’est justement Monseigneur
qui a attaché le grelot à cette malheureuse verrière en prouvant qu’elle
représente Gilbert le Mauvais, sire de Guermantes, le descendant direct de
Geneviève de Brabant qui était une demoiselle de Guermantes, recevant
l’absolution de Saint-Hilaire.»

—«Mais je ne vois pas où est Saint-Hilaire?

—«Mais si, dans le coin du vitrail vous n’avez jamais remarqué une dame en robe
jaune? Hé bien! c’est Saint-Hilaire qu’on appelle aussi, vous le savez, dans
certaines provinces, Saint-Illiers, Saint-Hélier, et même, dans le Jura,
Saint-Ylie. Ces diverses corruptions de sanctus Hilarius ne sont pas du reste
les plus curieuses de celles qui se sont produites dans les noms des
bienheureux. Ainsi votre patronne, ma bonne Eulalie, sancta Eulalia, savez-vous
ce qu’elle est devenue en Bourgogne? Saint-Eloi tout simplement: elle est
devenue un saint. Voyez-vous, Eulalie, qu’après votre mort on fasse de vous un
homme?»—«Monsieur le Curé a toujours le mot pour rigoler.»—«Le frère de Gilbert,
Charles le Bègue, prince pieux mais qui, ayant perdu de bonne heure son père,
Pépin l’Insensé, mort des suites de sa maladie mentale, exerçait le pouvoir
suprême avec toute la présomption d’une jeunesse à qui la discipline a manqué;
dès que la figure d’un particulier ne lui revenait pas dans une ville, il y
faisait massacrer jusqu’au dernier habitant. Gilbert voulant se venger de
Charles fit brûler l’église de Combray, la primitive église alors, celle que
Théodebert, en quittant avec sa cour la maison de campagne qu’il avait près
d’ici, à Thiberzy (Theodeberciacus), pour aller combattre les Burgondes, avait
promis de bâtir au-dessus du tombeau de Saint-Hilaire, si le Bienheureux lui
procurait la victoire. Il n’en reste que la crypte où Théodore a dû vous faire
descendre, puisque Gilbert brûla le reste. Ensuite il défit l’infortuné Charles
avec l’aide de Guillaume Le Conquérant (le curé prononçait Guilôme), ce qui fait
que beaucoup d’Anglais viennent pour visiter. Mais il ne semble pas avoir su se
concilier la sympathie des habitants de Combray, car ceux-ci se ruèrent sur lui
à la sortie de la messe et lui tranchèrent la tête. Du reste Théodore prête un
petit livre qui donne les explications.

«Mais ce qui est incontestablement le plus curieux dans notre église, c’est le
point de vue qu’on a du clocher et qui est grandiose. Certainement, pour vous
qui n’êtes pas très forte, je ne vous conseillerais pas de monter nos
quatre-vingt-dix-sept marches, juste la moitié du célèbre dôme de Milan. Il y a
de quoi fatiguer une personne bien portante, d’autant plus qu’on monte plié en
deux si on ne veut pas se casser la tête, et on ramasse avec ses effets toutes
les toiles d’araignées de l’escalier. En tous cas il faudrait bien vous couvrir,
ajoutait-il (sans apercevoir l’indignation que causait à ma tante l’idée qu’elle
fût capable de monter dans le clocher), car il fait un de ces courants d’air une
fois arrivé là-haut! Certaines personnes affirment y avoir ressenti le froid de
la mort. N’importe, le dimanche il y a toujours des sociétés qui viennent même
de très loin pour admirer la beauté du panorama et qui s’en retournent
enchantées. Tenez, dimanche prochain, si le temps se maintient, vous trouveriez
certainement du monde, comme ce sont les Rogations. Il faut avouer du reste
qu’on jouit de là d’un coup d’œil féerique, avec des sortes d’échappées sur la
plaine qui ont un cachet tout particulier. Quand le temps est clair on peut
distinguer jusqu’à Verneuil. Surtout on embrasse à la fois des choses qu’on ne
peut voir habituellement que l’une sans l’autre, comme le cours de la Vivonne et
les fossés de Saint-Assise-lès-Combray, dont elle est séparée par un rideau de
grands arbres, ou encore comme les différents canaux de Jouy-le-Vicomte
(Gaudiacus vice comitis comme vous savez). Chaque fois que je suis allé à
Jouy-le-Vicomte, j’ai bien vu un bout du canal, puis quand j’avais tourné une
rue j’en voyais un autre, mais alors je ne voyais plus le précédent. J’avais
beau les mettre ensemble par la pensée, cela ne me faisait pas grand effet. Du
clocher de Saint-Hilaire c’est autre chose, c’est tout un réseau où la localité
est prise. Seulement on ne distingue pas d’eau, on dirait de grandes fentes qui
coupent si bien la ville en quartiers, qu’elle est comme une brioche dont les
morceaux tiennent ensemble mais sont déjà découpés. Il faudrait pour bien faire
être à la fois dans le clocher de Saint-Hilaire et à Jouy-le-Vicomte.»

Le curé avait tellement fatigué ma tante qu’à peine était-il parti, elle était
obligée de renvoyer Eulalie.

—«Tenez, ma pauvre Eulalie, disait-elle d’une voix faible, en tirant une pièce
d’une petite bourse qu’elle avait à portée de sa main, voilà pour que vous ne
m’oubliiez pas dans vos prières.»

—«Ah! mais, madame Octave, je ne sais pas si je dois, vous savez bien que ce
n’est pas pour cela que je viens!» disait Eulalie avec la même hésitation et le
même embarras, chaque fois, que si c’était la première, et avec une apparence de
mécontentement qui égayait ma tante mais ne lui déplaisait pas, car si un jour
Eulalie, en prenant la pièce, avait un air un peu moins contrarié que de
coutume, ma tante disait:

—«Je ne sais pas ce qu’avait Eulalie; je lui ai pourtant donné la même chose que
d’habitude, elle n’avait pas l’air contente.»

— Je crois qu’elle n’a pourtant pas à se plaindre, soupirait Françoise, qui
avait une tendance à considérer comme de la menue monnaie tout ce que lui
donnait ma tante pour elle ou pour ses enfants, et comme des trésors follement
gaspillés pour une ingrate les piécettes mises chaque dimanche dans la main
d’Eulalie, mais si discrètement que Françoise n’arrivait jamais à les voir. Ce
n’est pas que l’argent que ma tante donnait à Eulalie, Françoise l’eût voulu
pour elle. Elle jouissait suffisamment de ce que ma tante possédait, sachant que
les richesses de la maîtresse du même coup élèvent et embellissent aux yeux de
tous sa servante; et qu’elle, Françoise, était insigne et glorifiée dans
Combray, Jouy-le-Vicomte et autres lieux, pour les nombreuses fermes de ma
tante, les visites fréquentes et prolongées du curé, le nombre singulier des
bouteilles d’eau de Vichy consommées. Elle n’était avare que pour ma tante; si
elle avait géré sa fortune, ce qui eût été son rêve, elle l’aurait préservée des
entreprises d’autrui avec une férocité maternelle. Elle n’aurait pourtant pas
trouvé grand mal à ce que ma tante, qu’elle savait incurablement généreuse, se
fût laissée aller à donner, si au moins ç’avait été à des riches. Peut-être
pensait-elle que ceux-là, n’ayant pas besoin des cadeaux de ma tante, ne
pouvaient être soupçonnés de l’aimer à cause d’eux. D’ailleurs offerts à des
personnes d’une grande position de fortune, à Mme Sazerat, à M. Swann, à M.
Legrandin, à Mme Goupil, à des personnes «de même rang» que ma tante et qui
«allaient bien ensemble», ils lui apparaissaient comme faisant partie des usages
de cette vie étrange et brillante des gens riches qui chassent, se donnent des
bals, se font des visites et qu’elle admirait en souriant. Mais il n’en allait
plus de même si les bénéficiaires de la générosité de ma tante étaient de ceux
que Françoise appelait «des gens comme moi, des gens qui ne sont pas plus que
moi» et qui étaient ceux qu’elle méprisait le plus à moins qu’ils ne
l’appelassent «Madame Françoise» et ne se considérassent comme étant «moins
qu’elle». Et quand elle vit que, malgré ses conseils, ma tante n’en faisait qu’à
sa tête et jetait l’argent — Françoise le croyait du moins — pour des créatures
indignes, elle commença à trouver bien petits les dons que ma tante lui faisait
en comparaison des sommes imaginaires prodiguées à Eulalie. Il n’y avait pas
dans les environs de Combray de ferme si conséquente que Françoise ne supposât
qu’Eulalie eût pu facilement l’acheter, avec tout ce que lui rapporteraient ses
visites. Il est vrai qu’Eulalie faisait la même estimation des richesses
immenses et cachées de Françoise. Habituellement, quand Eulalie était partie,
Françoise prophétisait sans bienveillance sur son compte. Elle la haïssait, mais
elle la craignait et se croyait tenue, quand elle était là, à lui faire «bon
visage». Elle se rattrapait après son départ, sans la nommer jamais à vrai dire,
mais en proférant des oracles sibyllins, des sentences d’un caractère général
telles que celles de l’Ecclésiaste, mais dont l’application ne pouvait échapper
à ma tante. Après avoir regardé par le coin du rideau si Eulalie avait refermé
la porte: «Les personnes flatteuses savent se faire bien venir et ramasser les
pépettes; mais patience, le bon Dieu les punit toutes par un beau jour»,
disait-elle, avec le regard latéral et l’insinuation de Joas pensant
exclusivement à Athalie quand il dit:

\begin{center}\textit{
Le bonheur des méchants comme un torrent s’écoule.
}\end{center}

Mais quand le curé était venu aussi et que sa visite interminable avait épuisé
les forces de ma tante, Françoise sortait de la chambre derrière Eulalie et
disait:

—«Madame Octave, je vous laisse reposer, vous avez l’air beaucoup fatiguée.»

Et ma tante ne répondait même pas, exhalant un soupir qui semblait devoir être
le dernier, les yeux clos, comme morte. Mais à peine Françoise était-elle
descendue que quatre coups donnés avec la plus grande violence retentissaient
dans la maison et ma tante, dressée sur son lit, criait:

—«Est-ce qu’Eulalie est déjà partie? Croyez-vous que j’ai oublié de lui demander
si Mme Goupil était arrivée à la messe avant l’élévation! Courez vite après
elle!»

Mais Françoise revenait n’ayant pu rattraper Eulalie.

—«C’est contrariant, disait ma tante en hochant la tête. La seule chose
importante que j’avais à lui demander!»

Ainsi passait la vie pour ma tante Léonie, toujours identique, dans la douce
uniformité de ce qu’elle appelait avec un dédain affecté et une tendresse
profonde, son «petit traintrain». Préservé par tout le monde, non seulement à la
maison, où chacun ayant éprouvé l’inutilité de lui conseiller une meilleure
hygiène, s’était peu à peu résigné à le respecter, mais même dans le village où,
à trois rues de nous, l’emballeur, avant de clouer ses caisses, faisait demander
à Françoise si ma tante ne «reposait pas» — ce traintrain fut pourtant troublé
une fois cette année-là. Comme un fruit caché qui serait parvenu à maturité sans
qu’on s’en aperçût et se détacherait spontanément, survint une nuit la
délivrance de la fille de cuisine. Mais ses douleurs étaient intolérables, et
comme il n’y avait pas de sage-femme à Combray, Françoise dut partir avant le
jour en chercher une à Thiberzy. Ma tante, à cause des cris de la fille de
cuisine, ne put reposer, et Françoise, malgré la courte distance, n’étant
revenue que très tard, lui manqua beaucoup. Aussi, ma mère me dit-elle dans la
matinée: «Monte donc voir si ta tante n’a besoin de rien.» J’entrai dans la
première pièce et, par la porte ouverte, vis ma tante, couchée sur le côté, qui
dormait; je l’entendis ronfler légèrement. J’allais m’en aller doucement mais
sans doute le bruit que j’avais fait était intervenu dans son sommeil et en
avait «changé la vitesse», comme on dit pour les automobiles, car la musique du
ronflement s’interrompit une seconde et reprit un ton plus bas, puis elle
s’éveilla et tourna à demi son visage que je pus voir alors; il exprimait une
sorte de terreur; elle venait évidemment d’avoir un rêve affreux; elle ne
pouvait me voir de la façon dont elle était placée, et je restais là ne sachant
si je devais m’avancer ou me retirer; mais déjà elle semblait revenue au
sentiment de la réalité et avait reconnu le mensonge des visions qui l’avaient
effrayée; un sourire de joie, de pieuse reconnaissance envers Dieu qui permet
que la vie soit moins cruelle que les rêves, éclaira faiblement son visage, et
avec cette habitude qu’elle avait prise de se parler à mi-voix à elle-même quand
elle se croyait seule, elle murmura: «Dieu soit loué! nous n’avons comme tracas
que la fille de cuisine qui accouche. Voilà-t-il pas que je rêvais que mon
pauvre Octave était ressuscité et qu’il voulait me faire faire une promenade
tous les jours!» Sa main se tendit vers son chapelet qui était sur la petite
table, mais le sommeil recommençant ne lui laissa pas la force de l’atteindre:
elle se rendormit, tranquillisée, et je sortis à pas de loup de la chambre sans
qu’elle ni personne eût jamais appris ce que j’avais entendu.

Quand je dis qu’en dehors d’événements très rares, comme cet accouchement, le
traintrain de ma tante ne subissait jamais aucune variation, je ne parle pas de
celles qui, se répétant toujours identiques à des intervalles réguliers,
n’introduisaient au sein de l’uniformité qu’une sorte d’uniformité secondaire.
C’est ainsi que tous les samedis, comme Françoise allait dans l’après-midi au
marché de Roussainville-le-Pin, le déjeuner était, pour tout le monde, une heure
plus tôt. Et ma tante avait si bien pris l’habitude de cette dérogation
hebdomadaire à ses habitudes, qu’elle tenait à cette habitude-là autant qu’aux
autres. Elle y était si bien «routinée», comme disait Françoise, que s’il lui
avait fallu un samedi, attendre pour déjeuner l’heure habituelle, cela l’eût
autant «dérangée» que si elle avait dû, un autre jour, avancer son déjeuner à
l’heure du samedi. Cette avance du déjeuner donnait d’ailleurs au samedi, pour
nous tous, une figure particulière, indulgente, et assez sympathique. Au moment
où d’habitude on a encore une heure à vivre avant la détente du repas, on savait
que, dans quelques secondes, on allait voir arriver des endives précoces, une
omelette de faveur, un bifteck immérité. Le retour de ce samedi asymétrique
était un de ces petits événements intérieurs, locaux, presque civiques qui, dans
les vies tranquilles et les sociétés fermées, créent une sorte de lien national
et deviennent le thème favori des conversations, des plaisanteries, des récits
exagérés à plaisir: il eût été le noyau tout prêt pour un cycle légendaire si
l’un de nous avait eu la tête épique. Dès le matin, avant d’être habillés, sans
raison, pour le plaisir d’éprouver la force de la solidarité, on se disait les
uns aux autres avec bonne humeur, avec cordialité, avec patriotisme: «Il n’y a
pas de temps à perdre, n’oublions pas que c’est samedi!» cependant que ma tante,
conférant avec Françoise et songeant que la journée serait plus longue que
d’habitude, disait: «Si vous leur faisiez un beau morceau de veau, comme c’est
samedi.» Si à dix heures et demie un distrait tirait sa montre en disant:
«Allons, encore une heure et demie avant le déjeuner», chacun était enchanté
d’avoir à lui dire: «Mais voyons, à quoi pensez-vous, vous oubliez que c’est
samedi!»; on en riait encore un quart d’heure après et on se promettait de
monter raconter cet oubli à ma tante pour l’amuser. Le visage du ciel même
semblait changé. Après le déjeuner, le soleil, conscient que c’était samedi,
flânait une heure de plus au haut du ciel, et quand quelqu’un, pensant qu’on
était en retard pour la promenade, disait: «Comment, seulement deux heures?» en
voyant passer les deux coups du clocher de Saint-Hilaire (qui ont l’habitude de
ne rencontrer encore personne dans les chemins désertés à cause du repas de midi
ou de la sieste, le long de la rivière vive et blanche que le pêcheur même a
abandonnée, et passent solitaires dans le ciel vacant où ne restent que quelques
nuages paresseux), tout le monde en chœur lui répondait: «Mais ce qui vous
trompe, c’est qu’on a déjeuné une heure plus tôt, vous savez bien que c’est
samedi!» La surprise d’un barbare (nous appelions ainsi tous les gens qui ne
savaient pas ce qu’avait de particulier le samedi) qui, étant venu à onze heures
pour parler à mon père, nous avait trouvés à table, était une des choses qui,
dans sa vie, avaient le plus égayé Françoise. Mais si elle trouvait amusant que
le visiteur interloqué ne sût pas que nous déjeunions plus tôt le samedi, elle
trouvait plus comique encore (tout en sympathisant du fond du cœur avec ce
chauvinisme étroit) que mon père, lui, n’eût pas eu l’idée que ce barbare
pouvait l’ignorer et eût répondu sans autre explication à son étonnement de nous
voir déjà dans la salle à manger: «Mais voyons, c’est samedi!» Parvenue à ce
point de son récit, elle essuyait des larmes d’hilarité et pour accroître le
plaisir qu’elle éprouvait, elle prolongeait le dialogue, inventait ce qu’avait
répondu le visiteur à qui ce «samedi» n’expliquait rien. Et bien loin de nous
plaindre de ses additions, elles ne nous suffisaient pas encore et nous disions:
«Mais il me semblait qu’il avait dit aussi autre chose. C’était plus long la
première fois quand vous l’avez raconté.» Ma grand’tante elle-même laissait son
ouvrage, levait la tête et regardait par-dessus son lorgnon.

Le samedi avait encore ceci de particulier que ce jour-là, pendant le mois de
mai, nous sortions après le dîner pour aller au «mois de Marie».

Comme nous y rencontrions parfois M. Vinteuil, très sévère pour «le genre
déplorable des jeunes gens négligés, dans les idées de l’époque actuelle», ma
mère prenait garde que rien ne clochât dans ma tenue, puis on partait pour
l’église. C’est au mois de Marie que je me souviens d’avoir commencé à aimer les
aubépines. N’étant pas seulement dans l’église, si sainte, mais où nous avions
le droit d’entrer, posées sur l’autel même, inséparables des mystères à la
célébration desquels elles prenaient part, elles faisaient courir au milieu des
flambeaux et des vases sacrés leurs branches attachées horizontalement les unes
aux autres en un apprêt de fête, et qu’enjolivaient encore les festons de leur
feuillage sur lequel étaient semés à profusion, comme sur une traîne de mariée,
de petits bouquets de boutons d’une blancheur éclatante. Mais, sans oser les
regarder qu’à la dérobée, je sentais que ces apprêts pompeux étaient vivants et
que c’était la nature elle-même qui, en creusant ces découpures dans les
feuilles, en ajoutant l’ornement suprême de ces blancs boutons, avait rendu
cette décoration digne de ce qui était à la fois une réjouissance populaire et
une solennité mystique. Plus haut s’ouvraient leurs corolles çà et là avec une
grâce insouciante, retenant si négligemment comme un dernier et vaporeux atour
le bouquet d’étamines, fines comme des fils de la Vierge, qui les embrumait tout
entières, qu’en suivant, qu’en essayant de mimer au fond de moi le geste de leur
efflorescence, je l’imaginais comme si ç’avait été le mouvement de tête étourdi
et rapide, au regard coquet, aux pupilles diminuées, d’une blanche jeune fille,
distraite et vive. M. Vinteuil était venu avec sa fille se placer à côté de
nous. D’une bonne famille, il avait été le professeur de piano des sœurs de ma
grand’mère et quand, après la mort de sa femme et un héritage qu’il avait fait,
il s’était retiré auprès de Combray, on le recevait souvent à la maison. Mais
d’une pudibonderie excessive, il cessa de venir pour ne pas rencontrer Swann qui
avait fait ce qu’il appelait «un mariage déplacé, dans le goût du jour». Ma
mère, ayant appris qu’il composait, lui avait dit par amabilité que, quand elle
irait le voir, il faudrait qu’il lui fît entendre quelque chose de lui. M.
Vinteuil en aurait eu beaucoup de joie, mais il poussait la politesse et la
bonté jusqu’à de tels scrupules que, se mettant toujours à la place des autres,
il craignait de les ennuyer et de leur paraître égoïste s’il suivait ou
seulement laissait deviner son désir. Le jour où mes parents étaient allés chez
lui en visite, je les avais accompagnés, mais ils m’avaient permis de rester
dehors et, comme la maison de M. Vinteuil, Montjouvain, était en contre-bas d’un
monticule buissonneux, où je m’étais caché, je m’étais trouvé de plain-pied avec
le salon du second étage, à cinquante centimètres de la fenêtre. Quand on était
venu lui annoncer mes parents, j’avais vu M. Vinteuil se hâter de mettre en
évidence sur le piano un morceau de musique. Mais une fois mes parents entrés,
il l’avait retiré et mis dans un coin. Sans doute avait-il craint de leur
laisser supposer qu’il n’était heureux de les voir que pour leur jouer de ses
compositions. Et chaque fois que ma mère était revenue à la charge au cours de
la visite, il avait répété plusieurs fois «Mais je ne sais qui a mis cela sur le
piano, ce n’est pas sa place», et avait détourné la conversation sur d’autres
sujets, justement parce que ceux-là l’intéressaient moins. Sa seule passion
était pour sa fille et celle-ci qui avait l’air d’un garçon paraissait si
robuste qu’on ne pouvait s’empêcher de sourire en voyant les précautions que son
père prenait pour elle, ayant toujours des châles supplémentaires à lui jeter
sur les épaules. Ma grand’mère faisait remarquer quelle expression douce
délicate, presque timide passait souvent dans les regards de cette enfant si
rude, dont le visage était semé de taches de son. Quand elle venait de prononcer
une parole elle l’entendait avec l’esprit de ceux à qui elle l’avait dite,
s’alarmait des malentendus possibles et on voyait s’éclairer, se découper comme
par transparence, sous la figure hommasse du «bon diable», les traits plus fins
d’une jeune fille éplorée.

Quand, au moment de quitter l’église, je m’agenouillai devant l’autel, je sentis
tout d’un coup, en me relevant, s’échapper des aubépines une odeur amère et
douce d’amandes, et je remarquai alors sur les fleurs de petites places plus
blondes, sous lesquelles je me figurai que devait être cachée cette odeur comme
sous les parties gratinées le goût d’une frangipane ou sous leurs taches de
rousseur celui des joues de Mlle Vinteuil. Malgré la silencieuse immobilité des
aubépines, cette intermittente ardeur était comme le murmure de leur vie intense
dont l’autel vibrait ainsi qu’une haie agreste visitée par de vivantes antennes,
auxquelles on pensait en voyant certaines étamines presque rousses qui
semblaient avoir gardé la virulence printanière, le pouvoir irritant, d’insectes
aujourd’hui métamorphosés en fleurs.

Nous causions un moment avec M. Vinteuil devant le porche en sortant de
l’église. Il intervenait entre les gamins qui se chamaillaient sur la place,
prenait la défense des petits, faisait des sermons aux grands. Si sa fille nous
disait de sa grosse voix combien elle avait été contente de nous voir, aussitôt
il semblait qu’en elle-même une sœur plus sensible rougissait de ce propos de
bon garçon étourdi qui avait pu nous faire croire qu’elle sollicitait d’être
invitée chez nous. Son père lui jetait un manteau sur les épaules, ils montaient
dans un petit buggy qu’elle conduisait elle-même et tous deux retournaient à
Montjouvain. Quant à nous, comme c’était le lendemain dimanche et qu’on ne se
lèverait que pour la grand’messe, s’il faisait clair de lune et que l’air fût
chaud, au lieu de nous faire rentrer directement, mon père, par amour de la
gloire, nous faisait faire par le calvaire une longue promenade, que le peu
d’aptitude de ma mère à s’orienter et à se reconnaître dans son chemin, lui
faisait considérer comme la prouesse d’un génie stratégique. Parfois nous
allions jusqu’au viaduc, dont les enjambées de pierre commençaient à la gare et
me représentaient l’exil et la détresse hors du monde civilisé parce que chaque
année en venant de Paris, on nous recommandait de faire bien attention, quand ce
serait Combray, de ne pas laisser passer la station, d’être prêts d’avance car
le train repartait au bout de deux minutes et s’engageait sur le viaduc au delà
des pays chrétiens dont Combray marquait pour moi l’extrême limite. Nous
revenions par le boulevard de la gare, où étaient les plus agréables villas de
la commune. Dans chaque jardin le clair de lune, comme Hubert Robert, semait ses
degrés rompus de marbre blanc, ses jets d’eau, ses grilles entr’ouvertes. Sa
lumière avait détruit le bureau du télégraphe. Il n’en subsistait plus qu’une
colonne à demi brisée, mais qui gardait la beauté d’une ruine immortelle. Je
traînais la jambe, je tombais de sommeil, l’odeur des tilleuls qui embaumait
m’apparaissait comme une récompense qu’on ne pouvait obtenir qu’au prix des plus
grandes fatigues et qui n’en valait pas la peine. De grilles fort éloignées les
unes des autres, des chiens réveillés par nos pas solitaires faisaient alterner
des aboiements comme il m’arrive encore quelquefois d’en entendre le soir, et
entre lesquels dut venir (quand sur son emplacement on créa le jardin public de
Combray) se réfugier le boulevard de la gare, car, où que je me trouve, dès
qu’ils commencent à retentir et à se répondre, je l’aperçois, avec ses tilleuls
et son trottoir éclairé par la lune.

Tout d’un coup mon père nous arrêtait et demandait à ma mère: «Où sommes-nous?»
Epuisée par la marche, mais fière de lui, elle lui avouait tendrement qu’elle
n’en savait absolument rien. Il haussait les épaules et riait. Alors, comme s’il
l’avait sortie de la poche de son veston avec sa clef, il nous montrait debout
devant nous la petite porte de derrière de notre jardin qui était venue avec le
coin de la rue du Saint-Esprit nous attendre au bout de ces chemins inconnus. Ma
mère lui disait avec admiration: «Tu es extraordinaire!» Et à partir de cet
instant, je n’avais plus un seul pas à faire, le sol marchait pour moi dans ce
jardin où depuis si longtemps mes actes avaient cessé d’être accompagnés
d’attention volontaire: l’Habitude venait de me prendre dans ses bras et me
portait jusqu’à mon lit comme un petit enfant.

Si la journée du samedi, qui commençait une heure plus tôt, et où elle était
privée de Françoise, passait plus lentement qu’une autre pour ma tante, elle en
attendait pourtant le retour avec impatience depuis le commencement de la
semaine, comme contenant toute la nouveauté et la distraction que fût encore
capable de supporter son corps affaibli et maniaque. Et ce n’est pas cependant
qu’elle n’aspirât parfois à quelque plus grand changement, qu’elle n’eût de ces
heures d’exception où l’on a soif de quelque chose d’autre que ce qui est, et où
ceux que le manque d’énergie ou d’imagination empêche de tirer d’eux-mêmes un
principe de rénovation, demandent à la minute qui vient, au facteur qui sonne,
de leur apporter du nouveau, fût-ce du pire, une émotion, une douleur; où la
sensibilité, que le bonheur a fait taire comme une harpe oisive, veut résonner
sous une main, même brutale, et dût-elle en être brisée; où la volonté, qui a si
difficilement conquis le droit d’être livrée sans obstacle à ses désirs, à ses
peines, voudrait jeter les rênes entre les mains d’événements impérieux,
fussent-ils cruels. Sans doute, comme les forces de ma tante, taries à la
moindre fatigue, ne lui revenaient que goutte à goutte au sein de son repos, le
réservoir était très long à remplir, et il se passait des mois avant qu’elle eût
ce léger trop-plein que d’autres dérivent dans l’activité et dont elle était
incapable de savoir et de décider comment user. Je ne doute pas qu’alors — comme
le désir de la remplacer par des pommes de terre béchamel finissait au bout de
quelque temps par naître du plaisir même que lui causait le retour quotidien de
la purée dont elle ne se «fatiguait» pas — elle ne tirât de l’accumulation de
ces jours monotones auxquels elle tenait tant, l’attente d’un cataclysme
domestique limité à la durée d’un moment mais qui la forcerait d’accomplir une
fois pour toutes un de ces changements dont elle reconnaissait qu’ils lui
seraient salutaires et auxquels elle ne pouvait d’elle-même se décider. Elle
nous aimait véritablement, elle aurait eu plaisir à nous pleurer; survenant à un
moment où elle se sentait bien et n’était pas en sueur, la nouvelle que la
maison était la proie d’un incendie où nous avions déjà tous péri et qui
n’allait plus bientôt laisser subsister une seule pierre des murs, mais auquel
elle aurait eu tout le temps d’échapper sans se presser, à condition de se lever
tout de suite, a dû souvent hanter ses espérances comme unissant aux avantages
secondaires de lui faire savourer dans un long regret toute sa tendresse pour
nous, et d’être la stupéfaction du village en conduisant notre deuil, courageuse
et accablée, moribonde debout, celui bien plus précieux de la forcer au bon
moment, sans temps à perdre, sans possibilité d’hésitation énervante, à aller
passer l’été dans sa jolie ferme de Mirougrain, où il y avait une chute d’eau.
Comme n’était jamais survenu aucun événement de ce genre, dont elle méditait
certainement la réussite quand elle était seule absorbée dans ses innombrables
jeux de patience (et qui l’eût désespérée au premier commencement de
réalisation, au premier de ces petits faits imprévus, de cette parole annonçant
une mauvaise nouvelle et dont on ne peut plus jamais oublier l’accent, de tout
ce qui porte l’empreinte de la mort réelle, bien différente de sa possibilité
logique et abstraite), elle se rabattait pour rendre de temps en temps sa vie
plus intéressante, à y introduire des péripéties imaginaires qu’elle suivait
avec passion. Elle se plaisait à supposer tout d’un coup que Françoise la
volait, qu’elle recourait à la ruse pour s’en assurer, la prenait sur le fait;
habituée, quand elle faisait seule des parties de cartes, à jouer à la fois son
jeu et le jeu de son adversaire, elle se prononçait à elle-même les excuses
embarrassées de Françoise et y répondait avec tant de feu et d’indignation que
l’un de nous, entrant à ces moments-là, la trouvait en nage, les yeux
étincelants, ses faux cheveux déplacés laissant voir son front chauve. Françoise
entendit peut-être parfois dans la chambre voisine de mordants sarcasmes qui
s’adressaient à elle et dont l’invention n’eût pas soulagé suffisamment ma
tante, s’ils étaient restés à l’état purement immatériel, et si en les murmurant
à mi-voix elle ne leur eût donné plus de réalité. Quelquefois, ce «spectacle
dans un lit» ne suffisait même pas à ma tante, elle voulait faire jouer ses
pièces. Alors, un dimanche, toutes portes mystérieusement fermées, elle confiait
à Eulalie ses doutes sur la probité de Françoise, son intention de se défaire
d’elle, et une autre fois, à Françoise ses soupçons de l’infidélité d’Eulalie, à
qui la porte serait bientôt fermée; quelques jours après elle était dégoûtée de
sa confidente de la veille et racoquinée avec le traître, lesquels d’ailleurs,
pour la prochaine représentation, échangeraient leurs emplois. Mais les soupçons
que pouvait parfois lui inspirer Eulalie, n’étaient qu’un feu de paille et
tombaient vite, faute d’aliment, Eulalie n’habitant pas la maison. Il n’en était
pas de même de ceux qui concernaient Françoise, que ma tante sentait
perpétuellement sous le même toit qu’elle, sans que, par crainte de prendre
froid si elle sortait de son lit, elle osât descendre à la cuisine se rendre
compte s’ils étaient fondés. Peu à peu son esprit n’eut plus d’autre occupation
que de chercher à deviner ce qu’à chaque moment pouvait faire, et chercher à lui
cacher, Françoise. Elle remarquait les plus furtifs mouvements de physionomie de
celle-ci, une contradiction dans ses paroles, un désir qu’elle semblait
dissimuler. Et elle lui montrait qu’elle l’avait démasquée, d’un seul mot qui
faisait pâlir Françoise et que ma tante semblait trouver, à enfoncer au cœur de
la malheureuse, un divertissement cruel. Et le dimanche suivant, une révélation
d’Eulalie — comme ces découvertes qui ouvrent tout d’un coup un champ
insoupçonné à une science naissante et qui se traînait dans l’ornière — prouvait
à ma tante qu’elle était dans ses suppositions bien au-dessous de la vérité.
«Mais Françoise doit le savoir maintenant que vous y avez donné une voiture».
—«Que je lui ai donné une voiture!» s’écriait ma tante. —«Ah! mais je ne sais
pas, moi, je croyais, je l’avais vue qui passait maintenant en calèche, fière
comme Artaban, pour aller au marché de Roussainville. J’avais cru que c’était
Mme Octave qui lui avait donné.» Peu à peu Françoise et ma tante, comme la bête
et le chasseur, ne cessaient plus de tâcher de prévenir les ruses l’une de
l’autre. Ma mère craignait qu’il ne se développât chez Françoise une véritable
haine pour ma tante qui l’offensait le plus durement qu’elle le pouvait. En tous
cas Françoise attachait de plus en plus aux moindres paroles, aux moindres
gestes de ma tante une attention extraordinaire. Quand elle avait quelque chose
à lui demander, elle hésitait longtemps sur la manière dont elle devait s’y
prendre. Et quand elle avait proféré sa requête, elle observait ma tante à la
dérobée, tâchant de deviner dans l’aspect de sa figure ce que celle-ci avait
pensé et déciderait. Et ainsi — tandis que quelque artiste lisant les Mémoires
du XVIIe siècle, et désirant de se rapprocher du grand Roi, croit marcher dans
cette voie en se fabriquant une généalogie qui le fait descendre d’une famille
historique ou en entretenant une correspondance avec un des souverains actuels
de l’Europe, tourne précisément le dos à ce qu’il a le tort de chercher sous des
formes identiques et par conséquent mortes — une vieille dame de province qui ne
faisait qu’obéir sincèrement à d’irrésistibles manies et à une méchanceté née de
l’oisiveté, voyait sans avoir jamais pensé à Louis XIV les occupations les plus
insignifiantes de sa journée, concernant son lever, son déjeuner, son repos,
prendre par leur singularité despotique un peu de l’intérêt de ce que
Saint-Simon appelait la «mécanique» de la vie à Versailles, et pouvait croire
aussi que ses silences, une nuance de bonne humeur ou de hauteur dans sa
physionomie, étaient de la part de Françoise l’objet d’un commentaire aussi
passionné, aussi craintif que l’étaient le silence, la bonne humeur, la hauteur
du Roi quand un courtisan, ou même les plus grands seigneurs, lui avaient remis
une supplique, au détour d’une allée, à Versailles.

Un dimanche, où ma tante avait eu la visite simultanée du curé et d’Eulalie, et
s’était ensuite reposée, nous étions tous montés lui dire bonsoir, et maman lui
adressait ses condoléances sur la mauvaise chance qui amenait toujours ses
visiteurs à la même heure:

—«Je sais que les choses se sont encore mal arrangées tantôt, Léonie, lui
dit-elle avec douceur, vous avez eu tout votre monde à la fois.»

Ce que ma grand’tante interrompit par: «Abondance de biens . . . » car depuis
que sa fille était malade elle croyait devoir la remonter en lui présentant
toujours tout par le bon côté. Mais mon père prenant la parole:

—«Je veux profiter, dit-il, de ce que toute la famille est réunie pour vous
faire un récit sans avoir besoin de le recommencer à chacun. J’ai peur que nous
ne soyons fâchés avec Legrandin: il m’a à peine dit bonjour ce matin.»

Je ne restai pas pour entendre le récit de mon père, car j’étais justement avec
lui après la messe quand nous avions rencontré M. Legrandin, et je descendis à
la cuisine demander le menu du dîner qui tous les jours me distrayait comme les
nouvelles qu’on lit dans un journal et m’excitait à la façon d’un programme de
fête. Comme M. Legrandin avait passé près de nous en sortant de l’église,
marchant à côté d’une châtelaine du voisinage que nous ne connaissions que de
vue, mon père avait fait un salut à la fois amical et réservé, sans que nous
nous arrêtions; M. Legrandin avait à peine répondu, d’un air étonné, comme s’il
ne nous reconnaissait pas, et avec cette perspective du regard particulière aux
personnes qui ne veulent pas être aimables et qui, du fond subitement prolongé
de leurs yeux, ont l’air de vous apercevoir comme au bout d’une route
interminable et à une si grande distance qu’elles se contentent de vous adresser
un signe de tête minuscule pour le proportionner à vos dimensions de
marionnette.

Or, la dame qu’accompagnait Legrandin était une personne vertueuse et
considérée; il ne pouvait être question qu’il fût en bonne fortune et gêné
d’être surpris, et mon père se demandait comment il avait pu mécontenter
Legrandin. «Je regretterais d’autant plus de le savoir fâché, dit mon père,
qu’au milieu de tous ces gens endimanchés il a, avec son petit veston droit, sa
cravate molle, quelque chose de si peu apprêté, de si vraiment simple, et un air
presque ingénu qui est tout à fait sympathique.» Mais le conseil de famille fut
unanimement d’avis que mon père s’était fait une idée, ou que Legrandin, à ce
moment-là, était absorbé par quelque pensée. D’ailleurs la crainte de mon père
fut dissipée dès le lendemain soir. Comme nous revenions d’une grande promenade,
nous aperçûmes près du Pont-Vieux Legrandin, qui à cause des fêtes, restait
plusieurs jours à Combray. Il vint à nous la main tendue: «Connaissez-vous,
monsieur le liseur, me demanda-t-il, ce vers de Paul Desjardins:

\begin{center}\textit{
Les bois sont déjà noirs, le ciel est encor bleu.
}\end{center}

N’est-ce pas la fine notation de cette heure-ci? Vous n’avez peut-être jamais lu
Paul Desjardins. Lisez-le, mon enfant; aujourd’hui il se mue, me dit-on, en
frère prêcheur, mais ce fut longtemps un aquarelliste limpide . . .

\begin{center}\textit{
Les bois sont déjà noirs, le ciel est encor bleu . . .
}\end{center}

Que le ciel reste toujours bleu pour vous, mon jeune ami; et même à l’heure, qui
vient pour moi maintenant, où les bois sont déjà noirs, où la nuit tombe vite,
vous vous consolerez comme je fais en regardant du côté du ciel.» Il sortit de
sa poche une cigarette, resta longtemps les yeux à l’horizon, «Adieu, les
camarades», nous dit-il tout à coup, et il nous quitta.

A cette heure où je descendais apprendre le menu, le dîner était déjà commencé,
et Françoise, commandant aux forces de la nature devenues ses aides, comme dans
les féeries où les géants se font engager comme cuisiniers, frappait la houille,
donnait à la vapeur des pommes de terre à étuver et faisait finir à point par le
feu les chefs-d’œuvre culinaires d’abord préparés dans des récipients de
céramiste qui allaient des grandes cuves, marmites, chaudrons et poissonnières,
aux terrines pour le gibier, moules à pâtisserie, et petits pots de crème en
passant par une collection complète de casserole de toutes dimensions. Je
m’arrêtais à voir sur la table, où la fille de cuisine venait de les écosser,
les petits pois alignés et nombrés comme des billes vertes dans un jeu; mais mon
ravissement était devant les asperges, trempées d’outremer et de rose et dont
l’épi, finement pignoché de mauve et d’azur, se dégrade insensiblement jusqu’au
pied — encore souillé pourtant du sol de leur plant — par des irisations qui ne
sont pas de la terre. Il me semblait que ces nuances célestes trahissaient les
délicieuses créatures qui s’étaient amusées à se métamorphoser en légumes et
qui, à travers le déguisement de leur chair comestible et ferme, laissaient
apercevoir en ces couleurs naissantes d’aurore, en ces ébauches d’arc-en-ciel,
en cette extinction de soirs bleus, cette essence précieuse que je reconnaissais
encore quand, toute la nuit qui suivait un dîner où j’en avais mangé, elles
jouaient, dans leurs farces poétiques et grossières comme une féerie de
Shakespeare, à changer mon pot de chambre en un vase de parfum.

La pauvre Charité de Giotto, comme l’appelait Swann, chargée par Françoise de
les «plumer», les avait près d’elle dans une corbeille, son air était
douloureux, comme si elle ressentait tous les malheurs de la terre; et les
légères couronnes d’azur qui ceignaient les asperges au-dessus de leurs tuniques
de rose étaient finement dessinées, étoile par étoile, comme le sont dans la
fresque les fleurs bandées autour du front ou piquées dans la corbeille de la
Vertu de Padoue. Et cependant, Françoise tournait à la broche un de ces poulets,
comme elle seule savait en rôtir, qui avaient porté loin dans Combray l’odeur de
ses mérites, et qui, pendant qu’elle nous les servait à table, faisaient
prédominer la douceur dans ma conception spéciale de son caractère, l’arôme de
cette chair qu’elle savait rendre si onctueuse et si tendre n’étant pour moi que
le propre parfum d’une de ses vertus.

Mais le jour où, pendant que mon père consultait le conseil de famille sur la
rencontre de Legrandin, je descendis à la cuisine, était un de ceux où la
Charité de Giotto, très malade de son accouchement récent, ne pouvait se lever;
Françoise, n’étant plus aidée, était en retard. Quand je fus en bas, elle était
en train, dans l’arrière-cuisine qui donnait sur la basse-cour, de tuer un
poulet qui, par sa résistance désespérée et bien naturelle, mais accompagnée par
Françoise hors d’elle, tandis qu’elle cherchait à lui fendre le cou sous
l’oreille, des cris de «sale bête! sale bête!», mettait la sainte douceur et
l’onction de notre servante un peu moins en lumière qu’il n’eût fait, au dîner
du lendemain, par sa peau brodée d’or comme une chasuble et son jus précieux
égoutté d’un ciboire. Quand il fut mort, Françoise recueillit le sang qui
coulait sans noyer sa rancune, eut encore un sursaut de colère, et regardant le
cadavre de son ennemi, dit une dernière fois: «Sale bête!» Je remontai tout
tremblant; j’aurais voulu qu’on mît Françoise tout de suite à la porte. Mais qui
m’eût fait des boules aussi chaudes, du café aussi parfumé, et même . . . ces
poulets? . . . Et en réalité, ce lâche calcul, tout le monde avait eu à le faire
comme moi. Car ma tante Léonie savait — ce que j’ignorais encore — que Françoise
qui, pour sa fille, pour ses neveux, aurait donné sa vie sans une plainte, était
pour d’autres êtres d’une dureté singulière. Malgré cela ma tante l’avait
gardée, car si elle connaissait sa cruauté, elle appréciait son service. Je
m’aperçus peu à peu que la douceur, la componction, les vertus de Françoise
cachaient des tragédies d’arrière-cuisine, comme l’histoire découvre que les
règnes des Rois et des Reines, qui sont représentés les mains jointes dans les
vitraux des églises, furent marqués d’incidents sanglants. Je me rendis compte
que, en dehors de ceux de sa parenté, les humains excitaient d’autant plus sa
pitié par leurs malheurs, qu’ils vivaient plus éloignés d’elle. Les torrents de
larmes qu’elle versait en lisant le journal sur les infortunes des inconnus se
tarissaient vite si elle pouvait se représenter la personne qui en était l’objet
d’une façon un peu précise. Une de ces nuits qui suivirent l’accouchement de la
fille de cuisine, celle-ci fut prise d’atroces coliques; maman l’entendit se
plaindre, se leva et réveilla Françoise qui, insensible, déclara que tous ces
cris étaient une comédie, qu’elle voulait «faire la maîtresse». Le médecin, qui
craignait ces crises, avait mis un signet, dans un livre de médecine que nous
avions, à la page où elles sont décrites et où il nous avait dit de nous
reporter pour trouver l’indication des premiers soins à donner. Ma mère envoya
Françoise chercher le livre en lui recommandant de ne pas laisser tomber le
signet. Au bout d’une heure, Françoise n’était pas revenue; ma mère indignée
crut qu’elle s’était recouchée et me dit d’aller voir moi-même dans la
bibliothèque. J’y trouvai Françoise qui, ayant voulu regarder ce que le signet
marquait, lisait la description clinique de la crise et poussait des sanglots
maintenant qu’il s’agissait d’une malade-type qu’elle ne connaissait pas. A
chaque symptôme douloureux mentionné par l’auteur du traité, elle s’écriait: «Hé
là! Sainte Vierge, est-il possible que le bon Dieu veuille faire souffrir ainsi
une malheureuse créature humaine? Hé! la pauvre!»

Mais dès que je l’eus appelée et qu’elle fut revenue près du lit de la Charité
de Giotto, ses larmes cessèrent aussitôt de couler; elle ne put reconnaître ni
cette agréable sensation de pitié et d’attendrissement qu’elle connaissait bien
et que la lecture des journaux lui avait souvent donnée, ni aucun plaisir de
même famille, dans l’ennui et dans l’irritation de s’être levée au milieu de la
nuit pour la fille de cuisine; et à la vue des mêmes souffrances dont la
description l’avait fait pleurer, elle n’eut plus que des ronchonnements de
mauvaise humeur, même d’affreux sarcasmes, disant, quand elle crut que nous
étions partis et ne pouvions plus l’entendre: «Elle n’avait qu’à ne pas faire ce
qu’il faut pour ça! ça lui a fait plaisir! qu’elle ne fasse pas de manières
maintenant. Faut-il tout de même qu’un garçon ait été abandonné du bon Dieu pour
aller avec ça. Ah! c’est bien comme on disait dans le patois de ma pauvre mère:

\begin{center}
\textit{«Qui du cul d’un chien s’amourose} \par
\textit{«Il lui paraît une rose.»}
\end{center}

Si, quand son petit-fils était un peu enrhumé du cerveau, elle partait la nuit,
même malade, au lieu de se coucher, pour voir s’il n’avait besoin de rien,
faisant quatre lieues à pied avant le jour afin d’être rentrée pour son travail,
en revanche ce même amour des siens et son désir d’assurer la grandeur future de
sa maison se traduisait dans sa politique à l’égard des autres domestiques par
une maxime constante qui fut de n’en jamais laisser un seul s’implanter chez ma
tante, qu’elle mettait d’ailleurs une sorte d’orgueil à ne laisser approcher par
personne, préférant, quand elle-même était malade, se relever pour lui donner
son eau de Vichy plutôt que de permettre l’accès de la chambre de sa maîtresse à
la fille de cuisine. Et comme cet hyménoptère observé par Fabre, la guêpe
fouisseuse, qui pour que ses petits après sa mort aient de la viande fraîche à
manger, appelle l’anatomie au secours de sa cruauté et, ayant capturé des
charançons et des araignées, leur perce avec un savoir et une adresse
merveilleux le centre nerveux d’où dépend le mouvement des pattes, mais non les
autres fonctions de la vie, de façon que l’insecte paralysé près duquel elle
dépose ses oeufs, fournisse aux larves, quand elles écloront un gibier docile,
inoffensif, incapable de fuite ou de résistance, mais nullement faisandé,
Françoise trouvait pour servir sa volonté permanente de rendre la maison
intenable à tout domestique, des ruses si savantes et si impitoyables que, bien
des années plus tard, nous apprîmes que si cet été-là nous avions mangé presque
tous les jours des asperges, c’était parce que leur odeur donnait à la pauvre
fille de cuisine chargée de les éplucher des crises d’asthme d’une telle
violence qu’elle fut obligée de finir par s’en aller.

Hélas! nous devions définitivement changer d’opinion sur Legrandin. Un des
dimanches qui suivit la rencontre sur le Pont-Vieux après laquelle mon père
avait dû confesser son erreur, comme la messe finissait et qu’avec le soleil et
le bruit du dehors quelque chose de si peu sacré entrait dans l’église que Mme
Goupil, Mme Percepied (toutes les personnes qui tout à l’heure, à mon arrivée un
peu en retard, étaient restées les yeux absorbés dans leur prière et que
j’aurais même pu croire ne m’avoir pas vu entrer si, en même temps, leurs pieds
n’avaient repoussé légèrement le petit banc qui m’empêchait de gagner ma chaise)
commençaient à s’entretenir avec nous à haute voix de sujets tout temporels
comme si nous étions déjà sur la place, nous vîmes sur le seuil brûlant du
porche, dominant le tumulte bariolé du marché, Legrandin, que le mari de cette
dame avec qui nous l’avions dernièrement rencontré, était en train de présenter
à la femme d’un autre gros propriétaire terrien des environs. La figure de
Legrandin exprimait une animation, un zèle extraordinaires; il fit un profond
salut avec un renversement secondaire en arrière, qui ramena brusquement son dos
au delà de la position de départ et qu’avait dû lui apprendre le mari de sa
sœur, Mme De Cambremer. Ce redressement rapide fit refluer en une sorte d’onde
fougueuse et musclée la croupe de Legrandin que je ne supposais pas si charnue;
et je ne sais pourquoi cette ondulation de pure matière, ce flot tout charnel,
sans expression de spiritualité et qu’un empressement plein de bassesse
fouettait en tempête, éveillèrent tout d’un coup dans mon esprit la possibilité
d’un Legrandin tout différent de celui que nous connaissions. Cette dame le pria
de dire quelque chose à son cocher, et tandis qu’il allait jusqu’à la voiture,
l’empreinte de joie timide et dévouée que la présentation avait marquée sur son
visage y persistait encore. Ravi dans une sorte de rêve, il souriait, puis il
revint vers la dame en se hâtant et, comme il marchait plus vite qu’il n’en
avait l’habitude, ses deux épaules oscillaient de droite et de gauche
ridiculement, et il avait l’air tant il s’y abandonnait entièrement en n’ayant
plus souci du reste, d’être le jouet inerte et mécanique du bonheur. Cependant,
nous sortions du porche, nous allions passer à côté de lui, il était trop bien
élevé pour détourner la tête, mais il fixa de son regard soudain chargé d’une
rêverie profonde un point si éloigné de l’horizon qu’il ne put nous voir et
n’eut pas à nous saluer. Son visage restait ingénu au-dessus d’un veston souple
et droit qui avait l’air de se sentir fourvoyé malgré lui au milieu d’un luxe
détesté. Et une lavallière à pois qu’agitait le vent de la Place continuait à
flotter sur Legrandin comme l’étendard de son fier isolement et de sa noble
indépendance. Au moment où nous arrivions à la maison, maman s’aperçut qu’on
avait oublié le Saint-Honoré et demanda à mon père de retourner avec moi sur nos
pas dire qu’on l’apportât tout de suite. Nous croisâmes près de l’église
Legrandin qui venait en sens inverse conduisant la même dame à sa voiture. Il
passa contre nous, ne s’interrompit pas de parler à sa voisine et nous fit du
coin de son œil bleu un petit signe en quelque sorte intérieur aux paupières et
qui, n’intéressant pas les muscles de son visage, put passer parfaitement
inaperçu de son interlocutrice; mais, cherchant à compenser par l’intensité du
sentiment le champ un peu étroit où il en circonscrivait l’expression, dans ce
coin d’azur qui nous était affecté il fit pétiller tout l’entrain de la bonne
grâce qui dépassa l’enjouement, frisa la malice; il subtilisa les finesses de
l’amabilité jusqu’aux clignements de la connivence, aux demi-mots, aux
sous-entendus, aux mystères de la complicité; et finalement exalta les
assurances d’amitié jusqu’aux protestations de tendresse, jusqu’à la déclaration
d’amour, illuminant alors pour nous seuls d’une langueur secrète et invisible à
la châtelaine, une prunelle énamourée dans un visage de glace.

Il avait précisément demandé la veille à mes parents de m’envoyer dîner ce
soir-là avec lui: «Venez tenir compagnie à votre vieil ami, m’avait-il dit.
Comme le bouquet qu’un voyageur nous envoie d’un pays où nous ne retournerons
plus, faites-moi respirer du lointain de votre adolescence ces fleurs des
printemps que j’ai traversés moi aussi il y a bien des années. Venez avec la
primevère, la barbe de chanoine, le bassin d’or, venez avec le sédum dont est
fait le bouquet de dilection de la flore balzacienne, avec la fleur du jour de
la Résurrection, la pâquerette et la boule de neige des jardins qui commence à
embaumer dans les allées de votre grand’tante quand ne sont pas encore fondues
les dernières boules de neige des giboulées de Pâques. Venez avec la glorieuse
vêture de soie du lis digne de Salomon, et l’émail polychrome des pensées, mais
venez surtout avec la brise fraîche encore des dernières gelées et qui va
entr’ouvrir, pour les deux papillons qui depuis ce matin attendent à la porte,
la première rose de Jérusalem.»

On se demandait à la maison si on devait m’envoyer tout de même dîner avec M.
Legrandin. Mais ma grand’mère refusa de croire qu’il eût été impoli. «Vous
reconnaissez vous-même qu’il vient là avec sa tenue toute simple qui n’est guère
celle d’un mondain.» Elle déclarait qu’en tous cas, et à tout mettre au pis,
s’il l’avait été, mieux valait ne pas avoir l’air de s’en être aperçu. A vrai
dire mon père lui-même, qui était pourtant le plus irrité contre l’attitude
qu’avait eue Legrandin, gardait peut-être un dernier doute sur le sens qu’elle
comportait. Elle était comme toute attitude ou action où se révèle le caractère
profond et caché de quelqu’un: elle ne se relie pas à ses paroles antérieures,
nous ne pouvons pas la faire confirmer par le témoignage du coupable qui
n’avouera pas; nous en sommes réduits à celui de nos sens dont nous nous
demandons, devant ce souvenir isolé et incohérent, s’ils n’ont pas été le jouet
d’une illusion; de sorte que de telles attitudes, les seules qui aient de
l’importance, nous laissent souvent quelques doutes.

Je dînai avec Legrandin sur sa terrasse; il faisait clair de lune: «Il y a une
jolie qualité de silence, n’est-ce pas, me dit-il; aux cœurs blessés comme l’est
le mien, un romancier que vous lirez plus tard, prétend que conviennent
seulement l’ombre et le silence. Et voyez-vous, mon enfant, il vient dans la vie
une heure dont vous êtes bien loin encore où les yeux las ne tolèrent plus
qu’une lumière, celle qu’une belle nuit comme celle-ci prépare et distille avec
l’obscurité, où les oreilles ne peuvent plus écouter de musique que celle que
joue le clair de lune sur la flûte du silence.» J’écoutais les paroles de M.
Legrandin qui me paraissaient toujours si agréables; mais troublé par le
souvenir d’une femme que j’avais aperçue dernièrement pour la première fois, et
pensant, maintenant que je savais que Legrandin était lié avec plusieurs
personnalités aristocratiques des environs, que peut-être il connaissait
celle-ci, prenant mon courage, je lui dis: «Est-ce que vous connaissez,
monsieur, la . . . les châtelaines de Guermantes», heureux aussi en prononçant
ce nom de prendre sur lui une sorte de pouvoir, par le seul fait de le tirer de
mon rêve et de lui donner une existence objective et sonore.

Mais à ce nom de Guermantes, je vis au milieu des yeux bleus de notre ami se
ficher une petite encoche brune comme s’ils venaient d’être percés par une
pointe invisible, tandis que le reste de la prunelle réagissait en sécrétant des
flots d’azur. Le cerne de sa paupière noircit, s’abaissa. Et sa bouche marquée
d’un pli amer se ressaissant plus vite sourit, tandis que le regard restait
douloureux, comme celui d’un beau martyr dont le corps est hérissé de flèches:
«Non, je ne les connais pas», dit-il, mais au lieu de donner à un renseignement
aussi simple, à une réponse aussi peu surprenante le ton naturel et courant qui
convenait, il le débita en appuyant sur les mots, en s’inclinant, en saluant de
la tête, à la fois avec l’insistance qu’on apporte, pour être cru, à une
affirmation invraisemblable — comme si ce fait qu’il ne connût pas les
Guermantes ne pouvait être l’effet que d’un hasard singulier — et aussi avec
l’emphase de quelqu’un qui, ne pouvant pas taire une situation qui lui est
pénible, préfère la proclamer pour donner aux autres l’idée que l’aveu qu’il
fait ne lui cause aucun embarras, est facile, agréable, spontané, que la
situation elle-même — l’absence de relations avec les Guermantes — pourrait bien
avoir été non pas subie, mais voulue par lui, résulter de quelque tradition de
famille, principe de morale ou voeu mystique lui interdisant nommément la
fréquentation des Guermantes. «Non, reprit-il, expliquant par ses paroles sa
propre intonation, non, je ne les connais pas, je n’ai jamais voulu, j’ai
toujours tenu à sauvegarder ma pleine indépendance; au fond je suis une tête
jacobine, vous le savez. Beaucoup de gens sont venus à la rescousse, on me
disait que j’avais tort de ne pas aller à Guermantes, que je me donnais l’air
d’un malotru, d’un vieil ours. Mais voilà une réputation qui n’est pas pour
m’effrayer, elle est si vraie! Au fond, je n’aime plus au monde que quelques
églises, deux ou trois livres, à peine davantage de tableaux, et le clair de
lune quand la brise de votre jeunesse apporte jusqu’à moi l’odeur des parterres
que mes vieilles prunelles ne distinguent plus.» Je ne comprenais pas bien que
pour ne pas aller chez des gens qu’on ne connaît pas, il fût nécessaire de tenir
à son indépendance, et en quoi cela pouvait vous donner l’air d’un sauvage ou
d’un ours. Mais ce que je comprenais c’est que Legrandin n’était pas tout à fait
véridique quand il disait n’aimer que les églises, le clair de lune et la
jeunesse; il aimait beaucoup les gens des châteaux et se trouvait pris devant
eux d’une si grande peur de leur déplaire qu’il n’osait pas leur laisser voir
qu’il avait pour amis des bourgeois, des fils de notaires ou d’agents de change,
préférant, si la vérité devait se découvrir, que ce fût en son absence, loin de
lui et «par défaut»; il était snob. Sans doute il ne disait jamais rien de tout
cela dans le langage que mes parents et moi-même nous aimions tant. Et si je
demandais: «Connaissez-vous les Guermantes?», Legrandin le causeur répondait:
«Non, je n’ai jamais voulu les connaître.» Malheureusement il ne le répondait
qu’en second, car un autre Legrandin qu’il cachait soigneusement au fond de lui,
qu’il ne montrait pas, parce que ce Legrandin-là savait sur le nôtre, sur son
snobisme, des histoires compromettantes, un autre Legrandin avait déjà répondu
par la blessure du regard, par le rictus de la bouche, par la gravité excessive
du ton de la réponse, par les mille flèches dont notre Legrandin s’était trouvé
en un instant lardé et alangui, comme un saint Sébastien du snobisme: «Hélas!
que vous me faites mal, non je ne connais pas les Guermantes, ne réveillez pas
la grande douleur de ma vie.» Et comme ce Legrandin enfant terrible, ce
Legrandin maître chanteur, s’il n’avait pas le joli langage de l’autre, avait le
verbe infiniment plus prompt, composé de ce qu’on appelle «réflexes», quand
Legrandin le causeur voulait lui imposer silence, l’autre avait déjà parlé et
notre ami avait beau se désoler de la mauvaise impression que les révélations de
son alter ego avaient dû produire, il ne pouvait qu’entreprendre de la pallier.

Et certes cela ne veut pas dire que M. Legrandin ne fût pas sincère quand il
tonnait contre les snobs. Il ne pouvait pas savoir, au moins par lui-même, qu’il
le fût, puisque nous ne connaissons jamais que les passions des autres, et que
ce que nous arrivons à savoir des nôtres, ce n’est que d’eux que nous avons pu
l’apprendre. Sur nous, elles n’agissent que d’une façon seconde, par
l’imagination qui substitue aux premiers mobiles des mobiles de relais qui sont
plus décents. Jamais le snobisme de Legrandin ne lui conseillait d’aller voir
souvent une duchesse. Il chargeait l’imagination de Legrandin de lui faire
apparaître cette duchesse comme parée de toutes les grâces. Legrandin se
rapprochait de la duchesse, s’estimant de céder à cet attrait de l’esprit et de
la vertu qu’ignorent les infâmes snobs. Seuls les autres savaient qu’il en était
un; car, grâce à l’incapacité où ils étaient de comprendre le travail
intermédiaire de son imagination, ils voyaient en face l’une de l’autre
l’activité mondaine de Legrandin et sa cause première.

Maintenant, à la maison, on n’avait plus aucune illusion sur M. Legrandin, et
nos relations avec lui s’étaient fort espacées. Maman s’amusait infiniment
chaque fois qu’elle prenait Legrandin en flagrant délit du péché qu’il n’avouait
pas, qu’il continuait à appeler le péché sans rémission, le snobisme. Mon père,
lui, avait de la peine à prendre les dédains de Legrandin avec tant de
détachement et de gaîté; et quand on pensa une année à m’envoyer passer les
grandes vacances à Balbec avec ma grand’mère, il dit: «Il faut absolument que
j’annonce à Legrandin que vous irez à Balbec, pour voir s’il vous offrira de
vous mettre en rapport avec sa sœur. Il ne doit pas se souvenir nous avoir dit
qu’elle demeurait à deux kilomètres de là.» Ma grand’mère qui trouvait qu’aux
bains de mer il faut être du matin au soir sur la plage à humer le sel et qu’on
n’y doit connaître personne, parce que les visites, les promenades sont autant
de pris sur l’air marin, demandait au contraire qu’on ne parlât pas de nos
projets à Legrandin, voyant déjà sa sœur, Mme de Cambremer, débarquant à l’hôtel
au moment où nous serions sur le point d’aller à la pêche et nous forçant à
rester enfermés pour la recevoir. Mais maman riait de ses craintes, pensant à
part elle que le danger n’était pas si menaçant, que Legrandin ne serait pas si
pressé de nous mettre en relations avec sa sœur. Or, sans qu’on eût besoin de
lui parler de Balbec, ce fut lui-même, Legrandin, qui, ne se doutant pas que
nous eussions jamais l’intention d’aller de ce côté, vint se mettre dans le
piège un soir où nous le rencontrâmes au bord de la Vivonne.

—«Il y a dans les nuages ce soir des violets et des bleus bien beaux, n’est-ce
pas, mon compagnon, dit-il à mon père, un bleu surtout plus floral qu’aérien, un
bleu de cinéraire, qui surprend dans le ciel. Et ce petit nuage rose n’a-t-il
pas aussi un teint de fleur, d’œillet ou d’hydrangéa? Il n’y a guère que dans la
Manche, entre Normandie et Bretagne, que j’ai pu faire de plus riches
observations sur cette sorte de règne végétal de l’atmosphère. Là-bas, près de
Balbec, près de ces lieux sauvages, il y a une petite baie d’une douceur
charmante où le coucher de soleil du pays d’Auge, le coucher de soleil rouge et
or que je suis loin de dédaigner, d’ailleurs, est sans caractère, insignifiant;
mais dans cette atmosphère humide et douce s’épanouissent le soir en quelques
instants de ces bouquets célestes, bleus et roses, qui sont incomparables et qui
mettent souvent des heures à se faner. D’autres s’effeuillent tout de suite et
c’est alors plus beau encore de voir le ciel entier que jonche la dispersion
d’innombrables pétales soufrés ou roses. Dans cette baie, dite d’opale, les
plages d’or semblent plus douces encore pour être attachées comme de blondes
Andromèdes à ces terribles rochers des côtes voisines, à ce rivage funèbre,
fameux par tant de naufrages, où tous les hivers bien des barques trépassent au
péril de la mer. Balbec! la plus antique ossature géologique de notre sol,
vraiment Ar-mor, la Mer, la fin de la terre, la région maudite qu’Anatole France
— un enchanteur que devrait lire notre petit ami — a si bien peinte, sous ses
brouillards éternels, comme le véritable pays des Cimmériens, dans l’Odyssée. De
Balbec surtout, où déjà des hôtels se construisent, superposés au sol antique et
charmant qu’ils n’altèrent pas, quel délice d’excursionner à deux pas dans ces
régions primitives et si belles.»

—«Ah! est-ce que vous connaissez quelqu’un à Balbec? dit mon père. Justement ce
petit-là doit y aller passer deux mois avec sa grand’mère et peut-être avec ma
femme.»

Legrandin pris au dépourvu par cette question à un moment où ses yeux étaient
fixés sur mon père, ne put les détourner, mais les attachant de seconde en
seconde avec plus d’intensité— et tout en souriant tristement — sur les yeux de
son interlocuteur, avec un air d’amitié et de franchise et de ne pas craindre de
le regarder en face, il sembla lui avoir traversé la figure comme si elle fût
devenue transparente, et voir en ce moment bien au delà derrière elle un nuage
vivement coloré qui lui créait un alibi mental et qui lui permettrait d’établir
qu’au moment où on lui avait demandé s’il connaissait quelqu’un à Balbec, il
pensait à autre chose et n’avait pas entendu la question. Habituellement de tels
regards font dire à l’interlocuteur: «A quoi pensez-vous donc?» Mais mon père
curieux, irrité et cruel, reprit:

—«Est-ce que vous avez des amis de ce côté-là, que vous connaissez si bien
Balbec?»

Dans un dernier effort désespéré, le regard souriant de Legrandin atteignit son
maximum de tendresse, de vague, de sincérité et de distraction, mais, pensant
sans doute qu’il n’y avait plus qu’à répondre, il nous dit:

—«J’ai des amis partout où il y a des groupes d’arbres blessés, mais non
vaincus, qui se sont rapprochés pour implorer ensemble avec une obstination
pathétique un ciel inclément qui n’a pas pitié d’eux.

—«Ce n’est pas cela que je voulais dire, interrompit mon père, aussi obstiné que
les arbres et aussi impitoyable que le ciel. Je demandais pour le cas où il
arriverait n’importe quoi à ma belle-mère et où elle aurait besoin de ne pas se
sentir là-bas en pays perdu, si vous y connaissez du monde?»

—«Là comme partout, je connais tout le monde et je ne connais personne, répondit
Legrandin qui ne se rendait pas si vite; beaucoup les choses et fort peu les
personnes. Mais les choses elles-mêmes y semblent des personnes, des personnes
rares, d’une essence délicate et que la vie aurait déçues. Parfois c’est un
castel que vous rencontrez sur la falaise, au bord du chemin où il s’est arrêté
pour confronter son chagrin au soir encore rose où monte la lune d’or et dont
les barques qui rentrent en striant l’eau diaprée hissent à leurs mâts la flamme
et portent les couleurs; parfois c’est une simple maison solitaire, plutôt
laide, l’air timide mais romanesque, qui cache à tous les yeux quelque secret
impérissable de bonheur et de désenchantement. Ce pays sans vérité, ajouta-t-il
avec une délicatesse machiavélique, ce pays de pure fiction est d’une mauvaise
lecture pour un enfant, et ce n’est certes pas lui que je choisirais et
recommanderais pour mon petit ami déjà si enclin à la tristesse, pour son cœur
prédisposé. Les climats de confidence amoureuse et de regret inutile peuvent
convenir au vieux désabusé que je suis, ils sont toujours malsains pour un
tempérament qui n’est pas formé. Croyez-moi, reprit-il avec insistance, les eaux
de cette baie, déjà à moitié bretonne, peuvent exercer une action sédative,
d’ailleurs discutable, sur un cœur qui n’est plus intact comme le mien, sur un
cœur dont la lésion n’est plus compensée. Elles sont contre-indiquées à votre
âge, petit garçon. Bonne nuit, voisins», ajouta-t-il en nous quittant avec cette
brusquerie évasive dont il avait l’habitude et, se retournant vers nous avec un
doigt levé de docteur, il résuma sa consultation: «Pas de Balbec avant cinquante
ans et encore cela dépend de l’état du cœur», nous cria-t-il.

Mon père lui en reparla dans nos rencontres ultérieures, le tortura de
questions, ce fut peine inutile: comme cet escroc érudit qui employait à
fabriquer de faux palimpsestes un labeur et une science dont la centième partie
eût suffi à lui assurer une situation plus lucrative, mais honorable, M.
Legrandin, si nous avions insisté encore, aurait fini par édifier toute une
éthique de paysage et une géographie céleste de la basse Normandie, plutôt que
de nous avouer qu’à deux kilomètres de Balbec habitait sa propre sœur, et d’être
obligé à nous offrir une lettre d’introduction qui n’eût pas été pour lui un tel
sujet d’effroi s’il avait été absolument certain — comme il aurait dû l’être en
effet avec l’expérience qu’il avait du caractère de ma grand’mère — que nous
n’en aurions pas profité.

%. . .
\PRLsep

Nous rentrions toujours de bonne heure de nos promenades pour pouvoir faire une
visite à ma tante Léonie avant le dîner. Au commencement de la saison où le jour
finit tôt, quand nous arrivions rue du Saint-Esprit, il y avait encore un reflet
du couchant sur les vitres de la maison et un bandeau de pourpre au fond des
bois du Calvaire qui se reflétait plus loin dans l’étang, rougeur qui,
accompagnée souvent d’un froid assez vif, s’associait, dans mon esprit, à la
rougeur du feu au-dessus duquel rôtissait le poulet qui ferait succéder pour moi
au plaisir poétique donné par la promenade, le plaisir de la gourmandise, de la
chaleur et du repos. Dans l’été, au contraire, quand nous rentrions, le soleil
ne se couchait pas encore; et pendant la visite que nous faisions chez ma tante
Léonie, sa lumière qui s’abaissait et touchait la fenêtre était arrêtée entre
les grands rideaux et les embrasses, divisée, ramifiée, filtrée, et incrustant
de petits morceaux d’or le bois de citronnier de la commode, illuminait
obliquement la chambre avec la délicatesse qu’elle prend dans les sous-bois.
Mais certains jours fort rares, quand nous rentrions, il y avait bien longtemps
que la commode avait perdu ses incrustations momentanées, il n’y avait plus
quand nous arrivions rue du Saint-Esprit nul reflet de couchant étendu sur les
vitres et l’étang au pied du calvaire avait perdu sa rougeur, quelquefois il
était déjà couleur d’opale et un long rayon de lune qui allait en s’élargissant
et se fendillait de toutes les rides de l’eau le traversait tout entier. Alors,
en arrivant près de la maison, nous apercevions une forme sur le pas de la porte
et maman me disait:

—«Mon dieu! voilà Françoise qui nous guette, ta tante est inquiète; aussi nous
rentrons trop tard.»

Et sans avoir pris le temps d’enlever nos affaires, nous montions vite chez ma
tante Léonie pour la rassurer et lui montrer que, contrairement à ce qu’elle
imaginait déjà, il ne nous était rien arrivé, mais que nous étions allés «du
côté de Guermantes» et, dame, quand on faisait cette promenade-là, ma tante
savait pourtant bien qu’on ne pouvait jamais être sûr de l’heure à laquelle on
serait rentré.

—«Là, Françoise, disait ma tante, quand je vous le disais, qu’ils seraient allés
du côté de Guermantes! Mon dieu! ils doivent avoir une faim! et votre gigot qui
doit être tout desséché après ce qu’il a attendu. Aussi est-ce une heure pour
rentrer! comment, vous êtes allés du côté de Guermantes!»

—«Mais je croyais que vous le saviez, Léonie, disait maman. Je pensais que
Françoise nous avait vus sortir par la petite porte du potager.»

Car il y avait autour de Combray deux «côtés» pour les promenades, et si opposés
qu’on ne sortait pas en effet de chez nous par la même porte, quand on voulait
aller d’un côté ou de l’autre: le côté de Méséglise-la-Vineuse, qu’on appelait
aussi le côté de chez Swann parce qu’on passait devant la propriété de M. Swann
pour aller par là, et le côté de Guermantes. De Méséglise-la-Vineuse, à vrai
dire, je n’ai jamais connu que le «côté» et des gens étrangers qui venaient le
dimanche se promener à Combray, des gens que, cette fois, ma tante elle-même et
nous tous ne «connaissions point» et qu’à ce signe on tenait pour «des gens qui
seront venus de Méséglise». Quant à Guermantes je devais un jour en connaître
davantage, mais bien plus tard seulement; et pendant toute mon adolescence, si
Méséglise était pour moi quelque chose d’inaccessible comme l’horizon, dérobé à
la vue, si loin qu’on allât, par les plis d’un terrain qui ne ressemblait déjà
plus à celui de Combray, Guermantes lui ne m’est apparu que comme le terme
plutôt idéal que réel de son propre «côté», une sorte d’expression géographique
abstraite comme la ligne de l’équateur, comme le pôle, comme l’orient. Alors,
«prendre par Guermantes» pour aller à Méséglise, ou le contraire, m’eût semblé
une expression aussi dénuée de sens que prendre par l’est pour aller à l’ouest.
Comme mon père parlait toujours du côté de Méséglise comme de la plus belle vue
de plaine qu’il connût et du côté de Guermantes comme du type de paysage de
rivière, je leur donnais, en les concevant ainsi comme deux entités, cette
cohésion, cette unité qui n’appartiennent qu’aux créations de notre esprit; la
moindre parcelle de chacun d’eux me semblait précieuse et manifester leur
excellence particulière, tandis qu’à côté d’eux, avant qu’on fût arrivé sur le
sol sacré de l’un ou de l’autre, les chemins purement matériels au milieu
desquels ils étaient posés comme l’idéal de la vue de plaine et l’idéal du
paysage de rivière, ne valaient pas plus la peine d’être regardés que par le
spectateur épris d’art dramatique, les petites rues qui avoisinent un théâtre.
Mais surtout je mettais entre eux, bien plus que leurs distances kilométriques
la distance qu’il y avait entre les deux parties de mon cerveau où je pensais à
eux, une de ces distances dans l’esprit qui ne font pas qu’éloigner, qui
séparent et mettent dans un autre plan. Et cette démarcation était rendue plus
absolue encore parce que cette habitude que nous avions de n’aller jamais vers
les deux côtés un même jour, dans une seule promenade, mais une fois du côté de
Méséglise, une fois du côté de Guermantes, les enfermait pour ainsi dire loin
l’un de l’autre, inconnaissables l’un à l’autre, dans les vases clos et sans
communication entre eux, d’après-midi différents.

Quand on voulait aller du côté de Méséglise, on sortait (pas trop tôt et même si
le ciel était couvert, parce que la promenade n’était pas bien longue et
n’entraînait pas trop) comme pour aller n’importe où, par la grande porte de la
maison de ma tante sur la rue du Saint-Esprit. On était salué par l’armurier, on
jetait ses lettres à la boîte, on disait en passant à Théodore, de la part de
Françoise, qu’elle n’avait plus d’huile ou de café, et l’on sortait de la ville
par le chemin qui passait le long de la barrière blanche du parc de M. Swann.
Avant d’y arriver, nous rencontrions, venue au-devant des étrangers, l’odeur de
ses lilas. Eux-mêmes, d’entre les petits cœurs verts et frais de leurs feuilles,
levaient curieusement au-dessus de la barrière du parc leurs panaches de plumes
mauves ou blanches que lustrait, même à l’ombre, le soleil où elles avaient
baigné. Quelques-uns, à demi cachés par la petite maison en tuiles appelée
maison des Archers, où logeait le gardien, dépassaient son pignon gothique de
leur rose minaret. Les Nymphes du printemps eussent semblé vulgaires, auprès de
ces jeunes houris qui gardaient dans ce jardin français les tons vifs et purs
des miniatures de la Perse. Malgré mon désir d’enlacer leur taille souple et
d’attirer à moi les boucles étoilées de leur tête odorante, nous passions sans
nous arrêter, mes parents n’allant plus à Tansonville depuis le mariage de
Swann, et, pour ne pas avoir l’air de regarder dans le parc, au lieu de prendre
le chemin qui longe sa clôture et qui monte directement aux champs, nous en
prenions un autre qui y conduit aussi, mais obliquement, et nous faisait
déboucher trop loin. Un jour, mon grand-père dit à mon père:

—«Vous rappelez-vous que Swann a dit hier que, comme sa femme et sa fille
partaient pour Reims, il en profiterait pour aller passer vingt-quatre heures à
Paris? Nous pourrions longer le parc, puisque ces dames ne sont pas là, cela
nous abrégerait d’autant.»

Nous nous arrêtâmes un moment devant la barrière. Le temps des lilas approchait
de sa fin; quelques-uns effusaient encore en hauts lustres mauves les bulles
délicates de leurs fleurs, mais dans bien des parties du feuillage où déferlait,
il y avait seulement une semaine, leur mousse embaumée, se flétrissait, diminuée
et noircie, une écume creuse, sèche et sans parfum. Mon grand-père montrait à
mon père en quoi l’aspect des lieux était resté le même, et en quoi il avait
changé, depuis la promenade qu’il avait faite avec M. Swann le jour de la mort
de sa femme, et il saisit cette occasion pour raconter cette promenade une fois
de plus.

Devant nous, une allée bordée de capucines montait en plein soleil vers le
château. A droite, au contraire, le parc s’étendait en terrain plat. Obscurcie
par l’ombre des grands arbres qui l’entouraient, une pièce d’eau avait été
creusée par les parents de Swann; mais dans ses créations les plus factices,
c’est sur la nature que l’homme travaille; certains lieux font toujours régner
autour d’eux leur empire particulier, arborent leurs insignes immémoriaux au
milieu d’un parc comme ils auraient fait loin de toute intervention humaine,
dans une solitude qui revient partout les entourer, surgie des nécessités de
leur exposition et superposée à l’œuvre humaine. C’est ainsi qu’au pied de
l’allée qui dominait l’étang artificiel, s’était composée sur deux rangs,
tressés de fleurs de myosotis et de pervenches, la couronne naturelle, délicate
et bleue qui ceint le front clair-obscur des eaux, et que le glaïeul, laissant
fléchir ses glaives avec un abandon royal, étendait sur l’eupatoire et la
grenouillette au pied mouillé, les fleurs de lis en lambeaux, violettes et
jaunes, de son sceptre lacustre.

Le départ de Mlle Swann qui — en m’ôtant la chance terrible de la voir
apparaître dans une allée, d’être connu et méprisé par la petite fille
privilégiée qui avait Bergotte pour ami et allait avec lui visiter des
cathédrales — me rendait la contemplation de Tansonville indifférente la
première fois où elle m’était permise, semblait au contraire ajouter à cette
propriété, aux yeux de mon grand-père et de mon père, des commodités, un
agrément passager, et, comme fait pour une excursion en pays de montagnes,
l’absence de tout nuage, rendre cette journée exceptionnellement propice à une
promenade de ce côté; j’aurais voulu que leurs calculs fussent déjoués, qu’un
miracle fît apparaître Mlle Swann avec son père, si près de nous, que nous
n’aurions pas le temps de l’éviter et serions obligés de faire sa connaissance.
Aussi, quand tout d’un coup, j’aperçus sur l’herbe, comme un signe de sa
présence possible, un koufin oublié à côté d’une ligne dont le bouchon flottait
sur l’eau, je m’empressai de détourner d’un autre côté, les regards de mon père
et de mon grand-père. D’ailleurs Swann nous ayant dit que c’était mal à lui de
s’absenter, car il avait pour le moment de la famille à demeure, la ligne
pouvait appartenir à quelque invité. On n’entendait aucun bruit de pas dans les
allées. Divisant la hauteur d’un arbre incertain, un invisible oiseau
s’ingéniait à faire trouver la journée courte, explorait d’une note prolongée,
la solitude environnante, mais il recevait d’elle une réplique si unanime, un
choc en retour si redoublé de silence et d’immobilité qu’on aurait dit qu’il
venait d’arrêter pour toujours l’instant qu’il avait cherché à faire passer plus
vite. La lumière tombait si implacable du ciel devenu fixe que l’on aurait voulu
se soustraire à son attention, et l’eau dormante elle-même, dont des insectes
irritaient perpétuellement le sommeil, rêvant sans doute de quelque Maelstrôm
imaginaire, augmentait le trouble où m’avait jeté la vue du flotteur de liège en
semblant l’entraîner à toute vitesse sur les étendues silencieuses du ciel
reflété; presque vertical il paraissait prêt à plonger et déjà je me demandais,
si, sans tenir compte du désir et de la crainte que j’avais de la connaître, je
n’avais pas le devoir de faire prévenir Mlle Swann que le poisson mordait —
quand il me fallut rejoindre en courant mon père et mon grand-père qui
m’appelaient, étonnés que je ne les eusse pas suivis dans le petit chemin qui
monte vers les champs et où ils s’étaient engagés. Je le trouvai tout
bourdonnant de l’odeur des aubépines. La haie formait comme une suite de
chapelles qui disparaissaient sous la jonchée de leurs fleurs amoncelées en
reposoir; au-dessous d’elles, le soleil posait à terre un quadrillage de clarté,
comme s’il venait de traverser une verrière; leur parfum s’étendait aussi
onctueux, aussi délimité en sa forme que si j’eusse été devant l’autel de la
Vierge, et les fleurs, aussi parées, tenaient chacune d’un air distrait son
étincelant bouquet d’étamines, fines et rayonnantes nervures de style flamboyant
comme celles qui à l’église ajouraient la rampe du jubé ou les meneaux du
vitrail et qui s’épanouissaient en blanche chair de fleur de fraisier. Combien
naïves et paysannes en comparaison sembleraient les églantines qui, dans
quelques semaines, monteraient elles aussi en plein soleil le même chemin
rustique, en la soie unie de leur corsage rougissant qu’un souffle défait.

Mais j’avais beau rester devant les aubépines à respirer, à porter devant ma
pensée qui ne savait ce qu’elle devait en faire, à perdre, à retrouver leur
invisible et fixe odeur, à m’unir au rythme qui jetait leurs fleurs, ici et là,
avec une allégresse juvénile et à des intervalles inattendus comme certains
intervalles musicaux, elles m’offraient indéfiniment le même charme avec une
profusion inépuisable, mais sans me laisser approfondir davantage, comme ces
mélodies qu’on rejoue cent fois de suite sans descendre plus avant dans leur
secret. Je me détournais d’elles un moment, pour les aborder ensuite avec des
forces plus fraîches. Je poursuivais jusque sur le talus qui, derrière la haie,
montait en pente raide vers les champs, quelque coquelicot perdu, quelques
bluets restés paresseusement en arrière, qui le décoraient çà et là de leurs
fleurs comme la bordure d’une tapisserie où apparaît clairsemé le motif agreste
qui triomphera sur le panneau; rares encore, espacés comme les maisons isolées
qui annoncent déjà l’approche d’un village, ils m’annonçaient l’immense étendue
où déferlent les blés, où moutonnent les nuages, et la vue d’un seul coquelicot
hissant au bout de son cordage et faisant cingler au vent sa flamme rouge,
au-dessus de sa bouée graisseuse et noire, me faisait battre le cœur, comme au
voyageur qui aperçoit sur une terre basse une première barque échouée que répare
un calfat, et s’écrie, avant de l’avoir encore vue: «La Mer!»

Puis je revenais devant les aubépines comme devant ces chefs-d’œuvre dont on
croit qu’on saura mieux les voir quand on a cessé un moment de les regarder,
mais j’avais beau me faire un écran de mes mains pour n’avoir qu’elles sous les
yeux, le sentiment qu’elles éveillaient en moi restait obscur et vague,
cherchant en vain à se dégager, à venir adhérer à leurs fleurs. Elles ne
m’aidaient pas à l’éclaircir, et je ne pouvais demander à d’autres fleurs de le
satisfaire. Alors, me donnant cette joie que nous éprouvons quand nous voyons de
notre peintre préféré une œuvre qui diffère de celles que nous connaissions, ou
bien si l’on nous mène devant un tableau dont nous n’avions vu jusque-là qu’une
esquisse au crayon, si un morceau entendu seulement au piano nous apparaît
ensuite revêtu des couleurs de l’orchestre, mon grand-père m’appelant et me
désignant la haie de Tansonville, me dit: «Toi qui aimes les aubépines, regarde
un peu cette épine rose; est-elle jolie!» En effet c’était une épine, mais rose,
plus belle encore que les blanches. Elle aussi avait une parure de fête — de ces
seules vraies fêtes que sont les fêtes religieuses, puisqu’un caprice contingent
ne les applique pas comme les fêtes mondaines à un jour quelconque qui ne leur
est pas spécialement destiné, qui n’a rien d’essentiellement férié — mais une
parure plus riche encore, car les fleurs attachées sur la branche, les unes
au-dessus des autres, de manière à ne laisser aucune place qui ne fût décorée,
comme des pompons qui enguirlandent une houlette rococo, étaient «en couleur»,
par conséquent d’une qualité supérieure selon l’esthétique de Combray si l’on en
jugeait par l’échelle des prix dans le «magasin» de la Place ou chez Camus où
étaient plus chers ceux des biscuits qui étaient roses. Moi-même j’appréciais
plus le fromage à la crème rose, celui où l’on m’avait permis d’écraser des
fraises. Et justement ces fleurs avaient choisi une de ces teintes de chose
mangeable, ou de tendre embellissement à une toilette pour une grande fête, qui,
parce qu’elles leur présentent la raison de leur supériorité, sont celles qui
semblent belles avec le plus d’évidence aux yeux des enfants, et à cause de
cela, gardent toujours pour eux quelque chose de plus vif et de plus naturel que
les autres teintes, même lorsqu’ils ont compris qu’elles ne promettaient rien à
leur gourmandise et n’avaient pas été choisies par la couturière. Et certes, je
l’avais tout de suite senti, comme devant les épines blanches mais avec plus
d’émerveillement, que ce n’était pas facticement, par un artifice de fabrication
humaine, qu’était traduite l’intention de festivité dans les fleurs, mais que
c’était la nature qui, spontanément, l’avait exprimée avec la naïveté d’une
commerçante de village travaillant pour un reposoir, en surchargeant l’arbuste
de ces rosettes d’un ton trop tendre et d’un pompadour provincial. Au haut des
branches, comme autant de ces petits rosiers aux pots cachés dans des papiers en
dentelles, dont aux grandes fêtes on faisait rayonner sur l’autel les minces
fusées, pullulaient mille petits boutons d’une teinte plus pâle qui, en
s’entr’ouvrant, laissaient voir, comme au fond d’une coupe de marbre rose, de
rouges sanguines et trahissaient plus encore que les fleurs, l’essence
particulière, irrésistible, de l’épine, qui, partout où elle bourgeonnait, où
elle allait fleurir, ne le pouvait qu’en rose. Intercalé dans la haie, mais
aussi différent d’elle qu’une jeune fille en robe de fête au milieu de personnes
en négligé qui resteront à la maison, tout prêt pour le mois de Marie, dont il
semblait faire partie déjà, tel brillait en souriant dans sa fraîche toilette
rose, l’arbuste catholique et délicieux.

La haie laissait voir à l’intérieur du parc une allée bordée de jasmins, de
pensées et de verveines entre lesquelles des giroflées ouvraient leur bourse
fraîche, du rose odorant et passé d’un cuir ancien de Cordoue, tandis que sur le
gravier un long tuyau d’arrosage peint en vert, déroulant ses circuits, dressait
aux points où il était percé au-dessus des fleurs, dont il imbibait les parfums,
l’éventail vertical et prismatique de ses gouttelettes multicolores. Tout à
coup, je m’arrêtai, je ne pus plus bouger, comme il arrive quand une vision ne
s’adresse pas seulement à nos regards, mais requiert des perceptions plus
profondes et dispose de notre être tout entier. Une fillette d’un blond roux qui
avait l’air de rentrer de promenade et tenait à la main une bêche de jardinage,
nous regardait, levant son visage semé de taches roses. Ses yeux noirs
brillaient et comme je ne savais pas alors, ni ne l’ai appris depuis, réduire en
ses éléments objectifs une impression forte, comme je n’avais pas, ainsi qu’on
dit, assez «d’esprit d’observation» pour dégager la notion de leur couleur,
pendant longtemps, chaque fois que je repensai à elle, le souvenir de leur éclat
se présentait aussitôt à moi comme celui d’un vif azur, puisqu’elle était
blonde: de sorte que, peut-être si elle n’avait pas eu des yeux aussi noirs — ce
qui frappait tant la première fois qu’on la voyait — je n’aurais pas été, comme
je le fus, plus particulièrement amoureux, en elle, de ses yeux bleus.

Je la regardais, d’abord de ce regard qui n’est pas que le porte-parole des
yeux, mais à la fenêtre duquel se penchent tous les sens, anxieux et pétrifiés,
le regard qui voudrait toucher, capturer, emmener le corps qu’il regarde et
l’âme avec lui; puis, tant j’avais peur que d’une seconde à l’autre mon
grand-père et mon père, apercevant cette jeune fille, me fissent éloigner en me
disant de courir un peu devant eux, d’un second regard, inconsciemment
supplicateur, qui tâchait de la forcer à faire attention à moi, à me connaître!
Elle jeta en avant et de côté ses pupilles pour prendre connaissance de mon
grand’père et de mon père, et sans doute l’idée qu’elle en rapporta fut celle
que nous étions ridicules, car elle se détourna et d’un air indifférent et
dédaigneux, se plaça de côté pour épargner à son visage d’être dans leur champ
visuel; et tandis que continuant à marcher et ne l’ayant pas aperçue, ils
m’avaient dépassé, elle laissa ses regards filer de toute leur longueur dans ma
direction, sans expression particulière, sans avoir l’air de me voir, mais avec
une fixité et un sourire dissimulé, que je ne pouvais interpréter d’après les
notions que l’on m’avait données sur la bonne éducation, que comme une preuve
d’outrageant mépris; et sa main esquissait en même temps un geste indécent,
auquel quand il était adressé en public à une personne qu’on ne connaissait pas,
le petit dictionnaire de civilité que je portais en moi ne donnait qu’un seul
sens, celui d’une intention insolente.

—«Allons, Gilberte, viens; qu’est-ce que tu fais, cria d’une voix perçante et
autoritaire une dame en blanc que je n’avais pas vue, et à quelque distance de
laquelle un Monsieur habillé de coutil et que je ne connaissais pas, fixait sur
moi des yeux qui lui sortaient de la tête; et cessant brusquement de sourire, la
jeune fille prit sa bêche et s’éloigna sans se retourner de mon côté, d’un air
docile, impénétrable et sournois.

Ainsi passa près de moi ce nom de Gilberte, donné comme un talisman qui me
permettait peut-être de retrouver un jour celle dont il venait de faire une
personne et qui, l’instant d’avant, n’était qu’une image incertaine. Ainsi
passa-t-il, proféré au-dessus des jasmins et des giroflées, aigre et frais comme
les gouttes de l’arrosoir vert; imprégnant, irisant la zone d’air pur qu’il
avait traversée — et qu’il isolait — du mystère de la vie de celle qu’il
désignait pour les êtres heureux qui vivaient, qui voyageaient avec elle;
déployant sous l’épinier rose, à hauteur de mon épaule, la quintessence de leur
familiarité, pour moi si douloureuse, avec elle, avec l’inconnu de sa vie où je
n’entrerais pas.

Un instant (tandis que nous nous éloignions et que mon grand-père murmurait: «Ce
pauvre Swann, quel rôle ils lui font jouer: on le fait partir pour qu’elle reste
seule avec son Charlus, car c’est lui, je l’ai reconnu! Et cette petite, mêlée à
toute cette infamie!») l’impression laissée en moi par le ton despotique avec
lequel la mère de Gilberte lui avait parlé sans qu’elle répliquât, en me la
montrant comme forcée d’obéir à quelqu’un, comme n’étant pas supérieure à tout,
calma un peu ma souffrance, me rendit quelque espoir et diminua mon amour. Mais
bien vite cet amour s’éleva de nouveau en moi comme une réaction par quoi mon
cœur humilié voulait se mettre de niveau avec Gilberte ou l’abaisser jusqu’à
lui. Je l’aimais, je regrettais de ne pas avoir eu le temps et l’inspiration de
l’offenser, de lui faire mal, et de la forcer à se souvenir de moi. Je la
trouvais si belle que j’aurais voulu pouvoir revenir sur mes pas, pour lui crier
en haussant les épaules: «Comme je vous trouve laide, grotesque, comme vous me
répugnez!» Cependant je m’éloignais, emportant pour toujours, comme premier type
d’un bonheur inaccessible aux enfants de mon espèce de par des lois naturelles
impossibles à transgresser, l’image d’une petite fille rousse, à la peau semée
de taches roses, qui tenait une bêche et qui riait en laissant filer sur moi de
longs regards sournois et inexpressifs. Et déjà le charme dont son nom avait
encensé cette place sous les épines roses où il avait été entendu ensemble par
elle et par moi, allait gagner, enduire, embaumer, tout ce qui l’approchait, ses
grands-parents que les miens avaient eu l’ineffable bonheur de connaître, la
sublime profession d’agent de change, le douloureux quartier des Champs-Élysées
qu’elle habitait à Paris.

«Léonie, dit mon grand-père en rentrant, j’aurais voulu t’avoir avec nous
tantôt. Tu ne reconnaîtrais pas Tansonville. Si j’avais osé, je t’aurais coupé
une branche de ces épines roses que tu aimais tant.» Mon grand-père racontait
ainsi notre promenade à ma tante Léonie, soit pour la distraire, soit qu’on
n’eût pas perdu tout espoir d’arriver à la faire sortir. Or elle aimait beaucoup
autrefois cette propriété, et d’ailleurs les visites de Swann avaient été les
dernières qu’elle avait reçues, alors qu’elle fermait déjà sa porte à tout le
monde. Et de même que quand il venait maintenant prendre de ses nouvelles (elle
était la seule personne de chez nous qu’il demandât encore à voir), elle lui
faisait répondre qu’elle était fatiguée, mais qu’elle le laisserait entrer la
prochaine fois, de même elle dit ce soir-là: «Oui, un jour qu’il fera beau,
j’irai en voiture jusqu’à la porte du parc.» C’est sincèrement qu’elle le
disait. Elle eût aimé revoir Swann et Tansonville; mais le désir qu’elle en
avait suffisait à ce qui lui restait de forces; sa réalisation les eût excédées.
Quelquefois le beau temps lui rendait un peu de vigueur, elle se levait,
s’habillait; la fatigue commençait avant qu’elle fût passée dans l’autre chambre
et elle réclamait son lit. Ce qui avait commencé pour elle — plus tôt seulement
que cela n’arrive d’habitude — c’est ce grand renoncement de la vieillesse qui
se prépare à la mort, s’enveloppe dans sa chrysalide, et qu’on peut observer, à
la fin des vies qui se prolongent tard, même entre les anciens amants qui se
sont le plus aimés, entre les amis unis par les liens les plus spirituels et qui
à partir d’une certaine année cessent de faire le voyage ou la sortie nécessaire
pour se voir, cessent de s’écrire et savent qu’ils ne communiqueront plus en ce
monde. Ma tante devait parfaitement savoir qu’elle ne reverrait pas Swann,
qu’elle ne quitterait plus jamais la maison, mais cette réclusion définitive
devait lui être rendue assez aisée pour la raison même qui selon nous aurait dû
la lui rendre plus douloureuse: c’est que cette réclusion lui était imposée par
la diminution qu’elle pouvait constater chaque jour dans ses forces, et qui, en
faisant de chaque action, de chaque mouvement, une fatigue, sinon une
souffrance, donnait pour elle à l’inaction, à l’isolement, au silence, la
douceur réparatrice et bénie du repos.

Ma tante n’alla pas voir la haie d’épines roses, mais à tous moments je
demandais à mes parents si elle n’irait pas, si autrefois elle allait souvent à
Tansonville, tâchant de les faire parler des parents et grands-parents de Mlle
Swann qui me semblaient grands comme des Dieux. Ce nom, devenu pour moi presque
mythologique, de Swann, quand je causais avec mes parents, je languissais du
besoin de le leur entendre dire, je n’osais pas le prononcer moi-même, mais je
les entraînais sur des sujets qui avoisinaient Gilberte et sa famille, qui la
concernaient, où je ne me sentais pas exilé trop loin d’elle; et je contraignais
tout d’un coup mon père, en feignant de croire par exemple que la charge de mon
grand-père avait été déjà avant lui dans notre famille, ou que la haie d’épines
roses que voulait voir ma tante Léonie se trouvait en terrain communal, à
rectifier mon assertion, à me dire, comme malgré moi, comme de lui-même: «Mais
non, cette charge-là était au père de Swann, cette haie fait partie du parc de
Swann.» Alors j’étais obligé de reprendre ma respiration, tant, en se posant sur
la place où il était toujours écrit en moi, pesait à m’étouffer ce nom qui, au
moment où je l’entendais, me paraissait plus plein que tout autre, parce qu’il
était lourd de toutes les fois où, d’avance, je l’avais mentalement proféré. Il
me causait un plaisir que j’étais confus d’avoir osé réclamer à mes parents, car
ce plaisir était si grand qu’il avait dû exiger d’eux pour qu’ils me le
procurassent beaucoup de peine, et sans compensation, puisqu’il n’était pas un
plaisir pour eux. Aussi je détournais la conversation par discrétion. Par
scrupule aussi. Toutes les séductions singulières que je mettais dans ce nom de
Swann, je les retrouvais en lui dès qu’ils le prononçaient. Il me semblait alors
tout d’un coup que mes parents ne pouvaient pas ne pas les ressentir, qu’ils se
trouvaient placés à mon point de vue, qu’ils apercevaient à leur tour,
absolvaient, épousaient mes rêves, et j’étais malheureux comme si je les avais
vaincus et dépravés.

Cette année-là, quand, un peu plus tôt que d’habitude, mes parents eurent fixé
le jour de rentrer à Paris, le matin du départ, comme on m’avait fait friser
pour être photographié, coiffer avec précaution un chapeau que je n’avais encore
jamais mis et revêtir une douillette de velours, après m’avoir cherché partout,
ma mère me trouva en larmes dans le petit raidillon, contigu à Tansonville, en
train de dire adieu aux aubépines, entourant de mes bras les branches piquantes,
et, comme une princesse de tragédie à qui pèseraient ces vains ornements, ingrat
envers l’importune main qui en formant tous ces nœuds avait pris soin sur mon
front d’assembler mes cheveux, foulant aux pieds mes papillotes arrachées et mon
chapeau neuf. Ma mère ne fut pas touchée par mes larmes, mais elle ne put
retenir un cri à la vue de la coiffe défoncée et de la douillette perdue. Je ne
l’entendis pas: «O mes pauvres petites aubépines, disais-je en pleurant, ce
n’est pas vous qui voudriez me faire du chagrin, me forcer à partir. Vous, vous
ne m’avez jamais fait de peine! Aussi je vous aimerai toujours.» Et, essuyant
mes larmes, je leur promettais, quand je serais grand, de ne pas imiter la vie
insensée des autres hommes et, même à Paris, les jours de printemps, au lieu
d’aller faire des visites et écouter des niaiseries, de partir dans la campagne
voir les premières aubépines.

Une fois dans les champs, on ne les quittait plus pendant tout le reste de la
promenade qu’on faisait du côté de Méséglise. Ils étaient perpétuellement
parcourus, comme par un chemineau invisible, par le vent qui était pour moi le
génie particulier de Combray. Chaque année, le jour de notre arrivée, pour
sentir que j’étais bien à Combray, je montais le retrouver qui courait dans les
sayons et me faisait courir à sa suite. On avait toujours le vent à côté de soi
du côté de Méséglise, sur cette plaine bombée où pendant des lieues il ne
rencontre aucun accident de terrain. Je savais que Mlle Swann allait souvent à
Laon passer quelques jours et, bien que ce fût à plusieurs lieues, la distance
se trouvant compensée par l’absence de tout obstacle, quand, par les chauds
après-midi, je voyais un même souffle, venu de l’extrême horizon, abaisser les
blés les plus éloignés, se propager comme un flot sur toute l’immense étendue et
venir se coucher, murmurant et tiède, parmi les sainfoins et les trèfles, à mes
pieds, cette plaine qui nous était commune à tous deux semblait nous rapprocher,
nous unir, je pensais que ce souffle avait passé auprès d’elle, que c’était
quelque message d’elle qu’il me chuchotait sans que je pusse le comprendre, et
je l’embrassais au passage. A gauche était un village qui s’appelait Champieu
(Campus Pagani, selon le curé). Sur la droite, on apercevait par delà les blés,
les deux clochers ciselés et rustiques de Saint-André-des-Champs, eux-mêmes
effilés, écailleux, imbriqués d’alvéoles, guillochés, jaunissants et grumeleux,
comme deux épis.

A intervalles symétriques, au milieu de l’inimitable ornementation de leurs
feuilles qu’on ne peut confondre avec la feuille d’aucun autre arbre fruitier,
les pommiers ouvraient leurs larges pétales de satin blanc ou suspendaient les
timides bouquets de leurs rougissants boutons. C’est du côté de Méséglise que
j’ai remarqué pour la première fois l’ombre ronde que les pommiers font sur la
terre ensoleillée, et aussi ces soies d’or impalpable que le couchant tisse
obliquement sous les feuilles, et que je voyais mon père interrompre de sa canne
sans les faire jamais dévier.

Parfois dans le ciel de l’après-midi passait la lune blanche comme une nuée,
furtive, sans éclat, comme une actrice dont ce n’est pas l’heure de jouer et
qui, de la salle, en toilette de ville, regarde un moment ses camarades,
s’effaçant, ne voulant pas qu’on fasse attention à elle. J’aimais à retrouver
son image dans des tableaux et dans des livres, mais ces œuvres d’art étaient
bien différentes — du moins pendant les premières années, avant que Bloch eût
accoutumé mes yeux et ma pensée à des harmonies plus subtiles — de celles où la
lune me paraîtrait belle aujourd’hui et où je ne l’eusse pas reconnue alors.
C’était, par exemple, quelque roman de Saintine, un paysage de Gleyre où elle
découpe nettement sur le ciel une faucille d’argent, de ces œuvres naïvement
incomplètes comme étaient mes propres impressions et que les sœurs de ma
grand’mère s’indignaient de me voir aimer. Elles pensaient qu’on doit mettre
devant les enfants, et qu’ils font preuve de goût en aimant d’abord, les œuvres
que, parvenu à la maturité, on admire définitivement. C’est sans doute qu’elles
se figuraient les mérites esthétiques comme des objets matériels qu’un œil
ouvert ne peut faire autrement que de percevoir, sans avoir eu besoin d’en mûrir
lentement des équivalents dans son propre cœur.

C’est du côté de Méséglise, à Montjouvain, maison située au bord d’une grande
mare et adossée à un talus buissonneux que demeurait M. Vinteuil. Aussi
croisait-on souvent sur la route sa fille, conduisant un buggy à toute allure. A
partir d’une certaine année on ne la rencontra plus seule, mais avec une amie
plus âgée, qui avait mauvaise réputation dans le pays et qui un jour s’installa
définitivement à Montjouvain. On disait: «Faut-il que ce pauvre M. Vinteuil soit
aveuglé par la tendresse pour ne pas s’apercevoir de ce qu’on raconte, et
permettre à sa fille, lui qui se scandalise d’une parole déplacée, de faire
vivre sous son toit une femme pareille. Il dit que c’est une femme supérieure,
un grand cœur et qu’elle aurait eu des dispositions extraordinaires pour la
musique si elle les avait cultivées. Il peut être sûr que ce n’est pas de
musique qu’elle s’occupe avec sa fille.» M. Vinteuil le disait; et il est en
effet remarquable combien une personne excite toujours d’admiration pour ses
qualités morales chez les parents de toute autre personne avec qui elle a des
relations charnelles. L’amour physique, si injustement décrié, force tellement
tout être à manifester jusqu’aux moindres parcelles qu’il possède de bonté,
d’abandon de soi, qu’elles resplendissent jusqu’aux yeux de l’entourage
immédiat. Le docteur Percepied à qui sa grosse voix et ses gros sourcils
permettaient de tenir tant qu’il voulait le rôle de perfide dont il n’avait pas
le physique, sans compromettre en rien sa réputation inébranlable et imméritée
de bourru bienfaisant, savait faire rire aux larmes le curé et tout le monde en
disant d’un ton rude: «Hé bien! il paraît qu’elle fait de la musique avec son
amie, Mlle Vinteuil. Ça a l’air de vous étonner. Moi je sais pas. C’est le père
Vinteuil qui m’a encore dit ça hier. Après tout, elle a bien le droit d’aimer la
musique, c’te fille. Moi je ne suis pas pour contrarier les vocations
artistiques des enfants. Vinteuil non plus à ce qu’il paraît. Et puis lui aussi
il fait de la musique avec l’amie de sa fille. Ah! sapristi on en fait une
musique dans c’te boîte-là. Mais qu’est-ce que vous avez à rire; mais ils font
trop de musique ces gens. L’autre jour j’ai rencontré le père Vinteuil près du
cimetière. Il ne tenait pas sur ses jambes.»

Pour ceux qui comme nous virent à cette époque M. Vinteuil éviter les personnes
qu’il connaissait, se détourner quand il les apercevait, vieillir en quelques
mois, s’absorber dans son chagrin, devenir incapable de tout effort qui n’avait
pas directement le bonheur de sa fille pour but, passer des journées entières
devant la tombe de sa femme — il eût été difficile de ne pas comprendre qu’il
était en train de mourir de chagrin, et de supposer qu’il ne se rendait pas
compte des propos qui couraient. Il les connaissait, peut-être même y
ajoutait-il foi. Il n’est peut-être pas une personne, si grande que soit sa
vertu, que la complexité des circonstances ne puisse amener à vivre un jour dans
la familiarité du vice qu’elle condamne le plus formellement — sans qu’elle le
reconnaisse d’ailleurs tout à fait sous le déguisement de faits particuliers
qu’il revêt pour entrer en contact avec elle et la faire souffrir: paroles
bizarres, attitude inexplicable, un certain soir, de tel être qu’elle a par
ailleurs tant de raisons pour aimer. Mais pour un homme comme M. Vinteuil il
devait entrer bien plus de souffrance que pour un autre dans la résignation à
une de ces situations qu’on croit à tort être l’apanage exclusif du monde de la
bohème: elles se produisent chaque fois qu’a besoin de se réserver la place et
la sécurité qui lui sont nécessaires, un vice que la nature elle-même fait
épanouir chez un enfant, parfois rien qu’en mêlant les vertus de son père et de
sa mère, comme la couleur de ses yeux. Mais de ce que M. Vinteuil connaissait
peut-être la conduite de sa fille, il ne s’ensuit pas que son culte pour elle en
eût été diminué. Les faits ne pénètrent pas dans le monde où vivent nos
croyances, ils n’ont pas fait naître celles-ci, ils ne les détruisent pas; ils
peuvent leur infliger les plus constants démentis sans les affaiblir, et une
avalanche de malheurs ou de maladies se succédant sans interruption dans une
famille, ne la fera pas douter de la bonté de son Dieu ou du talent de son
médecin. Mais quand M. Vinteuil songeait à sa fille et à lui-même du point de
vue du monde, du point de vue de leur réputation, quand il cherchait à se situer
avec elle au rang qu’ils occupaient dans l’estime générale, alors ce jugement
d’ordre social, il le portait exactement comme l’eût fait l’habitant de Combray
qui lui eût été le plus hostile, il se voyait avec sa fille dans le dernier
bas-fond, et ses manières en avaient reçu depuis peu cette humilité, ce respect
pour ceux qui se trouvaient au-dessus de lui et qu’il voyait d’en bas
(eussent-ils été fort au-dessous de lui jusque-là), cette tendance à chercher à
remonter jusqu’à eux, qui est une résultante presque mécanique de toutes les
déchéances. Un jour que nous marchions avec Swann dans une rue de Combray, M.
Vinteuil qui débouchait d’une autre, s’était trouvé trop brusquement en face de
nous pour avoir le temps de nous éviter; et Swann avec cette orgueilleuse
charité de l’homme du monde qui, au milieu de la dissolution de tous ses
préjugés moraux, ne trouve dans l’infamie d’autrui qu’une raison d’exercer
envers lui une bienveillance dont les témoignages chatouillent d’autant plus
l’amour-propre de celui qui les donne, qu’il les sent plus précieux à celui qui
les reçoit, avait longuement causé avec M. Vinteuil, à qui, jusque-là il
n’adressait pas la parole, et lui avait demandé avant de nous quitter s’il
n’enverrait pas un jour sa fille jouer à Tansonville. C’était une invitation
qui, il y a deux ans, eût indigné M. Vinteuil, mais qui, maintenant, le
remplissait de sentiments si reconnaissants qu’il se croyait obligé par eux, à
ne pas avoir l’indiscrétion de l’accepter. L’amabilité de Swann envers sa fille
lui semblait être en soi-même un appui si honorable et si délicieux qu’il
pensait qu’il valait peut-être mieux ne pas s’en servir, pour avoir la douceur
toute platonique de le conserver.

—«Quel homme exquis, nous dit-il, quand Swann nous eut quittés, avec la même
enthousiaste vénération qui tient de spirituelles et jolies bourgeoises en
respect et sous le charme d’une duchesse, fût-elle laide et sotte. Quel homme
exquis! Quel malheur qu’il ait fait un mariage tout à fait déplacé.»

Et alors, tant les gens les plus sincères sont mêlés d’hypocrisie et dépouillent
en causant avec une personne l’opinion qu’ils ont d’elle et expriment dès
qu’elle n’est plus là, mes parents déplorèrent avec M. Vinteuil le mariage de
Swann au nom de principes et de convenances auxquels (par cela même qu’ils les
invoquaient en commun avec lui, en braves gens de même acabit) ils avaient l’air
de sous-entendre qu’il n’était pas contrevenu à Montjouvain. M. Vinteuil
n’envoya pas sa fille chez Swann. Et celui-ci fût le premier à le regretter. Car
chaque fois qu’il venait de quitter M. Vinteuil, il se rappelait qu’il avait
depuis quelque temps un renseignement à lui demander sur quelqu’un qui portait
le même nom que lui, un de ses parents, croyait-il. Et cette fois-là il s’était
bien promis de ne pas oublier ce qu’il avait à lui dire, quand M. Vinteuil
enverrait sa fille à Tansonville.

Comme la promenade du côté de Méséglise était la moins longue des deux que nous
faisions autour de Combray et qu’à cause de cela on la réservait pour les temps
incertains, le climat du côté de Méséglise était assez pluvieux et nous ne
perdions jamais de vue la lisière des bois de Roussainville dans l’épaisseur
desquels nous pourrions nous mettre à couvert.

Souvent le soleil se cachait derrière une nuée qui déformait son ovale et dont
il jaunissait la bordure. L’éclat, mais non la clarté, était enlevé à la
campagne où toute vie semblait suspendue, tandis que le petit village de
Roussainville sculptait sur le ciel le relief de ses arêtes blanches avec une
précision et un fini accablants. Un peu de vent faisait envoler un corbeau qui
retombait dans le lointain, et, contre le ciel blanchissant, le lointain des
bois paraissait plus bleu, comme peint dans ces camaïeux qui décorent les
trumeaux des anciennes demeures.

Mais d’autres fois se mettait à tomber la pluie dont nous avait menacés le
capucin que l’opticien avait à sa devanture; les gouttes d’eau comme des oiseaux
migrateurs qui prennent leur vol tous ensemble, descendaient à rangs pressés du
ciel. Elles ne se séparent point, elles ne vont pas à l’aventure pendant la
rapide traversée, mais chacune tenant sa place, attire à elle celle qui la suit
et le ciel en est plus obscurci qu’au départ des hirondelles. Nous nous
réfugiions dans le bois. Quand leur voyage semblait fini, quelques-unes, plus
débiles, plus lentes, arrivaient encore. Mais nous ressortions de notre abri,
car les gouttes se plaisent aux feuillages, et la terre était déjà presque
séchée que plus d’une s’attardait à jouer sur les nervures d’une feuille, et
suspendue à la pointe, reposée, brillant au soleil, tout d’un coup se laissait
glisser de toute la hauteur de la branche et nous tombait sur le nez.

Souvent aussi nous allions nous abriter, pêle-mêle avec les Saints et les
Patriarches de pierre sous le porche de Saint-André-des-Champs. Que cette église
était française! Au-dessus de la porte, les Saints, les rois-chevaliers une
fleur de lys à la main, des scènes de noces et de funérailles, étaient
représentés comme ils pouvaient l’être dans l’âme de Françoise. Le sculpteur
avait aussi narré certaines anecdotes relatives à Aristote et à Virgile de la
même façon que Françoise à la cuisine parlait volontiers de saint Louis comme si
elle l’avait personnellement connu, et généralement pour faire honte par la
comparaison à mes grands-parents moins «justes». On sentait que les notions que
l’artiste médiéval et la paysanne médiévale (survivant au XlXe siècle) avaient
de l’histoire ancienne ou chrétienne, et qui se distinguaient par autant
d’inexactitude que de bonhomie, ils les tenaient non des livres, mais d’une
tradition à la fois antique et directe, ininterrompue, orale, déformée,
méconnaissable et vivante. Une autre personnalité de Combray que je
reconnaissais aussi, virtuelle et prophétisée, dans la sculpture gothique de
Saint-André-des-Champs c’était le jeune Théodore, le garçon de chez Camus.
Françoise sentait d’ailleurs si bien en lui un pays et un contemporain que,
quand ma tante Léonie était trop malade pour que Françoise pût suffire à la
retourner dans son lit, à la porter dans son fauteuil, plutôt que de laisser la
fille de cuisine monter se faire «bien voir» de ma tante, elle appelait
Théodore. Or, ce garçon qui passait et avec raison pour si mauvais sujet, était
tellement rempli de l’âme qui avait décoré Saint-André-des-Champs et notamment
des sentiments de respect que Françoise trouvait dus aux «pauvres malades», à
«sa pauvre maîtresse», qu’il avait pour soulever la tête de ma tante sur son
oreiller la mine naïve et zélée des petits anges des bas-reliefs, s’empressant,
un cierge à la main, autour de la Vierge défaillante, comme si les visages de
pierre sculptée, grisâtres et nus, ainsi que sont les bois en hiver, n’étaient
qu’un ensommeillement, qu’une réserve, prête à refleurir dans la vie en
innombrables visages populaires, révérends et futés comme celui de Théodore,
enluminés de la rougeur d’une pomme mûre. Non plus appliquée à la pierre comme
ces petits anges, mais détachée du porche, d’une stature plus qu’humaine, debout
sur un socle comme sur un tabouret qui lui évitât de poser ses pieds sur le sol
humide, une sainte avait les joues pleines, le sein ferme et qui gonflait la
draperie comme une grappe mûre dans un sac de crin, le front étroit, le nez
court et mutin, les prunelles enfoncées, l’air valide, insensible et courageux
des paysannes de la contrée. Cette ressemblance qui insinuait dans la statue une
douceur que je n’y avais pas cherchée, était souvent certifiée par quelque fille
des champs, venue comme nous se mettre à couvert et dont la présence, pareille à
celle de ces feuillages pariétaires qui ont poussé à côté des feuillages
sculptés, semblait destinée à permettre, par une confrontation avec la nature,
de juger de la vérité de l’œuvre d’art. Devant nous, dans le lointain, terre
promise ou maudite, Roussainville, dans les murs duquel je n’ai jamais pénétré,
Roussainville, tantôt, quand la pluie avait déjà cessé pour nous, continuait à
être châtié comme un village de la Bible par toutes les lances de l’orage qui
flagellaient obliquement les demeures de ses habitants, ou bien était déjà
pardonné par Dieu le Père qui faisait descendre vers lui, inégalement longues,
comme les rayons d’un ostensoir d’autel, les tiges d’or effrangées de son soleil
reparu.

Quelquefois le temps était tout à fait gâté, il fallait rentrer et rester
enfermé dans la maison. Çà et là au loin dans la campagne que l’obscurité et
l’humidité faisaient ressembler à la mer, des maisons isolées, accrochées au
flanc d’une colline plongée dans la nuit et dans l’eau, brillaient comme des
petits bateaux qui ont replié leurs voiles et sont immobiles au large pour toute
la nuit. Mais qu’importait la pluie, qu’importait l’orage! L’été, le mauvais
temps n’est qu’une humeur passagère, superficielle, du beau temps sous-jacent et
fixe, bien différent du beau temps instable et fluide de l’hiver et qui, au
contraire, installé sur la terre où il s’est solidifié en denses feuillages sur
lesquels la pluie peut s’égoutter sans compromettre la résistance de leur
permanente joie, a hissé pour toute la saison, jusque dans les rues du village,
aux murs des maisons et des jardins, ses pavillons de soie violette ou blanche.
Assis dans le petit salon, où j’attendais l’heure du dîner en lisant,
j’entendais l’eau dégoutter de nos marronniers, mais je savais que l’averse ne
faisait que vernir leurs feuilles et qu’ils promettaient de demeurer là, comme
des gages de l’été, toute la nuit pluvieuse, à assurer la continuité du beau
temps; qu’il avait beau pleuvoir, demain, au-dessus de la barrière blanche de
Tansonville, onduleraient, aussi nombreuses, de petites feuilles en forme de
cœur; et c’est sans tristesse que j’apercevais le peuplier de la rue des
Perchamps adresser à l’orage des supplications et des salutations désespérées;
c’est sans tristesse que j’entendais au fond du jardin les derniers roulements
du tonnerre roucouler dans les lilas.

Si le temps était mauvais dès le matin, mes parents renonçaient à la promenade
et je ne sortais pas. Mais je pris ensuite l’habitude d’aller, ces jours-là,
marcher seul du côté de Méséglise-la-Vineuse, dans l’automne où nous dûmes venir
à Combray pour la succession de ma tante Léonie, car elle était enfin morte,
faisant triompher à la fois ceux qui prétendaient que son régime affaiblissant
finirait par la tuer, et non moins les autres qui avaient toujours soutenu
qu’elle souffrait d’une maladie non pas imaginaire mais organique, à l’évidence
de laquelle les sceptiques seraient bien obligés de se rendre quand elle y
aurait succombé; et ne causant par sa mort de grande douleur qu’à un seul être,
mais à celui-là, sauvage. Pendant les quinze jours que dura la dernière maladie
de ma tante, Françoise ne la quitta pas un instant, ne se déshabilla pas, ne
laissa personne lui donner aucun soin, et ne quitta son corps que quand il fut
enterré. Alors nous comprîmes que cette sorte de crainte où Françoise avait vécu
des mauvaises paroles, des soupçons, des colères de ma tante avait développé
chez elle un sentiment que nous avions pris pour de la haine et qui était de la
vénération et de l’amour. Sa véritable maîtresse, aux décisions impossibles à
prévoir, aux ruses difficiles à déjouer, au bon cœur facile à fléchir, sa
souveraine, son mystérieux et tout-puissant monarque n’était plus. A côté d’elle
nous comptions pour bien peu de chose. Il était loin le temps où quand nous
avions commencé à venir passer nos vacances à Combray, nous possédions autant de
prestige que ma tante aux yeux de Françoise. Cet automne-là tout occupés des
formalités à remplir, des entretiens avec les notaires et avec les fermiers, mes
parents n’ayant guère de loisir pour faire des sorties que le temps d’ailleurs
contrariait, prirent l’habitude de me laisser aller me promener sans eux du côté
de Méséglise, enveloppé dans un grand plaid qui me protégeait contre la pluie et
que je jetais d’autant plus volontiers sur mes épaules que je sentais que ses
rayures écossaises scandalisaient Françoise, dans l’esprit de qui on n’aurait pu
faire entrer l’idée que la couleur des vêtements n’a rien à faire avec le deuil
et à qui d’ailleurs le chagrin que nous avions de la mort de ma tante plaisait
peu, parce que nous n’avions pas donné de grand repas funèbre, que nous ne
prenions pas un son de voix spécial pour parler d’elle, que même parfois je
chantonnais. Je suis sûr que dans un livre — et en cela j’étais bien moi-même
comme Françoise — cette conception du deuil d’après la Chanson de Roland et le
portail de Saint-André-des-Champs m’eût été sympathique. Mais dès que Françoise
était auprès de moi, un démon me poussait à souhaiter qu’elle fût en colère, je
saisissais le moindre prétexte pour lui dire que je regrettais ma tante parce
que c’était une bonne femme, malgré ses ridicules, mais nullement parce que
c’était ma tante, qu’elle eût pu être ma tante et me sembler odieuse, et sa mort
ne me faire aucune peine, propos qui m’eussent semblé ineptes dans un livre.

Si alors Françoise remplie comme un poète d’un flot de pensées confuses sur le
chagrin, sur les souvenirs de famille, s’excusait de ne pas savoir répondre à
mes théories et disait: «Je ne sais pas m’esprimer», je triomphais de cet aveu
avec un bon sens ironique et brutal digne du docteur Percepied; et si elle
ajoutait: «Elle était tout de même de la parentèse, il reste toujours le respect
qu’on doit à la parentèse», je haussais les épaules et je me disais: «Je suis
bien bon de discuter avec une illettrée qui fait des cuirs pareils», adoptant
ainsi pour juger Françoise le point de vue mesquin d’hommes dont ceux qui les
méprisent le plus dans l’impartialité de la méditation, sont fort capables de
tenir le rôle quand ils jouent une des scènes vulgaires de la vie.

Mes promenades de cet automne-là furent d’autant plus agréables que je les
faisais après de longues heures passées sur un livre. Quand j’étais fatigué
d’avoir lu toute la matinée dans la salle, jetant mon plaid sur mes épaules, je
sortais: mon corps obligé depuis longtemps de garder l’immobilité, mais qui
s’était chargé sur place d’animation et de vitesse accumulées, avait besoin
ensuite, comme une toupie qu’on lâche, de les dépenser dans toutes les
directions. Les murs des maisons, la haie de Tansonville, les arbres du bois de
Roussainville, les buissons auxquels s’adosse Montjouvain, recevaient des coups
de parapluie ou de canne, entendaient des cris joyeux, qui n’étaient, les uns et
les autres, que des idées confuses qui m’exaltaient et qui n’ont pas atteint le
repos dans la lumière, pour avoir préféré à un lent et difficile
éclaircissement, le plaisir d’une dérivation plus aisée vers une issue
immédiate. La plupart des prétendues traductions de ce que nous avons ressenti
ne font ainsi que nous en débarrasser en le faisant sortir de nous sous une
forme indistincte qui ne nous apprend pas à le connaître. Quand j’essaye de
faire le compte de ce que je dois au côté de Méséglise, des humbles découvertes
dont il fût le cadre fortuit ou le nécessaire inspirateur, je me rappelle que
c’est, cet automne-là, dans une de ces promenades, près du talus broussailleux
qui protège Montjouvain, que je fus frappé pour la première fois de ce désaccord
entre nos impressions et leur expression habituelle. Après une heure de pluie et
de vent contre lesquels j’avais lutté avec allégresse, comme j’arrivais au bord
de la mare de Montjouvain devant une petite cahute recouverte en tuiles où le
jardinier de M. Vinteuil serrait ses instruments de jardinage, le soleil venait
de reparaître, et ses dorures lavées par l’averse reluisaient à neuf dans le
ciel, sur les arbres, sur le mur de la cahute, sur son toit de tuile encore
mouillé, à la crête duquel se promenait une poule. Le vent qui soufflait tirait
horizontalement les herbes folles qui avaient poussé dans la paroi du mur, et
les plumes de duvet de la poule, qui, les unes et les autres se laissaient filer
au gré de son souffle jusqu’à l’extrémité de leur longueur, avec l’abandon de
choses inertes et légères. Le toit de tuile faisait dans la mare, que le soleil
rendait de nouveau réfléchissante, une marbrure rose, à laquelle je n’avais
encore jamais fait attention. Et voyant sur l’eau et à la face du mur un pâle
sourire répondre au sourire du ciel, je m’écriai dans mon enthousiasme en
brandissant mon parapluie refermé: «Zut, zut, zut, zut.» Mais en même temps je
sentis que mon devoir eût été de ne pas m’en tenir à ces mots opaques et de
tâcher de voir plus clair dans mon ravissement.

Et c’est à ce moment-là encore — grâce à un paysan qui passait, l’air déjà
d’être d’assez mauvaise humeur, qui le fut davantage quand il faillit recevoir
mon parapluie dans la figure, et qui répondit sans chaleur à mes «beau temps,
n’est-ce pas, il fait bon marcher» — que j’appris que les mêmes émotions ne se
produisent pas simultanément, dans un ordre préétabli, chez tous les hommes.
Plus tard chaque fois qu’une lecture un peu longue m’avait mis en humeur de
causer, le camarade à qui je brûlais d’adresser la parole venait justement de se
livrer au plaisir de la conversation et désirait maintenant qu’on le laissât
lire tranquille. Si je venais de penser à mes parents avec tendresse et de
prendre les décisions les plus sages et les plus propres à leur faire plaisir,
ils avaient employé le même temps à apprendre une peccadille que j’avais oubliée
et qu’ils me reprochaient sévèrement au moment où je m’élançais vers eux pour
les embrasser.

Parfois à l’exaltation que me donnait la solitude, s’en ajoutait une autre que
je ne savais pas en départager nettement, causée par le désir de voir surgir
devant moi une paysanne, que je pourrais serrer dans mes bras. Né brusquement,
et sans que j’eusse eu le temps de le rapporter exactement à sa cause, au milieu
de pensées très différentes, le plaisir dont il était accompagné ne me semblait
qu’un degré supérieur de celui qu’elles me donnaient. Je faisais un mérite de
plus à tout ce qui était à ce moment-là dans mon esprit, au reflet rose du toit
de tuile, aux herbes folles, au village de Roussainville où je désirais depuis
longtemps aller, aux arbres de son bois, au clocher de son église, de cet émoi
nouveau qui me les faisait seulement paraître plus désirables parce que je
croyais que c’était eux qui le provoquaient, et qui semblait ne vouloir que me
porter vers eux plus rapidement quand il enflait ma voile d’une brise puissante,
inconnue et propice. Mais si ce désir qu’une femme apparût ajoutait pour moi aux
charmes de la nature quelque chose de plus exaltant, les charmes de la nature,
en retour, élargissaient ce que celui de la femme aurait eu de trop restreint.
Il me semblait que la beauté des arbres c’était encore la sienne et que l’âme de
ces horizons, du village de Roussainville, des livres que je lisais cette
année-là, son baiser me la livrerait; et mon imagination reprenant des forces au
contact de ma sensualité, ma sensualité se répandant dans tous les domaines de
mon imagination, mon désir n’avait plus de limites. C’est qu’aussi — comme il
arrive dans ces moments de rêverie au milieu de la nature où l’action de
l’habitude étant suspendue, nos notions abstraites des choses mises de côté,
nous croyons d’une foi profonde, à l’originalité, à la vie individuelle du lieu
où nous nous trouvons — la passante qu’appelait mon désir me semblait être non
un exemplaire quelconque de ce type général: la femme, mais un produit
nécessaire et naturel de ce sol. Car en ce temps-là tout ce qui n’était pas moi,
la terre et les êtres, me paraissait plus précieux, plus important, doué d’une
existence plus réelle que cela ne paraît aux hommes faits. Et la terre et les
êtres je ne les séparais pas. J’avais le désir d’une paysanne de Méséglise ou de
Roussainville, d’une pêcheuse de Balbec, comme j’avais le désir de Méséglise et
de Balbec. Le plaisir qu’elles pouvaient me donner m’aurait paru moins vrai, je
n’aurais plus cru en lui, si j’en avais modifié à ma guise les conditions.
Connaître à Paris une pêcheuse de Balbec ou une paysanne de Méséglise c’eût été
recevoir des coquillages que je n’aurais pas vus sur la plage, une fougère que
je n’aurais pas trouvée dans les bois, c’eût été retrancher au plaisir que la
femme me donnerait tous ceux au milieu desquels l’avait enveloppée mon
imagination. Mais errer ainsi dans les bois de Roussainville sans une paysanne à
embrasser, c’était ne pas connaître de ces bois le trésor caché, la beauté
profonde. Cette fille que je ne voyais que criblée de feuillages, elle était
elle-même pour moi comme une plante locale d’une espèce plus élevée seulement
que les autres et dont la structure permet d’approcher de plus près qu’en elles,
la saveur profonde du pays. Je pouvais d’autant plus facilement le croire (et
que les caresses par lesquelles elle m’y ferait parvenir, seraient aussi d’une
sorte particulière et dont je n’aurais pas pu connaître le plaisir par une autre
qu’elle), que j’étais pour longtemps encore à l’âge où on ne l’a pas encore
abstrait ce plaisir de la possession des femmes différentes avec lesquelles on
l’a goûté, où on ne l’a pas réduit à une notion générale qui les fait considérer
dès lors comme les instruments interchangeables d’un plaisir toujours identique.
Il n’existe même pas, isolé, séparé et formulé dans l’esprit, comme le but qu’on
poursuit en s’approchant d’une femme, comme la cause du trouble préalable qu’on
ressent. A peine y songe-t-on comme à un plaisir qu’on aura; plutôt, on
l’appelle son charme à elle; car on ne pense pas à soi, on ne pense qu’à sortir
de soi. Obscurément attendu, immanent et caché, il porte seulement à un tel
paroxysme au moment où il s’accomplit, les autres plaisirs que nous causent les
doux regards, les baisers de celle qui est auprès de nous, qu’il nous apparaît
surtout à nous-même comme une sorte de transport de notre reconnaissance pour la
bonté de cœur de notre compagne et pour sa touchante prédilection à notre égard
que nous mesurons aux bienfaits, au bonheur dont elle nous comble.

Hélas, c’était en vain que j’implorais le donjon de Roussainville, que je lui
demandais de faire venir auprès de moi quelque enfant de son village, comme au
seul confident que j’avais eu de mes premiers désirs, quand au haut de notre
maison de Combray, dans le petit cabinet sentant l’iris, je ne voyais que sa
tour au milieu du carreau de la fenêtre entr’ouverte, pendant qu’avec les
hésitations héroïques du voyageur qui entreprend une exploration ou du désespéré
qui se suicide, défaillant, je me frayais en moi-même une route inconnue et que
je croyais mortelle, jusqu’au moment où une trace naturelle comme celle d’un
colimaçon s’ajoutait aux feuilles du cassis sauvage qui se penchaient jusqu’à
moi. En vain je le suppliais maintenant. En vain, tenant l’étendue dans le champ
de ma vision, je la drainais de mes regards qui eussent voulu en ramener une
femme. Je pouvais aller jusqu’au porche de Saint-André-des-Champs; jamais ne s’y
trouvait la paysanne que je n’eusse pas manqué d’y rencontrer si j’avais été
avec mon grand-père et dans l’impossibilité de lier conversation avec elle. Je
fixais indéfiniment le tronc d’un arbre lointain, de derrière lequel elle allait
surgir et venir à moi; l’horizon scruté restait désert, la nuit tombait, c’était
sans espoir que mon attention s’attachait, comme pour aspirer les créatures
qu’ils pouvaient recéler, à ce sol stérile, à cette terre épuisée; et ce n’était
plus d’allégresse, c’était de rage que je frappais les arbres du bois de
Roussainville d’entre lesquels ne sortait pas plus d’êtres vivants que s’ils
eussent été des arbres peints sur la toile d’un panorama, quand, ne pouvant me
résigner à rentrer à la maison avant d’avoir serré dans mes bras la femme que
j’avais tant désirée, j’étais pourtant obligé de reprendre le chemin de Combray
en m’avouant à moi-même qu’était de moins en moins probable le hasard qui l’eût
mise sur mon chemin. Et s’y fût-elle trouvée, d’ailleurs, eussé-je osé lui
parler? Il me semblait qu’elle m’eût considéré comme un fou; je cessais de
croire partagés par d’autres êtres, de croire vrais en dehors de moi les désirs
que je formais pendant ces promenades et qui ne se réalisaient pas. Ils ne
m’apparaissaient plus que comme les créations purement subjectives,
impuissantes, illusoires, de mon tempérament. Ils n’avaient plus de lien avec la
nature, avec la réalité qui dès lors perdait tout charme et toute signification
et n’était plus à ma vie qu’un cadre conventionnel comme l’est à la fiction d’un
roman le wagon sur la banquette duquel le voyageur le lit pour tuer le temps.

C’est peut-être d’une impression ressentie aussi auprès de Montjouvain, quelques
années plus tard, impression restée obscure alors, qu’est sortie, bien après,
l’idée que je me suis faite du sadisme. On verra plus tard que, pour de tout
autres raisons, le souvenir de cette impression devait jouer un rôle important
dans ma vie. C’était par un temps très chaud; mes parents qui avaient dû
s’absenter pour toute la journée, m’avaient dit de rentrer aussi tard que je
voudrais; et étant allé jusqu’à la mare de Montjouvain où j’aimais revoir les
reflets du toit de tuile, je m’étais étendu à l’ombre et endormi dans les
buissons du talus qui domine la maison, là où j’avais attendu mon père
autrefois, un jour qu’il était allé voir M. Vinteuil. Il faisait presque nuit
quand je m’éveillai, je voulus me lever, mais je vis Mlle Vinteuil (autant que
je pus la reconnaître, car je ne l’avais pas vue souvent à Combray, et seulement
quand elle était encore une enfant, tandis qu’elle commençait d’être une jeune
fille) qui probablement venait de rentrer, en face de moi, à quelques
centimètres de moi, dans cette chambre où son père avait reçu le mien et dont
elle avait fait son petit salon à elle. La fenêtre était entr’ouverte, la lampe
était allumée, je voyais tous ses mouvements sans qu’elle me vît, mais en m’en
allant j’aurais fait craquer les buissons, elle m’aurait entendu et elle aurait
pu croire que je m’étais caché là pour l’épier.

Elle était en grand deuil, car son père était mort depuis peu. Nous n’étions pas
allés la voir, ma mère ne l’avait pas voulu à cause d’une vertu qui chez elle
limitait seule les effets de la bonté: la pudeur; mais elle la plaignait
profondément. Ma mère se rappelant la triste fin de vie de M. Vinteuil, tout
absorbée d’abord par les soins de mère et de bonne d’enfant qu’il donnait à sa
fille, puis par les souffrances que celle-ci lui avait causées; elle revoyait le
visage torturé qu’avait eu le vieillard tous les derniers temps; elle savait
qu’il avait renoncé à jamais à achever de transcrire au net toute son œuvre des
dernières années, pauvres morceaux d’un vieux professeur de piano, d’un ancien
organiste de village dont nous imaginions bien qu’ils n’avaient guère de valeur
en eux-mêmes, mais que nous ne méprisions pas parce qu’ils en avaient tant pour
lui dont ils avaient été la raison de vivre avant qu’il les sacrifiât à sa
fille, et qui pour la plupart pas même notés, conservés seulement dans sa
mémoire, quelques-uns inscrits sur des feuillets épars, illisibles, resteraient
inconnus; ma mère pensait à cet autre renoncement plus cruel encore auquel M.
Vinteuil avait été contraint, le renoncement à un avenir de bonheur honnête et
respecté pour sa fille; quand elle évoquait toute cette détresse suprême de
l’ancien maître de piano de mes tantes, elle éprouvait un véritable chagrin et
songeait avec effroi à celui autrement amer que devait éprouver Mlle Vinteuil
tout mêlé du remords d’avoir à peu près tué son père. «Pauvre M. Vinteuil,
disait ma mère, il a vécu et il est mort pour sa fille, sans avoir reçu son
salaire. Le recevra-t-il après sa mort et sous quelle forme? Il ne pourrait lui
venir que d’elle.»

Au fond du salon de Mlle Vinteuil, sur la cheminée était posé un petit portrait
de son père que vivement elle alla chercher au moment où retentit le roulement
d’une voiture qui venait de la route, puis elle se jeta sur un canapé, et tira
près d’elle une petite table sur laquelle elle plaça le portrait, comme M.
Vinteuil autrefois avait mis à côté de lui le morceau qu’il avait le désir de
jouer à mes parents. Bientôt son amie entra. Mlle Vinteuil l’accueillit sans se
lever, ses deux mains derrière la tête et se recula sur le bord opposé du sofa
comme pour lui faire une place. Mais aussitôt elle sentit qu’elle semblait ainsi
lui imposer une attitude qui lui était peut-être importune. Elle pensa que son
amie aimerait peut-être mieux être loin d’elle sur une chaise, elle se trouva
indiscrète, la délicatesse de son cœur s’en alarma; reprenant toute la place sur
le sofa elle ferma les yeux et se mit à bâiller pour indiquer que l’envie de
dormir était la seule raison pour laquelle elle s’était ainsi étendue. Malgré la
familiarité rude et dominatrice qu’elle avait avec sa camarade, je reconnaissais
les gestes obséquieux et réticents, les brusques scrupules de son père. Bientôt
elle se leva, feignit de vouloir fermer les volets et de n’y pas réussir.

—«Laisse donc tout ouvert, j’ai chaud,» dit son amie.

—«Mais c’est assommant, on nous verra», répondit Mlle Vinteuil.

Mais elle devina sans doute que son amie penserait qu’elle n’avait dit ces mots
que pour la provoquer à lui répondre par certains autres qu’elle avait en effet
le désir d’entendre, mais que par discrétion elle voulait lui laisser
l’initiative de prononcer. Aussi son regard que je ne pouvais distinguer, dut-il
prendre l’expression qui plaisait tant à ma grand’mère, quand elle ajouta
vivement:

—«Quand je dis nous voir, je veux dire nous voir lire, c’est assommant, quelque
chose insignifiante qu’on fasse, de penser que des yeux vous voient.»

Par une générosité instinctive et une politesse involontaire elle taisait les
mots prémédités qu’elle avait jugés indispensables à la pleine réalisation de
son désir. Et à tous moments au fond d’elle-même une vierge timide et suppliante
implorait et faisait reculer un soudard fruste et vainqueur.

—«Oui, c’est probable qu’on nous regarde à cette heure-ci, dans cette campagne
fréquentée, dit ironiquement son amie. Et puis quoi? Ajouta-t-elle (en croyant
devoir accompagner d’un clignement d’yeux malicieux et tendre, ces mots qu’elle
récita par bonté, comme un texte, qu’elle savait être agréable à Mlle Vinteuil,
d’un ton qu’elle s’efforçait de rendre cynique), quand même on nous verrait ce
n’en est que meilleur.»

Mlle Vinteuil frémit et se leva. Son cœur scrupuleux et sensible ignorait
quelles paroles devaient spontanément venir s’adapter à la scène que ses sens
réclamaient. Elle cherchait le plus loin qu’elle pouvait de sa vraie nature
morale, à trouver le langage propre à la fille vicieuse qu’elle désirait d’être,
mais les mots qu’elle pensait que celle-ci eût prononcés sincèrement lui
paraissaient faux dans sa bouche. Et le peu qu’elle s’en permettait était dit
sur un ton guindé où ses habitudes de timidité paralysaient ses velléités
d’audace, et s’entremêlait de: «tu n’as pas froid, tu n’as pas trop chaud, tu
n’as pas envie d’être seule et de lire?»

—«Mademoiselle me semble avoir des pensées bien lubriques, ce soir», finit-elle
par dire, répétant sans doute une phrase qu’elle avait entendue autrefois dans
la bouche de son amie.

Dans l’échancrure de son corsage de crêpe Mlle Vinteuil sentit que son amie
piquait un baiser, elle poussa un petit cri, s’échappa, et elles se
poursuivirent en sautant, faisant voleter leurs larges manches comme des ailes
et gloussant et piaillant comme des oiseaux amoureux. Puis Mlle Vinteuil finit
par tomber sur le canapé, recouverte par le corps de son amie. Mais celle-ci
tournait le dos à la petite table sur laquelle était placé le portrait de
l’ancien professeur de piano. Mlle Vinteuil comprit que son amie ne le verrait
pas si elle n’attirait pas sur lui son attention, et elle lui dit, comme si elle
venait seulement de le remarquer:

—«Oh! ce portrait de mon père qui nous regarde, je ne sais pas qui a pu le
mettre là, j’ai pourtant dit vingt fois que ce n’était pas sa place.»

Je me souvins que c’étaient les mots que M. Vinteuil avait dits à mon père à
propos du morceau de musique. Ce portrait leur servait sans doute habituellement
pour des profanations rituelles, car son amie lui répondit par ces paroles qui
devaient faire partie de ses réponses liturgiques:

—«Mais laisse-le donc où il est, il n’est plus là pour nous embêter. Crois-tu
qu’il pleurnicherait, qu’il voudrait te mettre ton manteau, s’il te voyait là,
la fenêtre ouverte, le vilain singe.»

Mlle Vinteuil répondit par des paroles de doux reproche: «Voyons, voyons», qui
prouvaient la bonté de sa nature, non qu’elles fussent dictées par l’indignation
que cette façon de parler de son père eût pu lui causer (évidemment c’était là
un sentiment qu’elle s’était habituée, à l’aide de quels sophismes? à faire
taire en elle dans ces minutes-là), mais parce qu’elles étaient comme un frein
que pour ne pas se montrer égoïste elle mettait elle-même au plaisir que son
amie cherchait à lui procurer. Et puis cette modération souriante en répondant à
ces blasphèmes, ce reproche hypocrite et tendre, paraissaient peut-être à sa
nature franche et bonne, une forme particulièrement infâme, une forme doucereuse
de cette scélératesse qu’elle cherchait à s’assimiler. Mais elle ne put résister
à l’attrait du plaisir qu’elle éprouverait à être traitée avec douceur par une
personne si implacable envers un mort sans défense; elle sauta sur les genoux de
son amie, et lui tendit chastement son front à baiser comme elle aurait pu faire
si elle avait été sa fille, sentant avec délices qu’elles allaient ainsi toutes
deux au bout de la cruauté en ravissant à M. Vinteuil, jusque dans le tombeau,
sa paternité. Son amie lui prit la tête entre ses mains et lui déposa un baiser
sur le front avec cette docilité que lui rendait facile la grande affection
qu’elle avait pour Mlle Vinteuil et le désir de mettre quelque distraction dans
la vie si triste maintenant de l’orpheline.

—«Sais-tu ce que j’ai envie de lui faire à cette vieille horreur?» dit-elle en
prenant le portrait.

Et elle murmura à l’oreille de Mlle Vinteuil quelque chose que je ne pus
entendre.

—«Oh! tu n’oserais pas.»

—«Je n’oserais pas cracher dessus? sur ça?» dit l’amie avec une brutalité
voulue.

Je n’en entendis pas davantage, car Mlle Vinteuil, d’un air las, gauche,
affairé, honnête et triste, vint fermer les volets et la fenêtre, mais je savais
maintenant, pour toutes les souffrances que pendant sa vie M. Vinteuil avait
supportées à cause de sa fille, ce qu’après la mort il avait reçu d’elle en
salaire.

Et pourtant j’ai pensé depuis que si M. Vinteuil avait pu assister à cette
scène, il n’eût peut-être pas encore perdu sa foi dans le bon cœur de sa fille,
et peut-être même n’eût-il pas eu en cela tout à fait tort. Certes, dans les
habitudes de Mlle Vinteuil l’apparence du mal était si entière qu’on aurait eu
de la peine à la rencontrer réalisée à ce degré de perfection ailleurs que chez
une sadique; c’est à la lumière de la rampe des théâtres du boulevard plutôt que
sous la lampe d’une maison de campagne véritable qu’on peut voir une fille faire
cracher une amie sur le portrait d’un père qui n’a vécu que pour elle; et il n’y
a guère que le sadisme qui donne un fondement dans la vie à l’esthétique du
mélodrame. Dans la réalité, en dehors des cas de sadisme, une fille aurait
peut-être des manquements aussi cruels que ceux de Mlle Vinteuil envers la
mémoire et les volontés de son père mort, mais elle ne les résumerait pas
expressément en un acte d’un symbolisme aussi rudimentaire et aussi naïf; ce que
sa conduite aurait de criminel serait plus voilé aux yeux des autres et même à
ses yeux à elle qui ferait le mal sans se l’avouer. Mais, au-delà de
l’apparence, dans le cœur de Mlle Vinteuil, le mal, au début du moins, ne fut
sans doute pas sans mélange. Une sadique comme elle est l’artiste du mal, ce
qu’une créature entièrement mauvaise ne pourrait être car le mal ne lui serait
pas extérieur, il lui semblerait tout naturel, ne se distinguerait même pas
d’elle; et la vertu, la mémoire des morts, la tendresse filiale, comme elle n’en
aurait pas le culte, elle ne trouverait pas un plaisir sacrilège à les profaner.
Les sadiques de l’espèce de Mlle Vinteuil sont des êtres si purement
sentimentaux, si naturellement vertueux que même le plaisir sensuel leur paraît
quelque chose de mauvais, le privilège des méchants. Et quand ils se concèdent à
eux-mêmes de s’y livrer un moment, c’est dans la peau des méchants qu’ils
tâchent d’entrer et de faire entrer leur complice, de façon à avoir eu un moment
l’illusion de s’être évadés de leur âme scrupuleuse et tendre, dans le monde
inhumain du plaisir. Et je comprenais combien elle l’eût désiré en voyant
combien il lui était impossible d’y réussir. Au moment où elle se voulait si
différente de son père, ce qu’elle me rappelait c’était les façons de penser, de
dire, du vieux professeur de piano. Bien plus que sa photographie, ce qu’elle
profanait, ce qu’elle faisait servir à ses plaisirs mais qui restait entre eux
et elle et l’empêchait de les goûter directement, c’était la ressemblance de son
visage, les yeux bleus de sa mère à lui qu’il lui avait transmis comme un bijou
de famille, ces gestes d’amabilité qui interposaient entre le vice de Mlle
Vinteuil et elle une phraséologie, une mentalité qui n’était pas faite pour lui
et l’empêchait de le connaître comme quelque chose de très différent des
nombreux devoirs de politesse auxquels elle se consacrait d’habitude. Ce n’est
pas le mal qui lui donnait l’idée du plaisir, qui lui semblait agréable; c’est
le plaisir qui lui semblait malin. Et comme chaque fois qu’elle s’y adonnait il
s’accompagnait pour elle de ces pensées mauvaises qui le reste du temps étaient
absentes de son âme vertueuse, elle finissait par trouver au plaisir quelque
chose de diabolique, par l’identifier au Mal. Peut-être Mlle Vinteuil
sentait-elle que son amie n’était pas foncièrement mauvaise, et qu’elle n’était
pas sincère au moment où elle lui tenait ces propos blasphématoires. Du moins
avait-elle le plaisir d’embrasser sur son visage, des sourires, des regards,
feints peut-être, mais analogues dans leur expression vicieuse et basse à ceux
qu’aurait eus non un être de bonté et de souffrance, mais un être de cruauté et
de plaisir. Elle pouvait s’imaginer un instant qu’elle jouait vraiment les jeux
qu’eût joués avec une complice aussi dénaturée, une fille qui aurait ressenti en
effet ces sentiments barbares à l’égard de la mémoire de son père. Peut-être
n’eût-elle pas pensé que le mal fût un état si rare, si extraordinaire, si
dépaysant, où il était si reposant d’émigrer, si elle avait su discerner en elle
comme en tout le monde, cette indifférence aux souffrances qu’on cause et qui,
quelques autres noms qu’on lui donne, est la forme terrible et permanente de la
cruauté.

S’il était assez simple d’aller du côté de Méséglise, c’était une autre affaire
d’aller du côté de Guermantes, car la promenade était longue et l’on voulait
être sûr du temps qu’il ferait. Quand on semblait entrer dans une série de beaux
jours; quand Françoise désespérée qu’il ne tombât pas une goutte d’eau pour les
«pauvres récoltes», et ne voyant que de rares nuages blancs nageant à la surface
calme et bleue du ciel s’écriait en gémissant: «Ne dirait-on pas qu’on voit ni
plus ni moins des chiens de mer qui jouent en montrant là-haut leurs museaux?
Ah! ils pensent bien à faire pleuvoir pour les pauvres laboureurs! Et puis quand
les blés seront poussés, alors la pluie se mettra à tomber tout à petit patapon,
sans discontinuer, sans plus savoir sur quoi elle tombe que si c’était sur la
mer»; quand mon père avait reçu invariablement les mêmes réponses favorables du
jardinier et du baromètre, alors on disait au dîner: «Demain s’il fait le même
temps, nous irons du côté de Guermantes.» On partait tout de suite après
déjeuner par la petite porte du jardin et on tombait dans la rue des Perchamps,
étroite et formant un angle aigu, remplie de graminées au milieu desquelles deux
ou trois guêpes passaient la journée à herboriser, aussi bizarre que son nom
d’où me semblaient dériver ses particularités curieuses et sa personnalité
revêche, et qu’on chercherait en vain dans le Combray d’aujourd’hui où sur son
tracé ancien s’élève l’école. Mais ma rêverie (semblable à ces architectes
élèves de Viollet-le-Duc, qui, croyant retrouver sous un jubé Renaissance et un
autel du XVIIe siècle les traces d’un chœur roman, remettent tout l’édifice dans
l’état où il devait être au XIIe siècle) ne laisse pas une pierre du bâtiment
nouveau, reperce et «restitue» la rue des Perchamps. Elle a d’ailleurs pour ces
reconstitutions, des données plus précises que n’en ont généralement les
restaurateurs: quelques images conservées par ma mémoire, les dernières
peut-être qui existent encore actuellement, et destinées à être bientôt
anéanties, de ce qu’était le Combray du temps de mon enfance; et parce que c’est
lui-même qui les a tracées en moi avant de disparaître, émouvantes — si on peut
comparer un obscur portrait à ces effigies glorieuses dont ma grand’mère aimait
à me donner des reproductions — comme ces gravures anciennes de la Cène ou ce
tableau de Gentile Bellini dans lesquels l’on voit en un état qui n’existe plus
aujourd’hui le chef-d’œuvre de Vinci et le portail de Saint-Marc.

On passait, rue de l’Oiseau, devant la vieille hôtellerie de l’Oiseau flesché
dans la grande cour de laquelle entrèrent quelquefois au XVIIe siècle les
carrosses des duchesses de Montpensier, de Guermantes et de Montmorency quand
elles avaient à venir à Combray pour quelque contestation avec leurs fermiers,
pour une question d’hommage. On gagnait le mail entre les arbres duquel
apparaissait le clocher de Saint-Hilaire. Et j’aurais voulu pouvoir m’asseoir là
et rester toute la journée à lire en écoutant les cloches; car il faisait si
beau et si tranquille que, quand sonnait l’heure, on aurait dit non qu’elle
rompait le calme du jour mais qu’elle le débarrassait de ce qu’il contenait et
que le clocher avec l’exactitude indolente et soigneuse d’une personne qui n’a
rien d’autre à faire, venait seulement — pour exprimer et laisser tomber les
quelques gouttes d’or que la chaleur y avait lentement et naturellement amassées
— de presser, au moment voulu, la plénitude du silence.

Le plus grand charme du côté de Guermantes, c’est qu’on y avait presque tout le
temps à côté de soi le cours de la Vivonne. On la traversait une première fois,
dix minutes après avoir quitté la maison, sur une passerelle dite le Pont-Vieux.
Dès le lendemain de notre arrivée, le jour de Pâques, après le sermon s’il
faisait beau temps, je courais jusque-là, voir dans ce désordre d’un matin de
grande fête où quelques préparatifs somptueux font paraître plus sordides les
ustensiles de ménage qui traînent encore, la rivière qui se promenait déjà en
bleu-ciel entre les terres encore noires et nues, accompagnée seulement d’une
bande de coucous arrivés trop tôt et de primevères en avance, cependant que çà
et là une violette au bec bleu laissait fléchir sa tige sous le poids de la
goutte d’odeur qu’elle tenait dans son cornet. Le Pont-Vieux débouchait dans un
sentier de halage qui à cet endroit se tapissait l’été du feuillage bleu d’un
noisetier sous lequel un pêcheur en chapeau de paille avait pris racine. A
Combray où je savais quelle individualité de maréchal ferrant ou de garçon
épicier était dissimulée sous l’uniforme du suisse ou le surplis de l’enfant de
chœur, ce pêcheur est la seule personne dont je n’aie jamais découvert
l’identité. Il devait connaître mes parents, car il soulevait son chapeau quand
nous passions; je voulais alors demander son nom, mais on me faisait signe de me
taire pour ne pas effrayer le poisson. Nous nous engagions dans le sentier de
halage qui dominait le courant d’un talus de plusieurs pieds; de l’autre côté la
rive était basse, étendue en vastes prés jusqu’au village et jusqu’à la gare qui
en était distante. Ils étaient semés des restes, à demi enfouis dans l’herbe, du
château des anciens comtes de Combray qui au Moyen âge avait de ce côté le cours
de la Vivonne comme défense contre les attaques des sires de Guermantes et des
abbés de Martinville. Ce n’étaient plus que quelques fragments de tours bossuant
la prairie, à peine apparents, quelques créneaux d’où jadis l’arbalétrier
lançait des pierres, d’où le guetteur surveillait Novepont, Clairefontaine,
Martinville-le-Sec, Bailleau-l’Exempt, toutes terres vassales de Guermantes
entre lesquelles Combray était enclavé, aujourd’hui au ras de l’herbe, dominés
par les enfants de l’école des frères qui venaient là apprendre leurs leçons ou
jouer aux récréations; — passé presque descendu dans la terre, couché au bord de
l’eau comme un promeneur qui prend le frais, mais me donnant fort à songer, me
faisant ajouter dans le nom de Combray à la petite ville d’aujourd’hui une cité
très différente, retenant mes pensées par son visage incompréhensible et
d’autrefois qu’il cachait à demi sous les boutons d’or. Ils étaient fort
nombreux à cet endroit qu’ils avaient choisi pour leurs jeux sur l’herbe,
isolés, par couples, par troupes, jaunes comme un jaune d’oeuf, brillants
d’autant plus, me semblait-il, que ne pouvant dériver vers aucune velléité de
dégustation le plaisir que leur vue me causait, je l’accumulais dans leur
surface dorée, jusqu’à ce qu’il devînt assez puissant pour produire de l’inutile
beauté; et cela dès ma plus petite enfance, quand du sentier de halage je
tendais les bras vers eux sans pouvoir épeler complètement leur joli nom de
Princes de contes de fées français, venus peut-être il y a bien des siècles
d’Asie mais apatriés pour toujours au village, contents du modeste horizon,
aimant le soleil et le bord de l’eau, fidèles à la petite vue de la gare,
gardant encore pourtant comme certaines de nos vieilles toiles peintes, dans
leur simplicité populaire, un poétique éclat d’orient.

Je m’amusais à regarder les carafes que les gamins mettaient dans la Vivonne
pour prendre les petits poissons, et qui, remplies par la rivière, où elles sont
à leur tour encloses, à la fois «contenant» aux flancs transparents comme une
eau durcie, et «contenu» plongé dans un plus grand contenant de cristal liquide
et courant, évoquaient l’image de la fraîcheur d’une façon plus délicieuse et
plus irritante qu’elles n’eussent fait sur une table servie, en ne la montrant
qu’en fuite dans cette allitération perpétuelle entre l’eau sans consistance où
les mains ne pouvaient la capter et le verre sans fluidité où le palais ne
pourrait en jouir. Je me promettais de venir là plus tard avec des lignes;
j’obtenais qu’on tirât un peu de pain des provisions du goûter; j’en jetais dans
la Vivonne des boulettes qui semblaient suffire pour y provoquer un phénomène de
sursaturation, car l’eau se solidifiait aussitôt autour d’elles en grappes
ovoïdes de têtards inanitiés qu’elle tenait sans doute jusque-là en dissolution,
invisibles, tout près d’être en voie de cristallisation.

Bientôt le cours de la Vivonne s’obstrue de plantes d’eau. Il y en a d’abord
d’isolées comme tel nénufar à qui le courant au travers duquel il était placé
d’une façon malheureuse laissait si peu de repos que comme un bac actionné
mécaniquement il n’abordait une rive que pour retourner à celle d’où il était
venu, refaisant éternellement la double traversée. Poussé vers la rive, son
pédoncule se dépliait, s’allongeait, filait, atteignait l’extrême limite de sa
tension jusqu’au bord où le courant le reprenait, le vert cordage se repliait
sur lui-même et ramenait la pauvre plante à ce qu’on peut d’autant mieux appeler
son point de départ qu’elle n’y restait pas une seconde sans en repartir par une
répétition de la même manœuvre. Je la retrouvais de promenade en promenade,
toujours dans la même situation, faisant penser à certains neurasthéniques au
nombre desquels mon grand-père comptait ma tante Léonie, qui nous offrent sans
changement au cours des années le spectacle des habitudes bizarres qu’ils se
croient chaque fois à la veille de secouer et qu’ils gardent toujours; pris dans
l’engrenage de leurs malaises et de leurs manies, les efforts dans lesquels ils
se débattent inutilement pour en sortir ne font qu’assurer le fonctionnement et
faire jouer le déclic de leur diététique étrange, inéluctable et funeste. Tel
était ce nénufar, pareil aussi à quelqu’un de ces malheureux dont le tourment
singulier, qui se répète indéfiniment durant l’éternité, excitait la curiosité
de Dante et dont il se serait fait raconter plus longuement les particularités
et la cause par le supplicié lui-même, si Virgile, s’éloignant à grands pas, ne
l’avait forcé à le rattraper au plus vite, comme moi mes parents.

Mais plus loin le courant se ralentit, il traverse une propriété dont l’accès
était ouvert au public par celui à qui elle appartenait et qui s’y était complu
à des travaux d’horticulture aquatique, faisant fleurir, dans les petits étangs
que forme la Vivonne, de véritables jardins de nymphéas. Comme les rives étaient
à cet endroit très boisées, les grandes ombres des arbres donnaient à l’eau un
fond qui était habituellement d’un vert sombre mais que parfois, quand nous
rentrions par certains soirs rassérénés d’après-midi orageux, j’ai vu d’un bleu
clair et cru, tirant sur le violet, d’apparence cloisonnée et de goût japonais.
Çà et là, à la surface, rougissait comme une fraise une fleur de nymphéa au cœur
écarlate, blanc sur les bords. Plus loin, les fleurs plus nombreuses étaient
plus pâles, moins lisses, plus grenues, plus plissées, et disposées par le
hasard en enroulements si gracieux qu’on croyait voir flotter à la dérive, comme
après l’effeuillement mélancolique d’une fête galante, des roses mousseuses en
guirlandes dénouées. Ailleurs un coin semblait réservé aux espèces communes qui
montraient le blanc et rose proprets de la julienne, lavés comme de la
porcelaine avec un soin domestique, tandis qu’un peu plus loin, pressées les
unes contre les autres en une véritable plate-bande flottante, on eût dit des
pensées des jardins qui étaient venues poser comme des papillons leur ailes
bleuâtres et glacées, sur l’obliquité transparente de ce parterre d’eau; de ce
parterre céleste aussi: car il donnait aux fleurs un sol d’une couleur plus
précieuse, plus émouvante que la couleur des fleurs elles-mêmes; et, soit que
pendant l’après-midi il fît étinceler sous les nymphéas le kaléidoscope d’un
bonheur attentif, silencieux et mobile, ou qu’il s’emplît vers le soir, comme
quelque port lointain, du rose et de la rêverie du couchant, changeant sans
cesse pour rester toujours en accord, autour des corolles de teintes plus fixes,
avec ce qu’il y a de plus profond, de plus fugitif, de plus mystérieux — avec ce
qu’il y a d’infini — dans l’heure, il semblait les avoir fait fleurir en plein
ciel.

Au sortir de ce parc, la Vivonne redevient courante. Que de fois j’ai vu, j’ai
désiré imiter quand je serais libre de vivre à ma guise, un rameur, qui, ayant
lâché l’aviron, s’était couché à plat sur le dos, la tête en bas, au fond de sa
barque, et la laissant flotter à la dérive, ne pouvant voir que le ciel qui
filait lentement au-dessus de lui, portait sur son visage l’avant-goût du
bonheur et de la paix.

Nous nous asseyions entre les iris au bord de l’eau. Dans le ciel férié, flânait
longuement un nuage oisif. Par moments oppressée par l’ennui, une carpe se
dressait hors de l’eau dans une aspiration anxieuse. C’était l’heure du goûter.
Avant de repartir nous restions longtemps à manger des fruits, du pain et du
chocolat, sur l’herbe où parvenaient jusqu’à nous, horizontaux, affaiblis, mais
denses et métalliques encore, des sons de la cloche de Saint-Hilaire qui ne
s’étaient pas mélangés à l’air qu’ils traversaient depuis si longtemps, et
côtelés par la palpitation successive de toutes leurs lignes sonores, vibraient
en rasant les fleurs, à nos pieds.

Parfois, au bord de l’eau entourée de bois, nous rencontrions une maison dite de
plaisance, isolée, perdue, qui ne voyait rien, du monde, que la rivière qui
baignait ses pieds. Une jeune femme dont le visage pensif et les voiles élégants
n’étaient pas de ce pays et qui sans doute était venue, selon l’expression
populaire «s’enterrer» là, goûter le plaisir amer de sentir que son nom, le nom
surtout de celui dont elle n’avait pu garder le cœur, y était inconnu,
s’encadrait dans la fenêtre qui ne lui laissait pas regarder plus loin que la
barque amarrée près de la porte. Elle levait distraitement les yeux en entendant
derrière les arbres de la rive la voix des passants dont avant qu’elle eût
aperçu leur visage, elle pouvait être certaine que jamais ils n’avaient connu,
ni ne connaîtraient l’infidèle, que rien dans leur passé ne gardait sa marque,
que rien dans leur avenir n’aurait l’occasion de la recevoir. On sentait que,
dans son renoncement, elle avait volontairement quitté des lieux où elle aurait
pu du moins apercevoir celui qu’elle aimait, pour ceux-ci qui ne l’avaient
jamais vu. Et je la regardais, revenant de quelque promenade sur un chemin où
elle savait qu’il ne passerait pas, ôter de ses mains résignées de longs gants
d’une grâce inutile.

Jamais dans la promenade du côté de Guermantes nous ne pûmes remonter jusqu’aux
sources de la Vivonne, auxquelles j’avais souvent pensé et qui avaient pour moi
une existence si abstraite, si idéale, que j’avais été aussi surpris quand on
m’avait dit qu’elles se trouvaient dans le département, à une certaine distance
kilométrique de Combray, que le jour où j’avais appris qu’il y avait un autre
point précis de la terre où s’ouvrait, dans l’antiquité, l’entrée des Enfers.
Jamais non plus nous ne pûmes pousser jusqu’au terme que j’eusse tant souhaité
d’atteindre, jusqu’à Guermantes. Je savais que là résidaient des châtelains, le
duc et la duchesse de Guermantes, je savais qu’ils étaient des personnages réels
et actuellement existants, mais chaque fois que je pensais à eux, je me les
représentais tantôt en tapisserie, comme était la comtesse de Guermantes, dans
le «Couronnement d’Esther» de notre église, tantôt de nuances changeantes comme
était Gilbert le Mauvais dans le vitrail où il passait du vert chou au bleu
prune selon que j’étais encore à prendre de l’eau bénite ou que j’arrivais à nos
chaises, tantôt tout à fait impalpables comme l’image de Geneviève de Brabant,
ancêtre de la famille de Guermantes, que la lanterne magique promenait sur les
rideaux de ma chambre ou faisait monter au plafond — enfin toujours enveloppés
du mystère des temps mérovingiens et baignant comme dans un coucher de soleil
dans la lumière orangée qui émane de cette syllabe: «antes». Mais si malgré cela
ils étaient pour moi, en tant que duc et duchesse, des êtres réels, bien
qu’étranges, en revanche leur personne ducale se distendait démesurément,
s’immatérialisait, pour pouvoir contenir en elle ce Guermantes dont ils étaient
duc et duchesse, tout ce «côté de Guermantes» ensoleillé, le cours de la
Vivonne, ses nymphéas et ses grands arbres, et tant de beaux après-midi. Et je
savais qu’ils ne portaient pas seulement le titre de duc et de duchesse de
Guermantes, mais que depuis le XIVe siècle où, après avoir inutilement essayé de
vaincre leurs anciens seigneurs ils s’étaient alliés à eux par des mariages, ils
étaient comtes de Combray, les premiers des citoyens de Combray par conséquent
et pourtant les seuls qui n’y habitassent pas. Comtes de Combray, possédant
Combray au milieu de leur nom, de leur personne, et sans doute ayant
effectivement en eux cette étrange et pieuse tristesse qui était spéciale à
Combray; propriétaires de la ville, mais non d’une maison particulière,
demeurant sans doute dehors, dans la rue, entre ciel et terre, comme ce Gilbert
de Guermantes, dont je ne voyais aux vitraux de l’abside de Saint-Hilaire que
l’envers de laque noire, si je levais la tête quand j’allais chercher du sel
chez Camus.

Puis il arriva que sur le côté de Guermantes je passai parfois devant de petits
enclos humides où montaient des grappes de fleurs sombres. Je m’arrêtais,
croyant acquérir une notion précieuse, car il me semblait avoir sous les yeux un
fragment de cette région fluviatile, que je désirais tant connaître depuis que
je l’avais vue décrite par un de mes écrivains préférés. Et ce fut avec elle,
avec son sol imaginaire traversé de cours d’eau bouillonnants, que Guermantes,
changeant d’aspect dans ma pensée, s’identifia, quand j’eus entendu le docteur
Percepied nous parler des fleurs et des belles eaux vives qu’il y avait dans le
parc du château. Je rêvais que Mme de Guermantes m’y faisait venir, éprise pour
moi d’un soudain caprice; tout le jour elle y pêchait la truite avec moi. Et le
soir me tenant par la main, en passant devant les petits jardins de ses vassaux,
elle me montrait le long des murs bas, les fleurs qui y appuient leurs
quenouilles violettes et rouges et m’apprenait leurs noms. Elle me faisait lui
dire le sujet des poèmes que j’avais l’intention de composer. Et ces rêves
m’avertissaient que puisque je voulais un jour être un écrivain, il était temps
de savoir ce que je comptais écrire. Mais dès que je me le demandais, tâchant de
trouver un sujet où je pusse faire tenir une signification philosophique
infinie, mon esprit s’arrêtait de fonctionner, je ne voyais plus que le vide en
face de mon attention, je sentais que je n’avais pas de génie ou peut-être une
maladie cérébrale l’empêchait de naître. Parfois je comptais sur mon père pour
arranger cela. Il était si puissant, si en faveur auprès des gens en place qu’il
arrivait à nous faire transgresser les lois que Françoise m’avait appris à
considérer comme plus inéluctables que celles de la vie et de la mort, à faire
retarder d’un an pour notre maison, seule de tout le quartier, les travaux de
«ravalement», à obtenir du ministre pour le fils de Mme Sazerat qui voulait
aller aux eaux, l’autorisation qu’il passât le baccalauréat deux mois d’avance,
dans la série des candidats dont le nom commençait par un A au lieu d’attendre
le tour des S. Si j’étais tombé gravement malade, si j’avais été capturé par des
brigands, persuadé que mon père avait trop d’intelligences avec les puissances
suprêmes, de trop irrésistibles lettres de recommandation auprès du bon Dieu,
pour que ma maladie ou ma captivité pussent être autre chose que de vains
simulacres sans danger pour moi, j’aurais attendu avec calme l’heure inévitable
du retour à la bonne réalité, l’heure de la délivrance ou de la guérison;
peut-être cette absence de génie, ce trou noir qui se creusait dans mon esprit
quand je cherchais le sujet de mes écrits futurs, n’était-il aussi qu’une
illusion sans consistance, et cesserait-elle par l’intervention de mon père qui
avait dû convenir avec le Gouvernement et avec la Providence que je serais le
premier écrivain de l’époque. Mais d’autres fois tandis que mes parents
s’impatientaient de me voir rester en arrière et ne pas les suivre, ma vie
actuelle au lieu de me sembler une création artificielle de mon père et qu’il
pouvait modifier à son gré, m’apparaissait au contraire comme comprise dans une
réalité qui n’était pas faite pour moi, contre laquelle il n’y avait pas de
recours, au cœur de laquelle je n’avais pas d’allié, qui ne cachait rien au delà
d’elle-même. Il me semblait alors que j’existais de la même façon que les autres
hommes, que je vieillirais, que je mourrais comme eux, et que parmi eux j’étais
seulement du nombre de ceux qui n’ont pas de dispositions pour écrire. Aussi,
découragé, je renonçais à jamais à la littérature, malgré les encouragements que
m’avait donnés Bloch. Ce sentiment intime, immédiat, que j’avais du néant de ma
pensée, prévalait contre toutes les paroles flatteuses qu’on pouvait me
prodiguer, comme chez un méchant dont chacun vante les bonnes actions, les
remords de sa conscience.

Un jour ma mère me dit: «Puisque tu parles toujours de Mme de Guermantes, comme
le docteur Percepied l’a très bien soignée il y a quatre ans, elle doit venir à
Combray pour assister au mariage de sa fille. Tu pourras l’apercevoir à la
cérémonie.» C’était du reste par le docteur Percepied que j’avais le plus
entendu parler de Mme de Guermantes, et il nous avait même montré le numéro
d’une revue illustrée où elle était représentée dans le costume qu’elle portait
à un bal travesti chez la princesse de Léon.

Tout d’un coup pendant la messe de mariage, un mouvement que fit le suisse en se
déplaçant me permit de voir assise dans une chapelle une dame blonde avec un
grand nez, des yeux bleus et perçants, une cravate bouffante en soie mauve,
lisse, neuve et brillante, et un petit bouton au coin du nez. Et parce que dans
la surface de son visage rouge, comme si elle eût eu très chaud, je distinguais,
diluées et à peine perceptibles, des parcelles d’analogie avec le portrait qu’on
m’avait montré, parce que surtout les traits particuliers que je relevais en
elle, si j’essayais de les énoncer, se formulaient précisément dans les mêmes
termes: un grand nez, des yeux bleus, dont s’était servi le docteur Percepied
quand il avait décrit devant moi la duchesse de Guermantes, je me dis: cette
dame ressemble à Mme de Guermantes; or la chapelle où elle suivait la messe
était celle de Gilbert le Mauvais, sous les plates tombes de laquelle, dorées et
distendues comme des alvéoles de miel, reposaient les anciens comtes de Brabant,
et que je me rappelais être à ce qu’on m’avait dit réservée à la famille de
Guermantes quand quelqu’un de ses membres venait pour une cérémonie à Combray;
il ne pouvait vraisemblablement y avoir qu’une seule femme ressemblant au
portrait de Mme de Guermantes, qui fût ce jour-là, jour où elle devait justement
venir, dans cette chapelle: c’était elle! Ma déception était grande. Elle
provenait de ce que je n’avais jamais pris garde quand je pensais à Mme de
Guermantes, que je me la représentais avec les couleurs d’une tapisserie ou d’un
vitrail, dans un autre siècle, d’une autre matière que le reste des personnes
vivantes. Jamais je ne m’étais avisé qu’elle pouvait avoir une figure rouge, une
cravate mauve comme Mme Sazerat, et l’ovale de ses joues me fit tellement
souvenir de personnes que j’avais vues à la maison que le soupçon m’effleura,
pour se dissiper d’ailleurs aussitôt après, que cette dame en son principe
générateur, en toutes ses molécules, n’était peut-être pas substantiellement la
duchesse de Guermantes, mais que son corps, ignorant du nom qu’on lui
appliquait, appartenait à un certain type féminin, qui comprenait aussi des
femmes de médecins et de commerçants. «C’est cela, ce n’est que cela, Mme de
Guermantes!» disait la mine attentive et étonnée avec laquelle je contemplais
cette image qui naturellement n’avait aucun rapport avec celles qui sous le même
nom de Mme de Guermantes étaient apparues tant de fois dans mes songes, puisque,
elle, elle n’avait pas été comme les autres arbitrairement formée par moi, mais
qu’elle m’avait sauté aux yeux pour la première fois il y a un moment seulement,
dans l’église; qui n’était pas de la même nature, n’était pas colorable à
volonté comme elles qui se laissaient imbiber de la teinte orangée d’une
syllabe, mais était si réelle que tout, jusqu’à ce petit bouton qui s’enflammait
au coin du nez, certifiait son assujettissement aux lois de la vie, comme dans
une apothéose de théâtre, un plissement de la robe de la fée, un tremblement de
son petit doigt, dénoncent la présence matérielle d’une actrice vivante, là où
nous étions incertains si nous n’avions pas devant les yeux une simple
projection lumineuse.

Mais en même temps, sur cette image que le nez proéminent, les yeux perçants,
épinglaient dans ma vision (peut-être parce que c’était eux qui l’avaient
d’abord atteinte, qui y avaient fait la première encoche, au moment où je
n’avais pas encore le temps de songer que la femme qui apparaissait devant moi
pouvait être Mme de Guermantes), sur cette image toute récente, inchangeable,
j’essayais d’appliquer l’idée: «C’est Mme de Guermantes» sans parvenir qu’à la
faire manœuvrer en face de l’image, comme deux disques séparés par un
intervalle. Mais cette Mme de Guermantes à laquelle j’avais si souvent rêvé,
maintenant que je voyais qu’elle existait effectivement en dehors de moi, en
prit plus de puissance encore sur mon imagination qui, un moment paralysée au
contact d’une réalité si différente de ce qu’elle attendait, se mit à réagir et
à me dire: «Glorieux dès avant Charlemagne, les Guermantes avaient le droit de
vie et de mort sur leurs vassaux; la duchesse de Guermantes descend de Geneviève
de Brabant. Elle ne connaît, ni ne consentirait à connaître aucune des personnes
qui sont ici.»

Et —ô merveilleuse indépendance des regards humains, retenus au visage par une
corde si lâche, si longue, si extensible qu’ils peuvent se promener seuls loin
de lui — pendant que Mme de Guermantes était assise dans la chapelle au-dessus
des tombes de ses morts, ses regards flânaient çà et là, montaient le long des
piliers, s’arrêtaient même sur moi comme un rayon de soleil errant dans la nef,
mais un rayon de soleil qui, au moment où je reçus sa caresse, me sembla
conscient. Quant à Mme de Guermantes elle-même, comme elle restait immobile,
assise comme une mère qui semble ne pas voir les audaces espiègles et les
entreprises indiscrètes de ses enfants qui jouent et interpellent des personnes
qu’elle ne connaît pas, il me fût impossible de savoir si elle approuvait ou
blâmait dans le désœuvrement de son âme, le vagabondage de ses regards.

Je trouvais important qu’elle ne partît pas avant que j’eusse pu la regarder
suffisamment, car je me rappelais que depuis des années je considérais sa vue
comme éminemment désirable, et je ne détachais pas mes yeux d’elle, comme si
chacun de mes regards eût pu matériellement emporter et mettre en réserve en moi
le souvenir du nez proéminent, des joues rouges, de toutes ces particularités
qui me semblaient autant de renseignements précieux, authentiques et singuliers
sur son visage. Maintenant que me le faisaient trouver beau toutes les pensées
que j’y rapportais — et peut-être surtout, forme de l’instinct de conservation
des meilleures parties de nous-mêmes, ce désir qu’on a toujours de ne pas avoir
été déçu — la replaçant (puisque c’était une seule personne qu’elle et cette
duchesse de Guermantes que j’avais évoquée jusque-là) hors du reste de
l’humanité dans laquelle la vue pure et simple de son corps me l’avait fait un
instant confondre, je m’irritais en entendant dire autour de moi: «Elle est
mieux que Mme Sazerat, que Mlle Vinteuil», comme si elle leur eût été
comparable. Et mes regards s’arrêtant à ses cheveux blonds, à ses yeux bleus, à
l’attache de son cou et omettant les traits qui eussent pu me rappeler d’autres
visages, je m’écriais devant ce croquis volontairement incomplet: «Qu’elle est
belle! Quelle noblesse! Comme c’est bien une fière Guermantes, la descendante de
Geneviève de Brabant, que j’ai devant moi!» Et l’attention avec laquelle
j’éclairais son visage l’isolait tellement, qu’aujourd’hui si je repense à cette
cérémonie, il m’est impossible de revoir une seule des personnes qui y
assistaient sauf elle et le suisse qui répondit affirmativement quand je lui
demandai si cette dame était bien Mme de Guermantes. Mais elle, je la revois,
surtout au moment du défilé dans la sacristie qu’éclairait le soleil
intermittent et chaud d’un jour de vent et d’orage, et dans laquelle Mme de
Guermantes se trouvait au milieu de tous ces gens de Combray dont elle ne savait
même pas les noms, mais dont l’infériorité proclamait trop sa suprématie pour
qu’elle ne ressentît pas pour eux une sincère bienveillance et auxquels du reste
elle espérait imposer davantage encore à force de bonne grâce et de simplicité.
Aussi, ne pouvant émettre ces regards volontaires, chargés d’une signification
précise, qu’on adresse à quelqu’un qu’on connaît, mais seulement laisser ses
pensées distraites s’échapper incessamment devant elle en un flot de lumière
bleue qu’elle ne pouvait contenir, elle ne voulait pas qu’il pût gêner, paraître
dédaigner ces petites gens qu’il rencontrait au passage, qu’il atteignait à tous
moments. Je revois encore, au-dessus de sa cravate mauve, soyeuse et gonflée, le
doux étonnement de ses yeux auxquels elle avait ajouté sans oser le destiner à
personne mais pour que tous pussent en prendre leur part un sourire un peu
timide de suzeraine qui a l’air de s’excuser auprès de ses vassaux et de les
aimer. Ce sourire tomba sur moi qui ne la quittais pas des yeux. Alors me
rappelant ce regard qu’elle avait laissé s’arrêter sur moi, pendant la messe,
bleu comme un rayon de soleil qui aurait traversé le vitrail de Gilbert le
Mauvais, je me dis: «Mais sans doute elle fait attention à moi.» Je crus que je
lui plaisais, qu’elle penserait encore à moi quand elle aurait quitté l’église,
qu’à cause de moi elle serait peut-être triste le soir à Guermantes. Et aussitôt
je l’aimai, car s’il peut quelquefois suffire pour que nous aimions une femme
qu’elle nous regarde avec mépris comme j’avais cru qu’avait fait Mlle Swann et
que nous pensions qu’elle ne pourra jamais nous appartenir, quelquefois aussi il
peut suffire qu’elle nous regarde avec bonté comme faisait Mme de Guermantes et
que nous pensions qu’elle pourra nous appartenir. Ses yeux bleuissaient comme
une pervenche impossible à cueillir et que pourtant elle m’eût dédiée; et le
soleil menacé par un nuage, mais dardant encore de toute sa force sur la place
et dans la sacristie, donnait une carnation de géranium aux tapis rouges qu’on y
avait étendus par terre pour la solennité et sur lesquels s’avançait en souriant
Mme de Guermantes, et ajoutait à leur lainage un velouté rose, un épiderme de
lumière, cette sorte de tendresse, de sérieuse douceur dans la pompe et dans la
joie qui caractérisent certaines pages de Lohengrin, certaines peintures de
Carpaccio, et qui font comprendre que Baudelaire ait pu appliquer au son de la
trompette l’épithète de délicieux.

Combien depuis ce jour, dans mes promenades du côté de Guermantes, il me parut
plus affligeant encore qu’auparavant de n’avoir pas de dispositions pour les
lettres, et de devoir renoncer à être jamais un écrivain célèbre. Les regrets
que j’en éprouvais, tandis que je restais seul à rêver un peu à l’écart, me
faisaient tant souffrir, que pour ne plus les ressentir, de lui-même par une
sorte d’inhibition devant la douleur, mon esprit s’arrêtait entièrement de
penser aux vers, aux romans, à un avenir poétique sur lequel mon manque de
talent m’interdisait de compter. Alors, bien en dehors de toutes ces
préoccupations littéraires et ne s’y rattachant en rien, tout d’un coup un toit,
un reflet de soleil sur une pierre, l’odeur d’un chemin me faisaient arrêter par
un plaisir particulier qu’ils me donnaient, et aussi parce qu’ils avaient l’air
de cacher au delà de ce que je voyais, quelque chose qu’ils invitaient à venir
prendre et que malgré mes efforts je n’arrivais pas à découvrir. Comme je
sentais que cela se trouvait en eux, je restais là, immobile, à regarder, à
respirer, à tâcher d’aller avec ma pensée au delà de l’image ou de l’odeur. Et
s’il me fallait rattraper mon grand-père, poursuivre ma route, je cherchais à
les retrouver, en fermant les yeux; je m’attachais à me rappeler exactement la
ligne du toit, la nuance de la pierre qui, sans que je pusse comprendre
pourquoi, m’avaient semblé pleines, prêtes à s’entr’ouvrir, à me livrer ce dont
elles n’étaient qu’un couvercle. Certes ce n’était pas des impressions de ce
genre qui pouvaient me rendre l’espérance que j’avais perdue de pouvoir être un
jour écrivain et poète, car elles étaient toujours liées à un objet particulier
dépourvu de valeur intellectuelle et ne se rapportant à aucune vérité abstraite.
Mais du moins elles me donnaient un plaisir irraisonné, l’illusion d’une sorte
de fécondité et par là me distrayaient de l’ennui, du sentiment de mon
impuissance que j’avais éprouvés chaque fois que j’avais cherché un sujet
philosophique pour une grande œuvre littéraire. Mais le devoir de conscience
était si ardu que m’imposaient ces impressions de forme, de parfum ou de couleur
— de tâcher d’apercevoir ce qui se cachait derrière elles, que je ne tardais pas
à me chercher à moi-même des excuses qui me permissent de me dérober à ces
efforts et de m’épargner cette fatigue. Par bonheur mes parents m’appelaient, je
sentais que je n’avais pas présentement la tranquillité nécessaire pour
poursuivre utilement ma recherche, et qu’il valait mieux n’y plus penser jusqu’à
ce que je fusse rentré, et ne pas me fatiguer d’avance sans résultat. Alors je
ne m’occupais plus de cette chose inconnue qui s’enveloppait d’une forme ou d’un
parfum, bien tranquille puisque je la ramenais à la maison, protégée par le
revêtement d’images sous lesquelles je la trouverais vivante, comme les poissons
que les jours où on m’avait laissé aller à la pêche, je rapportais dans mon
panier couverts par une couche d’herbe qui préservait leur fraîcheur. Une fois à
la maison je songeais à autre chose et ainsi s’entassaient dans mon esprit
(comme dans ma chambre les fleurs que j’avais cueillies dans mes promenades ou
les objets qu’on m’avait donnés), une pierre où jouait un reflet, un toit, un
son de cloche, une odeur de feuilles, bien des images différentes sous
lesquelles il y a longtemps qu’est morte la réalité pressentie que je n’ai pas
eu assez de volonté pour arriver à découvrir. Une fois pourtant — où notre
promenade s’étant prolongée fort au delà de sa durée habituelle, nous avions été
bien heureux de rencontrer à mi-chemin du retour, comme l’après-midi finissait,
le docteur Percepied qui passait en voiture à bride abattue, nous avait reconnus
et fait monter avec lui — j’eus une impression de ce genre et ne l’abandonnai
pas sans un peu l’approfondir. On m’avait fait monter près du cocher, nous
allions comme le vent parce que le docteur avait encore avant de rentrer à
Combray à s’arrêter à Martinville-le-Sec chez un malade à la porte duquel il
avait été convenu que nous l’attendrions. Au tournant d’un chemin j’éprouvai
tout à coup ce plaisir spécial qui ne ressemblait à aucun autre, à apercevoir
les deux clochers de Martinville, sur lesquels donnait le soleil couchant et que
le mouvement de notre voiture et les lacets du chemin avaient l’air de faire
changer de place, puis celui de Vieuxvicq qui, séparé d’eux par une colline et
une vallée, et situé sur un plateau plus élevé dans le lointain, semblait
pourtant tout voisin d’eux.

En constatant, en notant la forme de leur flèche, le déplacement de leurs
lignes, l’ensoleillement de leur surface, je sentais que je n’allais pas au bout
de mon impression, que quelque chose était derrière ce mouvement, derrière cette
clarté, quelque chose qu’ils semblaient contenir et dérober à la fois.

Les clochers paraissaient si éloignés et nous avions l’air de si peu nous
rapprocher d’eux, que je fus étonné quand, quelques instants après, nous nous
arrêtâmes devant l’église de Martinville. Je ne savais pas la raison du plaisir
que j’avais eu à les apercevoir à l’horizon et l’obligation de chercher à
découvrir cette raison me semblait bien pénible; j’avais envie de garder en
réserve dans ma tête ces lignes remuantes au soleil et de n’y plus penser
maintenant. Et il est probable que si je l’avais fait, les deux clochers
seraient allés à jamais rejoindre tant d’arbres, de toits, de parfums, de sons,
que j’avais distingués des autres à cause de ce plaisir obscur qu’ils m’avaient
procuré et que je n’ai jamais approfondi. Je descendis causer avec mes parents
en attendant le docteur. Puis nous repartîmes, je repris ma place sur le siège,
je tournai la tête pour voir encore les clochers qu’un peu plus tard, j’aperçus
une dernière fois au tournant d’un chemin. Le cocher, qui ne semblait pas
disposé à causer, ayant à peine répondu à mes propos, force me fut, faute
d’autre compagnie, de me rabattre sur celle de moi-même et d’essayer de me
rappeler mes clochers. Bientôt leurs lignes et leurs surfaces ensoleillées,
comme si elles avaient été une sorte d’écorce, se déchirèrent, un peu de ce qui
m’était caché en elles m’apparut, j’eus une pensée qui n’existait pas pour moi
l’instant avant, qui se formula en mots dans ma tête, et le plaisir que m’avait
fait tout à l’heure éprouver leur vue s’en trouva tellement accru que, pris
d’une sorte d’ivresse, je ne pus plus penser à autre chose. A ce moment et comme
nous étions déjà loin de Martinville en tournant la tête je les aperçus de
nouveau, tout noirs cette fois, car le soleil était déjà couché. Par moments les
tournants du chemin me les dérobaient, puis ils se montrèrent une dernière fois
et enfin je ne les vis plus.

Sans me dire que ce qui était caché derrière les clochers de Martinville devait
être quelque chose d’analogue à une jolie phrase, puisque c’était sous la forme
de mots qui me faisaient plaisir, que cela m’était apparu, demandant un crayon
et du papier au docteur, je composai malgré les cahots de la voiture, pour
soulager ma conscience et obéir à mon enthousiasme, le petit morceau suivant que
j’ai retrouvé depuis et auquel je n’ai eu à faire subir que peu de changements:

«Seuls, s’élevant du niveau de la plaine et comme perdus en rase campagne,
montaient vers le ciel les deux clochers de Martinville. Bientôt nous en vîmes
trois: venant se placer en face d’eux par une volte hardie, un clocher
retardataire, celui de Vieuxvicq, les avait rejoints. Les minutes passaient,
nous allions vite et pourtant les trois clochers étaient toujours au loin devant
nous, comme trois oiseaux posés sur la plaine, immobiles et qu’on distingue au
soleil. Puis le clocher de Vieuxvicq s’écarta, prit ses distances, et les
clochers de Martinville restèrent seuls, éclairés par la lumière du couchant que
même à cette distance, sur leurs pentes, je voyais jouer et sourire. Nous avions
été si longs à nous rapprocher d’eux, que je pensais au temps qu’il faudrait
encore pour les atteindre quand, tout d’un coup, la voiture ayant tourné, elle
nous déposa à leurs pieds; et ils s’étaient jetés si rudement au-devant d’elle,
qu’on n’eut que le temps d’arrêter pour ne pas se heurter au porche. Nous
poursuivîmes notre route; nous avions déjà quitté Martinville depuis un peu de
temps et le village après nous avoir accompagnés quelques secondes avait
disparu, que restés seuls à l’horizon à nous regarder fuir, ses clochers et
celui de Vieuxvicq agitaient encore en signe d’adieu leurs cimes ensoleillées.
Parfois l’un s’effaçait pour que les deux autres pussent nous apercevoir un
instant encore; mais la route changea de direction, ils virèrent dans la lumière
comme trois pivots d’or et disparurent à mes yeux. Mais, un peu plus tard, comme
nous étions déjà près de Combray, le soleil étant maintenant couché, je les
aperçus une dernière fois de très loin qui n’étaient plus que comme trois fleurs
peintes sur le ciel au-dessus de la ligne basse des champs. Ils me faisaient
penser aussi aux trois jeunes filles d’une légende, abandonnées dans une
solitude où tombait déjà l’obscurité; et tandis que nous nous éloignions au
galop, je les vis timidement chercher leur chemin et après quelques gauches
trébuchements de leurs nobles silhouettes, se serrer les uns contre les autres,
glisser l’un derrière l’autre, ne plus faire sur le ciel encore rose qu’une
seule forme noire, charmante et résignée, et s’effacer dans la nuit.» Je ne
repensai jamais à cette page, mais à ce moment-là, quand, au coin du siège où le
cocher du docteur plaçait habituellement dans un panier les volailles qu’il
avait achetées au marché de Martinville, j’eus fini de l’écrire, je me trouvai
si heureux, je sentais qu’elle m’avait si parfaitement débarrassé de ces
clochers et de ce qu’ils cachaient derrière eux, que, comme si j’avais été
moi-même une poule et si je venais de pondre un oeuf, je me mis à chanter à
tue-tête.

Pendant toute la journée, dans ces promenades, j’avais pu rêver au plaisir que
ce serait d’être l’ami de la duchesse de Guermantes, de pêcher la truite, de me
promener en barque sur la Vivonne, et, avide de bonheur, ne demander en ces
moments-là rien d’autre à la vie que de se composer toujours d’une suite
d’heureux après-midi. Mais quand sur le chemin du retour j’avais aperçu sur la
gauche une ferme, assez distante de deux autres qui étaient au contraire très
rapprochées, et à partir de laquelle pour entrer dans Combray il n’y avait plus
qu’à prendre une allée de chênes bordée d’un côté de prés appartenant chacun à
un petit clos et plantés à intervalles égaux de pommiers qui y portaient, quand
ils étaient éclairés par le soleil couchant, le dessin japonais de leurs ombres,
brusquement mon cœur se mettait à battre, je savais qu’avant une demi-heure nous
serions rentrés, et que, comme c’était de règle les jours où nous étions allés
du côté de Guermantes et où le dîner était servi plus tard, on m’enverrait me
coucher sitôt ma soupe prise, de sorte que ma mère, retenue à table comme s’il y
avait du monde à dîner, ne monterait pas me dire bonsoir dans mon lit. La zone
de tristesse où je venais d’entrer était aussi distincte de la zone, où je
m’élançais avec joie il y avait un moment encore que dans certains ciels une
bande rose est séparée comme par une ligne d’une bande verte ou d’une bande
noire. On voit un oiseau voler dans le rose, il va en atteindre la fin, il
touche presque au noir, puis il y est entré. Les désirs qui tout à l’heure
m’entouraient, d’aller à Guermantes, de voyager, d’être heureux, j’étais
maintenant tellement en dehors d’eux que leur accomplissement ne m’eût fait
aucun plaisir. Comme j’aurais donné tout cela pour pouvoir pleurer toute la nuit
dans les bras de maman! Je frissonnais, je ne détachais pas mes yeux angoissés
du visage de ma mère, qui n’apparaîtrait pas ce soir dans la chambre où je me
voyais déjà par la pensée, j’aurais voulu mourir. Et cet état durerait jusqu’au
lendemain, quand les rayons du matin, appuyant, comme le jardinier, leurs
barreaux au mur revêtu de capucines qui grimpaient jusqu’à ma fenêtre, je
sauterais à bas du lit pour descendre vite au jardin, sans plus me rappeler que
le soir ramènerait jamais l’heure de quitter ma mère. Et de la sorte c’est du
côté de Guermantes que j’ai appris à distinguer ces états qui se succèdent en
moi, pendant certaines périodes, et vont jusqu’à se partager chaque journée,
l’un revenant chasser l’autre, avec la ponctualité de la fièvre; contigus, mais
si extérieurs l’un à l’autre, si dépourvus de moyens de communication entre eux,
que je ne puis plus comprendre, plus même me représenter dans l’un, ce que j’ai
désiré, ou redouté, ou accompli dans l’autre.

Aussi le côté de Méséglise et le côté de Guermantes restent-ils pour moi liés à
bien des petits événements de celle de toutes les diverses vies que nous menons
parallèlement, qui est la plus pleine de péripéties, la plus riche en épisodes,
je veux dire la vie intellectuelle. Sans doute elle progresse en nous
insensiblement et les vérités qui en ont changé pour nous le sens et l’aspect,
qui nous ont ouvert de nouveaux chemins, nous en préparions depuis longtemps la
découverte; mais c’était sans le savoir; et elles ne datent pour nous que du
jour, de la minute où elles nous sont devenues visibles. Les fleurs qui jouaient
alors sur l’herbe, l’eau qui passait au soleil, tout le paysage qui environna
leur apparition continue à accompagner leur souvenir de son visage inconscient
ou distrait; et certes quand ils étaient longuement contemplés par cet humble
passant, par cet enfant qui rêvait — comme l’est un roi, par un mémorialiste
perdu dans la foule — ce coin de nature, ce bout de jardin n’eussent pu penser
que ce serait grâce à lui qu’ils seraient appelés à survivre en leurs
particularités les plus éphémères; et pourtant ce parfum d’aubépine qui butine
le long de la haie où les églantiers le remplaceront bientôt, un bruit de pas
sans écho sur le gravier d’une allée, une bulle formée contre une plante
aquatique par l’eau de la rivière et qui crève aussitôt, mon exaltation les a
portés et a réussi à leur faire traverser tant d’années successives, tandis
qu’alentour les chemins se sont effacés et que sont morts ceux qui les foulèrent
et le souvenir de ceux qui les foulèrent. Parfois ce morceau de paysage amené
ainsi jusqu’à aujourd’hui se détache si isolé de tout, qu’il flotte incertain
dans ma pensée comme une Délos fleurie, sans que je puisse dire de quel pays, de
quel temps — peut-être tout simplement de quel rêve — il vient. Mais c’est
surtout comme à des gisements profonds de mon sol mental, comme aux terrains
résistants sur lesquels je m’appuie encore, que je dois penser au côté de
Méséglise et au côté de Guermantes. C’est parce que je croyais aux choses, aux
êtres, tandis que je les parcourais, que les choses, les êtres qu’ils m’ont fait
connaître, sont les seuls que je prenne encore au sérieux et qui me donnent
encore de la joie. Soit que la foi qui crée soit tarie en moi, soit que la
réalité ne se forme que dans la mémoire, les fleurs qu’on me montre aujourd’hui
pour la première fois ne me semblent pas de vraies fleurs. Le côté de Méséglise
avec ses lilas, ses aubépines, ses bluets, ses coquelicots, ses pommiers, le
côté de Guermantes avec sa rivière à têtards, ses nymphéas et ses boutons d’or,
ont constitué à tout jamais pour moi la figure des pays où j’aimerais vivre, où
j’exige avant tout qu’on puisse aller à la pêche, se promener en canot, voir des
ruines de fortifications gothiques et trouver au milieu des blés, ainsi qu’était
Saint-André-des-Champs, une église monumentale, rustique et dorée comme une
meule; et les bluets, les aubépines, les pommiers qu’il m’arrive quand je voyage
de rencontrer encore dans les champs, parce qu’ils sont situés à la même
profondeur, au niveau de mon passé, sont immédiatement en communication avec mon
cœur. Et pourtant, parce qu’il y a quelque chose d’individuel dans les lieux,
quand me saisit le désir de revoir le côté de Guermantes, on ne le satisferait
pas en me menant au bord d’une rivière où il y aurait d’aussi beaux, de plus
beaux nymphéas que dans la Vivonne, pas plus que le soir en rentrant — à l’heure
où s’éveillait en moi cette angoisse qui plus tard émigre dans l’amour, et peut
devenir à jamais inséparable de lui — je n’aurais souhaité que vînt me dire
bonsoir une mère plus belle et plus intelligente que la mienne. Non; de même que
ce qu’il me fallait pour que je pusse m’endormir heureux, avec cette paix sans
trouble qu’aucune maîtresse n’a pu me donner depuis puisqu’on doute d’elles
encore au moment où on croit en elles, et qu’on ne possède jamais leur cœur
comme je recevais dans un baiser celui de ma mère, tout entier, sans la réserve
d’une arrère-pensée, sans le reliquat d’une intention qui ne fut pas pour moi —
c’est que ce fût elle, c’est qu’elle inclinât vers moi ce visage où il y avait
au-dessous de l’œil quelque chose qui était, paraît-il, un défaut, et que
j’aimais à l’égal du reste, de même ce que je veux revoir, c’est le côté de
Guermantes que j’ai connu, avec la ferme qui est peu éloignée des deux suivantes
serrées l’une contre l’autre, à l’entrée de l’allée des chênes; ce sont ces
prairies où, quand le soleil les rend réfléchissantes comme une mare, se
dessinent les feuilles des pommiers, c’est ce paysage dont parfois, la nuit dans
mes rêves, l’individualité m’étreint avec une puissance presque fantastique et
que je ne peux plus retrouver au réveil. Sans doute pour avoir à jamais
indissolublement uni en moi des impressions différentes rien que parce qu’ils me
les avaient fait éprouver en même temps, le côté de Méséglise ou le côté de
Guermantes m’ont exposé, pour l’avenir, à bien des déceptions et même à bien des
fautes. Car souvent j’ai voulu revoir une personne sans discerner que c’était
simplement parce qu’elle me rappelait une haie d’aubépines, et j’ai été induit à
croire, à faire croire à un regain d’affection, par un simple désir de voyage.
Mais par là même aussi, et en restant présents en celles de mes impressions
d’aujourd’hui auxquelles ils peuvent se relier, ils leur donnent des assises, de
la profondeur, une dimension de plus qu’aux autres. Ils leur ajoutent aussi un
charme, une signification qui n’est que pour moi. Quand par les soirs d’été le
ciel harmonieux gronde comme une bête fauve et que chacun boude l’orage, c’est
au côté de Méséglise que je dois de rester seul en extase à respirer, à travers
le bruit de la pluie qui tombe, l’odeur d’invisibles et persistants lilas.

%. . .
\PRLsep

C’est ainsi que je restais souvent jusqu’au matin à songer au temps de Combray,
à mes tristes soirées sans sommeil, à tant de jours aussi dont l’image m’avait
été plus récemment rendue par la saveur — ce qu’on aurait appelé à Combray le
«parfum»— d’une tasse de thé, et par association de souvenirs à ce que, bien des
années après avoir quitté cette petite ville, j’avais appris, au sujet d’un
amour que Swann avait eu avant ma naissance, avec cette précision dans les
détails plus facile à obtenir quelquefois pour la vie de personnes mortes il y a
des siècles que pour celle de nos meilleurs amis, et qui semble impossible comme
semblait impossible de causer d’une ville à une autre — tant qu’on ignore le
biais par lequel cette impossibilité a été tournée. Tous ces souvenirs ajoutés
les uns aux autres ne formaient plus qu’une masse, mais non sans qu’on ne pût
distinguer entre eux — entre les plus anciens, et ceux plus récents, nés d’un
parfum, puis ceux qui n’étaient que les souvenirs d’une autre personne de qui je
les avais appris — sinon des fissures, des failles véritables, du moins ces
veinures, ces bigarrures de coloration, qui dans certaines roches, dans certains
marbres, révèlent des différences d’origine, d’âge, de «formation».

Certes quand approchait le matin, il y avait bien longtemps qu’était dissipée la
brève incertitude de mon réveil. Je savais dans quelle chambre je me trouvais
effectivement, je l’avais reconstruite autour de moi dans l’obscurité, et — soit
en m’orientant par la seule mémoire, soit en m’aidant, comme indication, d’une
faible lueur aperçue, au pied de laquelle je plaçais les rideaux de la croisée —
je l’avais reconstruite tout entière et meublée comme un architecte et un
tapissier qui gardent leur ouverture primitive aux fenêtres et aux portes,
j’avais reposé les glaces et remis la commode à sa place habituelle. Mais à
peine le jour — et non plus le reflet d’une dernière braise sur une tringle de
cuivre que j’avais pris pour lui — traçait-il dans l’obscurité, et comme à la
craie, sa première raie blanche et rectificative, que la fenêtre avec ses
rideaux, quittait le cadre de la porte où je l’avais située par erreur, tandis
que pour lui faire place, le bureau que ma mémoire avait maladroitement installé
là se sauvait à toute vitesse, poussant devant lui la cheminée et écartant le
mur mitoyen du couloir; une courette régnait à l’endroit où il y a un instant
encore s’étendait le cabinet de toilette, et la demeure que j’avais rebâtie dans
les ténèbres était allée rejoindre les demeures entrevues dans le tourbillon du
réveil, mise en fuite par ce pâle signe qu’avait tracé au-dessus des rideaux le
doigt levé du jour.


%%%%%%%%%%%%%%%%%%%%%%%%%%%%%%%%%%%%%%%%%%%%%%%%
\part{Un Amour de Swann}


Pour faire partie du «petit noyau», du «petit groupe», du «petit clan» des
Verdurin, une condition était suffisante mais elle était nécessaire: il fallait
adhérer tacitement à un Credo dont un des articles était que le jeune pianiste,
protégé par Mme Verdurin cette année-là et dont elle disait: «Ça ne devrait pas
être permis de savoir jouer Wagner comme ça!», «enfonçait» à la fois Planté et
Rubinstein et que le docteur Cottard avait plus de diagnostic que Potain. Toute
«nouvelle recrue» à qui les Verdurin ne pouvaient pas persuader que les soirées
des gens qui n’allaient pas chez eux étaient ennuyeuses comme la pluie, se
voyait immédiatement exclue. Les femmes étant à cet égard plus rebelles que les
hommes à déposer toute curiosité mondaine et l’envie de se renseigner par
soi-même sur l’agrément des autres salons, et les Verdurin sentant d’autre part
que cet esprit d’examen et ce démon de frivolité pouvaient par contagion devenir
fatal à l’orthodoxie de la petite église, ils avaient été amenés à rejeter
successivement tous les «fidèles» du sexe féminin.

En dehors de la jeune femme du docteur, ils étaient réduits presque uniquement
cette année-là (bien que Mme Verdurin fût elle-même vertueuse et d’une
respectable famille bourgeoise excessivement riche et entièrement obscure avec
laquelle elle avait peu à peu cessé volontairement toute relation) à une
personne presque du demi-monde, Mme de Crécy, que Mme Verdurin appelait par son
petit nom, Odette, et déclarait être «un amour» et à la tante du pianiste,
laquelle devait avoir tiré le cordon; personnes ignorantes du monde et à la
naïveté de qui il avait été si facile de faire accroire que la princesse de
Sagan et la duchesse de Guermantes étaient obligées de payer des malheureux pour
avoir du monde à leurs dîners, que si on leur avait offert de les faire inviter
chez ces deux grandes dames, l’ancienne concierge et la cocotte eussent
dédaigneusement refusé.

Les Verdurin n’invitaient pas à dîner: on avait chez eux «son couvert mis». Pour
la soirée, il n’y avait pas de programme. Le jeune pianiste jouait, mais
seulement si «ça lui chantait», car on ne forçait personne et comme disait M.
Verdurin: «Tout pour les amis, vivent les camarades!» Si le pianiste voulait
jouer la chevauchée de la Walkyrie ou le prélude de Tristan, Mme Verdurin
protestait, non que cette musique lui déplût, mais au contraire parce qu’elle
lui causait trop d’impression. «Alors vous tenez à ce que j’aie ma migraine?
Vous savez bien que c’est la même chose chaque fois qu’il joue ça. Je sais ce
qui m’attend! Demain quand je voudrai me lever, bonsoir, plus personne!» S’il ne
jouait pas, on causait, et l’un des amis, le plus souvent leur peintre favori
d’alors, «lâchait», comme disait M. Verdurin, «une grosse faribole qui faisait
s’esclaffer tout le monde», Mme Verdurin surtout, à qui — tant elle avait
l’habitude de prendre au propre les expressions figurées des émotions qu’elle
éprouvait — le docteur Cottard (un jeune débutant à cette époque) dut un jour
remettre sa mâchoire qu’elle avait décrochée pour avoir trop ri.

L’habit noir était défendu parce qu’on était entre «copains» et pour ne pas
ressembler aux «ennuyeux» dont on se garait comme de la peste et qu’on
n’invitait qu’aux grandes soirées, données le plus rarement possible et
seulement si cela pouvait amuser le peintre ou faire connaître le musicien. Le
reste du temps on se contentait de jouer des charades, de souper en costumes,
mais entre soi, en ne mêlant aucun étranger au petit «noyau».

Mais au fur et à mesure que les «camarades» avaient pris plus de place dans la
vie de Mme Verdurin, les ennuyeux, les réprouvés, ce fut tout ce qui retenait
les amis loin d’elle, ce qui les empêchait quelquefois d’être libres, ce fut la
mère de l’un, la profession de l’autre, la maison de campagne ou la mauvaise
santé d’un troisième. Si le docteur Cottard croyait devoir partir en sortant de
table pour retourner auprès d’un malade en danger: «Qui sait, lui disait Mme
Verdurin, cela lui fera peut-être beaucoup plus de bien que vous n’alliez pas le
déranger ce soir; il passera une bonne nuit sans vous; demain matin vous irez de
bonne heure et vous le trouverez guéri.» Dès le commencement de décembre elle
était malade à la pensée que les fidèles «lâcheraient» pour le jour de Noël et
le 1er janvier. La tante du pianiste exigeait qu’il vînt dîner ce jour-là en
famille chez sa mère à elle:

—«Vous croyez qu’elle en mourrait, votre mère, s’écria durement Mme Verdurin, si
vous ne dîniez pas avec elle le jour de l’an, comme en province!»

Ses inquiétudes renaissaient à la semaine sainte:

—«Vous, Docteur, un savant, un esprit fort, vous venez naturellement le vendredi
saint comme un autre jour?» dit-elle à Cottard la première année, d’un ton
assuré comme si elle ne pouvait douter de la réponse. Mais elle tremblait en
attendant qu’il l’eût prononcée, car s’il n’était pas venu, elle risquait de se
trouver seule.

—«Je viendrai le vendredi saint . . . vous faire mes adieux car nous allons
passer les fêtes de Pâques en Auvergne.»

—«En Auvergne? pour vous faire manger par les puces et la vermine, grand bien
vous fasse!»

Et après un silence:

—«Si vous nous l’aviez dit au moins, nous aurions tâché d’organiser cela et de
faire le voyage ensemble dans des conditions confortables.»

De même si un «fidèle» avait un ami, ou une «habituée» un flirt qui serait
capable de faire «lâcher» quelquefois, les Verdurin qui ne s’effrayaient pas
qu’une femme eût un amant pourvu qu’elle l’eût chez eux, l’aimât en eux, et ne
le leur préférât pas, disaient: «Eh bien! amenez-le votre ami.» Et on
l’engageait à l’essai, pour voir s’il était capable de ne pas avoir de secrets
pour Mme Verdurin, s’il était susceptible d’être agrégé au «petit clan». S’il ne
l’était pas on prenait à part le fidèle qui l’avait présenté et on lui rendait
le service de le brouiller avec son ami ou avec sa maîtresse. Dans le cas
contraire, le «nouveau» devenait à son tour un fidèle. Aussi quand cette
année-là, la demi-mondaine raconta à M. Verdurin qu’elle avait fait la
connaissance d’un homme charmant, M. Swann, et insinua qu’il serait très heureux
d’être reçu chez eux, M. Verdurin transmit-il séance tenante la requête à sa
femme. (Il n’avait jamais d’avis qu’après sa femme, dont son rôle particulier
était de mettre à exécution les désirs, ainsi que les désirs des fidèles, avec
de grandes ressources d’ingéniosité.)

— Voici Mme de Crécy qui a quelque chose à te demander. Elle désirerait te
présenter un de ses amis, M. Swann. Qu’en dis-tu?

—«Mais voyons, est-ce qu’on peut refuser quelque chose à une petite perfection
comme ça. Taisez-vous, on ne vous demande pas votre avis, je vous dis que vous
êtes une perfection.»

—«Puisque vous le voulez, répondit Odette sur un ton de marivaudage, et elle
ajouta: vous savez que je ne suis pas «fishing for compliments».

—«Eh bien! amenez-le votre ami, s’il est agréable.»

Certes le «petit noyau» n’avait aucun rapport avec la société où fréquentait
Swann, et de purs mondains auraient trouvé que ce n’était pas la peine d’y
occuper comme lui une situation exceptionnelle pour se faire présenter chez les
Verdurin. Mais Swann aimait tellement les femmes, qu’à partir du jour où il
avait connu à peu près toutes celles de l’aristocratie et où elles n’avaient
plus rien eu à lui apprendre, il n’avait plus tenu à ces lettres de
naturalisation, presque des titres de noblesse, que lui avait octroyées le
faubourg Saint-Germain, que comme à une sorte de valeur d’échange, de lettre de
crédit dénuée de prix en elle-même, mais lui permettant de s’improviser une
situation dans tel petit trou de province ou tel milieu obscur de Paris, où la
fille du hobereau ou du greffier lui avait semblé jolie. Car le désir ou l’amour
lui rendait alors un sentiment de vanité dont il était maintenant exempt dans
l’habitude de la vie (bien que ce fût lui sans doute qui autrefois l’avait
dirigé vers cette carrière mondaine où il avait gaspillé dans les plaisirs
frivoles les dons de son esprit et fait servir son érudition en matière d’art à
conseiller les dames de la société dans leurs achats de tableaux et pour
l’ameublement de leurs hôtels), et qui lui faisait désirer de briller, aux yeux
d’une inconnue dont il s’était épris, d’une élégance que le nom de Swann à lui
tout seul n’impliquait pas. Il le désirait surtout si l’inconnue était d’humble
condition. De même que ce n’est pas à un autre homme intelligent qu’un homme
intelligent aura peur de paraître bête, ce n’est pas par un grand seigneur,
c’est par un rustre qu’un homme élégant craindra de voir son élégance méconnue.
Les trois quarts des frais d’esprit et des mensonges de vanité qui ont été
prodigués depuis que le monde existe par des gens qu’ils ne faisaient que
diminuer, l’ont été pour des inférieurs. Et Swann qui était simple et négligent
avec une duchesse, tremblait d’être méprisé, posait, quand il était devant une
femme de chambre.

Il n’était pas comme tant de gens qui par paresse, ou sentiment résigné de
l’obligation que crée la grandeur sociale de rester attaché à un certain rivage,
s’abstiennent des plaisirs que la réalité leur présente en dehors de la position
mondaine où ils vivent cantonnés jusqu’à leur mort, se contentant de finir par
appeler plaisirs, faute de mieux, une fois qu’ils sont parvenus à s’y habituer,
les divertissements médiocres ou les supportables ennuis qu’elle renferme.
Swann, lui, ne cherchait pas à trouver jolies les femmes avec qui il passait son
temps, mais à passer son temps avec les femmes qu’il avait d’abord trouvées
jolies. Et c’était souvent des femmes de beauté assez vulgaire, car les qualités
physiques qu’il recherchait sans s’en rendre compte étaient en complète
opposition avec celles qui lui rendaient admirables les femmes sculptées ou
peintes par les maîtres qu’il préférait. La profondeur, la mélancolie de
l’expression, glaçaient ses sens que suffisait au contraire à éveiller une chair
saine, plantureuse et rose.

Si en voyage il rencontrait une famille qu’il eût été plus élégant de ne pas
chercher à connaître, mais dans laquelle une femme se présentait à ses yeux
parée d’un charme qu’il n’avait pas encore connu, rester dans son «quant à soi»
et tromper le désir qu’elle avait fait naître, substituer un plaisir différent
au plaisir qu’il eût pu connaître avec elle, en écrivant à une ancienne
maîtresse de venir le rejoindre, lui eût semblé une aussi lâche abdication
devant la vie, un aussi stupide renoncement à un bonheur nouveau, que si au lieu
de visiter le pays, il s’était confiné dans sa chambre en regardant des vues de
Paris. Il ne s’enfermait pas dans l’édifice de ses relations, mais en avait
fait, pour pouvoir le reconstruire à pied d’œuvre sur de nouveaux frais partout
où une femme lui avait plu, une de ces tentes démontables comme les explorateurs
en emportent avec eux. Pour ce qui n’en était pas transportable ou échangeable
contre un plaisir nouveau, il l’eût donné pour rien, si enviable que cela parût
à d’autres. Que de fois son crédit auprès d’une duchesse, fait du désir accumulé
depuis des années que celle-ci avait eu de lui être agréable sans en avoir
trouvé l’occasion, il s’en était défait d’un seul coup en réclamant d’elle par
une indiscrète dépêche une recommandation télégraphique qui le mît en relation
sur l’heure avec un de ses intendants dont il avait remarqué la fille à la
campagne, comme ferait un affamé qui troquerait un diamant contre un morceau de
pain. Même, après coup, il s’en amusait, car il y avait en lui, rachetée par de
rares délicatesses, une certaine muflerie. Puis, il appartenait à cette
catégorie d’hommes intelligents qui ont vécu dans l’oisiveté et qui cherchent
une consolation et peut-être une excuse dans l’idée que cette oisiveté offre à
leur intelligence des objets aussi dignes d’intérêt que pourrait faire l’art ou
l’étude, que la «Vie» contient des situations plus intéressantes, plus
romanesques que tous les romans. Il l’assurait du moins et le persuadait
aisément aux plus affinés de ses amis du monde notamment au baron de Charlus,
qu’il s’amusait à égayer par le récit des aventures piquantes qui lui
arrivaient, soit qu’ayant rencontré en chemin de fer une femme qu’il avait
ensuite ramenée chez lui il eût découvert qu’elle était la sœur d’un souverain
entre les mains de qui se mêlaient en ce moment tous les fils de la politique
européenne, au courant de laquelle il se trouvait ainsi tenu d’une façon très
agréable, soit que par le jeu complexe des circonstances, il dépendît du choix
qu’allait faire le conclave, s’il pourrait ou non devenir l’amant d’une
cuisinière.

Ce n’était pas seulement d’ailleurs la brillante phalange de vertueuses
douairières, de généraux, d’académiciens, avec lesquels il était
particulièrement lié, que Swann forçait avec tant de cynisme à lui servir
d’entremetteurs. Tous ses amis avaient l’habitude de recevoir de temps en temps
des lettres de lui où un mot de recommandation ou d’introduction leur était
demandé avec une habileté diplomatique qui, persistant à travers les amours
successives et les prétextes différents, accusait, plus que n’eussent fait les
maladresses, un caractère permanent et des buts identiques. Je me suis souvent
fait raconter bien des années plus tard, quand je commençai à m’intéresser à son
caractère à cause des ressemblances qu’en de tout autres parties il offrait avec
le mien, que quand il écrivait à mon grand-père (qui ne l’était pas encore, car
c’est vers l’époque de ma naissance que commença la grande liaison de Swann et
elle interrompit longtemps ces pratiques) celui-ci, en reconnaissant sur
l’enveloppe l’écriture de son ami, s’écriait: «Voilà Swann qui va demander
quelque chose: à la garde!» Et soit méfiance, soit par le sentiment
inconsciemment diabolique qui nous pousse à n’offrir une chose qu’aux gens qui
n’en ont pas envie, mes grands-parents opposaient une fin de non-recevoir
absolue aux prières les plus faciles à satisfaire qu’il leur adressait, comme de
le présenter à une jeune fille qui dînait tous les dimanches à la maison, et
qu’ils étaient obligés, chaque fois que Swann leur en reparlait, de faire
semblant de ne plus voir, alors que pendant toute la semaine on se demandait qui
on pourrait bien inviter avec elle, finissant souvent par ne trouver personne,
faute de faire signe à celui qui en eût été si heureux.

Quelquefois tel couple ami de mes grands-parents et qui jusque-là s’était plaint
de ne jamais voir Swann, leur annonçait avec satisfaction et peut-être un peu le
désir d’exciter l’envie, qu’il était devenu tout ce qu’il y a de plus charmant
pour eux, qu’il ne les quittait plus. Mon grand-père ne voulait pas troubler
leur plaisir mais regardait ma grand’mère en fredonnant:

\begin{center}
  \textit{«Quel est donc ce mystère}\par
  \textit{Je ne puis rien comprendre.»}
\end{center}

\noindent ou:

\begin{center}\textit{
«Vision fugitive . . . »
}\end{center}

\noindent ou:

\begin{center}
  \textit{«Dans ces affaires}\par 
  \textit{Le mieux est de ne rien voir.»}
\end{center}

Quelques mois après, si mon grand-père demandait au nouvel ami de Swann: «Et
Swann, le voyez-vous toujours beaucoup?» la figure de l’interlocuteur
s’allongeait: «Ne prononcez jamais son nom devant moi!»—«Mais je croyais que
vous étiez si liés . . . » Il avait été ainsi pendant quelques mois le familier
de cousins de ma grand’mère, dînant presque chaque jour chez eux. Brusquement il
cessa de venir, sans avoir prévenu. On le crut malade, et la cousine de ma
grand’mère allait envoyer demander de ses nouvelles quand à l’office elle trouva
une lettre de lui qui traînait par mégarde dans le livre de comptes de la
cuisinière. Il y annonçait à cette femme qu’il allait quitter Paris, qu’il ne
pourrait plus venir. Elle était sa maîtresse, et au moment de rompre, c’était
elle seule qu’il avait jugé utile d’avertir.

Quand sa maîtresse du moment était au contraire une personne mondaine ou du
moins une personne qu’une extraction trop humble ou une situation trop
irrégulière n’empêchait pas qu’il fît recevoir dans le monde, alors pour elle il
y retournait, mais seulement dans l’orbite particulier où elle se mouvait ou
bien où il l’avait entraînée. «Inutile de compter sur Swann ce soir, disait-on,
vous savez bien que c’est le jour d’Opéra de son Américaine.» Il la faisait
inviter dans les salons particulièrement fermés où il avait ses habitudes, ses
dîners hebdomadaires, son poker; chaque soir, après qu’un léger crépelage ajouté
à la brosse de ses cheveux roux avait tempéré de quelque douceur la vivacité de
ses yeux verts, il choisissait une fleur pour sa boutonnière et partait pour
retrouver sa maîtresse à dîner chez l’une ou l’autre des femmes de sa coterie;
et alors, pensant à l’admiration et à l’amitié que les gens à la mode pour qui
il faisait la pluie et le beau temps et qu’il allait retrouver là, lui
prodigueraient devant la femme qu’il aimait, il retrouvait du charme à cette vie
mondaine sur laquelle il s’était blasé, mais dont la matière, pénétrée et
colorée chaudement d’une flamme insinuée qui s’y jouait, lui semblait précieuse
et belle depuis qu’il y avait incorporé un nouvel amour.

Mais tandis que chacune de ces liaisons, ou chacun de ces flirts, avait été la
réalisation plus ou moins complète d’un rêve né de la vue d’un visage ou d’un
corps que Swann avait, spontanément, sans s’y efforcer, trouvés charmants, en
revanche quand un jour au théâtre il fut présenté à Odette de Crécy par un de
ses amis d’autrefois, qui lui avait parlé d’elle comme d’une femme ravissante
avec qui il pourrait peut-être arriver à quelque chose, mais en la lui donnant
pour plus difficile qu’elle n’était en réalité afin de paraître lui-même avoir
fait quelque chose de plus aimable en la lui faisant connaître, elle était
apparue à Swann non pas certes sans beauté, mais d’un genre de beauté qui lui
était indifférent, qui ne lui inspirait aucun désir, lui causait même une sorte
de répulsion physique, de ces femmes comme tout le monde a les siennes,
différentes pour chacun, et qui sont l’opposé du type que nos sens réclament.
Pour lui plaire elle avait un profil trop accusé, la peau trop fragile, les
pommettes trop saillantes, les traits trop tirés. Ses yeux étaient beaux mais si
grands qu’ils fléchissaient sous leur propre masse, fatiguaient le reste de son
visage et lui donnaient toujours l’air d’avoir mauvaise mine ou d’être de
mauvaise humeur. Quelque temps après cette présentation au théâtre, elle lui
avait écrit pour lui demander à voir ses collections qui l’intéressaient tant,
«elle, ignorante qui avait le goût des jolies choses», disant qu’il lui semblait
qu’elle le connaîtrait mieux, quand elle l’aurait vu dans «son home» où elle
l’imaginait «si confortable avec son thé et ses livres», quoiqu’elle ne lui eût
pas caché sa surprise qu’il habitât ce quartier qui devait être si triste et
«qui était si peu smart pour lui qui l’était tant». Et après qu’il l’eut laissée
venir, en le quittant elle lui avait dit son regret d’être restée si peu dans
cette demeure où elle avait été heureuse de pénétrer, parlant de lui comme s’il
avait été pour elle quelque chose de plus que les autres êtres qu’elle
connaissait et semblant établir entre leurs deux personnes une sorte de trait
d’union romanesque qui l’avait fait sourire. Mais à l’âge déjà un peu désabusé
dont approchait Swann et où l’on sait se contenter d’être amoureux pour le
plaisir de l’être sans trop exiger de réciprocité, ce rapprochement des cœurs,
s’il n’est plus comme dans la première jeunesse le but vers lequel tend
nécessairement l’amour, lui reste uni en revanche par une association d’idées si
forte, qu’il peut en devenir la cause, s’il se présente avant lui. Autrefois on
rêvait de posséder le cœur de la femme dont on était amoureux; plus tard sentir
qu’on possède le cœur d’une femme peut suffire à vous en rendre amoureux. Ainsi,
à l’âge où il semblerait, comme on cherche surtout dans l’amour un plaisir
subjectif, que la part du goût pour la beauté d’une femme devait y être la plus
grande, l’amour peut naître — l’amour le plus physique — sans qu’il y ait eu, à
sa base, un désir préalable. A cette époque de la vie, on a déjà été atteint
plusieurs fois par l’amour; il n’évolue plus seul suivant ses propres lois
inconnues et fatales, devant notre cœur étonné et passif. Nous venons à son
aide, nous le faussons par la mémoire, par la suggestion. En reconnaissant un de
ses symptômes, nous nous rappelons, nous faisons renaître les autres. Comme nous
possédons sa chanson, gravée en nous tout entière, nous n’avons pas besoin
qu’une femme nous en dise le début — rempli par l’admiration qu’inspire la
beauté — pour en trouver la suite. Et si elle commence au milieu — là où les
cœurs se rapprochent, où l’on parle de n’exister plus que l’un pour l’autre —
nous avons assez l’habitude de cette musique pour rejoindre tout de suite notre
partenaire au passage où elle nous attend.

Odette de Crécy retourna voir Swann, puis rapprocha ses visites; et sans doute
chacune d’elles renouvelait pour lui la déception qu’il éprouvait à se retrouver
devant ce visage dont il avait un peu oublié les particularités dans
l’intervalle, et qu’il ne s’était rappelé ni si expressif ni, malgré sa
jeunesse, si fané; il regrettait, pendant qu’elle causait avec lui, que la
grande beauté qu’elle avait ne fût pas du genre de celles qu’il aurait
spontanément préférées. Il faut d’ailleurs dire que le visage d’Odette
paraissait plus maigre et plus proéminent parce que le front et le haut des
joues, cette surface unie et plus plane était recouverte par la masse de cheveux
qu’on portait, alors, prolongés en «devants», soulevés en «crêpés», répandus en
mèches folles le long des oreilles; et quant à son corps qui était admirablement
fait, il était difficile d’en apercevoir la continuité (à cause des modes de
l’époque et quoiqu’elle fût une des femmes de Paris qui s’habillaient le mieux),
tant le corsage, s’avançant en saillie comme sur un ventre imaginaire et
finissant brusquement en pointe pendant que par en dessous commençait à s’enfler
le ballon des doubles jupes, donnait à la femme l’air d’être composée de pièces
différentes mal emmanchées les unes dans les autres; tant les ruchés, les
volants, le gilet suivaient en toute indépendance, selon la fantaisie de leur
dessin ou la consistance de leur étoffe, la ligne qui les conduisait aux nœuds,
aux bouillons de dentelle, aux effilés de jais perpendiculaires, ou qui les
dirigeait le long du busc, mais ne s’attachaient nullement à l’être vivant, qui
selon que l’architecture de ces fanfreluches se rapprochait ou s’écartait trop
de la sienne, s’y trouvait engoncé ou perdu.

Mais, quand Odette était partie, Swann souriait en pensant qu’elle lui avait dit
combien le temps lui durerait jusqu’à ce qu’il lui permît de revenir; il se
rappelait l’air inquiet, timide avec lequel elle l’avait une fois prié que ce ne
fût pas dans trop longtemps, et les regards qu’elle avait eus à ce moment-là,
fixés sur lui en une imploration craintive, et qui la faisaient touchante sous
le bouquet de fleurs de pensées artificielles fixé devant son chapeau rond de
paille blanche, à brides de velours noir. «Et vous, avait-elle dit, vous ne
viendriez pas une fois chez moi prendre le thé?» Il avait allégué des travaux en
train, une étude — en réalité abandonnée depuis des années — sur Ver Meer de
Delft. «Je comprends que je ne peux rien faire, moi chétive, à côté de grands
savants comme vous autres, lui avait-elle répondu. Je serais comme la grenouille
devant l’aréopage. Et pourtant j’aimerais tant m’instruire, savoir, être
initiée. Comme cela doit être amusant de bouquiner, de fourrer son nez dans de
vieux papiers, avait-elle ajouté avec l’air de contentement de soi-même que
prend une femme élégante pour affirmer que sa joie est de se livrer sans crainte
de se salir à une besogne malpropre, comme de faire la cuisine en «mettant
elle-même les mains à la pâte». «Vous allez vous moquer de moi, ce peintre qui
vous empêche de me voir (elle voulait parler de Ver Meer), je n’avais jamais
entendu parler de lui; vit-il encore? Est-ce qu’on peut voir de ses œuvres à
Paris, pour que je puisse me représenter ce que vous aimez, deviner un peu ce
qu’il y a sous ce grand front qui travaille tant, dans cette tête qu’on sent
toujours en train de réfléchir, me dire: voilà, c’est à cela qu’il est en train
de penser. Quel rêve ce serait d’être mêlée à vos travaux!» Il s’était excusé
sur sa peur des amitiés nouvelles, ce qu’il avait appelé, par galanterie, sa
peur d’être malheureux. «Vous avez peur d’une affection? comme c’est drôle, moi
qui ne cherche que cela, qui donnerais ma vie pour en trouver une, avait-elle
dit d’une voix si naturelle, si convaincue, qu’il en avait été remué. Vous avez
dû souffrir par une femme. Et vous croyez que les autres sont comme elle. Elle
n’a pas su vous comprendre; vous êtes un être si à part. C’est cela que j’ai
aimé d’abord en vous, j’ai bien senti que vous n’étiez pas comme tout le
monde.»—«Et puis d’ailleurs vous aussi, lui avait-il dit, je sais bien ce que
c’est que les femmes, vous devez avoir des tas d’occupations, être peu
libre.»—«Moi, je n’ai jamais rien à faire! Je suis toujours libre, je le serai
toujours pour vous. A n’importe quelle heure du jour ou de la nuit où il
pourrait vous être commode de me voir, faites-moi chercher, et je serai trop
heureuse d’accourir. Le ferez-vous? Savez-vous ce qui serait gentil, ce serait
de vous faire présenter à Mme Verdurin chez qui je vais tous les soirs.
Croyez-vous! si on s’y retrouvait et si je pensais que c’est un peu pour moi que
vous y êtes!»

Et sans doute, en se rappelant ainsi leurs entretiens, en pensant ainsi à elle
quand il était seul, il faisait seulement jouer son image entre beaucoup
d’autres images de femmes dans des rêveries romanesques; mais si, grâce à une
circonstance quelconque (ou même peut-être sans que ce fût grâce à elle, la
circonstance qui se présente au moment où un état, latent jusque-là, se déclare,
pouvant n’avoir influé en rien sur lui) l’image d’Odette de Crécy venait à
absorber toutes ces rêveries, si celles-ci n’étaient plus séparables de son
souvenir, alors l’imperfection de son corps ne garderait plus aucune importance,
ni qu’il eût été, plus ou moins qu’un autre corps, selon le goût de Swann,
puisque devenu le corps de celle qu’il aimait, il serait désormais le seul qui
fût capable de lui causer des joies et des tourments.

Mon grand-père avait précisément connu, ce qu’on n’aurait pu dire d’aucun de
leurs amis actuels, la famille de ces Verdurin. Mais il avait perdu toute
relation avec celui qu’il appelait le «jeune Verdurin» et qu’il considérait, un
peu en gros, comme tombé— tout en gardant de nombreux millions — dans la bohème
et la racaille. Un jour il reçut une lettre de Swann lui demandant s’il ne
pourrait pas le mettre en rapport avec les Verdurin: «A la garde! à la garde!
s’était écrié mon grand-père, ça ne m’étonne pas du tout, c’est bien par là que
devait finir Swann. Joli milieu! D’abord je ne peux pas faire ce qu’il me
demande parce que je ne connais plus ce monsieur. Et puis ça doit cacher une
histoire de femme, je ne me mêle pas de ces affaires-là. Ah bien! nous allons
avoir de l’agrément si Swann s’affuble des petits Verdurin.»

Et sur la réponse négative de mon grand-père, c’est Odette qui avait amené
elle-même Swann chez les Verdurin.

Les Verdurin avaient eu à dîner, le jour où Swann y fit ses débuts, le docteur
et Mme Cottard, le jeune pianiste et sa tante, et le peintre qui avait alors
leur faveur, auxquels s’étaient joints dans la soirée quelques autres fidèles.

Le docteur Cottard ne savait jamais d’une façon certaine de quel ton il devait
répondre à quelqu’un, si son interlocuteur voulait rire ou était sérieux. Et à
tout hasard il ajoutait à toutes ses expressions de physionomie l’offre d’un
sourire conditionnel et provisoire dont la finesse expectante le disculperait du
reproche de naïveté, si le propos qu’on lui avait tenu se trouvait avoir été
facétieux. Mais comme pour faire face à l’hypothèse opposée il n’osait pas
laisser ce sourire s’affirmer nettement sur son visage, on y voyait flotter
perpétuellement une incertitude où se lisait la question qu’il n’osait pas
poser: «Dites-vous cela pour de bon?» Il n’était pas plus assuré de la façon
dont il devait se comporter dans la rue, et même en général dans la vie, que
dans un salon, et on le voyait opposer aux passants, aux voitures, aux
événements un malicieux sourire qui ôtait d’avance à son attitude toute
impropriété puisqu’il prouvait, si elle n’était pas de mise, qu’il le savait
bien et que s’il avait adopté celle-là, c’était par plaisanterie.

Sur tous les points cependant où une franche question lui semblait permise, le
docteur ne se faisait pas faute de s’efforcer de restreindre le champ de ses
doutes et de compléter son instruction.

C’est ainsi que, sur les conseils qu’une mère prévoyante lui avait donnés quand
il avait quitté sa province, il ne laissait jamais passer soit une locution ou
un nom propre qui lui étaient inconnus, sans tâcher de se faire documenter sur
eux.

Pour les locutions, il était insatiable de renseignements, car, leur supposant
parfois un sens plus précis qu’elles n’ont, il eût désiré savoir ce qu’on
voulait dire exactement par celles qu’il entendait le plus souvent employer: la
beauté du diable, du sang bleu, une vie de bâtons de chaise, le quart d’heure de
Rabelais, être le prince des élégances, donner carte blanche, être réduit à
quia, etc., et dans quels cas déterminés il pouvait à son tour les faire figurer
dans ses propos. A leur défaut il plaçait des jeux de mots qu’il avait appris.
Quant aux noms de personnes nouveaux qu’on prononçait devant lui il se
contentait seulement de les répéter sur un ton interrogatif qu’il pensait
suffisant pour lui valoir des explications qu’il n’aurait pas l’air de demander.

Comme le sens critique qu’il croyait exercer sur tout lui faisait complètement
défaut, le raffinement de politesse qui consiste à affirmer, à quelqu’un qu’on
oblige, sans souhaiter d’en être cru, que c’est à lui qu’on a obligation, était
peine perdue avec lui, il prenait tout au pied de la lettre. Quel que fût
l’aveuglement de Mme Verdurin à son égard, elle avait fini, tout en continuant à
le trouver très fin, par être agacée de voir que quand elle l’invitait dans une
avant-scène à entendre Sarah Bernhardt, lui disant, pour plus de grâce: «Vous
êtes trop aimable d’être venu, docteur, d’autant plus que je suis sûre que vous
avez déjà souvent entendu Sarah Bernhardt, et puis nous sommes peut-être trop
près de la scène», le docteur Cottard qui était entré dans la loge avec un
sourire qui attendait pour se préciser ou pour disparaître que quelqu’un
d’autorisé le renseignât sur la valeur du spectacle, lui répondait: «En effet on
est beaucoup trop près et on commence à être fatigué de Sarah Bernhardt. Mais
vous m’avez exprimé le désir que je vienne. Pour moi vos désirs sont des ordres.
Je suis trop heureux de vous rendre ce petit service. Que ne ferait-on pas pour
vous être agréable, vous êtes si bonne!» Et il ajoutait: «Sarah Bernhardt c’est
bien la Voix d’Or, n’est-ce pas? On écrit souvent aussi qu’elle brûle les
planches. C’est une expression bizarre, n’est-ce pas?» dans l’espoir de
commentaires qui ne venaient point.

«Tu sais, avait dit Mme Verdurin à son mari, je crois que nous faisons fausse
route quand par modestie nous déprécions ce que nous offrons au docteur. C’est
un savant qui vit en dehors de l’existence pratique, il ne connaît pas par
lui-même la valeur des choses et il s’en rapporte à ce que nous lui en
disons.»—«Je n’avais pas osé te le dire, mais je l’avais remarqué», répondit M.
Verdurin. Et au jour de l’an suivant, au lieu d’envoyer au docteur Cottard un
rubis de trois mille francs en lui disant que c’était bien peu de chose, M.
Verdurin acheta pour trois cents francs une pierre reconstituée en laissant
entendre qu’on pouvait difficilement en voir d’aussi belle.

Quand Mme Verdurin avait annoncé qu’on aurait, dans la soirée, M. Swann:
«Swann?» s’était écrié le docteur d’un accent rendu brutal par la surprise, car
la moindre nouvelle prenait toujours plus au dépourvu que quiconque cet homme
qui se croyait perpétuellement préparé à tout. Et voyant qu’on ne lui répondait
pas: «Swann? Qui ça, Swann!» hurla-t-il au comble d’une anxiété qui se détendit
soudain quand Mme Verdurin eut dit: «Mais l’ami dont Odette nous avait
parlé.»—«Ah! bon, bon, ça va bien», répondit le docteur apaisé. Quant au peintre
il se réjouissait de l’introduction de Swann chez Mme Verdurin, parce qu’il le
supposait amoureux d’Odette et qu’il aimait à favoriser les liaisons. «Rien ne
m’amuse comme de faire des mariages, confia-t-il, dans l’oreille, au docteur
Cottard, j’en ai déjà réussi beaucoup, même entre femmes!»

En disant aux Verdurin que Swann était très «smart», Odette leur avait fait
craindre un «ennuyeux». Il leur fit au contraire une excellente impression dont
à leur insu sa fréquentation dans la société élégante était une des causes
indirectes. Il avait en effet sur les hommes même intelligents qui ne sont
jamais allés dans le monde, une des supériorités de ceux qui y ont un peu vécu,
qui est de ne plus le transfigurer par le désir ou par l’horreur qu’il inspire à
l’imagination, de le considérer comme sans aucune importance. Leur amabilité,
séparée de tout snobisme et de la peur de paraître trop aimable, devenue
indépendante, a cette aisance, cette grâce des mouvements de ceux dont les
membres assouplis exécutent exactement ce qu’ils veulent, sans participation
indiscrète et maladroite du reste du corps. La simple gymnastique élémentaire de
l’homme du monde tendant la main avec bonne grâce au jeune homme inconnu qu’on
lui présente et s’inclinant avec réserve devant l’ambassadeur à qui on le
présente, avait fini par passer sans qu’il en fût conscient dans toute
l’attitude sociale de Swann, qui vis-à-vis de gens d’un milieu inférieur au sien
comme étaient les Verdurin et leurs amis, fit instinctivement montre d’un
empressement, se livra à des avances, dont, selon eux, un ennuyeux se fût
abstenu. Il n’eut un moment de froideur qu’avec le docteur Cottard: en le voyant
lui cligner de l’œil et lui sourire d’un air ambigu avant qu’ils se fussent
encore parlé (mimique que Cottard appelait «laisser venir»), Swann crut que le
docteur le connaissait sans doute pour s’être trouvé avec lui en quelque lieu de
plaisir, bien que lui-même y allât pourtant fort peu, n’ayant jamais vécu dans
le monde de la noce. Trouvant l’allusion de mauvais goût, surtout en présence
d’Odette qui pourrait en prendre une mauvaise idée de lui, il affecta un air
glacial. Mais quand il apprit qu’une dame qui se trouvait près de lui était Mme
Cottard, il pensa qu’un mari aussi jeune n’aurait pas cherché à faire allusion
devant sa femme à des divertissements de ce genre; et il cessa de donner à l’air
entendu du docteur la signification qu’il redoutait. Le peintre invita tout de
suite Swann à venir avec Odette à son atelier, Swann le trouva gentil.
«Peut-être qu’on vous favorisera plus que moi, dit Mme Verdurin, sur un ton qui
feignait d’être piqué, et qu’on vous montrera le portrait de Cottard (elle
l’avait commandé au peintre). Pensez bien, «monsieur» Biche, rappela-t-elle au
peintre, à qui c’était une plaisanterie consacrée de dire monsieur, à rendre le
joli regard, le petit côté fin, amusant, de l’œil. Vous savez que ce que je veux
surtout avoir, c’est son sourire, ce que je vous ai demandé c’est le portrait de
son sourire. Et comme cette expression lui sembla remarquable elle la répéta
très haut pour être sûre que plusieurs invités l’eussent entendue, et même, sous
un prétexte vague, en fit d’abord rapprocher quelques-uns. Swann demanda à faire
la connaissance de tout le monde, même d’un vieil ami des Verdurin, Saniette, à
qui sa timidité, sa simplicité et son bon cœur avaient fait perdre partout la
considération que lui avaient value sa science d’archiviste, sa grosse fortune,
et la famille distinguée dont il sortait. Il avait dans la bouche, en parlant,
une bouillie qui était adorable parce qu’on sentait qu’elle trahissait moins un
défaut de la langue qu’une qualité de l’âme, comme un reste de l’innocence du
premier âge qu’il n’avait jamais perdue. Toutes les consonnes qu’il ne pouvait
prononcer figuraient comme autant de duretés dont il était incapable. En
demandant à être présenté à M. Saniette, Swann fit à Mme Verdurin l’effet de
renverser les rôles (au point qu’en réponse, elle dit en insistant sur la
différence: «Monsieur Swann, voudriez-vous avoir la bonté de me permettre de
vous présenter notre ami Saniette»), mais excita chez Saniette une sympathie
ardente que d’ailleurs les Verdurin ne révélèrent jamais à Swann, car Saniette
les agaçait un peu et ils ne tenaient pas à lui faire des amis. Mais en revanche
Swann les toucha infiniment en croyant devoir demander tout de suite à faire la
connaissance de la tante du pianiste. En robe noire comme toujours, parce
qu’elle croyait qu’en noir on est toujours bien et que c’est ce qu’il y a de
plus distingué, elle avait le visage excessivement rouge comme chaque fois
qu’elle venait de manger. Elle s’inclina devant Swann avec respect, mais se
redressa avec majesté. Comme elle n’avait aucune instruction et avait peur de
faire des fautes de français, elle prononçait exprès d’une manière confuse,
pensant que si elle lâchait un cuir il serait estompé d’un tel vague qu’on ne
pourrait le distinguer avec certitude, de sorte que sa conversation n’était
qu’un graillonnement indistinct duquel émergeaient de temps à autre les rares
vocables dont elle se sentait sûre. Swann crut pouvoir se moquer légèrement
d’elle en parlant à M. Verdurin lequel au contraire fut piqué.

—«C’est une si excellente femme, répondit-il. Je vous accorde qu’elle n’est pas
étourdissante; mais je vous assure qu’elle est agréable quand on cause seul avec
elle. «Je n’en doute pas, s’empressa de concéder Swann. Je voulais dire qu’elle
ne me semblait pas «éminente» ajouta-t-il en détachant cet adjectif, et en somme
c’est plutôt un compliment!» «Tenez, dit M. Verdurin, je vais vous étonner, elle
écrit d’une manière charmante. Vous n’avez jamais entendu son neveu? c’est
admirable, n’est-ce pas, docteur? Voulez-vous que je lui demande de jouer
quelque chose, Monsieur Swann?»

—«Mais ce sera un bonheur . . ., commençait à répondre Swann, quand le docteur
l’interrompit d’un air moqueur. En effet ayant retenu que dans la conversation
l’emphase, l’emploi de formes solennelles, était suranné, dès qu’il entendait un
mot grave dit sérieusement comme venait de l’être le mot «bonheur», il croyait
que celui qui l’avait prononcé venait de se montrer prudhommesque. Et si, de
plus, ce mot se trouvait figurer par hasard dans ce qu’il appelait un vieux
cliché, si courant que ce mot fût d’ailleurs, le docteur supposait que la phrase
commencée était ridicule et la terminait ironiquement par le lieu commun qu’il
semblait accuser son interlocuteur d’avoir voulu placer, alors que celui-ci n’y
avait jamais pensé.

—«Un bonheur pour la France!» s’écria-t-il malicieusement en levant les bras
avec emphase.

M. Verdurin ne put s’empêcher de rire.

—«Qu’est-ce qu’ils ont à rire toutes ces bonnes gens-là, on a l’air de ne pas
engendrer la mélancolie dans votre petit coin là-bas, s’écria Mme Verdurin. Si
vous croyez que je m’amuse, moi, à rester toute seule en pénitence»,
ajouta-t-elle sur un ton dépité, en faisant l’enfant.

Mme Verdurin était assise sur un haut siège suédois en sapin ciré, qu’un
violoniste de ce pays lui avait donné et qu’elle conservait quoiqu’il rappelât
la forme d’un escabeau et jurât avec les beaux meubles anciens qu’elle avait,
mais elle tenait à garder en évidence les cadeaux que les fidèles avaient
l’habitude de lui faire de temps en temps, afin que les donateurs eussent le
plaisir de les reconnaître quand ils venaient. Aussi tâchait-elle de persuader
qu’on s’en tînt aux fleurs et aux bonbons, qui du moins se détruisent; mais elle
n’y réussissait pas et c’était chez elle une collection de chauffe-pieds, de
coussins, de pendules, de paravents, de baromètres, de potiches, dans une
accumulation de redites et un disparate d’étrennes.

De ce poste élevé elle participait avec entrain à la conversation des fidèles et
s’égayait de leurs «fumisteries», mais depuis l’accident qui était arrivé à sa
mâchoire, elle avait renoncé à prendre la peine de pouffer effectivement et se
livrait à la place à une mimique conventionnelle qui signifiait sans fatigue ni
risques pour elle, qu’elle riait aux larmes. Au moindre mot que lâchait un
habitué contre un ennuyeux ou contre un ancien habitué rejeté au camp des
ennuyeux — et pour le plus grand désespoir de M. Verdurin qui avait eu longtemps
la prétention d’être aussi aimable que sa femme, mais qui riant pour de bon
s’essoufflait vite et avait été distancé et vaincu par cette ruse d’une
incessante et fictive hilarité — elle poussait un petit cri, fermait entièrement
ses yeux d’oiseau qu’une taie commençait à voiler, et brusquement, comme si elle
n’eût eu que le temps de cacher un spectacle indécent ou de parer à un accès
mortel, plongeant sa figure dans ses mains qui la recouvraient et n’en
laissaient plus rien voir, elle avait l’air de s’efforcer de réprimer,
d’anéantir un rire qui, si elle s’y fût abandonnée, l’eût conduite à
l’évanouissement. Telle, étourdie par la gaieté des fidèles, ivre de
camaraderie, de médisance et d’assentiment, Mme Verdurin, juchée sur son
perchoir, pareille à un oiseau dont on eût trempé le colifichet dans du vin
chaud, sanglotait d’amabilité.

Cependant, M. Verdurin, après avoir demandé à Swann la permission d’allumer sa
pipe («ici on ne se gêne pas, on est entre camarades»), priait le jeune artiste
de se mettre au piano.

—«Allons, voyons, ne l’ennuie pas, il n’est pas ici pour être tourmenté, s’écria
Mme Verdurin, je ne veux pas qu’on le tourmente moi!»

—«Mais pourquoi veux-tu que ça l’ennuie, dit M. Verdurin, M. Swann ne connaît
peut-être pas la sonate en fa dièse que nous avons découverte, il va nous jouer
l’arrangement pour piano.»

—«Ah! non, non, pas ma sonate! cria Mme Verdurin, je n’ai pas envie à force de
pleurer de me fiche un rhume de cerveau avec névralgies faciales, comme la
dernière fois; merci du cadeau, je ne tiens pas à recommencer; vous êtes bons
vous autres, on voit bien que ce n’est pas vous qui garderez le lit huit jours!»

Cette petite scène qui se renouvelait chaque fois que le pianiste allait jouer
enchantait les amis aussi bien que si elle avait été nouvelle, comme une preuve
de la séduisante originalité de la «Patronne» et de sa sensibilité musicale.
Ceux qui étaient près d’elle faisaient signe à ceux qui plus loin fumaient ou
jouaient aux cartes, de se rapprocher, qu’il se passait quelque chose, leur
disant, comme on fait au Reichstag dans les moments intéressants: «Écoutez,
écoutez.» Et le lendemain on donnait des regrets à ceux qui n’avaient pas pu
venir en leur disant que la scène avait été encore plus amusante que d’habitude.

— Eh bien! voyons, c’est entendu, dit M. Verdurin, il ne jouera que l’andante.

—«Que l’andante, comme tu y vas» s’écria Mme Verdurin. «C’est justement
l’andante qui me casse bras et jambes. Il est vraiment superbe le Patron! C’est
comme si dans la «Neuvième» il disait: nous n’entendrons que le finale, ou dans
«les Maîtres» que l’ouverture.»

Le docteur cependant, poussait Mme Verdurin à laisser jouer le pianiste, non pas
qu’il crût feints les troubles que la musique lui donnait — il y reconnaissait
certains états neurasthéniques — mais par cette habitude qu’ont beaucoup de
médecins, de faire fléchir immédiatement la sévérité de leurs prescriptions dès
qu’est en jeu, chose qui leur semble beaucoup plus importante, quelque réunion
mondaine dont ils font partie et dont la personne à qui ils conseillent
d’oublier pour une fois sa dyspepsie, ou sa grippe, est un des facteurs
essentiels.

— Vous ne serez pas malade cette fois-ci, vous verrez, lui dit-il en cherchant à
la suggestionner du regard. Et si vous êtes malade nous vous soignerons.

— Bien vrai? répondit Mme Verdurin, comme si devant l’espérance d’une telle
faveur il n’y avait plus qu’à capituler. Peut-être aussi à force de dire qu’elle
serait malade, y avait-il des moments où elle ne se rappelait plus que c’était
un mensonge et prenait une âme de malade. Or ceux-ci, fatigués d’être toujours
obligés de faire dépendre de leur sagesse la rareté de leurs accès, aiment se
laisser aller à croire qu’ils pourront faire impunément tout ce qui leur plaît
et leur fait mal d’habitude, à condition de se remettre en les mains d’un être
puissant, qui, sans qu’ils aient aucune peine à prendre, d’un mot ou d’une
pilule, les remettra sur pied.

Odette était allée s’asseoir sur un canapé de tapisserie qui était près du
piano:

— Vous savez, j’ai ma petite place, dit-elle à Mme Verdurin.

Celle-ci, voyant Swann sur une chaise, le fit lever:

—«Vous n’êtes pas bien là, allez donc vous mettre à côté d’Odette, n’est-ce pas
Odette, vous ferez bien une place à M. Swann?»

—«Quel joli beauvais, dit avant de s’asseoir Swann qui cherchait à être
aimable.»

—«Ah! je suis contente que vous appréciiez mon canapé, répondit Mme Verdurin. Et
je vous préviens que si vous voulez en voir d’aussi beau, vous pouvez y renoncer
tout de suite. Jamais ils n’ont rien fait de pareil. Les petites chaises aussi
sont des merveilles. Tout à l’heure vous regarderez cela. Chaque bronze
correspond comme attribut au petit sujet du siège; vous savez, vous avez de quoi
vous amuser si vous voulez regarder cela, je vous promets un bon moment. Rien
que les petites frises des bordures, tenez là, la petite vigne sur fond rouge de
l’Ours et les Raisins. Est-ce dessiné? Qu’est-ce que vous en dites, je crois
qu’ils le savaient plutôt, dessiner! Est-elle assez appétissante cette vigne?
Mon mari prétend que je n’aime pas les fruits parce que j’en mange moins que
lui. Mais non, je suis plus gourmande que vous tous, mais je n’ai pas besoin de
me les mettre dans la bouche puisque je jouis par les yeux. Qu’est ce que vous
avez tous à rire? demandez au docteur, il vous dira que ces raisins-là me
purgent. D’autres font des cures de Fontainebleau, moi je fais ma petite cure de
Beauvais. Mais, monsieur Swann, vous ne partirez pas sans avoir touché les
petits bronzes des dossiers. Est-ce assez doux comme patine? Mais non, à pleines
mains, touchez-les bien.

— Ah! si madame Verdurin commence à peloter les bronzes, nous n’entendrons pas
de musique ce soir, dit le peintre.

—«Taisez-vous, vous êtes un vilain. Au fond, dit-elle en se tournant vers Swann,
on nous défend à nous autres femmes des choses moins voluptueuses que cela. Mais
il n’y a pas une chair comparable à cela! Quand M. Verdurin me faisait l’honneur
d’être jaloux de moi — allons, sois poli au moins, ne dis pas que tu ne l’as
jamais été . . . —»

—«Mais je ne dis absolument rien. Voyons docteur je vous prends à témoin: est-ce
que j’ai dit quelque chose?»

Swann palpait les bronzes par politesse et n’osait pas cesser tout de suite.

— Allons, vous les caresserez plus tard; maintenant c’est vous qu’on va
caresser, qu’on va caresser dans l’oreille; vous aimez cela, je pense; voilà un
petit jeune homme qui va s’en charger.

Or quand le pianiste eut joué, Swann fut plus aimable encore avec lui qu’avec
les autres personnes qui se trouvaient là. Voici pourquoi:

L’année précédente, dans une soirée, il avait entendu une œuvre musicale
exécutée au piano et au violon. D’abord, il n’avait goûté que la qualité
matérielle des sons sécrétés par les instruments. Et ç’avait déjà été un grand
plaisir quand au-dessous de la petite ligne du violon mince, résistante, dense
et directrice, il avait vu tout d’un coup chercher à s’élever en un clapotement
liquide, la masse de la partie de piano, multiforme, indivise, plane et
entrechoquée comme la mauve agitation des flots que charme et bémolise le clair
de lune. Mais à un moment donné, sans pouvoir nettement distinguer un contour,
donner un nom à ce qui lui plaisait, charmé tout d’un coup, il avait cherché à
recueillir la phrase ou l’harmonie — il ne savait lui-même — qui passait et qui
lui avait ouvert plus largement l’âme, comme certaines odeurs de roses circulant
dans l’air humide du soir ont la propriété de dilater nos narines. Peut-être
est-ce parce qu’il ne savait pas la musique qu’il avait pu éprouver une
impression aussi confuse, une de ces impressions qui sont peut-être pourtant les
seules purement musicales, inattendues, entièrement originales, irréductibles à
tout autre ordre d’impressions. Une impression de ce genre pendant un instant,
est pour ainsi dire sine materia. Sans doute les notes que nous entendons alors,
tendent déjà, selon leur hauteur et leur quantité, à couvrir devant nos yeux des
surfaces de dimensions variées, à tracer des arabesques, à nous donner des
sensations de largeur, de ténuité, de stabilité, de caprice. Mais les notes sont
évanouies avant que ces sensations soient assez formées en nous pour ne pas être
submergées par celles qu’éveillent déjà les notes suivantes ou même simultanées.
Et cette impression continuerait à envelopper de sa liquidité et de son «fondu»
les motifs qui par instants en émergent, à peine discernables, pour plonger
aussitôt et disparaître, connus seulement par le plaisir particulier qu’ils
donnent, impossibles à décrire, à se rappeler, à nommer, ineffables — si la
mémoire, comme un ouvrier qui travaille à établir des fondations durables au
milieu des flots, en fabriquant pour nous des fac-similés de ces phrases
fugitives, ne nous permettait de les comparer à celles qui leur succèdent et de
les différencier. Ainsi à peine la sensation délicieuse que Swann avait
ressentie était-elle expirée, que sa mémoire lui en avait fourni séance tenante
une transcription sommaire et provisoire, mais sur laquelle il avait jeté les
yeux tandis que le morceau continuait, si bien que quand la même impression
était tout d’un coup revenue, elle n’était déjà plus insaisissable. Il s’en
représentait l’étendue, les groupements symétriques, la graphie, la valeur
expressive; il avait devant lui cette chose qui n’est plus de la musique pure,
qui est du dessin, de l’architecture, de la pensée, et qui permet de se rappeler
la musique. Cette fois il avait distingué nettement une phrase s’élevant pendant
quelques instants au-dessus des ondes sonores. Elle lui avait proposé aussitôt
des voluptés particulières, dont il n’avait jamais eu l’idée avant de
l’entendre, dont il sentait que rien autre qu’elle ne pourrait les lui faire
connaître, et il avait éprouvé pour elle comme un amour inconnu.

D’un rythme lent elle le dirigeait ici d’abord, puis là, puis ailleurs, vers un
bonheur noble, inintelligible et précis. Et tout d’un coup au point où elle
était arrivée et d’où il se préparait à la suivre, après une pause d’un instant,
brusquement elle changeait de direction et d’un mouvement nouveau, plus rapide,
menu, mélancolique, incessant et doux, elle l’entraînait avec elle vers des
perspectives inconnues. Puis elle disparut. Il souhaita passionnément la revoir
une troisième fois. Et elle reparut en effet mais sans lui parler plus
clairement, en lui causant même une volupté moins profonde. Mais rentré chez lui
il eut besoin d’elle, il était comme un homme dans la vie de qui une passante
qu’il a aperçue un moment vient de faire entrer l’image d’une beauté nouvelle
qui donne à sa propre sensibilité une valeur plus grande, sans qu’il sache
seulement s’il pourra revoir jamais celle qu’il aime déjà et dont il ignore
jusqu’au nom.

Même cet amour pour une phrase musicale sembla un instant devoir amorcer chez
Swann la possibilité d’une sorte de rajeunissement. Depuis si longtemps il avait
renoncé à appliquer sa vie à un but idéal et la bornait à la poursuite de
satisfactions quotidiennes, qu’il croyait, sans jamais se le dire formellement,
que cela ne changerait plus jusqu’à sa mort; bien plus, ne se sentant plus
d’idées élevées dans l’esprit, il avait cessé de croire à leur réalité, sans
pouvoir non plus la nier tout à fait. Aussi avait-il pris l’habitude de se
réfugier dans des pensées sans importance qui lui permettaient de laisser de
côté le fond des choses. De même qu’il ne se demandait pas s’il n’eût pas mieux
fait de ne pas aller dans le monde, mais en revanche savait avec certitude que
s’il avait accepté une invitation il devait s’y rendre et que s’il ne faisait
pas de visite après il lui fallait laisser des cartes, de même dans sa
conversation il s’efforçait de ne jamais exprimer avec cœur une opinion intime
sur les choses, mais de fournir des détails matériels qui valaient en quelque
sorte par eux-mêmes et lui permettaient de ne pas donner sa mesure. Il était
extrêmement précis pour une recette de cuisine, pour la date de la naissance ou
de la mort d’un peintre, pour la nomenclature de ses œuvres. Parfois, malgré
tout, il se laissait aller à émettre un jugement sur une œuvre, sur une manière
de comprendre la vie, mais il donnait alors à ses paroles un ton ironique comme
s’il n’adhérait pas tout entier à ce qu’il disait. Or, comme certains
valétudinaires chez qui tout d’un coup, un pays où ils sont arrivés, un régime
différent, quelquefois une évolution organique, spontanée et mystérieuse,
semblent amener une telle régression de leur mal qu’ils commencent à envisager
la possibilité inespérée de commencer sur le tard une vie toute différente,
Swann trouvait en lui, dans le souvenir de la phrase qu’il avait entendue, dans
certaines sonates qu’il s’était fait jouer, pour voir s’il ne l’y découvrirait
pas, la présence d’une de ces réalités invisibles auxquelles il avait cessé de
croire et auxquelles, comme si la musique avait eu sur la sécheresse morale dont
il souffrait une sorte d’influence élective, il se sentait de nouveau le désir
et presque la force de consacrer sa vie. Mais n’étant pas arrivé à savoir de qui
était l’œuvre qu’il avait entendue, il n’avait pu se la procurer et avait fini
par l’oublier. Il avait bien rencontré dans la semaine quelques personnes qui se
trouvaient comme lui à cette soirée et les avait interrogées; mais plusieurs
étaient arrivées après la musique ou parties avant; certaines pourtant étaient
là pendant qu’on l’exécutait mais étaient allées causer dans un autre salon, et
d’autres restées à écouter n’avaient pas entendu plus que les premières. Quant
aux maîtres de maison ils savaient que c’était une œuvre nouvelle que les
artistes qu’ils avaient engagés avaient demandé à jouer; ceux-ci étant partis en
tournée, Swann ne put pas en savoir davantage. Il avait bien des amis musiciens,
mais tout en se rappelant le plaisir spécial et intraduisible que lui avait fait
la phrase, en voyant devant ses yeux les formes qu’elle dessinait, il était
pourtant incapable de la leur chanter. Puis il cessa d’y penser.

Or, quelques minutes à peine après que le petit pianiste avait commencé de jouer
chez Mme Verdurin, tout d’un coup après une note haute longuement tenue pendant
deux mesures, il vit approcher, s’échappant de sous cette sonorité prolongée et
tendue comme un rideau sonore pour cacher le mystère de son incubation, il
reconnut, secrète, bruissante et divisée, la phrase aérienne et odorante qu’il
aimait. Et elle était si particulière, elle avait un charme si individuel et
qu’aucun autre n’aurait pu remplacer, que ce fut pour Swann comme s’il eût
rencontré dans un salon ami une personne qu’il avait admirée dans la rue et
désespérait de jamais retrouver. A la fin, elle s’éloigna, indicatrice,
diligente, parmi les ramifications de son parfum, laissant sur le visage de
Swann le reflet de son sourire. Mais maintenant il pouvait demander le nom de
son inconnue (on lui dit que c’était l’andante de la sonate pour piano et violon
de Vinteuil), il la tenait, il pourrait l’avoir chez lui aussi souvent qu’il
voudrait, essayer d’apprendre son langage et son secret.

Aussi quand le pianiste eut fini, Swann s’approcha-t-il de lui pour lui exprimer
une reconnaissance dont la vivacité plut beaucoup à Mme Verdurin.

— Quel charmeur, n’est-ce pas, dit-elle à Swann; la comprend-il assez, sa
sonate, le petit misérable? Vous ne saviez pas que le piano pouvait atteindre à
ça. C’est tout excepté du piano, ma parole! Chaque fois j’y suis reprise, je
crois entendre un orchestre. C’est même plus beau que l’orchestre, plus complet.

Le jeune pianiste s’inclina, et, souriant, soulignant les mots comme s’il avait
fait un trait d’esprit:

—«Vous êtes très indulgente pour moi», dit-il.

Et tandis que Mme Verdurin disait à son mari: «Allons, donne-lui de l’orangeade,
il l’a bien méritée», Swann racontait à Odette comment il avait été amoureux de
cette petite phrase. Quand Mme Verdurin, ayant dit d’un peu loin: «Eh bien! il
me semble qu’on est en train de vous dire de belles choses, Odette», elle
répondit: «Oui, de très belles» et Swann trouva délicieuse sa simplicité.
Cependant il demandait des renseignements sur Vinteuil, sur son œuvre, sur
l’époque de sa vie où il avait composé cette sonate, sur ce qu’avait pu
signifier pour lui la petite phrase, c’est cela surtout qu’il aurait voulu
savoir.

Mais tous ces gens qui faisaient profession d’admirer ce musicien (quand Swann
avait dit que sa sonate était vraiment belle, Mme Verdurin s’était écriée: «Je
vous crois un peu qu’elle est belle! Mais on n’avoue pas qu’on ne connaît pas la
sonate de Vinteuil, on n’a pas le droit de ne pas la connaître», et le peintre
avait ajouté: «Ah! c’est tout à fait une très grande machine, n’est-ce pas. Ce
n’est pas si vous voulez la chose «cher» et «public», n’est-ce pas, mais c’est
la très grosse impression pour les artistes»), ces gens semblaient ne s’être
jamais posé ces questions car ils furent incapables d’y répondre.

Même à une ou deux remarques particulières que fit Swann sur sa phrase préférée:
—«Tiens, c’est amusant, je n’avais jamais fait attention; je vous dirai que je
n’aime pas beaucoup chercher la petite bête et m’égarer dans des pointes
d’aiguille; on ne perd pas son temps à couper les cheveux en quatre ici, ce
n’est pas le genre de la maison», répondit Mme Verdurin, que le docteur Cottard
regardait avec une admiration béate et un zèle studieux se jouer au milieu de ce
flot d’expressions toutes faites. D’ailleurs lui et Mme Cottard avec une sorte
de bon sens comme en ont aussi certaines gens du peuple se gardaient bien de
donner une opinion ou de feindre l’admiration pour une musique qu’ils
s’avouaient l’un à l’autre, une fois rentrés chez eux, ne pas plus comprendre
que la peinture de «M. Biche». Comme le public ne connaît du charme, de la
grâce, des formes de la nature que ce qu’il en a puisé dans les poncifs d’un art
lentement assimilé, et qu’un artiste original commence par rejeter ces poncifs,
M. et Mme Cottard, image en cela du public, ne trouvaient ni dans la sonate de
Vinteuil, ni dans les portraits du peintre, ce qui faisait pour eux l’harmonie
de la musique et la beauté de la peinture. Il leur semblait quand le pianiste
jouait la sonate qu’il accrochait au hasard sur le piano des notes que ne
reliaient pas en effet les formes auxquelles ils étaient habitués, et que le
peintre jetait au hasard des couleurs sur ses toiles. Quand, dans celles-ci, ils
pouvaient reconnaître une forme, ils la trouvaient alourdie et vulgarisée
(c’est-à-dire dépourvue de l’élégance de l’école de peinture à travers laquelle
ils voyaient dans la rue même, les êtres vivants), et sans vérité, comme si M.
Biche n’eût pas su comment était construite une épaule et que les femmes n’ont
pas les cheveux mauves.

Pourtant les fidèles s’étant dispersés, le docteur sentit qu’il y avait là une
occasion propice et pendant que Mme Verdurin disait un dernier mot sur la sonate
de Vinteuil, comme un nageur débutant qui se jette à l’eau pour apprendre, mais
choisit un moment où il n’y a pas trop de monde pour le voir:

— Alors, c’est ce qu’on appelle un musicien di primo cartello! s’écria-t-il avec
une brusque résolution.

Swann apprit seulement que l’apparition récente de la sonate de Vinteuil avait
produit une grande impression dans une école de tendances très avancées mais
était entièrement inconnue du grand public.

— Je connais bien quelqu’un qui s’appelle Vinteuil, dit Swann, en pensant au
professeur de piano des sœurs de ma grand’mère.

— C’est peut-être lui, s’écria Mme Verdurin.

— Oh! non, répondit Swann en riant. Si vous l’aviez vu deux minutes, vous ne
vous poseriez pas la question.

— Alors poser la question c’est la résoudre? dit le docteur.

— Mais ce pourrait être un parent, reprit Swann, cela serait assez triste, mais
enfin un homme de génie peut être le cousin d’une vieille bête. Si cela était,
j’avoue qu’il n’y a pas de supplice que je ne m’imposerais pour que la vieille
bête me présentât à l’auteur de la sonate: d’abord le supplice de fréquenter la
vieille bête, et qui doit être affreux.

Le peintre savait que Vinteuil était à ce moment très malade et que le docteur
Potain craignait de ne pouvoir le sauver.

— Comment, s’écria Mme Verdurin, il y a encore des gens qui se font soigner par
Potain!

— Ah! madame Verdurin, dit Cottard, sur un ton de marivaudage, vous oubliez que
vous parlez d’un de mes confères, je devrais dire un de mes maîtres.

Le peintre avait entendu dire que Vinteuil était menacé d’aliénation mentale. Et
il assurait qu’on pouvait s’en apercevoir à certains passages de sa sonate.
Swann ne trouva pas cette remarque absurde, mais elle le troubla; car une œuvre
de musique pure ne contenant aucun des rapports logiques dont l’altération dans
le langage dénonce la folie, la folie reconnue dans une sonate lui paraissait
quelque chose d’aussi mystérieux que la folie d’une chienne, la folie d’un
cheval, qui pourtant s’observent en effet.

— Laissez-moi donc tranquille avec vos maîtres, vous en savez dix fois autant
que lui, répondit Mme Verdurin au docteur Cottard, du ton d’une personne qui a
le courage de ses opinions et tient bravement tête à ceux qui ne sont pas du
même avis qu’elle. Vous ne tuez pas vos malades, vous, au moins!

— Mais, Madame, il est de l’Académie, répliqua le docteur d’un ton air ironique.
Si un malade préfère mourir de la main d’un des princes de la science . . .
C’est beaucoup plus chic de pouvoir dire: «C’est Potain qui me soigne.»

— Ah! c’est plus chic? dit Mme Verdurin. Alors il y a du chic dans les maladies,
maintenant? je ne savais pas ça . . . Ce que vous m’amusez, s’écria-t-elle tout
à coup en plongeant sa figure dans ses mains. Et moi, bonne bête qui discutais
sérieusement sans m’apercevoir que vous me faisiez monter à l’arbre.

Quant à M. Verdurin, trouvant que c’était un peu fatigant de se mettre à rire
pour si peu, il se contenta de tirer une bouffée de sa pipe en songeant avec
tristesse qu’il ne pouvait plus rattraper sa femme sur le terrain de
l’amabilité.

— Vous savez que votre ami nous plaît beaucoup, dit Mme Verdurin à Odette au
moment où celle-ci lui souhaitait le bonsoir. Il est simple, charmant; si vous
n’avez jamais à nous présenter que des amis comme cela, vous pouvez les amener.

M. Verdurin fit remarquer que pourtant Swann n’avait pas apprécié la tante du
pianiste.

— Il s’est senti un peu dépaysé, cet homme, répondit Mme Verdurin, tu ne
voudrais pourtant pas que, la première fois, il ait déjà le ton de la maison
comme Cottard qui fait partie de notre petit clan depuis plusieurs années. La
première fois ne compte pas, c’était utile pour prendre langue. Odette, il est
convenu qu’il viendra nous retrouver demain au Châtelet. Si vous alliez le
prendre?

— Mais non, il ne veut pas.

— Ah! enfin, comme vous voudrez. Pourvu qu’il n’aille pas lâcher au dernier
moment!

A la grande surprise de Mme Verdurin, il ne lâcha jamais. Il allait les
rejoindre n’importe où, quelquefois dans les restaurants de banlieue où on
allait peu encore, car ce n’était pas la saison, plus souvent au théâtre, que
Mme Verdurin aimait beaucoup, et comme un jour, chez elle, elle dit devant lui
que pour les soirs de premières, de galas, un coupe-file leur eût été fort
utile, que cela les avait beaucoup gênés de ne pas en avoir le jour de
l’enterrement de Gambetta, Swann qui ne parlait jamais de ses relations
brillantes, mais seulement de celles mal cotées qu’il eût jugé peu délicat de
cacher, et au nombre desquelles il avait pris dans le faubourg Saint-Germain
l’habitude de ranger les relations avec le monde officiel, répondit:

— Je vous promets de m’en occuper, vous l’aurez à temps pour la reprise des
Danicheff, je déjeune justement demain avec le Préfet de police à l’Elysée.

— Comment ça, à l’Elysée? cria le docteur Cottard d’une voix tonnante.

— Oui, chez M. Grévy, répondit Swann, un peu gêné de l’effet que sa phrase avait
produit.

Et le peintre dit au docteur en manière de plaisanterie:

—Ça vous prend souvent?

Généralement, une fois l’explication donnée, Cottard disait: «Ah! bon, bon, ça
va bien» et ne montrait plus trace d’émotion.

Mais cette fois-ci, les derniers mots de Swann, au lieu de lui procurer
l’apaisement habituel, portèrent au comble son étonnement qu’un homme avec qui
il dînait, qui n’avait ni fonctions officielles, ni illustration d’aucune sorte,
frayât avec le Chef de l’État.

— Comment ça, M. Grévy? vous connaissez M. Grévy? dit-il à Swann de l’air
stupide et incrédule d’un municipal à qui un inconnu demande à voir le Président
de la République et qui, comprenant par ces mots «à qui il a affaire», comme
disent les journaux, assure au pauvre dément qu’il va être reçu à l’instant et
le dirige sur l’infirmerie spéciale du dépôt.

— Je le connais un peu, nous avons des amis communs (il n’osa pas dire que
c’était le prince de Galles), du reste il invite très facilement et je vous
assure que ces déjeuners n’ont rien d’amusant, ils sont d’ailleurs très simples,
on n’est jamais plus de huit à table, répondit Swann qui tâchait d’effacer ce
que semblaient avoir de trop éclatant aux yeux de son interlocuteur, des
relations avec le Président de la République.

Aussitôt Cottard, s’en rapportant aux paroles de Swann, adopta cette opinion, au
sujet de la valeur d’une invitation chez M. Grévy, que c’était chose fort peu
recherchée et qui courait les rues. Dès lors il ne s’étonna plus que Swann,
aussi bien qu’un autre, fréquentât l’Elysée, et même il le plaignait un peu
d’aller à des déjeuners que l’invité avouait lui-même être ennuyeux.

—«Ah! bien, bien, ça va bien», dit-il sur le ton d’un douanier, méfiant tout à
l’heure, mais qui, après vos explications, vous donne son visa et vous laisse
passer sans ouvrir vos malles.

—«Ah! je vous crois qu’ils ne doivent pas être amusants ces déjeuners, vous avez
de la vertu d’y aller, dit Mme Verdurin, à qui le Président de la République
apparaissait comme un ennuyeux particulièrement redoutable parce qu’il disposait
de moyens de séduction et de contrainte qui, employés à l’égard des fidèles,
eussent été capables de les faire lâcher. Il paraît qu’il est sourd comme un pot
et qu’il mange avec ses doigts.»

—«En effet, alors, cela ne doit pas beaucoup vous amuser d’y aller», dit le
docteur avec une nuance de commisération; et, se rappelant le chiffre de huit
convives: «Sont-ce des déjeuners intimes?» demanda-t-il vivement avec un zèle de
linguiste plus encore qu’une curiosité de badaud.

Mais le prestige qu’avait à ses yeux le Président de la République finit
pourtant par triompher et de l’humilité de Swann et de la malveillance de Mme
Verdurin, et à chaque dîner, Cottard demandait avec intérêt: «Verrons-nous ce
soir M. Swann? Il a des relations personnelles avec M. Grévy. C’est bien ce
qu’on appelle un gentleman?» Il alla même jusqu’à lui offrir une carte
d’invitation pour l’exposition dentaire.

—«Vous serez admis avec les personnes qui seront avec vous, mais on ne laisse
pas entrer les chiens. Vous comprenez je vous dis cela parce que j’ai eu des
amis qui ne le savaient pas et qui s’en sont mordu les doigts.»

Quant à M. Verdurin il remarqua le mauvais effet qu’avait produit sur sa femme
cette découverte que Swann avait des amitiés puissantes dont il n’avait jamais
parlé.

Si l’on n’avait pas arrangé une partie au dehors, c’est chez les Verdurin que
Swann retrouvait le petit noyau, mais il ne venait que le soir et n’acceptait
presque jamais à dîner malgré les instances d’Odette.

—«Je pourrais même dîner seule avec vous, si vous aimiez mieux cela», lui
disait-elle.

—«Et Mme Verdurin?»

—«Oh! ce serait bien simple. Je n’aurais qu’à dire que ma robe n’a pas été
prête, que mon cab est venu en retard. Il y a toujours moyen de s’arranger.

—«Vous êtes gentille.»

Mais Swann se disait que s’il montrait à Odette (en consentant seulement à la
retrouver après dîner), qu’il y avait des plaisirs qu’il préférait à celui
d’être avec elle, le goût qu’elle ressentait pour lui ne connaîtrait pas de
longtemps la satiété. Et, d’autre part, préférant infiniment à celle d’Odette,
la beauté d’une petite ouvrière fraîche et bouffie comme une rose et dont il
était épris, il aimait mieux passer le commencement de la soirée avec elle,
étant sûr de voir Odette ensuite. C’est pour les mêmes raisons qu’il n’acceptait
jamais qu’Odette vînt le chercher pour aller chez les Verdurin. La petite
ouvrière l’attendait près de chez lui à un coin de rue que son cocher Rémi
connaissait, elle montait à côté de Swann et restait dans ses bras jusqu’au
moment où la voiture l’arrêtait devant chez les Verdurin. A son entrée, tandis
que Mme Verdurin montrant des roses qu’il avait envoyées le matin lui disait:
«Je vous gronde» et lui indiquait une place à côté d’Odette, le pianiste jouait
pour eux deux, la petite phrase de Vinteuil qui était comme l’air national de
leur amour. Il commençait par la tenue des trémolos de violon que pendant
quelques mesures on entend seuls, occupant tout le premier plan, puis tout d’un
coup ils semblaient s’écarter et comme dans ces tableaux de Pieter De Hooch,
qu’approfondit le cadre étroit d’une porte entr’ouverte, tout au loin, d’une
couleur autre, dans le velouté d’une lumière interposée, la petite phrase
apparaissait, dansante, pastorale, intercalée, épisodique, appartenant à un
autre monde. Elle passait à plis simples et immortels, distribuant çà et là les
dons de sa grâce, avec le même ineffable sourire; mais Swann y croyait
distinguer maintenant du désenchantement. Elle semblait connaître la vanité de
ce bonheur dont elle montrait la voie. Dans sa grâce légère, elle avait quelque
chose d’accompli, comme le détachement qui succède au regret. Mais peu lui
importait, il la considérait moins en elle-même — en ce qu’elle pouvait exprimer
pour un musicien qui ignorait l’existence et de lui et d’Odette quand il l’avait
composée, et pour tous ceux qui l’entendraient dans des siècles — que comme un
gage, un souvenir de son amour qui, même pour les Verdurin que pour le petit
pianiste, faisait penser à Odette en même temps qu’à lui, les unissait; c’était
au point que, comme Odette, par caprice, l’en avait prié, il avait renoncé à son
projet de se faire jouer par un artiste la sonate entière, dont il continua à ne
connaître que ce passage. «Qu’avez-vous besoin du reste? lui avait-elle dit.
C’est ça notre morceau.» Et même, souffrant de songer, au moment où elle passait
si proche et pourtant à l’infini, que tandis qu’elle s’adressait à eux, elle ne
les connaissait pas, il regrettait presque qu’elle eût une signification, une
beauté intrinsèque et fixe, étrangère à eux, comme en des bijoux donnés, ou même
en des lettres écrites par une femme aimée, nous en voulons à l’eau de la gemme,
et aux mots du langage, de ne pas être faits uniquement de l’essence d’une
liaison passagère et d’un être particulier.

Souvent il se trouvait qu’il s’était tant attardé avec la jeune ouvrière avant
d’aller chez les Verdurin, qu’une fois la petite phrase jouée par le pianiste,
Swann s’apercevait qu’il était bientôt l’heure qu’Odette rentrât. Il la
reconduisait jusqu’à la porte de son petit hôtel, rue La Pérouse, derrière l’Arc
de Triomphe. Et c’était peut-être à cause de cela, pour ne pas lui demander
toutes les faveurs, qu’il sacrifiait le plaisir moins nécessaire pour lui de la
voir plus tôt, d’arriver chez les Verdurin avec elle, à l’exercice de ce droit
qu’elle lui reconnaissait de partir ensemble et auquel il attachait plus de
prix, parce que, grâce à cela, il avait l’impression que personne ne la voyait,
ne se mettait entre eux, ne l’empêchait d’être encore avec lui, après qu’il
l’avait quittée.

Ainsi revenait-elle dans la voiture de Swann; un soir comme elle venait d’en
descendre et qu’il lui disait à demain, elle cueillit précipitamment dans le
petit jardin qui précédait la maison un dernier chrysanthème et le lui donna
avant qu’il fût reparti. Il le tint serré contre sa bouche pendant le retour, et
quand au bout de quelques jours la fleur fut fanée, il l’enferma précieusement
dans son secrétaire.

Mais il n’entrait jamais chez elle. Deux fois seulement, dans l’après-midi, il
était allé participer à cette opération capitale pour elle «prendre le thé».
L’isolement et le vide de ces courtes rues (faites presque toutes de petits
hôtels contigus, dont tout à coup venait rompre la monotonie quelque sinistre
échoppe, témoignage historique et reste sordide du temps où ces quartiers
étaient encore mal famés), la neige qui était restée dans le jardin et aux
arbres, le négligé de la saison, le voisinage de la nature, donnaient quelque
chose de plus mystérieux à la chaleur, aux fleurs qu’il avait trouvées en
entrant.

Laissant à gauche, au rez-de-chaussée surélevé, la chambre à coucher d’Odette
qui donnait derrière sur une petite rue parallèle, un escalier droit entre des
murs peints de couleur sombre et d’où tombaient des étoffes orientales, des fils
de chapelets turcs et une grande lanterne japonaise suspendue à une cordelette
de soie (mais qui, pour ne pas priver les visiteurs des derniers conforts de la
civilisation occidentale s’éclairait au gaz), montait au salon et au petit
salon. Ils étaient précédés d’un étroit vestibule dont le mur quadrillé d’un
treillage de jardin, mais doré, était bordé dans toute sa longueur d’une caisse
rectangulaire où fleurissaient comme dans une serre une rangée de ces gros
chrysanthèmes encore rares à cette époque, mais bien éloignés cependant de ceux
que les horticulteurs réussirent plus tard à obtenir. Swann était agacé par la
mode qui depuis l’année dernière se portait sur eux, mais il avait eu plaisir,
cette fois, à voir la pénombre de la pièce zébrée de rose, d’orangér et de blanc
par les rayons odorants de ces astres éphémères qui s’allument dans les jours
gris. Odette l’avait reçu en robe de chambre de soie rose, le cou et les bras
nus. Elle l’avait fait asseoir près d’elle dans un des nombreux retraits
mystérieux qui étaient ménagés dans les enfoncements du salon, protégés par
d’immenses palmiers contenus dans des cache-pot de Chine, ou par des paravents
auxquels étaient fixés des photographies, des nœuds de rubans et des éventails.
Elle lui avait dit: «Vous n’êtes pas confortable comme cela, attendez, moi je
vais bien vous arranger», et avec le petit rire vaniteux qu’elle aurait eu pour
quelque invention particulière à elle, avait installé derrière la tête de Swann,
sous ses pieds, des coussins de soie japonaise qu’elle pétrissait comme si elle
avait été prodigue de ces richesses et insoucieuse de leur valeur. Mais quand le
valet de chambre était venu apporter successivement les nombreuses lampes qui,
presque toutes enfermées dans des potiches chinoises, brûlaient isolées ou par
couples, toutes sur des meubles différents comme sur des autels et qui dans le
crépuscule déjà presque nocturne de cette fin d’après-midi d’hiver avaient fait
reparaître un coucher de soleil plus durable, plus rose et plus humain — faisant
peut-être rêver dans la rue quelque amoureux arrêté devant le mystère de la
présence que décelaient et cachaient à la fois les vitres rallumées — elle avait
surveillé sévèrement du coin de l’œil le domestique pour voir s’il les posait
bien à leur place consacrée. Elle pensait qu’en en mettant une seule là où il ne
fallait pas, l’effet d’ensemble de son salon eût été détruit, et son portrait,
placé sur un chevalet oblique drapé de peluche, mal éclairé. Aussi suivait-elle
avec fièvre les mouvements de cet homme grossier et le réprimanda-t-elle
vivement parce qu’il avait passé trop près de deux jardinières qu’elle se
réservait de nettoyer elle-même dans sa peur qu’on ne les abîmât et qu’elle alla
regarder de près pour voir s’il ne les avait pas écornées. Elle trouvait à tous
ses bibelots chinois des formes «amusantes», et aussi aux orchidées, aux
catleyas surtout, qui étaient, avec les chrysanthèmes, ses fleurs préférées,
parce qu’ils avaient le grand mérite de ne pas ressembler à des fleurs, mais
d’être en soie, en satin. «Celle-là a l’air d’être découpée dans la doublure de
mon manteau», dit-elle à Swann en lui montrant une orchidée, avec une nuance
d’estime pour cette fleur si «chic», pour cette sœur élégante et imprévue que la
nature lui donnait, si loin d’elle dans l’échelle des êtres et pourtant
raffinée, plus digne que bien des femmes qu’elle lui fit une place dans son
salon. En lui montrant tour à tour des chimères à langues de feu décorant une
potiche ou brodées sur un écran, les corolles d’un bouquet d’orchidées, un
dromadaire d’argent niellé aux yeux incrustés de rubis qui voisinait sur la
cheminée avec un crapaud de jade, elle affectait tour à tour d’avoir peur de la
méchanceté, ou de rire de la cocasserie des monstres, de rougir de l’indécence
des fleurs et d’éprouver un irrésistible désir d’aller embrasser le dromadaire
et le crapaud qu’elle appelait: «chéris». Et ces affectations contrastaient avec
la sincérité de certaines de ses dévotions, notamment à Notre-Dame du Laghet qui
l’avait jadis, quand elle habitait Nice, guérie d’une maladie mortelle et dont
elle portait toujours sur elle une médaille d’or à laquelle elle attribuait un
pouvoir sans limites. Odette fit à Swann «son» thé, lui demanda: «Citron ou
crème?» et comme il répondit «crème», lui dit en riant: «Un nuage!» Et comme il
le trouvait bon: «Vous voyez que je sais ce que vous aimez.» Ce thé en effet
avait paru à Swann quelque chose de précieux comme à elle-même et l’amour a
tellement besoin de se trouver une justification, une garantie de durée, dans
des plaisirs qui au contraire sans lui n’en seraient pas et finissent avec lui,
que quand il l’avait quittée à sept heures pour rentrer chez lui s’habiller,
pendant tout le trajet qu’il fit dans son coupé, ne pouvant contenir la joie que
cet après-midi lui avait causée, il se répétait: «Ce serait bien agréable
d’avoir ainsi une petite personne chez qui on pourrait trouver cette chose si
rare, du bon thé.» Une heure après, il reçut un mot d’Odette, et reconnut tout
de suite cette grande écriture dans laquelle une affectation de raideur
britannique imposait une apparence de discipline à des caractères informes qui
eussent signifié peut-être pour des yeux moins prévenus le désordre de la
pensée, l’insuffisance de l’éducation, le manque de franchise et de volonté.
Swann avait oublié son étui à cigarettes chez Odette. «Que n’y avez-vous oublié
aussi votre cœur, je ne vous aurais pas laissé le reprendre.»

Une seconde visite qu’il lui fit eut plus d’importance peut-être. En se rendant
chez elle ce jour-là comme chaque fois qu’il devait la voir d’avance, il se la
représentait; et la nécessité où il était pour trouver jolie sa figure de
limiter aux seules pommettes roses et fraîches, les joues qu’elle avait si
souvent jaunes, languissantes, parfois piquées de petits points rouges,
l’affligeait comme une preuve que l’idéal est inaccessible et le bonheur
médiocre. Il lui apportait une gravure qu’elle désirait voir. Elle était un peu
souffrante; elle le reçut en peignoir de crêpe de Chine mauve, ramenant sur sa
poitrine, comme un manteau, une étoffe richement brodée. Debout à côté de lui,
laissant couler le long de ses joues ses cheveux qu’elle avait dénoués,
fléchissant une jambe dans une attitude légèrement dansante pour pouvoir se
pencher sans fatigue vers la gravure qu’elle regardait, en inclinant la tête, de
ses grands yeux, si fatigués et maussades quand elle ne s’animait pas, elle
frappa Swann par sa ressemblance avec cette figure de Zéphora, la fille de
Jéthro, qu’on voit dans une fresque de la chapelle Sixtine. Swann avait toujours
eu ce goût particulier d’aimer à retrouver dans la peinture des maîtres non pas
seulement les caractères généraux de la réalité qui nous entoure, mais ce qui
semble au contraire le moins susceptible de généralité, les traits individuels
des visages que nous connaissons: ainsi, dans la matière d’un buste du doge
Loredan par Antoine Rizzo, la saillie des pommettes, l’obliquité des sourcils,
enfin la ressemblance criante de son cocher Rémi; sous les couleurs d’un
Ghirlandajo, le nez de M. de Palancy; dans un portrait de Tintoret,
l’envahissement du gras de la joue par l’implantation des premiers poils des
favoris, la cassure du nez, la pénétration du regard, la congestion des
paupières du docteur du Boulbon. Peut-être ayant toujours gardé un remords
d’avoir borné sa vie aux relations mondaines, à la conversation, croyait-il
trouver une sorte d’indulgent pardon à lui accordé par les grands artistes, dans
ce fait qu’ils avaient eux aussi considéré avec plaisir, fait entrer dans leur
œuvre, de tels visages qui donnent à celle-ci un singulier certificat de réalité
et de vie, une saveur moderne; peut-être aussi s’était-il tellement laissé
gagner par la frivolité des gens du monde qu’il éprouvait le besoin de trouver
dans une œuvre ancienne ces allusions anticipées et rajeunissantes à des noms
propres d’aujourd’hui. Peut-être au contraire avait-il gardé suffisamment une
nature d’artiste pour que ces caractéristiques individuelles lui causassent du
plaisir en prenant une signification plus générale, dès qu’il les apercevait
déracinées, délivrées, dans la ressemblance d’un portrait plus ancien avec un
original qu’il ne représentait pas. Quoi qu’il en soit et peut-être parce que la
plénitude d’impressions qu’il avait depuis quelque temps et bien qu’elle lui fût
venue plutôt avec l’amour de la musique, avait enrichi même son goût pour la
peinture, le plaisir fut plus profond et devait exercer sur Swann une influence
durable, qu’il trouva à ce moment-là dans la ressemblance d’Odette avec la
Zéphora de ce Sandro di Mariano auquel on ne donne plus volontiers son surnom
populaire de Botticelli depuis que celui-ci évoque au lieu de l’œuvre véritable
du peintre l’idée banale et fausse qui s’en est vulgarisée. Il n’estima plus le
visage d’Odette selon la plus ou moins bonne qualité de ses joues et d’après la
douceur purement carnée qu’il supposait devoir leur trouver en les touchant avec
ses lèvres si jamais il osait l’embrasser, mais comme un écheveau de lignes
subtiles et belles que ses regards dévidèrent, poursuivant la courbe de leur
enroulement, rejoignant la cadence de la nuque à l’effusion des cheveux et à la
flexion des paupières, comme en un portrait d’elle en lequel son type devenait
intelligible et clair.

Il la regardait; un fragment de la fresque apparaissait dans son visage et dans
son corps, que dès lors il chercha toujours à y retrouver soit qu’il fût auprès
d’Odette, soit qu’il pensât seulement à elle, et bien qu’il ne tînt sans doute
au chef-d’œuvre florentin que parce qu’il le retrouvait en elle, pourtant cette
ressemblance lui conférait à elle aussi une beauté, la rendait plus précieuse.
Swann se reprocha d’avoir méconnu le prix d’un être qui eût paru adorable au
grand Sandro, et il se félicita que le plaisir qu’il avait à voir Odette trouvât
une justification dans sa propre culture esthétique. Il se dit qu’en associant
la pensée d’Odette à ses rêves de bonheur il ne s’était pas résigné à un
pis-aller aussi imparfait qu’il l’avait cru jusqu’ici, puisqu’elle contentait en
lui ses goûts d’art les plus raffinés. Il oubliait qu’Odette n’était pas plus
pour cela une femme selon son désir, puisque précisément son désir avait
toujours été orienté dans un sens opposé à ses goûts esthétiques. Le mot
d’«œuvre florentine» rendit un grand service à Swann. Il lui permit, comme un
titre, de faire pénétrer l’image d’Odette dans un monde de rêves, où elle
n’avait pas eu accès jusqu’ici et où elle s’imprégna de noblesse. Et tandis que
la vue purement charnelle qu’il avait eue de cette femme, en renouvelant
perpétuellement ses doutes sur la qualité de son visage, de son corps, de toute
sa beauté, affaiblissait son amour, ces doutes furent détruits, cet amour assuré
quand il eut à la place pour base les données d’une esthétique certaine; sans
compter que le baiser et la possession qui semblaient naturels et médiocres
s’ils lui étaient accordés par une chair abîmée, venant couronner l’adoration
d’une pièce de musée, lui parurent devoir être surnaturels et délicieux.

Et quand il était tenté de regretter que depuis des mois il ne fît plus que voir
Odette, il se disait qu’il était raisonnable de donner beaucoup de son temps à
un chef-d’œuvre inestimable, coulé pour une fois dans une matière différente et
particulièrement savoureuse, en un exemplaire rarissime qu’il contemplait tantôt
avec l’humilité, la spiritualité et le désintéressement d’un artiste, tantôt
avec l’orgueil, l’égoïsme et la sensualité d’un collectionneur.

Il plaça sur sa table de travail, comme une photographie d’Odette, une
reproduction de la fille de Jéthro. Il admirait les grands yeux, le délicat
visage qui laissait deviner la peau imparfaite, les boucles merveilleuses des
cheveux le long des joues fatiguées, et adaptant ce qu’il trouvait beau
jusque-là d’une façon esthétique à l’idée d’une femme vivante, il le
transformait en mérites physiques qu’il se félicitait de trouver réunis dans un
être qu’il pourrait posséder. Cette vague sympathie qui nous porte vers un
chef-d’œuvre que nous regardons, maintenant qu’il connaissait l’original charnel
de la fille de Jéthro, elle devenait un désir qui suppléa désormais à celui que
le corps d’Odette ne lui avait pas d’abord inspiré. Quand il avait regardé
longtemps ce Botticelli, il pensait à son Botticelli à lui qu’il trouvait plus
beau encore et approchant de lui la photographie de Zéphora, il croyait serrer
Odette contre son cœur.

Et cependant ce n’était pas seulement la lassitude d’Odette qu’il s’ingéniait à
prévenir, c’était quelquefois aussi la sienne propre; sentant que depuis
qu’Odette avait toutes facilités pour le voir, elle semblait n’avoir pas
grand’chose à lui dire, il craignait que les façons un peu insignifiantes,
monotones, et comme définitivement fixées, qui étaient maintenant les siennes
quand ils étaient ensemble, ne finissent par tuer en lui cet espoir romanesque
d’un jour où elle voudrait déclarer sa passion, qui seul l’avait rendu et gardé
amoureux. Et pour renouveler un peu l’aspect moral, trop figé, d’Odette, et dont
il avait peur de se fatiguer, il lui écrivait tout d’un coup une lettre pleine
de déceptions feintes et de colères simulées qu’il lui faisait porter avant le
dîner. Il savait qu’elle allait être effrayée, lui répondre et il espérait que
dans la contraction que la peur de le perdre ferait subir à son âme,
jailliraient des mots qu’elle ne lui avait encore jamais dits; et en effet c’est
de cette façon qu’il avait obtenu les lettres les plus tendres qu’elle lui eût
encore écrites dont l’une, qu’elle lui avait fait porter à midi de la «Maison
Dorée» (c’était le jour de la fête de Paris-Murcie donnée pour les inondés de
Murcie), commençait par ces mots: «Mon ami, ma main tremble si fort que je peux
à peine écrire», et qu’il avait gardée dans le même tiroir que la fleur séchée
du chrysanthème. Ou bien si elle n’avait pas eu le temps de lui écrire, quand il
arriverait chez les Verdurin, elle irait vivement à lui et lui dirait: «J’ai à
vous parler», et il contemplerait avec curiosité sur son visage et dans ses
paroles ce qu’elle lui avait caché jusque-là de son cœur.

Rien qu’en approchant de chez les Verdurin quand il apercevait, éclairées par
des lampes, les grandes fenêtres dont on ne fermait jamais les volets, il
s’attendrissait en pensant à l’être charmant qu’il allait voir épanoui dans leur
lumière d’or. Parfois les ombres des invités se détachaient minces et noires, en
écran, devant les lampes, comme ces petites gravures qu’on intercale de place en
place dans un abat-jour translucide dont les autres feuillets ne sont que
clarté. Il cherchait à distinguer la silhouette d’Odette. Puis, dès qu’il était
arrivé, sans qu’il s’en rendit compte, ses yeux brillaient d’une telle joie que
M. Verdurin disait au peintre: «Je crois que ça chauffe.» Et la présence
d’Odette ajoutait en effet pour Swann à cette maison ce dont n’était pourvue
aucune de celles où il était reçu: une sorte d’appareil sensitif, de réseau
nerveux qui se ramifiait dans toutes les pièces et apportait des excitations
constantes à son cœur.

Ainsi le simple fonctionnement de cet organisme social qu’était le petit «clan»
prenait automatiquement pour Swann des rendez-vous quotidiens avec Odette et lui
permettait de feindre une indifférence à la voir, ou même un désir de ne plus la
voir, qui ne lui faisait pas courir de grands risques, puisque, quoi qu’il lui
eût écrit dans la journée, il la verrait forcément le soir et la ramènerait chez
elle.

Mais une fois qu’ayant songé avec maussaderie à cet inévitable retour ensemble,
il avait emmené jusqu’au bois sa jeune ouvrière pour retarder le moment d’aller
chez les Verdurin, il arriva chez eux si tard qu’Odette, croyant qu’il ne
viendrait plus, était partie. En voyant qu’elle n’était plus dans le salon,
Swann ressentit une souffrance au cœur; il tremblait d’être privé d’un plaisir
qu’il mesurait pour la première fois, ayant eu jusque-là cette certitude de le
trouver quand il le voulait, qui pour tous les plaisirs nous diminue ou même
nous empêche d’apercevoir aucunement leur grandeur.

—«As-tu vu la tête qu’il a fait quand il s’est aperçu qu’elle n’était pas là?
dit M. Verdurin à sa femme, je crois qu’on peut dire qu’il est pincé!»

—«La tête qu’il a fait?» demanda avec violence le docteur Cottard qui, étant
allé un instant voir un malade, revenait chercher sa femme et ne savait pas de
qui on parlait.

—«Comment vous n’avez pas rencontré devant la porte le plus beau des Swann»?

—«Non. M. Swann est venu»?

— Oh! un instant seulement. Nous avons eu un Swann très agité, très nerveux.
Vous comprenez, Odette était partie.

—«Vous voulez dire qu’elle est du dernier bien avec lui, qu’elle lui a fait voir
l’heure du berger», dit le docteur, expérimentant avec prudence le sens de ces
expressions.

—«Mais non, il n’y a absolument rien, et entre nous, je trouve qu’elle a bien
tort et qu’elle se conduit comme une fameuse cruche, qu’elle est du reste.»

—«Ta, ta, ta, dit M. Verdurin, qu’est-ce que tu en sais qu’il n’y a rien, nous
n’avons pas été y voir, n’est-ce pas.»

—«A moi, elle me l’aurait dit, répliqua fièrement Mme Verdurin. Je vous dis
qu’elle me raconte toutes ses petites affaires! Comme elle n’a plus personne en
ce moment, je lui ai dit qu’elle devrait coucher avec lui. Elle prétend qu’elle
ne peut pas, qu’elle a bien eu un fort béguin pour lui mais qu’il est timide
avec elle, que cela l’intimide à son tour, et puis qu’elle ne l’aime pas de
cette manière-là, que c’est un être idéal, qu’elle a peur de déflorer le
sentiment qu’elle a pour lui, est-ce que je sais, moi. Ce serait pourtant
absolument ce qu’il lui faut.»

—«Tu me permettras de ne pas être de ton avis, dit M. Verdurin, il ne me revient
qu’à demi ce monsieur; je le trouve poseur.»

Mme Verdurin s’immobilisa, prit une expression inerte comme si elle était
devenue une statue, fiction qui lui permit d’être censée ne pas avoir entendu ce
mot insupportable de poseur qui avait l’air d’impliquer qu’on pouvait «poser»
avec eux, donc qu’on était «plus qu’eux».

—«Enfin, s’il n’y a rien, je ne pense pas que ce soit que ce monsieur la croit
vertueuse, dit ironiquement M. Verdurin. Et après tout, on ne peut rien dire,
puisqu’il a l’air de la croire intelligente. Je ne sais si tu as entendu ce
qu’il lui débitait l’autre soir sur la sonate de Vinteuil; j’aime Odette de tout
mon cœur, mais pour lui faire des théories d’esthétique, il faut tout de même
être un fameux jobard!»

—«Voyons, ne dites pas du mal d’Odette, dit Mme Verdurin en faisant l’enfant.
Elle est charmante.»

—«Mais cela ne l’empêche pas d’être charmante; nous ne disons pas du mal d’elle,
nous disons que ce n’est pas une vertu ni une intelligence. Au fond, dit-il au
peintre, tenez-vous tant que ça à ce qu’elle soit vertueuse? Elle serait
peut-être beaucoup moins charmante, qui sait?»

Sur le palier, Swann avait été rejoint par le maître d’hôtel qui ne se trouvait
pas là au moment où il était arrivé et avait été chargé par Odette de lui dire —
mais il y avait bien une heure déjà — au cas où il viendrait encore, qu’elle
irait probablement prendre du chocolat chez Prévost avant de rentrer. Swann
partit chez Prévost, mais à chaque pas sa voiture était arrêtée par d’autres ou
par des gens qui traversaient, odieux obstacles qu’il eût été heureux de
renverser si le procès-verbal de l’agent ne l’eût retardé plus encore que le
passage du piéton. Il comptait le temps qu’il mettait, ajoutait quelques
secondes à toutes les minutes pour être sûr de ne pas les avoir faites trop
courtes, ce qui lui eût laissé croire plus grande qu’elle n’était en réalité sa
chance d’arriver assez tôt et de trouver encore Odette. Et à un moment, comme un
fiévreux qui vient de dormir et qui prend conscience de l’absurdité des
rêvasseries qu’il ruminait sans se distinguer nettement d’elles, Swann tout d’un
coup aperçut en lui l’étrangeté des pensées qu’il roulait depuis le moment où on
lui avait dit chez les Verdurin qu’Odette était déjà partie, la nouveauté de la
douleur au cœur dont il souffrait, mais qu’il constata seulement comme s’il
venait de s’éveiller. Quoi? toute cette agitation parce qu’il ne verrait Odette
que demain, ce que précisément il avait souhaité, il y a une heure, en se
rendant chez Mme Verdurin. Il fut bien obligé de constater que dans cette même
voiture qui l’emmenait chez Prévost, il n’était plus le même, et qu’il n’était
plus seul, qu’un être nouveau était là avec lui, adhérent, amalgamé à lui,
duquel il ne pourrait peut-être pas se débarrasser, avec qui il allait être
obligé d’user de ménagements comme avec un maître ou avec une maladie. Et
pourtant depuis un moment qu’il sentait qu’une nouvelle personne s’était ainsi
ajoutée à lui, sa vie lui paraissait plus intéressante. C’est à peine s’il se
disait que cette rencontre possible chez Prévost (de laquelle l’attente
saccageait, dénudait à ce point les moments qui la précédaient qu’il ne trouvait
plus une seule idée, un seul souvenir derrière lequel il pût faire reposer son
esprit), il était probable pourtant, si elle avait lieu, qu’elle serait comme
les autres, fort peu de chose. Comme chaque soir, dès qu’il serait avec Odette,
jetant furtivement sur son changeant visage un regard aussitôt détourné de peur
qu’elle n’y vît l’avance d’un désir et ne crût plus à son désintéressement, il
cesserait de pouvoir penser à elle, trop occupé à trouver des prétextes qui lui
permissent de ne pas la quitter tout de suite et de s’assurer, sans avoir l’air
d’y tenir, qu’il la retrouverait le lendemain chez les Verdurin: c’est-à-dire de
prolonger pour l’instant et de renouveler un jour de plus la déception et la
torture que lui apportait la vaine présence de cette femme qu’il approchait sans
oser l’étreindre.

Elle n’était pas chez Prévost; il voulut chercher dans tous les restaurants des
boulevards. Pour gagner du temps, pendant qu’il visitait les uns, il envoya dans
les autres son cocher Rémi (le doge Loredan de Rizzo) qu’il alla attendre
ensuite — n’ayant rien trouvé lui-même —à l’endroit qu’il lui avait désigné. La
voiture ne revenait pas et Swann se représentait le moment qui approchait, à la
fois comme celui où Rémi lui dirait: «Cette dame est là», et comme celui où Rémi
lui dirait, «cette dame n’était dans aucun des cafés.» Et ainsi il voyait la fin
de la soirée devant lui, une et pourtant alternative, précédée soit par la
rencontre d’Odette qui abolirait son angoisse, soit, par le renoncement forcé à
la trouver ce soir, par l’acceptation de rentrer chez lui sans l’avoir vue.

Le cocher revint, mais, au moment où il s’arrêta devant Swann, celui-ci ne lui
dit pas: «Avez-vous trouvé cette dame?» mais: «Faites-moi donc penser demain à
commander du bois, je crois que la provision doit commencer à s’épuiser.»
Peut-être se disait-il que si Rémi avait trouvé Odette dans un café où elle
l’attendait, la fin de la soirée néfaste était déjà anéantie par la réalisation
commencée de la fin de soirée bienheureuse et qu’il n’avait pas besoin de se
presser d’atteindre un bonheur capturé et en lieu sûr, qui ne s’échapperait
plus. Mais aussi c’était par force d’inertie; il avait dans l’âme le manque de
souplesse que certains êtres ont dans le corps, ceux-là qui au moment d’éviter
un choc, d’éloigner une flamme de leur habit, d’accomplir un mouvement urgent,
prennent leur temps, commencent par rester une seconde dans la situation où ils
étaient auparavant comme pour y trouver leur point d’appui, leur élan. Et sans
doute si le cocher l’avait interrompu en lui disant: «Cette dame est là», il eut
répondu: «Ah! oui, c’est vrai, la course que je vous avais donnée, tiens je
n’aurais pas cru», et aurait continué à lui parler provision de bois pour lui
cacher l’émotion qu’il avait eue et se laisser à lui-même le temps de rompre
avec l’inquiétude et de se donner au bonheur.

Mais le cocher revint lui dire qu’il ne l’avait trouvée nulle part, et ajouta
son avis, en vieux serviteur:

— Je crois que Monsieur n’a plus qu’à rentrer.

Mais l’indifférence que Swann jouait facilement quand Rémi ne pouvait plus rien
changer à la réponse qu’il apportait tomba, quand il le vit essayer de le faire
renoncer à son espoir et à sa recherche:

—«Mais pas du tout, s’écria-t-il, il faut que nous trouvions cette dame; c’est
de la plus haute importance. Elle serait extrêmement ennuyée, pour une affaire,
et froissée, si elle ne m’avait pas vu.»

—«Je ne vois pas comment cette dame pourrait être froissée, répondit Rémi,
puisque c’est elle qui est partie sans attendre Monsieur, qu’elle a dit qu’elle
allait chez Prévost et qu’elle n’y était pas,»

D’ailleurs on commençait à éteindre partout. Sous les arbres des boulevards,
dans une obscurité mystérieuse, les passants plus rares erraient, à peine
reconnaissables. Parfois l’ombre d’une femme qui s’approchait de lui, lui
murmurant un mot à l’oreille, lui demandant de la ramener, fit tressaillir
Swann. Il frôlait anxieusement tous ces corps obscurs comme si parmi les
fantômes des morts, dans le royaume sombre, il eût cherché Eurydice.

De tous les modes de production de l’amour, de tous les agents de dissémination
du mal sacré, il est bien l’un des plus efficaces, ce grand souffle d’agitation
qui parfois passe sur nous. Alors l’être avec qui nous nous plaisons à ce
moment-là, le sort en est jeté, c’est lui que nous aimerons. Il n’est même pas
besoin qu’il nous plût jusque-là plus ou même autant que d’autres. Ce qu’il
fallait, c’est que notre goût pour lui devint exclusif. Et cette condition-là
est réalisée quand —à ce moment où il nous fait défaut —à la recherche des
plaisirs que son agrément nous donnait, s’est brusquement substitué en nous un
besoin anxieux, qui a pour objet cet être même, un besoin absurde, que les lois
de ce monde rendent impossible à satisfaire et difficile à guérir — le besoin
insensé et douloureux de le posséder.

Swann se fit conduire dans les derniers restaurants; c’est la seule hypothèse du
bonheur qu’il avait envisagée avec calme; il ne cachait plus maintenant son
agitation, le prix qu’il attachait à cette rencontre et il promit en cas de
succès une récompense à son cocher, comme si en lui inspirant le désir de
réussir qui viendrait s’ajouter à celui qu’il en avait lui-même, il pouvait
faire qu’Odette, au cas où elle fût déjà rentrée se coucher, se trouvât pourtant
dans un restaurant du boulevard. Il poussa jusqu’à la Maison Dorée, entra deux
fois chez Tortoni et, sans l’avoir vue davantage, venait de ressortir du Café
Anglais, marchant à grands pas, l’air hagard, pour rejoindre sa voiture qui
l’attendait au coin du boulevard des Italiens, quand il heurta une personne qui
venait en sens contraire: c’était Odette; elle lui expliqua plus tard que
n’ayant pas trouvé de place chez Prévost, elle était allée souper à la Maison
Dorée dans un enfoncement où il ne l’avait pas découverte, et elle regagnait sa
voiture.

Elle s’attendait si peu à le voir qu’elle eut un mouvement d’effroi. Quant à
lui, il avait couru Paris non parce qu’il croyait possible de la rejoindre, mais
parce qu’il lui était trop cruel d’y renoncer. Mais cette joie que sa raison
n’avait cessé d’estimer, pour ce soir, irréalisable, ne lui en paraissait
maintenant que plus réelle; car, il n’y avait pas collaboré par la prévision des
vraisemblances, elle lui restait extérieure; il n’avait pas besoin de tirer de
son esprit pour la lui fournir — c’est d’elle-même qu’émanait, c’est elle-même
qui projetait vers lui — cette vérité qui rayonnait au point de dissiper comme
un songe l’isolement qu’il avait redouté, et sur laquelle il appuyait, il
reposait, sans penser, sa rêverie heureuse. Ainsi un voyageur arrivé par un beau
temps au bord de la Méditerranée, incertain de l’existence des pays qu’il vient
de quitter, laisse éblouir sa vue, plutôt qu’il ne leur jette des regards, par
les rayons qu’émet vers lui l’azur lumineux et résistant des eaux.

Il monta avec elle dans la voiture qu’elle avait et dit à la sienne de suivre.

Elle tenait à la main un bouquet de catleyas et Swann vit, sous sa fanchon de
dentelle, qu’elle avait dans les cheveux des fleurs de cette même orchidée
attachées à une aigrette en plumes de cygnes. Elle était habillée sous sa
mantille, d’un flot de velours noir qui, par un rattrapé oblique, découvrait en
un large triangle le bas d’une jupe de faille blanche et laissait voir un
empiècement, également de faille blanche, à l’ouverture du corsage décolleté, où
étaient enfoncées d’autres fleurs de catleyas. Elle était à peine remise de la
frayeur que Swann lui avait causée quand un obstacle fit faire un écart au
cheval. Ils furent vivement déplacés, elle avait jeté un cri et restait toute
palpitante, sans respiration.

—«Ce n’est rien, lui dit-il, n’ayez pas peur.»

Et il la tenait par l’épaule, l’appuyant contre lui pour la maintenir; puis il
lui dit:

— Surtout, ne me parlez pas, ne me répondez que par signes pour ne pas vous
essouffler encore davantage. Cela ne vous gêne pas que je remette droites les
fleurs de votre corsage qui ont été déplacées par le choc. J’ai peur que vous ne
les perdiez, je voudrais les enfoncer un peu.

Elle, qui n’avait pas été habituée à voir les hommes faire tant de façons avec
elle, dit en souriant:

—«Non, pas du tout, ça ne me gêne pas.»

Mais lui, intimidé par sa réponse, peut-être aussi pour avoir l’air d’avoir été
sincère quand il avait pris ce prétexte, ou même, commençant déjà à croire qu’il
l’avait été, s’écria:

—«Oh! non, surtout, ne parlez pas, vous allez encore vous essouffler, vous
pouvez bien me répondre par gestes, je vous comprendrai bien. Sincèrement je ne
vous gêne pas? Voyez, il y a un peu . . . je pense que c’est du pollen qui s’est
répandu sur vous, vous permettez que je l’essuie avec ma main? Je ne vais pas
trop fort, je ne suis pas trop brutal? Je vous chatouille peut-être un peu? mais
c’est que je ne voudrais pas toucher le velours de la robe pour ne pas le
friper. Mais, voyez-vous, il était vraiment nécessaire de les fixer ils seraient
tombés; et comme cela, en les enfonçant un peu moi-même . . . Sérieusement, je
ne vous suis pas désagréable? Et en les respirant pour voir s’ils n’ont vraiment
pas d’odeur non plus? Je n’en ai jamais senti, je peux? dites la vérité.»?

Souriant, elle haussa légèrement les épaules, comme pour dire «vous êtes fou,
vous voyez bien que ça me plaît».

Il élevait son autre main le long de la joue d’Odette; elle le regarda fixement,
de l’air languissant et grave qu’ont les femmes du maître florentin avec
lesquelles il lui avait trouvé de la ressemblance; amenés au bord des paupières,
ses yeux brillants, larges et minces, comme les leurs, semblaient prêts à se
détacher ainsi que deux larmes. Elle fléchissait le cou comme on leur voit faire
à toutes, dans les scènes païennes comme dans les tableaux religieux. Et, en une
attitude qui sans doute lui était habituelle, qu’elle savait convenable à ces
moments-là et qu’elle faisait attention à ne pas oublier de prendre, elle
semblait avoir besoin de toute sa force pour retenir son visage, comme si une
force invisible l’eût attiré vers Swann. Et ce fut Swann, qui, avant qu’elle le
laissât tomber, comme malgré elle, sur ses lèvres, le retint un instant, à
quelque distance, entre ses deux mains. Il avait voulu laisser à sa pensée le
temps d’accourir, de reconnaître le rêve qu’elle avait si longtemps caressé et
d’assister à sa réalisation, comme une parente qu’on appelle pour prendre sa
part du succès d’un enfant qu’elle a beaucoup aimé. Peut-être aussi Swann
attachait-il sur ce visage d’Odette non encore possédée, ni même encore
embrassée par lui, qu’il voyait pour la dernière fois, ce regard avec lequel, un
jour de départ, on voudrait emporter un paysage qu’on va quitter pour toujours.

Mais il était si timide avec elle, qu’ayant fini par la posséder ce soir-là, en
commençant par arranger ses catleyas, soit crainte de la froisser, soit peur de
paraître rétrospectivement avoir menti, soit manque d’audace pour formuler une
exigence plus grande que celle-là (qu’il pouvait renouveler puisqu’elle n’avait
pas fiché Odette la première fois), les jours suivants il usa du même prétexte.
Si elle avait des catleyas à son corsage, il disait: «C’est malheureux, ce soir,
les catleyas n’ont pas besoin d’être arrangés, ils n’ont pas été déplacés comme
l’autre soir; il me semble pourtant que celui-ci n’est pas très droit. Je peux
voir s’ils ne sentent pas plus que les autres?» Ou bien, si elle n’en avait pas:
«Oh! pas de catleyas ce soir, pas moyen de me livrer à mes petits arrangements.»
De sorte que, pendant quelque temps, ne fut pas changé l’ordre qu’il avait suivi
le premier soir, en débutant par des attouchements de doigts et de lèvres sur la
gorge d’Odette et que ce fut par eux encore que commençaient chaque fois ses
caresses; et, bien plus tard quand l’arrangement (ou le simulacre d’arrangement)
des catleyas, fut depuis longtemps tombé en désuétude, la métaphore «faire
catleya», devenue un simple vocable qu’ils employaient sans y penser quand ils
voulaient signifier l’acte de la possession physique — où d’ailleurs l’on ne
possède rien — survécut dans leur langage, où elle le commémorait, à cet usage
oublié. Et peut-être cette manière particulière de dire «faire l’amour» ne
signifiait-elle pas exactement la même chose que ses synonymes. On a beau être
blasé sur les femmes, considérer la possession des plus différentes comme
toujours la même et connue d’avance, elle devient au contraire un plaisir
nouveau s’il s’agit de femmes assez difficiles — ou crues telles par nous — pour
que nous soyons obligés de la faire naître de quelque épisode imprévu de nos
relations avec elles, comme avait été la première fois pour Swann l’arrangement
des catleyas. Il espérait en tremblant, ce soir-là (mais Odette, se disait-il,
si elle était dupe de sa ruse, ne pouvait le deviner), que c’était la possession
de cette femme qui allait sortir d’entre leurs larges pétales mauves; et le
plaisir qu’il éprouvait déjà et qu’Odette ne tolérait peut-être, pensait-il, que
parce qu’elle ne l’avait pas reconnu, lui semblait, à cause de cela — comme il
put paraître au premier homme qui le goûta parmi les fleurs du paradis terrestre
— un plaisir qui n’avait pas existé jusque-là, qu’il cherchait à créer, un
plaisir — ainsi que le nom spécial qu’il lui donna en garda la trace —
entièrement particulier et nouveau.

Maintenant, tous les soirs, quand il l’avait ramenée chez elle, il fallait qu’il
entrât et souvent elle ressortait en robe de chambre et le conduisait jusqu’à sa
voiture, l’embrassait aux yeux du cocher, disant: «Qu’est-ce que cela peut me
faire, que me font les autres?» Les soirs où il n’allait pas chez les Verdurin
(ce qui arrivait parfois depuis qu’il pouvait la voir autrement), les soirs de
plus en plus rares où il allait dans le monde, elle lui demandait de venir chez
elle avant de rentrer, quelque heure qu’il fût. C’était le printemps, un
printemps pur et glacé. En sortant de soirée, il montait dans sa victoria,
étendait une couverture sur ses jambes, répondait aux amis qui s’en allaient en
même temps que lui et lui demandaient de revenir avec eux qu’il ne pouvait pas,
qu’il n’allait pas du même côté, et le cocher partait au grand trot sachant où
on allait. Eux s’étonnaient, et de fait, Swann n’était plus le même. On ne
recevait plus jamais de lettre de lui où il demandât à connaître une femme. Il
ne faisait plus attention à aucune, s’abstenait d’aller dans les endroits où on
en rencontre. Dans un restaurant, à la campagne, il avait l’attitude inversée de
celle à quoi, hier encore, on l’eût reconnu et qui avait semblé devoir toujours
être la sienne. Tant une passion est en nous comme un caractère momentané et
différent qui se substitue à l’autre et abolit les signes jusque-là invariables
par lesquels il s’exprimait! En revanche ce qui était invariable maintenant,
c’était que où que Swann se trouvât, il ne manquât pas d’aller rejoindre Odette.
Le trajet qui le séparait d’elle était celui qu’il parcourait inévitablement et
comme la pente même irrésistible et rapide de sa vie. A vrai dire, souvent resté
tard dans le monde, il aurait mieux aimé rentrer directement chez lui sans faire
cette longue course et ne la voir que le lendemain; mais le fait même de se
déranger à une heure anormale pour aller chez elle, de deviner que les amis qui
le quittaient se disaient: «Il est très tenu, il y a certainement une femme qui
le force à aller chez elle à n’importe quelle heure», lui faisait sentir qu’il
menait la vie des hommes qui ont une affaire amoureuse dans leur existence, et
en qui le sacrifice qu’ils font de leur repos et de leurs intérêts à une rêverie
voluptueuse fait naître un charme intérieur. Puis sans qu’il s’en rendît compte,
cette certitude qu’elle l’attendait, qu’elle n’était pas ailleurs avec d’autres,
qu’il ne reviendrait pas sans l’avoir vue, neutralisait cette angoisse oubliée
mais toujours prête à renaître qu’il avait éprouvée le soir où Odette n’était
plus chez les Verdurin et dont l’apaisement actuel était si doux que cela
pouvait s’appeler du bonheur. Peut-être était-ce à cette angoisse qu’il était
redevable de l’importance qu’Odette avait prise pour lui. Les êtres nous sont
d’habitude si indifférents, que quand nous avons mis dans l’un d’eux de telles
possibilités de souffrance et de joie, pour nous il nous semble appartenir à un
autre univers, il s’entoure de poésie, il fait de notre vie comme une étendue
émouvante où il sera plus ou moins rapproché de nous. Swann ne pouvait se
demander sans trouble ce qu’Odette deviendrait pour lui dans les années qui
allaient venir. Parfois, en voyant, de sa victoria, dans ces belles nuits
froides, la lune brillante qui répandait sa clarté entre ses yeux et les rues
désertes, il pensait à cette autre figure claire et légèrement rosée comme celle
de la lune, qui, un jour, avait surgi dans sa pensée et, depuis projetait sur le
monde la lumière mystérieuse dans laquelle il le voyait. S’il arrivait après
l’heure où Odette envoyait ses domestiques se coucher, avant de sonner à la
porte du petit jardin, il allait d’abord dans la rue, où donnait au
rez-de-chaussée, entre les fenêtres toutes pareilles, mais obscures, des hôtels
contigus, la fenêtre, seule éclairée, de sa chambre. Il frappait au carreau, et
elle, avertie, répondait et allait l’attendre de l’autre côté, à la porte
d’entrée. Il trouvait ouverts sur son piano quelques-uns des morceaux qu’elle
préférait: la Valse des Roses ou Pauvre fou de Tagliafico (qu’on devait, selon
sa volonté écrite, faire exécuter à son enterrement), il lui demandait de jouer
à la place la petite phrase de la sonate de Vinteuil, bien qu’Odette jouât fort
mal, mais la vision la plus belle qui nous reste d’une œuvre est souvent celle
qui s’éleva au-dessus des sons faux tirés par des doigts malhabiles, d’un piano
désaccordé. La petite phrase continuait à s’associer pour Swann à l’amour qu’il
avait pour Odette. Il sentait bien que cet amour, c’était quelque chose qui ne
correspondait à rien d’extérieur, de constatable par d’autres que lui; il se
rendait compte que les qualités d’Odette ne justifiaient pas qu’il attachât tant
de prix aux moments passés auprès d’elle. Et souvent, quand c’était
l’intelligence positive qui régnait seule en Swann, il voulait cesser de
sacrifier tant d’intérêts intellectuels et sociaux à ce plaisir imaginaire. Mais
la petite phrase, dès qu’il l’entendait, savait rendre libre en lui l’espace qui
pour elle était nécessaire, les proportions de l’âme de Swann s’en trouvaient
changées; une marge y était réservée à une jouissance qui elle non plus ne
correspondait à aucun objet extérieur et qui pourtant au lieu d’être purement
individuelle comme celle de l’amour, s’imposait à Swann comme une réalité
supérieure aux choses concrètes. Cette soif d’un charme inconnu, la petite
phrase l’éveillait en lui, mais ne lui apportait rien de précis pour l’assouvir.
De sorte que ces parties de l’âme de Swann où la petite phrase avait effacé le
souci des intérêts matériels, les considérations humaines et valables pour tous,
elle les avait laissées vacantes et en blanc, et il était libre d’y inscrire le
nom d’Odette. Puis à ce que l’affection d’Odette pouvait avoir d’un peu court et
décevant, la petite phrase venait ajouter, amalgamer son essence mystérieuse. A
voir le visage de Swann pendant qu’il écoutait la phrase, on aurait dit qu’il
était en train d’absorber un anesthésique qui donnait plus d’amplitude à sa
respiration. Et le plaisir que lui donnait la musique et qui allait bientôt
créer chez lui un véritable besoin, ressemblait en effet, à ces moments-là, au
plaisir qu’il aurait eu à expérimenter des parfums, à entrer en contact avec un
monde pour lequel nous ne sommes pas faits, qui nous semble sans forme parce que
nos yeux ne le perçoivent pas, sans signification parce qu’il échappe à notre
intelligence, que nous n’atteignons que par un seul sens. Grand repos,
mystérieuse rénovation pour Swann — pour lui dont les yeux quoique délicats
amateurs de peinture, dont l’esprit quoique fin observateur de mœurs, portaient
à jamais la trace indélébile de la sécheresse de sa vie — de se sentir
transformé en une créature étrangère à l’humanité, aveugle, dépourvue de
facultés logiques, presque une fantastique licorne, une créature chimérique ne
percevant le monde que par l’ouïe. Et comme dans la petite phrase il cherchait
cependant un sens où son intelligence ne pouvait descendre, quelle étrange
ivresse il avait à dépouiller son âme la plus intérieure de tous les secours du
raisonnement et à la faire passer seule dans le couloir, dans le filtre obscur
du son. Il commençait à se rendre compte de tout ce qu’il y avait de douloureux,
peut-être même de secrètement inapaisé au fond de la douceur de cette phrase,
mais il ne pouvait pas en souffrir. Qu’importait qu’elle lui dît que l’amour est
fragile, le sien était si fort! Il jouait avec la tristesse qu’elle répandait,
il la sentait passer sur lui, mais comme une caresse qui rendait plus profond et
plus doux le sentiment qu’il avait de son bonheur. Il la faisait rejouer dix
fois, vingt fois à Odette, exigeant qu’en même temps elle ne cessât pas de
l’embrasser. Chaque baiser appelle un autre baiser. Ah! dans ces premiers temps
où l’on aime, les baisers naissent si naturellement! Ils foisonnent si pressés
les uns contre les autres; et l’on aurait autant de peine à compter les baisers
qu’on s’est donnés pendant une heure que les fleurs d’un champ au mois de mai.
Alors elle faisait mine de s’arrêter, disant: «Comment veux-tu que je joue comme
cela si tu me tiens, je ne peux tout faire à la fois, sache au moins ce que tu
veux, est-ce que je dois jouer la phrase ou faire des petites caresses», lui se
fâchait et elle éclatait d’un rire qui se changeait et retombait sur lui, en une
pluie de baisers. Ou bien elle le regardait d’un air maussade, il revoyait un
visage digne de figurer dans la Vie de Moïse de Botticelli, il l’y situait, il
donnait au cou d’Odette l’inclinaison nécessaire; et quand il l’avait bien
peinte à la détrempe, au XVe siècle, sur la muraille de la Sixtine, l’idée
qu’elle était cependant restée là, près du piano, dans le moment actuel, prête à
être embrassée et possédée, l’idée de sa matérialité et de sa vie venait
l’enivrer avec une telle force que, l’œil égaré, les mâchoires tendues comme
pour dévorer, il se précipitait sur cette vierge de Botticelli et se mettait à
lui pincer les joues. Puis, une fois qu’il l’avait quittée, non sans être rentré
pour l’embrasser encore parce qu’il avait oublié d’emporter dans son souvenir
quelque particularité de son odeur ou de ses traits, tandis qu’il revenait dans
sa victoria, bénissant Odette de lui permettre ces visites quotidiennes, dont il
sentait qu’elles ne devaient pas lui causer à elle une bien grande joie, mais
qui en le preservant de devenir jaloux — en lui ôtant l’occasion de souffrir de
nouveau du mal qui s’était déclaré en lui le soir où il ne l’avait pas trouvée
chez les Verdurin — l’aideraient à arriver, sans avoir plus d’autres de ces
crises dont la première avait été si douloureuse et resterait la seule, au bout
de ces heures singulières de sa vie, heures presque enchantées, à la façon de
celles où il traversait Paris au clair de lune. Et, remarquant, pendant ce
retour, que l’astre était maintenant déplacé par rapport à lui, et presque au
bout de l’horizon, sentant que son amour obéissait, lui aussi, à des lois
immuables et naturelles, il se demandait si cette période où il était entré
durerait encore longtemps, si bientôt sa pensée ne verrait plus le cher visage
qu’occupant une position lointaine et diminuée, et près de cesser de répandre du
charme. Car Swann en trouvait aux choses, depuis qu’il était amoureux, comme au
temps où, adolescent, il se croyait artiste; mais ce n’était plus le même
charme, celui-ci c’est Odette seule qui le leur conférait. Il sentait renaître
en lui les inspirations de sa jeunesse qu’une vie frivole avait dissipées, mais
elles portaient toutes le reflet, la marque d’un être particulier; et, dans les
longues heures qu’il prenait maintenant un plaisir délicat à passer chez lui,
seul avec son âme en convalescence, il redevenait peu à peu lui-même, mais à une
autre.

Il n’allait chez elle que le soir, et il ne savait rien de l’emploi de son temps
pendant le jour, pas plus que de son passé, au point qu’il lui manquait même ce
petit renseignement initial qui, en nous permettant de nous imaginer ce que nous
ne savons pas, nous donne envie de le connaître. Aussi ne se demandait-il pas ce
qu’elle pouvait faire, ni quelle avait été sa vie. Il souriait seulement
quelquefois en pensant qu’il y a quelques années, quand il ne la connaissait
pas, on lui avait parlé d’une femme, qui, s’il se rappelait bien, devait
certainement être elle, comme d’une fille, d’une femme entretenue, une de ces
femmes auxquelles il attribuait encore, comme il avait peu vécu dans leur
société, le caractère entier, foncièrement pervers, dont les dota longtemps
l’imagination de certains romanciers. Il se disait qu’il n’y a souvent qu’à
prendre le contre-pied des réputations que fait le monde pour juger exactement
une personne, quand, à un tel caractère, il opposait celui d’Odette, bonne,
naïve, éprise d’idéal, presque si incapable de ne pas dire la vérité, que,
l’ayant un jour priée, pour pouvoir dîner seul avec elle, d’écrire aux Verdurin
qu’elle était souffrante, le lendemain, il l’avait vue, devant Mme Verdurin qui
lui demandait si elle allait mieux, rougir, balbutier et refléter malgré elle,
sur son visage, le chagrin, le supplice que cela lui était de mentir, et, tandis
qu’elle multipliait dans sa réponse les détails inventés sur sa prétendue
indisposition de la veille, avoir l’air de faire demander pardon par ses regards
suppliants et sa voix désolée de la fausseté de ses paroles.

Certains jours pourtant, mais rares, elle venait chez lui dans l’après-midi,
interrompre sa rêverie ou cette étude sur Ver Meer à laquelle il s’était remis
dernièrement. On venait lui dire que Mme de Crécy était dans son petit salon. Il
allait l’y retrouver, et quand il ouvrait la porte, au visage rosé d’Odette, dès
qu’elle avait aperçu Swann, venait — changeant la forme de sa bouche, le regard
de ses yeux, le modelé de ses joues — se mélanger un sourire. Une fois seul, il
revoyait ce sourire, celui qu’elle avait eu la veille, un autre dont elle
l’avait accueilli telle ou telle fois, celui qui avait été sa réponse, en
voiture, quand il lui avait demandé s’il lui était désagréable en redressant les
catleyas; et la vie d’Odette pendant le reste du temps, comme il n’en
connaissait rien, lui apparaissait avec son fond neutre et sans couleur,
semblable à ces feuilles d’études de Watteau, où on voit çà et là, à toutes les
places, dans tous les sens, dessinés aux trois crayons sur le papier chamois,
d’innombrables sourires. Mais, parfois, dans un coin de cette vie que Swann
voyait toute vide, si même son esprit lui disait qu’elle ne l’était pas, parce
qu’il ne pouvait pas l’imaginer, quelque ami, qui, se doutant qu’ils s’aimaient,
ne se fût pas risqué à lui rien dire d’elle que d’insignifiant, lui décrivait la
silhouette d’Odette, qu’il avait aperçue, le matin même, montant à pied la rue
Abbatucci dans une «visite» garnie de skunks, sous un chapeau «à la Rembrandt»
et un bouquet de violettes à son corsage. Ce simple croquis bouleversait Swann
parce qu’il lui faisait tout d’un coup apercevoir qu’Odette avait une vie qui
n’était pas tout entière à lui; il voulait savoir à qui elle avait cherché à
plaire par cette toilette qu’il ne lui connaissait pas; il se promettait de lui
demander où elle allait, à ce moment-là, comme si dans toute la vie incolore —
presque inexistante, parce qu’elle lui était invisible — de sa maîtresse, il n’y
avait qu’une seule chose en dehors de tous ces sourires adressés à lui: sa
démarche sous un chapeau à la Rembrandt, avec un bouquet de violettes au
corsage.

Sauf en lui demandant la petite phrase de Vinteuil au lieu de la Valse des
Roses, Swann ne cherchait pas à lui faire jouer plutôt des choses qu’il aimât,
et pas plus en musique qu’en littérature, à corriger son mauvais goût. Il se
rendait bien compte qu’elle n’était pas intelligente. En lui disant qu’elle
aimerait tant qu’il lui parlât des grands poètes, elle s’était imaginé qu’elle
allait connaître tout de suite des couplets héroïques et romanesques dans le
genre de ceux du vicomte de Borelli, en plus émouvant encore. Pour Ver Meer de
Delft, elle lui demanda s’il avait souffert par une femme, si c’était une femme
qui l’avait inspiré, et Swann lui ayant avoué qu’on n’en savait rien, elle
s’était désintéressée de ce peintre. Elle disait souvent: «Je crois bien, la
poésie, naturellement, il n’y aurait rien de plus beau si c’était vrai, si les
poètes pensaient tout ce qu’ils disent. Mais bien souvent, il n’y a pas plus
intéressé que ces gens-là. J’en sais quelque chose, j’avais une amie qui a aimé
une espèce de poète. Dans ses vers il ne parlait que de l’amour, du ciel, des
étoiles. Ah! ce qu’elle a été refaite! Il lui a croqué plus de trois cent mille
francs.» Si alors Swann cherchait à lui apprendre en quoi consistait la beauté
artistique, comment il fallait admirer les vers ou les tableaux, au bout d’un
instant, elle cessait d’écouter, disant: «Oui . . . je ne me figurais pas que
c’était comme cela.» Et il sentait qu’elle éprouvait une telle déception qu’il
préférait mentir en lui disant que tout cela n’était rien, que ce n’était encore
que des bagatelles, qu’il n’avait pas le temps d’aborder le fond, qu’il y avait
autre chose. Mais elle lui disait vivement: «Autre chose? quoi? . . . Dis-le
alors», mais il ne le disait pas, sachant combien cela lui paraîtrait mince et
différent de ce qu’elle espérait, moins sensationnel et moins touchant, et
craignant que, désillusionnée de l’art, elle ne le fût en même temps de l’amour.

Et en effet elle trouvait Swann, intellectuellement, inférieur à ce qu’elle
aurait cru. «Tu gardes toujours ton sang-froid, je ne peux te définir.» Elle
s’émerveillait davantage de son indifférence à l’argent, de sa gentillesse pour
chacun, de sa délicatesse. Et il arrive en effet souvent pour de plus grands que
n’était Swann, pour un savant, pour un artiste, quand il n’est pas méconnu par
ceux qui l’entourent, que celui de leurs sentiments qui prouve que la
supériorité de son intelligence s’est imposée à eux, ce n’est pas leur
admiration pour ses idées, car elles leur échappent, mais leur respect pour sa
bonté. C’est aussi du respect qu’inspirait à Odette la situation qu’avait Swann
dans le monde, mais elle ne désirait pas qu’il cherchât à l’y faire recevoir.
Peut-être sentait-elle qu’il ne pourrait pas y réussir, et même craignait-elle,
que rien qu’en parlant d’elle, il ne provoquât des révélations qu’elle
redoutait. Toujours est-il qu’elle lui avait fait promettre de ne jamais
prononcer son nom. La raison pour laquelle elle ne voulait pas aller dans le
monde, lui avait-elle dit, était une brouille qu’elle avait eue autrefois avec
une amie qui, pour se venger, avait ensuite dit du mal d’elle. Swann objectait:
«Mais tout le monde n’a pas connu ton amie.»—«Mais si, ça fait la tache d’huile,
le monde est si méchant.» D’une part Swann ne comprit pas cette histoire, mais
d’autre part il savait que ces propositions: «Le monde est si méchant», «un
propos calomnieux fait la tache d’huile», sont généralement tenues pour vraies;
il devait y avoir des cas auxquels elles s’appliquaient. Celui d’Odette était-il
l’un de ceux-là? Il se le demandait, mais pas longtemps, car il était sujet, lui
aussi, à cette lourdeur d’esprit qui s’appesantissait sur son père, quand il se
posait un problème difficile. D’ailleurs, ce monde qui faisait si peur à Odette,
ne lui inspirait peut-être pas de grands désirs, car pour qu’elle se le
représentât bien nettement, il était trop éloigné de celui qu’elle connaissait.
Pourtant, tout en étant restée à certains égards vraiment simple (elle avait par
exemple gardé pour amie une petite couturière retirée dont elle grimpait presque
chaque jour l’escalier raide, obscur et fétide), elle avait soif de chic, mais
ne s’en faisait pas la même idée que les gens du monde. Pour eux, le chic est
une émanation de quelques personnes peu nombreuses qui le projettent jusqu’à un
degré assez éloigné

— et plus ou moins affaibli dans la mesure où l’on est distant du centre de leur
intimité — dans le cercle de leurs amis ou des amis de leurs amis dont les noms
forment une sorte de répertoire. Les gens du monde le possèdent dans leur
mémoire, ils ont sur ces matières une érudition d’où ils ont extrait une sorte
de goût, de tact, si bien que Swann par exemple, sans avoir besoin de faire
appel à son savoir mondain, s’il lisait dans un journal les noms des personnes
qui se trouvaient à un dîner pouvait dire immédiatement la nuance du chic de ce
dîner, comme un lettré, à la simple lecture d’une phrase, apprécie exactement la
qualité littéraire de son auteur. Mais Odette faisait partie des personnes
(extrêmement nombreuses quoi qu’en pensent les gens du monde, et comme il y en a
dans toutes les classes de la société), qui ne possèdent pas ces notions,
imaginent un chic tout autre, qui revêt divers aspects selon le milieu auquel
elles appartiennent, mais a pour caractère particulier — que ce soit celui dont
rêvait Odette, ou celui devant lequel s’inclinait Mme Cottard — d’être
directement accessible à tous. L’autre, celui des gens du monde, l’est à vrai
dire aussi, mais il y faut quelque délai. Odette disait de quelqu’un:

—«Il ne va jamais que dans les endroits chics.»

Et si Swann lui demandait ce qu’elle entendait par là, elle lui répondait avec
un peu de mépris:

—«Mais les endroits chics, parbleu! Si, à ton âge, il faut t’apprendre ce que
c’est que les endroits chics, que veux-tu que je te dise, moi, par exemple, le
dimanche matin, l’avenue de l’Impératrice, à cinq heures le tour du Lac, le
jeudi l’Éden Théâtre, le vendredi l’Hippodrome, les bals . . . »

— Mais quels bals?

—«Mais les bals qu’on donne à Paris, les bals chics, je veux dire. Tiens,
Herbinger, tu sais, celui qui est chez un coulissier? mais si, tu dois savoir,
c’est un des hommes les plus lancés de Paris, ce grand jeune homme blond qui est
tellement snob, il a toujours une fleur à la boutonnière, une raie dans le dos,
des paletots clairs; il est avec ce vieux tableau qu’il promène à toutes les
premières. Eh bien! il a donné un bal, l’autre soir, il y avait tout ce qu’il y
a de chic à Paris. Ce que j’aurais aimé y aller! mais il fallait présenter sa
carte d’invitation à la porte et je n’avais pas pu en avoir. Au fond j’aime
autant ne pas y être allée, c’était une tuerie, je n’aurais rien vu. C’est
plutôt pour pouvoir dire qu’on était chez Herbinger. Et tu sais, moi, la
gloriole! Du reste, tu peux bien te dire que sur cent qui racontent qu’elles y
étaient, il y a bien la moitié dont ça n’est pas vrai . . . Mais ça m’étonne que
toi, un homme si «pschutt», tu n’y étais pas.»

Mais Swann ne cherchait nullement à lui faire modifier cette conception du chic;
pensant que la sienne n’était pas plus vraie, était aussi sotte, dénuée
d’importance, il ne trouvait aucun intérêt à en instruire sa maîtresse, si bien
qu’après des mois elle ne s’intéressait aux personnes chez qui il allait que
pour les cartes de pesage, de concours hippique, les billets de première qu’il
pouvait avoir par elles. Elle souhaitait qu’il cultivât des relations si utiles
mais elle était par ailleurs, portée à les croire peu chic, depuis qu’elle avait
vu passer dans la rue la marquise de Villeparisis en robe de laine noire, avec
un bonnet à brides.

— Mais elle a l’air d’une ouvreuse, d’une vieille concierge, darling! Ça, une
marquise! Je ne suis pas marquise, mais il faudrait me payer bien cher pour me
faire sortir nippée comme ça!

Elle ne comprenait pas que Swann habitât l’hôtel du quai d’Orléans que, sans
oser le lui avouer, elle trouvait indigne de lui.

Certes, elle avait la prétention d’aimer les «antiquités» et prenait un air ravi
et fin pour dire qu’elle adorait passer toute une journée à «bibeloter», à
chercher «du bric-à-brac», des choses «du temps». Bien qu’elle s’entêtât dans
une sorte de point d’honneur (et semblât pratiquer quelque précepte familial) en
ne répondant jamais aux questions et en ne «rendant pas de comptes» sur l’emploi
de ses journées, elle parla une fois à Swann d’une amie qui l’avait invitée et
chez qui tout était «de l’époque». Mais Swann ne put arriver à lui faire dire
quelle était cette époque. Pourtant, après avoir réfléchi, elle répondit que
c’était «moyenâgeux». Elle entendait par là qu’il y avait des boiseries. Quelque
temps après, elle lui reparla de son amie et ajouta, sur le ton hésitant et de
l’air entendu dont on cite quelqu’un avec qui on a dîné la veille et dont on
n’avait jamais entendu le nom, mais que vos amphitryons avaient l’air de
considérer comme quelqu’un de si célèbre qu’on espère que l’interlocuteur saura
bien de qui vous voulez parler: «Elle a une salle à manger . . . du . . .
dix-huitième!» Elle trouvait du reste cela affreux, nu, comme si la maison
n’était pas finie, les femmes y paraissaient affreuses et la mode n’en prendrait
jamais. Enfin, une troisième fois, elle en reparla et montra à Swann l’adresse
de l’homme qui avait fait cette salle à manger et qu’elle avait envie de faire
venir, quand elle aurait de l’argent pour voir s’il ne pourrait pas lui en
faire, non pas certes une pareille, mais celle qu’elle rêvait et que,
malheureusement, les dimensions de son petit hôtel ne comportaient pas, avec de
hauts dressoirs, des meubles Renaissance et des cheminées comme au château de
Blois. Ce jour-là, elle laissa échapper devant Swann ce qu’elle pensait de son
habitation du quai d’Orléans; comme il avait critiqué que l’amie d’Odette donnât
non pas dans le Louis XVI, car, disait-il, bien que cela ne se fasse pas, cela
peut être charmant, mais dans le faux ancien: «Tu ne voudrais pas qu’elle vécût
comme toi au milieu de meubles cassés et de tapis usés», lui dit-elle, le
respect humain de la bourgeoise l’emportant encore chez elle sur le
dilettantisme de la cocotte.

De ceux qui aimaient à bibeloter, qui aimaient les vers, méprisaient les bas
calculs, rêvaient d’honneur et d’amour, elle faisait une élite supérieure au
reste de l’humanité. Il n’y avait pas besoin qu’on eût réellement ces goûts
pourvu qu’on les proclamât; d’un homme qui lui avait avoué à dîner qu’il aimait
à flâner, à se salir les doigts dans les vieilles boutiques, qu’il ne serait
jamais apprécié par ce siècle commercial, car il ne se souciait pas de ses
intérêts et qu’il était pour cela d’un autre temps, elle revenait en disant:
«Mais c’est une âme adorable, un sensible, je ne m’en étais jamais doutée!» et
elle se sentait pour lui une immense et soudaine amitié. Mais, en revanche ceux,
qui comme Swann, avaient ces goûts, mais n’en parlaient pas, la laissaient
froide. Sans doute elle était obligée d’avouer que Swann ne tenait pas à
l’argent, mais elle ajoutait d’un air boudeur: «Mais lui, ça n’est pas la même
chose»; et en effet, ce qui parlait à son imagination, ce n’était pas la
pratique du désintéressement, c’en était le vocabulaire.

Sentant que souvent il ne pouvait pas réaliser ce qu’elle rêvait, il cherchait
du moins à ce qu’elle se plût avec lui, à ne pas contrecarrer ces idées
vulgaires, ce mauvais goût qu’elle avait en toutes choses, et qu’il aimait
d’ailleurs comme tout ce qui venait d’elle, qui l’enchantaient même, car c’était
autant de traits particuliers grâce auxquels l’essence de cette femme lui
apparaissait, devenait visible. Aussi, quand elle avait l’air heureux parce
qu’elle devait aller à la Reine Topaze, ou que son regard devenait sérieux,
inquiet et volontaire, si elle avait peur de manquer la rite des fleurs ou
simplement l’heure du thé, avec muffins et toasts, au «Thé de la Rue Royale» où
elle croyait que l’assiduité était indispensable pour consacrer la réputation
d’élégance d’une femme, Swann, transporté comme nous le sommes par le naturel
d’un enfant ou par la vérité d’un portrait qui semble sur le point de parler,
sentait si bien l’âme de sa maîtresse affleurer à son visage qu’il ne pouvait
résister à venir l’y toucher avec ses lèvres. «Ah! elle veut qu’on la mène à la
fête des fleurs, la petite Odette, elle veut se faire admirer, eh bien, on l’y
mènera, nous n’avons qu’à nous incliner.» Comme la vue de Swann était un peu
basse, il dut se résigner à se servir de lunettes pour travailler chez lui, et à
adopter, pour aller dans le monde, le monocle qui le défigurait moins. La
première fois qu’elle lui en vit un dans l’œil, elle ne put contenir sa joie:
«Je trouve que pour un homme, il n’y a pas à dire, ça a beaucoup de chic! Comme
tu es bien ainsi! tu as l’air d’un vrai gentleman. Il ne te manque qu’un titre!»
ajouta-t-elle, avec une nuance de regret. Il aimait qu’Odette fût ainsi, de même
que, s’il avait été épris d’une Bretonne, il aurait été heureux de la voir en
coiffe et de lui entendre dire qu’elle croyait aux revenants. Jusque-là, comme
beaucoup d’hommes chez qui leur goût pour les arts se développe indépendamment
de la sensualité, une disparate bizarre avait existé entre les satisfactions
qu’il accordait à l’un et à l’autre, jouissant, dans la compagnie de femmes de
plus en plus grossières, des séductions d’œuvres de plus en plus raffinées,
emmenant une petite bonne dans une baignoire grillée à la représentation d’une
pièce décadente qu’il avait envie d’entendre ou à une exposition de peinture
impressionniste, et persuadé d’ailleurs qu’une femme du monde cultivée n’y eut
pas compris davantage, mais n’aurait pas su se taire aussi gentiment. Mais, au
contraire, depuis qu’il aimait Odette, sympathiser avec elle, tâcher de n’avoir
qu’une âme à eux deux lui était si doux, qu’il cherchait à se plaire aux choses
qu’elle aimait, et il trouvait un plaisir d’autant plus profond non seulement à
imiter ses habitudes, mais à adopter ses opinions, que, comme elles n’avaient
aucune racine dans sa propre intelligence, elles lui rappelaient seulement son
amour, à cause duquel il les avait préférées. S’il retournait à Serge Panine,
s’il recherchait les occasions d’aller voir conduire Olivier Métra, c’était pour
la douceur d’être initié dans toutes les conceptions d’Odette, de se sentir de
moitié dans tous ses goûts. Ce charme de le rapprocher d’elle, qu’avaient les
ouvrages ou les lieux qu’elle aimait, lui semblait plus mystérieux que celui qui
est intrinsèque à de plus beaux, mais qui ne la lui rappelaient pas. D’ailleurs,
ayant laissé s’affaiblir les croyances intellectuelles de sa jeunesse, et son
scepticisme d’homme du monde ayant à son insu pénétré jusqu’à elles, il pensait
(ou du moins il avait si longtemps pensé cela qu’il le disait encore) que les
objets de nos goûts n’ont pas en eux une valeur absolue, mais que tout est
affaire d’époque, de classe, consiste en modes, dont les plus vulgaires valent
celles qui passent pour les plus distinguées. Et comme il jugeait que
l’importance attachée par Odette à avoir des cartes pour le vernissage n’était
pas en soi quelque chose de plus ridicule que le plaisir qu’il avait autrefois à
déjeuner chez le prince de Galles, de même, il ne pensait pas que l’admiration
qu’elle professait pour Monte-Carlo ou pour le Righi fût plus déraisonnable que
le goût qu’il avait, lui, pour la Hollande qu’elle se figurait laide et pour
Versailles qu’elle trouvait triste. Aussi, se privait-il d’y aller, ayant
plaisir à se dire que c’était pour elle, qu’il voulait ne sentir, n’aimer
qu’avec elle.

Comme tout ce qui environnait Odette et n’était en quelque sorte que le mode
selon lequel il pouvait la voir, causer avec elle, il aimait la société des
Verdurin. Là, comme au fond de tous les divertissements, repas, musique, jeux,
soupers costumés, parties de campagne, parties de théâtre, même les rares
«grandes soirées» données pour les «ennuyeux», il y avait la présence d’Odette,
la vue d’Odette, la conversation avec Odette, dont les Verdurin faisaient à
Swann, en l’invitant, le don inestimable, il se plaisait mieux que partout
ailleurs dans le «petit noyau», et cherchait à lui attribuer des mérites réels,
car il s’imaginait ainsi que par goût il le fréquenterait toute sa vie. Or,
n’osant pas se dire, par peur de ne pas le croire, qu’il aimerait toujours
Odette, du moins en cherchant á supposer qu’il fréquenterait toujours les
Verdurin (proposition qui, a priori, soulevait moins d’objections de principe de
la part de son intelligence), il se voyait dans l’avenir continuant à rencontrer
chaque soir Odette; cela ne revenait peut-être pas tout à fait au même que
l’aimer toujours, mais, pour le moment, pendant qu’il l’aimait, croire qu’il ne
cesserait pas un jour de la voir, c’est tout ce qu’il demandait. «Quel charmant
milieu, se disait-il. Comme c’est au fond la vraie vie qu’on mène là! Comme on y
est plus intelligent, plus artiste que dans le monde. Comme Mme Verdurin, malgré
de petites exagérations un peu risibles, a un amour sincère de la peinture, de
la musique! quelle passion pour les œuvres, quel désir de faire plaisir aux
artistes! Elle se fait une idée inexacte des gens du monde; mais avec cela que
le monde n’en a pas une plus fausse encore des milieux artistes! Peut-être
n’ai-je pas de grands besoins intellectuels à assouvir dans la conversation,
mais je me plais parfaitement bien avec Cottard, quoiqu’il fasse des calembours
ineptes. Et quant au peintre, si sa prétention est déplaisante quand il cherche
à étonner, en revanche c’est une des plus belles intelligences que j’aie
connues. Et puis surtout, là, on se sent libre, on fait ce qu’on veut sans
contrainte, sans cérémonie. Quelle dépense de bonne humeur il se fait par jour
dans ce salon-là! Décidément, sauf quelques rares exceptions, je n’irai plus
jamais que dans ce milieu. C’est là que j’aurai de plus en plus mes habitudes et
ma vie.»

Et comme les qualités qu’il croyait intrinsèques aux Verdurin n’étaient que le
reflet sur eux de plaisirs qu’avait goûtés chez eux son amour pour Odette, ces
qualités devenaient plus sérieuses, plus profondes, plus vitales, quand ces
plaisirs l’étaient aussi. Comme Mme Verdurin donnait parfois à Swann ce qui seul
pouvait constituer pour lui le bonheur; comme, tel soir où il se sentait anxieux
parce qu’Odette avait causé avec un invité plus qu’avec un autre, et où, irrité
contre elle, il ne voulait pas prendre l’initiative de lui demander si elle
reviendrait avec lui, Mme Verdurin lui apportait la paix et la joie en disant
spontanément: «Odette, vous allez ramener M. Swann, n’est-ce pas»? comme cet été
qui venait et où il s’était d’abord demandé avec inquiétude si Odette ne
s’absenterait pas sans lui, s’il pourrait continuer à la voir tous les jours,
Mme Verdurin allait les inviter à le passer tous deux chez elle à la campagne —
Swann laissant à son insu la reconnaissance et l’intérêt s’infiltrer dans son
intelligence et influer sur ses idées, allait jusqu’à proclamer que Mme Verdurin
était une grande âme. De quelques gens exquis ou éminents que tel de ses anciens
camarades de l’école du Louvre lui parlât: «Je préfère cent fois les Verdurin,
lui répondait-il.» Et, avec une solennité qui était nouvelle chez lui: «Ce sont
des êtres magnanimes, et la magnanimité est, au fond, la seule chose qui importe
et qui distingue ici-bas. Vois-tu, il n’y a que deux classes d’êtres: les
magnanimes et les autres; et je suis arrivé à un âge où il faut prendre parti,
décider une fois pour toutes qui on veut aimer et qui on veut dédaigner, se
tenir à ceux qu’on aime et, pour réparer le temps qu’on a gâché avec les autres,
ne plus les quitter jusqu’à sa mort. Eh bien! ajoutait-il avec cette légère
émotion qu’on éprouve quand même sans bien s’en rendre compte, on dit une chose
non parce qu’elle est vraie, mais parce qu’on a plaisir à la dire et qu’on
l’écoute dans sa propre voix comme si elle venait d’ailleurs que de nous-mêmes,
le sort en est jeté, j’ai choisi d’aimer les seuls cœurs magnanimes et de ne
plus vivre que dans la magnanimité. Tu me demandes si Mme Verdurin est
véritablement intelligente. Je t’assure qu’elle m’a donné les preuves d’une
noblesse de cœur, d’une hauteur d’âme où, que veux-tu, on n’atteint pas sans une
hauteur égale de pensée. Certes elle a la profonde intelligence des arts. Mais
ce n’est peut-être pas là qu’elle est le plus admirable; et telle petite action
ingénieusement, exquisement bonne, qu’elle a accomplie pour moi, telle géniale
attention, tel geste familièrement sublime, révèlent une compréhension plus
profonde de l’existence que tous les traités de philosophie.»

Il aurait pourtant pu se dire qu’il y avait des anciens amis de ses parents
aussi simples que les Verdurin, des camarades de sa jeunesse aussi épris d’art,
qu’il connaissait d’autres êtres d’un grand cœur, et que, pourtant, depuis qu’il
avait opté pour la simplicité, les arts et la magnanimité, il ne les voyait plus
jamais. Mais ceux-là ne connaissaient pas Odette, et, s’ils l’avaient connue, ne
se seraient pas souciés de la rapprocher de lui.

Ainsi il n’y avait sans doute pas, dans tout le milieu Verdurin, un seul fidèle
qui les aimât ou crût les aimer autant que Swann. Et pourtant, quand M. Verdurin
avait dit que Swann ne lui revenait pas, non seulement il avait exprimé sa
propre pensée, mais il avait deviné celle de sa femme. Sans doute Swann avait
pour Odette une affection trop particulière et dont il avait négligé de faire de
Mme Verdurin la confidente quotidienne: sans doute la discrétion même avec
laquelle il usait de l’hospitalité des Verdurin, s’abstenant souvent de venir
dîner pour une raison qu’ils ne soupçonnaient pas et à la place de laquelle ils
voyaient le désir de ne pas manquer une invitation chez des «ennuyeux», sans
doute aussi, et malgré toutes les précautions qu’il avait prises pour la leur
cacher, la découverte progressive qu’ils faisaient de sa brillante situation
mondaine, tout cela contribuait à leur irritation contre lui. Mais la raison
profonde en était autre. C’est qu’ils avaient très vite senti en lui un espace
réservé, impénétrable, où il continuait à professer silencieusement pour
lui-même que la princesse de Sagan n’était pas grotesque et que les
plaisanteries de Cottard n’étaient pas drôles, enfin et bien que jamais il ne se
départît de son amabilité et ne se révoltât contre leurs dogmes, une
impossibilité de les lui imposer, de l’y convertir entièrement, comme ils n’en
avaient jamais rencontré une pareille chez personne. Ils lui auraient pardonné
de fréquenter des ennuyeux (auxquels d’ailleurs, dans le fond de son cœur, il
préférait mille fois les Verdurin et tout le petit noyau) s’il avait consenti,
pour le bon exemple, à les renier en présence des fidèles. Mais c’est une
abjuration qu’ils comprirent qu’on ne pourrait pas lui arracher.

Quelle différence avec un «nouveau» qu’Odette leur avait demandé d’inviter,
quoiqu’elle ne l’eût rencontré que peu de fois, et sur lequel ils fondaient
beaucoup d’espoir, le comte de Forcheville! (Il se trouva qu’il était justement
le beau-frère de Saniette, ce qui remplit d’étonnement les fidèles: le vieil
archiviste avait des manières si humbles qu’ils l’avaient toujours cru d’un rang
social inférieur au leur et ne s’attendaient pas à apprendre qu’il appartenait à
un monde riche et relativement aristocratique.) Sans doute Forcheville était
grossièrement snob, alors que Swann ne l’était pas; sans doute il était bien
loin de placer, comme lui, le milieu des Verdurin au-dessus de tous les autres.
Mais il n’avait pas cette délicatesse de nature qui empêchait Swann de
s’associer aux critiques trop manifestement fausses que dirigeait Mme Verdurin
contre des gens qu’il connaissait. Quant aux tirades prétentieuses et vulgaires
que le peintre lançait à certains jours, aux plaisanteries de commis voyageur
que risquait Cottard et auxquelles Swann, qui les aimait l’un et l’autre,
trouvait facilement des excuses mais n’avait pas le courage et l’hypocrisie
d’applaudir, Forcheville était au contraire d’un niveau intellectuel qui lui
permettait d’être abasourdi, émerveillé par les unes, sans d’ailleurs les
comprendre, et de se délecter aux autres. Et justement le premier dîner chez les
Verdurin auquel assista Forcheville, mit en lumière toutes ces différences, fit
ressortir ses qualités et précipita la disgrâce de Swann.

Il y avait, à ce dîner, en dehors des habitués, un professeur de la Sorbonne,
Brichot, qui avait rencontré M. et Mme Verdurin aux eaux et si ses fonctions
universitaires et ses travaux d’érudition n’avaient pas rendu très rares ses
moments de liberté, serait volontiers venu souvent chez eux. Car il avait cette
curiosité, cette superstition de la vie, qui unie à un certain scepticisme
relatif à l’objet de leurs études, donne dans n’importe quelle profession, à
certains hommes intelligents, médecins qui ne croient pas à la médecine,
professeurs de lycée qui ne croient pas au thème latin, la réputation d’esprits
larges, brillants, et même supérieurs. Il affectait, chez Mme Verdurin, de
chercher ses comparaisons dans ce qu’il y avait de plus actuel quand il parlait
de philosophie et d’histoire, d’abord parce qu’il croyait qu’elles ne sont
qu’une préparation à la vie et qu’il s’imaginait trouver en action dans le petit
clan ce qu’il n’avait connu jusqu’ici que dans les livres, puis peut-être aussi
parce que, s’étant vu inculquer autrefois, et ayant gardé à son insu, le respect
de certains sujets, il croyait dépouiller l’universitaire en prenant avec eux
des hardiesses qui, au contraire, ne lui paraissaient telles, que parce qu’il
l’était resté.

Dès le commencement du repas, comme M. de Forcheville, placé à la droite de Mme
Verdurin qui avait fait pour le «nouveau» de grands frais de toilette, lui
disait: «C’est original, cette robe blanche», le docteur qui n’avait cessé de
l’observer, tant il était curieux de savoir comment était fait ce qu’il appelait
un «de», et qui cherchait une occasion d’attirer son attention et d’entrer plus
en contact avec lui, saisit au vol le mot «blanche» et, sans lever le nez de son
assiette, dit: «blanche? Blanche de Castille?», puis sans bouger la tête lança
furtivement de droite et de gauche des regards incertains et souriants. Tandis
que Swann, par l’effort douloureux et vain qu’il fit pour sourire, témoigna
qu’il jugeait ce calembour stupide, Forcheville avait montré à la fois qu’il en
goûtait la finesse et qu’il savait vivre, en contenant dans de justes limites
une gaieté dont la franchise avait charmé Mme Verdurin.

— Qu’est-ce que vous dites d’un savant comme cela? avait-elle demandé à
Forcheville. Il n’y a pas moyen de causer sérieusement deux minutes avec lui.
Est-ce que vous leur en dites comme cela, à votre hôpital? avait-elle ajouté en
se tournant vers le docteur, ça ne doit pas être ennuyeux tous les jours, alors.
Je vois qu’il va falloir que je demande à m’y faire admettre.

— Je crois avoir entendu que le docteur parlait de cette vieille chipie de
Blanche de Castille, si j’ose m’exprimer ainsi. N’est-il pas vrai, madame?
demanda Brichot à Mme Verdurin qui, pâmant, les yeux fermés, précipita sa figure
dans ses mains d’où s’échappèrent des cris étouffés.

«Mon Dieu, Madame, je ne voudrais pas alarmer les âmes respectueuses s’il y en a
autour de cette table, sub rosa . . . Je reconnais d’ailleurs que notre
ineffable république athénienne —ô combien! — pourrait honorer en cette
capétienne obscurantiste le premier des préfets de police à poigne. Si fait, mon
cher hôte, si fait, reprit-il de sa voix bien timbrée qui détachait chaque
syllabe, en réponse à une objection de M. Verdurin. La chronique de Saint-Denis
dont nous ne pouvons contester la sûreté d’information ne laisse aucun doute à
cet égard. Nulle ne pourrait être mieux choisie comme patronne par un
prolétariat laïcisateur que cette mère d’un saint à qui elle en fit d’ailleurs
voir de saumâtres, comme dit Suger et autres saint Bernard; car avec elle chacun
en prenait pour son grade.

— Quel est ce monsieur? demanda Forcheville à Mme Verdurin, il a l’air d’être de
première force.

— Comment, vous ne connaissez pas le fameux Brichot? il est célèbre dans toute
l’Europe.

— Ah! c’est Bréchot, s’écria Forcheville qui n’avait pas bien entendu, vous m’en
direz tant, ajouta-t-il tout en attachant sur l’homme célèbre des yeux
écarquillés. C’est toujours intéressant de dîner avec un homme en vue. Mais,
dites-moi, vous nous invitez-là avec des convives de choix. On ne s’ennuie pas
chez vous.

— Oh! vous savez ce qu’il y a surtout, dit modestement Mme Verdurin, c’est
qu’ils se sentent en confiance. Ils parlent de ce qu’ils veulent, et la
conversation rejaillit en fusées. Ainsi Brichot, ce soir, ce n’est rien: je l’ai
vu, vous savez, chez moi, éblouissant, à se mettre à genoux devant; eh bien!
chez les autres, ce n’est plus le même homme, il n’a plus d’esprit, il faut lui
arracher les mots, il est même ennuyeux.

— C’est curieux! dit Forcheville étonné.

Un genre d’esprit comme celui de Brichot aurait été tenu pour stupidité pure
dans la coterie où Swann avait passé sa jeunesse, bien qu’il soit compatible
avec une intelligence réelle. Et celle du professeur, vigoureuse et bien
nourrie, aurait probablement pu être enviée par bien des gens du monde que Swann
trouvait spirituels. Mais ceux-ci avaient fini par lui inculquer si bien leurs
goûts et leurs répugnances, au moins en tout ce qui touche à la vie mondaine et
même en celle de ses parties annexes qui devrait plutôt relever du domaine de
l’intelligence: la conversation, que Swann ne put trouver les plaisanteries de
Brichot que pédantesques, vulgaires et grasses à écœurer. Puis il était choqué,
dans l’habitude qu’il avait des bonnes manières, par le ton rude et militaire
qu’affectait, en s’adressant à chacun, l’universitaire cocardier. Enfin,
peut-être avait-il surtout perdu, ce soir-là, de son indulgence en voyant
l’amabilité que Mme Verdurin déployait pour ce Forcheville qu’Odette avait eu la
singulière idée d’amener. Un peu gênée vis-à-vis de Swann, elle lui avait
demandé en arrivant:

— Comment trouvez-vous mon invité?

Et lui, s’apercevant pour la première fois que Forcheville qu’il connaissait
depuis longtemps pouvait plaire à une femme et était assez bel homme, avait
répondu: «Immonde!» Certes, il n’avait pas l’idée d’être jaloux d’Odette, mais
il ne se sentait pas aussi heureux que d’habitude et quand Brichot, ayant
commencé à raconter l’histoire de la mère de Blanche de Castille qui «avait été
avec Henri Plantagenet des années avant de l’épouser», voulut s’en faire
demander la suite par Swann en lui disant: «n’est-ce pas, monsieur Swann?» sur
le ton martial qu’on prend pour se mettre à la portée d’un paysan ou pour donner
du cœur à un troupier, Swann coupa l’effet de Brichot à la grande fureur de la
maîtresse de la maison, en répondant qu’on voulût bien l’excuser de s’intéresser
si peu à Blanche de Castille, mais qu’il avait quelque chose à demander au
peintre. Celui-ci, en effet, était allé dans l’après-midi visiter l’exposition
d’un artiste, ami de Mme Verdurin qui était mort récemment, et Swann aurait
voulu savoir par lui (car il appréciait son goût) si vraiment il y avait dans
ces dernières œuvres plus que la virtuosité qui stupéfiait déjà dans les
précédentes.

— A ce point de vue-là, c’était extraordinaire, mais cela ne semblait pas d’un
art, comme on dit, très «élevé», dit Swann en souriant.

—Élevé . . . à la hauteur d’une institution, interrompit Cottard en levant les
bras avec une gravité simulée.

Toute la table éclata de rire.

— Quand je vous disais qu’on ne peut pas garder son sérieux avec lui, dit Mme
Verdurin à Forcheville. Au moment où on s’y attend le moins, il vous sort une
calembredaine.

Mais elle remarqua que seul Swann ne s’était pas déridé. Du reste il n’était pas
très content que Cottard fît rire de lui devant Forcheville. Mais le peintre, au
lieu de répondre d’une façon intéressante à Swann, ce qu’il eût probablement
fait s’il eût été seul avec lui, préféra se faire admirer des convives en
plaçant un morceau sur l’habileté du maître disparu.

— Je me suis approché, dit-il, pour voir comment c’était fait, j’ai mis le nez
dessus. Ah! bien ouiche! on ne pourrait pas dire si c’est fait avec de la colle,
avec du rubis, avec du savon, avec du bronze, avec du soleil, avec du caca!

— Et un font douze, s’écria trop tard le docteur dont personne ne comprit
l’interruption.

—«Ça a l’air fait avec rien, reprit le peintre, pas plus moyen de découvrir le
truc que dans la Ronde ou les Régentes et c’est encore plus fort comme patte que
Rembrandt et que Hals. Tout y est, mais non, je vous jure.»

Et comme les chanteurs parvenus à la note la plus haute qu’ils puissent donner
continuent en voix de tête, piano, il se contenta de murmurer, et en riant,
comme si en effet cette peinture eût été dérisoire à force de beauté:

—«Ça sent bon, ça vous prend à la tête, ça vous coupe la respiration, ça vous
fait des chatouilles, et pas mèche de savoir avec quoi c’est fait, c’en est
sorcier, c’est de la rouerie, c’est du miracle (éclatant tout à fait de rire):
c’en est malhonnête!» En s’arrêtant, redressant gravement la tête, prenant une
note de basse profonde qu’il tâcha de rendre harmonieuse, il ajouta: «et c’est
si loyal!»

Sauf au moment où il avait dit: «plus fort que la Ronde», blasphème qui avait
provoqué une protestation de Mme Verdurin qui tenait «la Ronde» pour le plus
grand chef-d’œuvre de l’univers avec «la Neuvième» et «la Samothrace», et à:
«fait avec du caca» qui avait fait jeter à Forcheville un coup d’œil circulaire
sur la table pour voir si le mot passait et avait ensuite amené sur sa bouche un
sourire prude et conciliant, tous les convives, excepté Swann, avaient attaché
sur le peintre des regards fascinés par l’admiration.

—«Ce qu’il m’amuse quand il s’emballe comme ça, s’écria, quand il eut terminé,
Mme Verdurin, ravie que la table fût justement si intéressante le jour où M. de
Forcheville venait pour la première fois. Et toi, qu’est-ce que tu as à rester
comme cela, bouche bée comme une grande bête? dit-elle à son mari. Tu sais
pourtant qu’il parle bien; on dirait que c’est la première fois qu’il vous
entend. Si vous l’aviez vu pendant que vous parliez, il vous buvait. Et demain
il nous récitera tout ce que vous avez dit sans manger un mot.»

— Mais non, c’est pas de la blague, dit le peintre, enchanté de son succès, vous
avez l’air de croire que je fais le boniment, que c’est du chiqué; je vous y
mènerai voir, vous direz si j’ai exagéré, je vous fiche mon billet que vous
revenez plus emballée que moi!

— Mais nous ne croyons pas que vous exagérez, nous voulons seulement que vous
mangiez, et que mon mari mange aussi; redonnez de la sole normande à Monsieur,
vous voyez bien que la sienne est froide. Nous ne sommes pas si pressés, vous
servez comme s’il y avait le feu, attendez donc un peu pour donner la salade.

Mme Cottard qui était modeste et parlait peu, savait pourtant ne pas manquer
d’assurance quand une heureuse inspiration lui avait fait trouver un mot juste.
Elle sentait qu’il aurait du succès, cela la mettait en confiance, et ce qu’elle
en faisait était moins pour briller que pour être utile à la carrière de son
mari. Aussi ne laissa-t-elle pas échapper le mot de salade que venait de
prononcer Mme Verdurin.

— Ce n’est pas de la salade japonaise? dit-elle à mi-voix en se tournant vers
Odette.

Et ravie et confuse de l’à-propos et de la hardiesse qu’il y avait à faire ainsi
une allusion discrète, mais claire, à la nouvelle et retentissante pièce de
Dumas, elle éclata d’un rire charmant d’ingénue, peu bruyant, mais si
irrésistible qu’elle resta quelques instants sans pouvoir le maîtriser. «Qui est
cette dame? elle a de l’esprit», dit Forcheville.

—«Non, mais nous vous en ferons si vous venez tous dîner vendredi.»

— Je vais vous paraître bien provinciale, monsieur, dit Mme Cottard à Swann,
mais je n’ai pas encore vu cette fameuse Francillon dont tout le monde parle. Le
docteur y est allé (je me rappelle même qu’il m’a dit avoir eu le très grand
plaisir de passer la soirée avec vous) et j’avoue que je n’ai pas trouvé
raisonnable qu’il louât des places pour y retourner avec moi. Évidemment, au
Théâtre-Français, on ne regrette jamais sa soirée, c’est toujours si bien joué,
mais comme nous avons des amis très aimables (Mme Cottard prononçait rarement un
nom propre et se contentait de dire «des amis à nous», «une de mes amies», par
«distinction», sur un ton factice, et avec l’air d’importance d’une personne qui
ne nomme que qui elle veut) qui ont souvent des loges et ont la bonne idée de
nous emmener à toutes les nouveautés qui en valent la peine, je suis toujours
sûre de voir Francillon un peu plus tôt ou un peu plus tard, et de pouvoir me
former une opinion. Je dois pourtant confesser que je me trouve assez sotte,
car, dans tous les salons où je vais en visite, on ne parle naturellement que de
cette malheureuse salade japonaise. On commence même à en être un peu fatigué,
ajouta-t-elle en voyant que Swann n’avait pas l’air aussi intéressé qu’elle
aurait cru par une si brûlante actualité. Il faut avouer pourtant que cela donne
quelquefois prétexte à des idées assez amusantes. Ainsi j’ai une de mes amies
qui est très originale, quoique très jolie femme, très entourée, très lancée, et
qui prétend qu’elle a fait faire chez elle cette salade japonaise, mais en
faisant mettre tout ce qu’Alexandre Dumas fils dit dans la pièce. Elle avait
invité quelques amies à venir en manger. Malheureusement je n’étais pas des
élues. Mais elle nous l’a raconté tantôt, à son jour; il paraît que c’était
détestable, elle nous a fait rire aux larmes. Mais vous savez, tout est dans la
manière de raconter, dit-elle en voyant que Swann gardait un air grave.

Et supposant que c’était peut-être parce qu’il n’aimait pas Francillon:

— Du reste, je crois que j’aurai une déception. Je ne crois pas que cela vaille
Serge Panine, l’idole de Mme de Crécy. Voilà au moins des sujets qui ont du
fond, qui font réfléchir; mais donner une recette de salade sur la scène du
Théâtre-Français! Tandis que Serge Panine! Du reste, comme tout ce qui vient de
la plume de Georges Ohnet, c’est toujours si bien écrit. Je ne sais pas si vous
connaissez Le Maître de Forges que je préférerais encore à Serge Panine.

—«Pardonnez-moi, lui dit Swann d’un air ironique, mais j’avoue que mon manque
d’admiration est à peu près égal pour ces deux chefs-d’œuvre.»

—«Vraiment, qu’est-ce que vous leur reprochez? Est-ce un parti pris?
Trouvez-vous peut-être que c’est un peu triste? D’ailleurs, comme je dis
toujours, il ne faut jamais discuter sur les romans ni sur les pièces de
théâtre. Chacun a sa manière de voir et vous pouvez trouver détestable ce que
j’aime le mieux.»

Elle fut interrompue par Forcheville qui interpellait Swann. En effet, tandis
que Mme Cottard parlait de Francillon, Forcheville avait exprimé à Mme Verdurin
son admiration pour ce qu’il avait appelé le petit «speech» du peintre.

— Monsieur a une facilité de parole, une mémoire! avait-il dit à Mme Verdurin
quand le peintre eut terminé, comme j’en ai rarement rencontré. Bigre! je
voudrais bien en avoir autant. Il ferait un excellent prédicateur. On peut dire
qu’avec M. Bréchot, vous avez là deux numéros qui se valent, je ne sais même pas
si comme platine, celui-ci ne damerait pas encore le pion au professeur. Ça
vient plus naturellement, c’est moins recherché. Quoiqu’il ait chemin faisant
quelques mots un peu réalistes, mais c’est le goût du jour, je n’ai pas souvent
vu tenir le crachoir avec une pareille dextérité, comme nous disions au
régiment, où pourtant j’avais un camarade que justement monsieur me rappelait un
peu. A propos de n’importe quoi, je ne sais que vous dire, sur ce verre, par
exemple, il pouvait dégoiser pendant des heures, non, pas à propos de ce verre,
ce que je dis est stupide; mais à propos de la bataille de Waterloo, de tout ce
que vous voudrez et il nous envoyait chemin faisant des choses auxquelles vous
n’auriez jamais pensé. Du reste Swann était dans le même régiment; il a dû le
connaître.»

— Vous voyez souvent M. Swann? demanda Mme Verdurin.

— Mais non, répondit M. de Forcheville et comme pour se rapprocher plus aisément
d’Odette, il désirait être agréable à Swann, voulant saisir cette occasion, pour
le flatter, de parler de ses belles relations, mais d’en parler en homme du
monde sur un ton de critique cordiale et n’avoir pas l’air de l’en féliciter
comme d’un succès inespéré: «N’est-ce pas, Swann? je ne vous vois jamais.
D’ailleurs, comment faire pour le voir? Cet animal-là est tout le temps fourré
chez les La Trémoïlle, chez les Laumes, chez tout ça! . . . » Imputation
d’autant plus fausse d’ailleurs que depuis un an Swann n’allait plus guère que
chez les Verdurin. Mais le seul nom de personnes qu’ils ne connaissaient pas
était accueilli chez eux par un silence réprobateur. M. Verdurin, craignant la
pénible impression que ces noms d’«ennuyeux», surtout lancés ainsi sans tact à
la face de tous les fidèles, avaient dû produire sur sa femme, jeta sur elle à
la dérobée un regard plein d’inquiète sollicitude. Il vit alors que dans sa
résolution de ne pas prendre acte, de ne pas avoir été touchée par la nouvelle
qui venait de lui être notifiée, de ne pas seulement rester muette, mais d’avoir
été sourde comme nous l’affectons, quand un ami fautif essaye de glisser dans la
conversation une excuse que ce serait avoir l’air d’admettre que de l’avoir
écoutée sans protester, ou quand on prononce devant nous le nom défendu d’un
ingrat, Mme Verdurin, pour que son silence n’eût pas l’air d’un consentement,
mais du silence ignorant des choses inanimées, avait soudain dépouillé son
visage de toute vie, de toute motilité; son front bombé n’était plus qu’une
belle étude de ronde bosse où le nom de ces La Trémoïlle chez qui était toujours
fourré Swann, n’avait pu pénétrer; son nez légèrement froncé laissait voir une
échancrure qui semblait calquée sur la vie. On eût dit que sa bouche
entr’ouverte allait parler. Ce n’était plus qu’une cire perdue, qu’un masque de
plâtre, qu’une maquette pour un monument, qu’un buste pour le Palais de
l’Industrie devant lequel le public s’arrêterait certainement pour admirer
comment le sculpteur, en exprimant l’imprescriptible dignité des Verdurin
opposée à celle des La Trémoïlle et des Laumes qu’ils valent certes ainsi que
tous les ennuyeux de la terre, était arrivé à donner une majesté presque papale
à la blancheur et à la rigidité de la pierre. Mais le marbre finit par s’animer
et fit entendre qu’il fallait ne pas être dégoûté pour aller chez ces gens-là,
car la femme était toujours ivre et le mari si ignorant qu’il disait collidor
pour corridor.

—«On me paierait bien cher que je ne laisserais pas entrer ça chez moi», conclut
Mme Verdurin, en regardant Swann d’un air impérieux.

Sans doute elle n’espérait pas qu’il se soumettrait jusqu’à imiter la sainte
simplicité de la tante du pianiste qui venait de s’écrier:

— Voyez-vous ça? Ce qui m’étonne, c’est qu’ils trouvent encore des personnes qui
consentent à leur causer; il me semble que j’aurais peur: un mauvais coup est si
vite reçu! Comment y a-t-il encore du peuple assez brute pour leur courir après.

Que ne répondait-il du moins comme Forcheville: «Dame, c’est une duchesse; il y
a des gens que ça impressionne encore», ce qui avait permis au moins à Mme
Verdurin de répliquer: «Grand bien leur fasse!» Au lieu de cela, Swann se
contenta de rire d’un air qui signifiait qu’il ne pouvait même pas prendre au
sérieux une pareille extravagance. M. Verdurin, continuant à jeter sur sa femme
des regards furtifs, voyait avec tristesse et comprenait trop bien qu’elle
éprouvait la colère d’un grand inquisiteur qui ne parvient pas à extirper
l’hérésie, et pour tâcher d’amener Swann à une rétractation, comme le courage de
ses opinions paraît toujours un calcul et une lâcheté aux yeux de ceux à
l’encontre de qui il s’exerce, M. Verdurin l’interpella:

— Dites donc franchement votre pensée, nous n’irons pas le leur répéter.

A quoi Swann répondit:

— Mais ce n’est pas du tout par peur de la duchesse (si c’est des La Trémoïlle
que vous parlez). Je vous assure que tout le monde aime aller chez elle. Je ne
vous dis pas qu’elle soit «profonde» (il prononça profonde, comme si ç’avait été
un mot ridicule, car son langage gardait la trace d’habitudes d’esprit qu’une
certaine rénovation, marquée par l’amour de la musique, lui avait momentanément
fait perdre — il exprimait parfois ses opinions avec chaleur —) mais, très
sincèrement, elle est intelligente et son mari est un véritable lettré. Ce sont
des gens charmants.

Si bien que Mme Verdurin, sentant que, par ce seul infidèle, elle serait
empêchée de réaliser l’unité morale du petit noyau, ne put pas s’empêcher dans
sa rage contre cet obstiné qui ne voyait pas combien ses paroles la faisaient
souffrir, de lui crier du fond du cœur:

— Trouvez-le si vous voulez, mais du moins ne nous le dites pas.

— Tout dépend de ce que vous appelez intelligence, dit Forcheville qui voulait
briller à son tour. Voyons, Swann, qu’entendez-vous par intelligence?

— Voilà! s’écria Odette, voilà les grandes choses dont je lui demande de me
parler, mais il ne veut jamais.

— Mais si . . . protesta Swann.

— Cette blague! dit Odette.

— Blague à tabac? demanda le docteur.

— Pour vous, reprit Forcheville, l’intelligence, est-ce le bagout du monde, les
personnes qui savent s’insinuer?

— Finissez votre entremets qu’on puisse enlever votre assiette, dit Mme Verdurin
d’un ton aigre en s’adressant à Saniette, lequel absorbé dans des réflexions,
avait cessé de manger. Et peut-être un peu honteuse du ton qu’elle avait pris:
«Cela ne fait rien, vous avez votre temps, mais, si je vous le dis, c’est pour
les autres, parce que cela empêche de servir.»

— Il y a, dit Brichot en martelant les syllabes, une définition bien curieuse de
l’intelligence dans ce doux anarchiste de Fénelon . . .

— Ecoutez! dit à Forcheville et au docteur Mme Verdurin, il va nous dire la
définition de l’intelligence par Fénelon, c’est intéressant, on n’a pas toujours
l’occasion d’apprendre cela.

Mais Brichot attendait que Swann eût donné la sienne. Celui-ci ne répondit pas
et en se dérobant fit manquer la brillante joute que Mme Verdurin se réjouissait
d’offrir à Forcheville.

— Naturellement, c’est comme avec moi, dit Odette d’un ton boudeur, je ne suis
pas fâchée de voir que je ne suis pas la seule qu’il ne trouve pas à la hauteur.

— Ces de La Trémouaille que Mme Verdurin nous a montrés comme si peu
recommandables, demanda Brichot, en articulant avec force, descendent-ils de
ceux que cette bonne snob de Mme de Sévigné avouait être heureuse de connaître
parce que cela faisait bien pour ses paysans? Il est vrai que la marquise avait
une autre raison, et qui pour elle devait primer celle-là, car gendelettre dans
l’âme, elle faisait passer la copie avant tout. Or dans le journal qu’elle
envoyait régulièrement à sa fille, c’est Mme de la Trémouaille, bien documentée
par ses grandes alliances, qui faisait la politique étrangère.

— Mais non, je ne crois pas que ce soit la même famille, dit à tout hasard Mme
Verdurin.

Saniette qui, depuis qu’il avait rendu précipitamment au maître d’hôtel son
assiette encore pleine, s’était replongé dans un silence méditatif, en sortit
enfin pour raconter en riant l’histoire d’un dîner qu’il avait fait avec le duc
de La Trémoïlle et d’où il résultait que celui-ci ne savait pas que George Sand
était le pseudonyme d’une femme. Swann qui avait de la sympathie pour Saniette
crut devoir lui donner sur la culture du duc des détails montrant qu’une telle
ignorance de la part de celui-ci était matériellement impossible; mais tout d’un
coup il s’arrêta, il venait de comprendre que Saniette n’avait pas besoin de ces
preuves et savait que l’histoire était fausse pour la raison qu’il venait de
l’inventer il y avait un moment. Cet excellent homme souffrait d’être trouvé si
ennuyeux par les Verdurin; et ayant conscience d’avoir été plus terne encore à
ce dîner que d’habitude, il n’avait voulu le laisser finir sans avoir réussi à
amuser. Il capitula si vite, eut l’air si malheureux de voir manqué l’effet sur
lequel il avait compté et répondit d’un ton si lâche à Swann pour que celui-ci
ne s’acharnât pas à une réfutation désormais inutile: «C’est bon, c’est bon; en
tous cas, même si je me trompe, ce n’est pas un crime, je pense» que Swann
aurait voulu pouvoir dire que l’histoire était vraie et délicieuse. Le docteur
qui les avait écoutés eut l’idée que c’était le cas de dire: «Se non e vero»,
mais il n’était pas assez sûr des mots et craignit de s’embrouiller.

Après le dîner Forcheville alla de lui-même vers le docteur.

—«Elle n’a pas dû être mal, Mme Verdurin, et puis c’est une femme avec qui on
peut causer, pour moi tout est là. Évidemment elle commence à avoir un peu de
bouteille. Mais Mme de Crécy voilà une petite femme qui a l’air intelligente,
ah! saperlipopette, on voit tout de suite qu’elle a l’œil américain, celle-là!
Nous parlons de Mme de Crécy, dit-il à M. Verdurin qui s’approchait, la pipe à
la bouche. Je me figure que comme corps de femme . . . »

—«J’aimerais mieux l’avoir dans mon lit que le tonnerre», dit précipitamment
Cottard qui depuis quelques instants attendait en vain que Forcheville reprît
haleine pour placer cette vieille plaisanterie dont il craignait que ne revînt
pas l’à-propos si la conversation changeait de cours, et qu’il débita avec cet
excès de spontanéité et d’assurance qui cherche à masquer la froideur et l’émoi
inséparables d’une récitation. Forcheville la connaissait, il la comprit et s’en
amusa. Quant à M. Verdurin, il ne marchanda pas sa gaieté, car il avait trouvé
depuis peu pour la signifier un symbole autre que celui dont usait sa femme,
mais aussi simple et aussi clair. A peine avait-il commencé à faire le mouvement
de tête et d’épaules de quelqu’un qui s’esclaffle qu’aussitôt il se mettait à
tousser comme si, en riant trop fort, il avait avalé la fumée de sa pipe. Et la
gardant toujours au coin de sa bouche, il prolongeait indéfiniment le simulacre
de suffocation et d’hilarité. Ainsi lui et Mme Verdurin, qui en face, écoutant
le peintre qui lui racontait une histoire, fermait les yeux avant de précipiter
son visage dans ses mains, avaient l’air de deux masques de théâtre qui
figuraient différemment la gaieté.

M. Verdurin avait d’ailleurs fait sagement en ne retirant pas sa pipe de sa
bouche, car Cottard qui avait besoin de s’éloigner un instant fit à mi-voix une
plaisanterie qu’il avait apprise depuis peu et qu’il renouvelait chaque fois
qu’il avait à aller au même endroit: «Il faut que j’aille entretenir un instant
le duc d’Aumale», de sorte que la quinte de M. Verdurin recommença.

— Voyons, enlève donc ta pipe de ta bouche, tu vois bien que tu vas t’étouffer à
te retenir de rire comme ça, lui dit Mme Verdurin qui venait offrir des
liqueurs.

—«Quel homme charmant que votre mari, il a de l’esprit comme quatre, déclara
Forcheville à Mme Cottard. Merci madame. Un vieux troupier comme moi, ça ne
refuse jamais la goutte.»

—«M. de Forcheville trouve Odette charmante», dit M. Verdurin à sa femme.

— Mais justement elle voudrait déjeuner une fois avec vous. Nous allons combiner
ça, mais il ne faut pas que Swann le sache. Vous savez, il met un peu de froid.
Ça ne vous empêchera pas de venir dîner, naturellement, nous espérons vous avoir
très souvent. Avec la belle saison qui vient, nous allons souvent dîner en plein
air. Cela ne vous ennuie pas les petits dîners au Bois? bien, bien, ce sera très
gentil. Est-ce que vous n’allez pas travailler de votre métier, vous!
cria-t-elle au petit pianiste, afin de faire montre, devant un nouveau de
l’importance de Forcheville, à la fois de son esprit et de son pouvoir
tyrannique sur les fidèles.

— M. de Forcheville était en train de me dire du mal de toi, dit Mme Cottard à
son mari quand il rentra au salon.

Et lui, poursuivant l’idée de la noblesse de Forcheville qui l’occupait depuis
le commencement du dîner, lui dit:

—«Je soigne en ce moment une baronne, la baronne Putbus, les Putbus étaient aux
Croisades, n’est-ce pas? Ils ont, en Poméranie, un lac qui est grand comme dix
fois la place de la Concorde. Je la soigne pour de l’arthrite sèche, c’est une
femme charmante. Elle connaît du reste Mme Verdurin, je crois.

Ce qui permit à Forcheville, quand il se retrouva, un moment après, seul avec
Mme Cottard, de compléter le jugement favorable qu’il avait porté sur son mari:

— Et puis il est intéressant, on voit qu’il connaît du monde. Dame, ça sait tant
de choses, les médecins.

— Je vais jouer la phrase de la Sonate pour M. Swann? dit le pianiste.

— Ah! bigre! ce n’est pas au moins le «Serpent à Sonates»? demanda M. de
Forcheville pour faire de l’effet.

Mais le docteur Cottard, qui n’avait jamais entendu ce calembour, ne le comprit
pas et crut à une erreur de M. de Forcheville. Il s’approcha vivement pour la
rectifier:

—«Mais non, ce n’est pas serpent à sonates qu’on dit, c’est serpent à
sonnettes», dit-il d’un ton zélé, impatient et triomphal.

Forcheville lui expliqua le calembour. Le docteur rougit.

— Avouez qu’il est drôle, docteur?

— Oh! je le connais depuis si longtemps, répondit Cottard.

Mais ils se turent; sous l’agitation des trémolos de violon qui la protégeaient
de leur tenue frémissante à deux octaves de là— et comme dans un pays de
montagne, derrière l’immobilité apparente et vertigineuse d’une cascade, on
aperçoit, deux cents pieds plus bas, la forme minuscule d’une promeneuse — la
petite phrase venait d’apparaître, lointaine, gracieuse, protégée par le long
déferlement du rideau transparent, incessant et sonore. Et Swann, en son cœur,
s’adressa à elle comme à une confidente de son amour, comme à une amie d’Odette
qui devrait bien lui dire de ne pas faire attention à ce Forcheville.

— Ah! vous arrivez tard, dit Mme Verdurin à un fidèle qu’elle n’avait invité
qu’en «cure-dents», «nous avons eu «un» Brichot incomparable, d’une éloquence!
Mais il est parti. N’est-ce pas, monsieur Swann? Je crois que c’est la première
fois que vous vous rencontriez avec lui, dit-elle pour lui faire remarquer que
c’était à elle qu’il devait de le connaître. «N’est-ce pas, il a été délicieux,
notre Brichot?»

Swann s’inclina poliment.

— Non? il ne vous a pas intéressé? lui demanda sèchement Mme Verdurin.

—«Mais si, madame, beaucoup, j’ai été ravi. Il est peut-être un peu péremptoire
et un peu jovial pour mon goût. Je lui voudrais parfois un peu d’hésitations et
de douceur, mais on sent qu’il sait tant de choses et il a l’air d’un bien brave
homme.

Tour le monde se retira fort tard. Les premiers mots de Cottard à sa femme
furent:

— J’ai rarement vu Mme Verdurin aussi en verve que ce soir.

— Qu’est-ce que c’est exactement que cette Mme Verdurin, un demi-castor? dit
Forcheville au peintre à qui il proposa de revenir avec lui.

Odette le vit s’éloigner avec regret, elle n’osa pas ne pas revenir avec Swann,
mais fut de mauvaise humeur en voiture, et quand il lui demanda s’il devait
entrer chez elle, elle lui dit: «Bien entendu» en haussant les épaules avec
impatience. Quand tous les invités furent partis, Mme Verdurin dit à son mari:

— As-tu remarqué comme Swann a ri d’un rire niais quand nous avons parlé de Mme
La Trémoïlle?»

Elle avait remarqué que devant ce nom Swann et Forcheville avaient plusieurs
fois supprimé la particule. Ne doutant pas que ce fût pour montrer qu’ils
n’étaient pas intimidés par les titres, elle souhaitait d’imiter leur fierté,
mais n’avait pas bien saisi par quelle forme grammaticale elle se traduisait.
Aussi sa vicieuse façon de parler l’emportant sur son intransigeance
républicaine, elle disait encore les de La Trémoïlle ou plutôt par une
abréviation en usage dans les paroles des chansons de café-concert et les
légendes des caricaturistes et qui dissimulait le de, les d’La Trémoïlle, mais
elle se rattrapait en disant: «Madame La Trémoïlle.» «La Duchesse, comme dit
Swann», ajouta-t-elle ironiquement avec un sourire qui prouvait qu’elle ne
faisait que citer et ne prenait pas à son compte une dénomination aussi naïve et
ridicule.

— Je te dirai que je l’ai trouvé extrêmement bête.

Et M. Verdurin lui répondit:

— Il n’est pas franc, c’est un monsieur cauteleux, toujours entre le zist et le
zest. Il veut toujours ménager la chèvre et le chou. Quelle différence avec
Forcheville. Voilà au moins un homme qui vous dit carrément sa façon de penser.
Ça vous plaît ou ça ne vous plaît pas. Ce n’est pas comme l’autre qui n’est
jamais ni figue ni raisin. Du reste Odette a l’air de préférer joliment le
Forcheville, et je lui donne raison. Et puis enfin puisque Swann veut nous la
faire à l’homme du monde, au champion des duchesses, au moins l’autre a son
titre; il est toujours comte de Forcheville, ajouta-t-il d’un air délicat, comme
si, au courant de l’histoire de ce comté, il en soupesait minutieusement la
valeur particulière.

— Je te dirai, dit Mme Verdurin, qu’il a cru devoir lancer contre Brichot
quelques insinuations venimeuses et assez ridicules. Naturellement, comme il a
vu que Brichot était aimé dans la maison, c’était une manière de nous atteindre,
de bêcher notre dîner. On sent le bon petit camarade qui vous débinera en
sortant.

— Mais je te l’ai dit, répondit M. Verdurin, c’est le raté, le petit individu
envieux de tout ce qui est un peu grand.

En réalité il n’y avait pas un fidèle qui ne fût plus malveillant que Swann;
mais tous ils avaient la précaution d’assaisonner leurs médisances de
plaisanteries connues, d’une petite pointe d’émotion et de cordialité; tandis
que la moindre réserve que se permettait Swann, dépouillée des formules de
convention telles que: «Ce n’est pas du mal que nous disons» et auxquelles il
dédaignait de s’abaisser, paraissait une perfidie. Il y a des auteurs originaux
dont la moindre hardiesse révolte parce qu’ils n’ont pas d’abord flatté les
goûts du public et ne lui ont pas servi les lieux communs auxquels il est
habitué; c’est de la même manière que Swann indignait M. Verdurin. Pour Swann
comme pour eux, c’était la nouveauté de son langage qui faisait croire à là
noirceur de ses intentions.

Swann ignorait encore la disgrâce dont il était menacé chez les Verdurin et
continuait à voir leurs ridicules en beau, au travers de son amour.

Il n’avait de rendez-vous avec Odette, au moins le plus souvent, que le soir;
mais le jour, ayant peur de la fatiguer de lui en allant chez elle, il aurait
aimé du moins ne pas cesser d’occuper sa pensée, et à tous moments il cherchait
à trouver une occasion d’y intervenir, mais d’une façon agréable pour elle. Si,
à la devanture d’un fleuriste ou d’un joaillier, la vue d’un arbuste ou d’un
bijou le charmait, aussitôt il pensait à les envoyer à Odette, imaginant le
plaisir qu’ils lui avaient procuré, ressenti par elle, venant accroître la
tendresse qu’elle avait pour lui, et les faisait porter immédiatement rue La
Pérouse, pour ne pas retarder l’instant où, comme elle recevrait quelque chose
de lui, il se sentirait en quelque sorte près d’elle. Il voulait surtout qu’elle
les reçût avant de sortir pour que la reconnaissance qu’elle éprouverait lui
valût un accueil plus tendre quand elle le verrait chez les Verdurin, ou même,
qui sait, si le fournisseur faisait assez diligence, peut-être une lettre
qu’elle lui enverrait avant le dîner, ou sa venue à elle en personne chez lui,
en une visite supplémentaire, pour le remercier. Comme jadis quand il
expérimentait sur la nature d’Odette les réactions du dépit, il cherchait par
celles de la gratitude à tirer d’elle des parcelles intimes de sentiment qu’elle
ne lui avait pas révélées encore.

Souvent elle avait des embarras d’argent et, pressée par une dette, le priait de
lui venir en aide. Il en était heureux comme de tout ce qui pouvait donner à
Odette une grande idée de l’amour qu’il avait pour elle, ou simplement une
grande idée de son influence, de l’utilité dont il pouvait lui être. Sans doute
si on lui avait dit au début: «c’est ta situation qui lui plaît», et maintenant:
«c’est pour ta fortune qu’elle t’aime», il ne l’aurait pas cru, et n’aurait pas
été d’ailleurs très mécontent qu’on se la figurât tenant à lui — qu’on les
sentît unis l’un à l’autre — par quelque chose d’aussi fort que le snobisme ou
l’argent. Mais, même s’il avait pensé que c’était vrai, peut-être n’eût-il pas
souffert de découvrir à l’amour d’Odette pour lui cet état plus durable que
l’agrément ou les qualités qu’elle pouvait lui trouver: l’intérêt, l’intérêt qui
empêcherait de venir jamais le jour où elle aurait pu être tentée de cesser de
le voir. Pour l’instant, en la comblant de présents, en lui rendant des
services, il pouvait se reposer sur des avantages extérieurs à sa personne, à
son intelligence, du soin épuisant de lui plaire par lui-même. Et cette volupté
d’être amoureux, de ne vivre que d’amour, de la réalité de laquelle il doutait
parfois, le prix dont en somme il la payait, en dilettante de sensations
immatérielles, lui en augmentait la valeur — comme on voit des gens incertains
si le spectacle de la mer et le bruit de ses vagues sont délicieux, s’en
convaincre ainsi que de la rare qualité de leurs goûts désintéressés, en louant
cent francs par jour la chambre d’hôtel qui leur permet de les goûter.

Un jour que des réflexions de ce genre le ramenaient encore au souvenir du temps
où on lui avait parlé d’Odette comme d’une femme entretenue, et où une fois de
plus il s’amusait à opposer cette personnification étrange: la femme entretenue
— chatoyant amalgame d’éléments inconnus et diaboliques, serti, comme une
apparition de Gustave Moreau, de fleurs vénéneuses entrelacées à des joyaux
précieux — et cette Odette sur le visage de qui il avait vu passer les mêmes
sentiments de pitié pour un malheureux, de révolte contre une injustice, de
gratitude pour un bienfait, qu’il avait vu éprouver autrefois par sa propre
mère, par ses amis, cette Odette dont les propos avaient si souvent trait aux
choses qu’il connaissait le mieux lui-même, à ses collections, à sa chambre, à
son vieux domestique, au banquier chez qui il avait ses titres, il se trouva que
cette dernière image du banquier lui rappela qu’il aurait à y prendre de
l’argent. En effet, si ce mois-ci il venait moins largement à l’aide d’Odette
dans ses difficultés matérielles qu’il n’avait fait le mois dernier où il lui
avait donné cinq mille francs, et s’il ne lui offrait pas une rivière de
diamants qu’elle désirait, il ne renouvellerait pas en elle cette admiration
qu’elle avait pour sa générosité, cette reconnaissance, qui le rendaient si
heureux, et même il risquerait de lui faire croire que son amour pour elle,
comme elle en verrait les manifestations devenir moins grandes, avait diminué.
Alors, tout d’un coup, il se demanda si cela, ce n’était pas précisément
l’«entretenir» (comme si, en effet, cette notion d’entretenir pouvait être
extraite d’éléments non pas mystérieux ni pervers, mais appartenant au fond
quotidien et privé de sa vie, tels que ce billet de mille francs, domestique et
familier, déchiré et recollé, que son valet de chambre, après lui avoir payé les
comptes du mois et le terme, avait serré dans le tiroir du vieux bureau où Swann
l’avait repris pour l’envoyer avec quatre autres à Odette) et si on ne pouvait
pas appliquer à Odette, depuis qu’il la connaissait (car il ne soupçonna pas un
instant qu’elle eût jamais pu recevoir d’argent de personne avant lui), ce mot
qu’il avait cru si inconciliable avec elle, de «femme entretenue». Il ne put
approfondir cette idée, car un accès d’une paresse d’esprit, qui était chez lui
congénitale, intermittente et providentielle, vint à ce moment éteindre toute
lumière dans son intelligence, aussi brusquement que, plus tard, quand on eut
installé partout l’éclairage électrique, on put couper l’électricité dans une
maison. Sa pensée tâtonna un instant dans l’obscurité, il retira ses lunettes,
en essuya les verres, se passa la main sur les yeux, et ne revit la lumière que
quand il se retrouva en présence d’une idée toute différente, à savoir qu’il
faudrait tâcher d’envoyer le mois prochain six ou sept mille francs à Odette au
lieu de cinq, à cause de la surprise et de la joie que cela lui causerait.

Le soir, quand il ne restait pas chez lui à attendre l’heure de retrouver Odette
chez les Verdurin ou plutôt dans un des restaurants d’été qu’ils affectionnaient
au Bois et surtout à Saint-Cloud, il allait dîner dans quelqu’une de ces maisons
élégantes dont il était jadis le convive habituel. Il ne voulait pas perdre
contact avec des gens qui — savait-on? pourraient peut-être un jour être utiles
à Odette, et grâce auxquels en attendant il réussissait souvent à lui être
agréable. Puis l’habitude qu’il avait eue longtemps du monde, du luxe, lui en
avait donné, en même temps que le dédain, le besoin, de sorte qu’à partir du
moment où les réduits les plus modestes lui étaient apparus exactement sur le
même pied que les plus princières demeures, ses sens étaient tellement
accoutumés aux secondes qu’il eût éprouvé quelque malaise à se trouver dans les
premiers. Il avait la même considération —à un degré d’identité qu’ils
n’auraient pu croire — pour des petits bourgeois qui faisaient danser au
cinquième étage d’un escalier D, palier à gauche, que pour la princesse de Parme
qui donnait les plus belles fêtes de Paris; mais il n’avait pas la sensation
d’être au bal en se tenant avec les pères dans la chambre à coucher de la
maîtresse de la maison et la vue des lavabos recouverts de serviettes, des lits
transformés en vestiaires, sur le couvre-pied desquels s’entassaient les
pardessus et les chapeaux lui donnait la même sensation d’étouffement que peut
causer aujourd’hui à des gens habitués à vingt ans d’électricité l’odeur d’une
lampe qui charbonne ou d’une veilleuse qui file.

Le jour où il dînait en ville, il faisait atteler pour sept heures et demie; il
s’habillait tout en songeant à Odette et ainsi il ne se trouvait pas seul, car
la pensée constante d’Odette donnait aux moments où il était loin d’elle le même
charme particulier qu’à ceux où elle était là. Il montait en voiture, mais il
sentait que cette pensée y avait sauté en même temps et s’installait sur ses
genoux comme une bête aimée qu’on emmène partout et qu’il garderait avec lui à
table, à l’insu des convives. Il la caressait, se réchauffait à elle, et
éprouvant une sorte de langueur, se laissait aller à un léger frémissement qui
crispait son cou et son nez, et était nouveau chez lui, tout en fixant à sa
boutonnière le bouquet d’ancolies. Se sentant souffrant et triste depuis quelque
temps, surtout depuis qu’Odette avait présenté Forcheville aux Verdurin, Swann
aurait aimé aller se reposer un peu à la campagne. Mais il n’aurait pas eu le
courage de quitter Paris un seul jour pendant qu’Odette y était. L’air était
chaud; c’étaient les plus beaux jours du printemps. Et il avait beau traverser
une ville de pierre pour se rendre en quelque hôtel clos, ce qui était sans
cesse devant ses yeux, c’était un parc qu’il possédait près de Combray, où, dès
quatre heures, avant d’arriver au plant d’asperges, grâce au vent qui vient des
champs de Méséglise, on pouvait goûter sous une charmille autant de fraîcheur
qu’au bord de l’étang cerné de myosotis et de glaïeuls, et où, quand il dînait,
enlacées par son jardinier, couraient autour de la table les groseilles et les
roses.

Après dîner, si le rendez-vous au bois ou à Saint-Cloud était de bonne heure, il
partait si vite en sortant de table — surtout si la pluie menaçait de tomber et
de faire rentrer plus tôt les «fidèles» — qu’une fois la princesse des Laumes
(chez qui on avait dîné tard et que Swann avait quittée avant qu’on servît le
café pour rejoindre les Verdurin dans l’île du Bois) dit:

—«Vraiment, si Swann avait trente ans de plus et une maladie de la vessie, on
l’excuserait de filer ainsi. Mais tout de même il se moque du monde.»

Il se disait que le charme du printemps qu’il ne pouvait pas aller goûter à
Combray, il le trouverait du moins dans l’île des Cygnes ou à Saint-Cloud. Mais
comme il ne pouvait penser qu’à Odette, il ne savait même pas, s’il avait senti
l’odeur des feuilles, s’il y avait eu du clair de lune. Il était accueilli par
la petite phrase de la Sonate jouée dans le jardin sur le piano du restaurant.
S’il n’y en avait pas là, les Verdurin prenaient une grande peine pour en faire
descendre un d’une chambre ou d’une salle à manger: ce n’est pas que Swann fût
rentré en faveur auprès d’eux, au contraire. Mais l’idée d’organiser un plaisir
ingénieux pour quelqu’un, même pour quelqu’un qu’ils n’aimaient pas, développait
chez eux, pendant les moments nécessaires à ces préparatifs, des sentiments
éphémères et occasionnels de sympathie et de cordialité. Parfois il se disait
que c’était un nouveau soir de printemps de plus qui passait, il se contraignait
à faire attention aux arbres, au ciel. Mais l’agitation où le mettait la
présence d’Odette, et aussi un léger malaise fébrile qui ne le quittait guère
depuis quelque temps, le privait du calme et du bien-être qui sont le fond
indispensable aux impressions que peut donner la nature.

Un soir où Swann avait accepté de dîner avec les Verdurin, comme pendant le
dîner il venait de dire que le lendemain il avait un banquet d’anciens
camarades, Odette lui avait répondu en pleine table, devant Forcheville, qui
était maintenant un des fidèles, devant le peintre, devant Cottard:

—«Oui, je sais que vous avez votre banquet, je ne vous verrai donc que chez moi,
mais ne venez pas trop tard.»

Bien que Swann n’eût encore jamais pris bien sérieusement ombrage de l’amitié
d’Odette pour tel ou tel fidèle, il éprouvait une douceur profonde à l’entendre
avouer ainsi devant tous, avec cette tranquille impudeur, leurs rendez-vous
quotidiens du soir, la situation privilégiée qu’il avait chez elle et la
préférence pour lui qui y était impliquée. Certes Swann avait souvent pensé
qu’Odette n’était à aucun degré une femme remarquable; et la suprématie qu’il
exerçait sur un être qui lui était si inférieur n’avait rien qui dût lui
paraître si flatteur à voir proclamer à la face des «fidèles», mais depuis qu’il
s’était aperçu qu’à beaucoup d’hommes Odette semblait une femme ravissante et
désirable, le charme qu’avait pour eux son corps avait éveillé en lui un besoin
douloureux de la maîtriser entièrement dans les moindres parties de son cœur. Et
il avait commencé d’attacher un prix inestimable à ces moments passés chez elle
le soir, où il l’asseyait sur ses genoux, lui faisait dire ce qu’elle pensait
d’une chose, d’une autre, où il recensait les seuls biens à la possession
desquels il tînt maintenant sur terre. Aussi, après ce dîner, la prenant à part,
il ne manqua pas de la remercier avec effusion, cherchant à lui enseigner selon
les degrés de la reconnaissance qu’il lui témoignait, l’échelle des plaisirs
qu’elle pouvait lui causer, et dont le suprême était de le garantir, pendant le
temps que son amour durerait et l’y rendrait vulnérable, des atteintes de la
jalousie.

Quand il sortit le lendemain du banquet, il pleuvait à verse, il n’avait à sa
disposition que sa victoria; un ami lui proposa de le reconduire chez lui en
coupé, et comme Odette, par le fait qu’elle lui avait demandé de venir, lui
avait donné la certitude qu’elle n’attendait personne, c’est l’esprit tranquille
et le cœur content que, plutôt que de partir ainsi dans la pluie, il serait
rentré chez lui se coucher. Mais peut-être, si elle voyait qu’il n’avait pas
l’air de tenir à passer toujours avec elle, sans aucune exception, la fin de la
soirée, négligerait-elle de la lui réserver, justement une fois où il l’aurait
particulièrement désiré.

Il arriva chez elle après onze heures, et, comme il s’excusait de n’avoir pu
venir plus tôt, elle se plaignit que ce fût en effet bien tard, l’orage l’avait
rendue souffrante, elle se sentait mal à la tête et le prévint qu’elle ne le
garderait pas plus d’une demi-heure, qu’à minuit, elle le renverrait; et, peu
après, elle se sentit fatiguée et désira s’endormir.

— Alors, pas de catleyas ce soir? lui dit-il, moi qui espérais un bon petit
catleya.

Et d’un air un peu boudeur et nerveux, elle lui répondit:

—«Mais non, mon petit, pas de catleyas ce soir, tu vois bien que je suis
souffrante!»

—«Cela t’aurait peut-être fait du bien, mais enfin je n’insiste pas.»

Elle le pria d’éteindre la lumière avant de s’en aller, il referma lui-même les
rideaux du lit et partit. Mais quand il fut rentré chez lui, l’idée lui vint
brusquement que peut-être Odette attendait quelqu’un ce soir, qu’elle avait
seulement simulé la fatigue et qu’elle ne lui avait demandé d’éteindre que pour
qu’il crût qu’elle allait s’endormir, qu’aussitôt qu’il avait été parti, elle
l’avait rallumée, et fait rentrer celui qui devait passer la nuit auprès d’elle.
Il regarda l’heure. Il y avait à peu près une heure et demie qu’il l’avait
quittée, il ressortit, prit un fiacre et se fit arrêter tout près de chez elle,
dans une petite rue perpendiculaire à celle sur laquelle donnait derrière son
hôtel et où il allait quelquefois frapper à la fenêtre de sa chambre à coucher
pour qu’elle vînt lui ouvrir; il descendit de voiture, tout était désert et noir
dans ce quartier, il n’eut que quelques pas à faire à pied et déboucha presque
devant chez elle. Parmi l’obscurité de toutes les fenêtres éteintes depuis
longtemps dans la rue, il en vit une seule d’où débordait — entre les volets qui
en pressaient la pulpe mystérieuse et dorée — la lumière qui remplissait la
chambre et qui, tant d’autres soirs, du plus loin qu’il l’apercevait, en
arrivant dans la rue le réjouissait et lui annonçait: «elle est là qui t’attend»
et qui maintenant, le torturait en lui disant: «elle est là avec celui qu’elle
attendait». Il voulait savoir qui; il se glissa le long du mur jusqu’à la
fenêtre, mais entre les lames obliques des volets il ne pouvait rien voir; il
entendait seulement dans le silence de la nuit le murmure d’une conversation.
Certes, il souffrait de voir cette lumière dans l’atmosphère d’or de laquelle se
mouvait derrière le châssis le couple invisible et détesté, d’entendre ce
murmure qui révélait la présence de celui qui était venu après son départ, la
fausseté d’Odette, le bonheur qu’elle était en train de goûter avec lui.

Et pourtant il était content d’être venu: le tourment qui l’avait forcé de
sortir de chez lui avait perdu de son acuité en perdant de son vague, maintenant
que l’autre vie d’Odette, dont il avait eu, à ce moment-là, le brusque et
impuissant soupçon, il la tenait là, éclairée en plein par la lampe, prisonnière
sans le savoir dans cette chambre où, quand il le voudrait, il entrerait la
surprendre et la capturer; ou plutôt il allait frapper aux volets comme il
faisait souvent quand il venait très tard; ainsi du moins, Odette apprendrait
qu’il avait su, qu’il avait vu la lumière et entendu la causerie, et lui, qui,
tout à l’heure, se la représentait comme se riant avec l’autre de ses illusions,
maintenant, c’était eux qu’il voyait, confiants dans leur erreur, trompés en
somme par lui qu’ils croyaient bien loin d’ici et qui, lui, savait déjà qu’il
allait frapper aux volets. Et peut-être, ce qu’il ressentait en ce moment de
presque agréable, c’était autre chose aussi que l’apaisement d’un doute et d’une
douleur: un plaisir de l’intelligence. Si, depuis qu’il était amoureux, les
choses avaient repris pour lui un peu de l’intérêt délicieux qu’il leur trouvait
autrefois, mais seulement là où elles étaient éclairées par le souvenir
d’Odette, maintenant, c’était une autre faculté de sa studieuse jeunesse que sa
jalousie ranimait, la passion de la vérité, mais d’une vérité, elle aussi,
interposée entre lui et sa maîtresse, ne recevant sa lumière que d’elle, vérité
tout individuelle qui avait pour objet unique, d’un prix infini et presque d’une
beauté désintéressée, les actions d’Odette, ses relations, ses projets, son
passé. A toute autre époque de sa vie, les petits faits et gestes quotidiens
d’une personne avaient toujours paru sans valeur à Swann: si on lui en faisait
le commérage, il le trouvait insignifiant, et, tandis qu’il l’écoutait, ce
n’était que sa plus vulgaire attention qui y était intéressée; c’était pour lui
un des moments où il se sentait le plus médiocre. Mais dans cette étrange
période de l’amour, l’individuel prend quelque chose de si profond, que cette
curiosité qu’il sentait s’éveiller en lui à l’égard des moindres occupations
d’une femme, c’était celle qu’il avait eue autrefois pour l’Histoire. Et tout ce
dont il aurait eu honte jusqu’ici, espionner devant une fenêtre, qui sait,
demain, peut-être faire parler habilement les indifférents, soudoyer les
domestiques, écouter aux portes, ne lui semblait plus, aussi bien que le
déchiffrement des textes, la comparaison des témoignages et l’interprétation des
monuments, que des méthodes d’investigation scientifique d’une véritable valeur
intellectuelle et appropriées à la recherche de la vérité.

Sur le point de frapper contre les volets, il eut un moment de honte en pensant
qu’Odette allait savoir qu’il avait eu des soupçons, qu’il était revenu, qu’il
s’était posté dans la rue. Elle lui avait dit souvent l’horreur qu’elle avait
des jaloux, des amants qui espionnent. Ce qu’il allait faire était bien
maladroit, et elle allait le détester désormais, tandis qu’en ce moment encore,
tant qu’il n’avait pas frappé, peut-être, même en le trompant, l’aimait-elle.
Que de bonheurs possibles dont on sacrifie ainsi la réalisation à l’impatience
d’un plaisir immédiat. Mais le désir de connaître la vérité était plus fort et
lui sembla plus noble. Il savait que la réalité de circonstances qu’il eût donné
sa vie pour restituer exactement, était lisible derrière cette fenêtre striée de
lumière, comme sous la couverture enluminée d’or d’un de ces manuscrits précieux
à la richesse artistique elle-même desquels le savant qui les consulte ne peut
rester indifférent. Il éprouvait une volupté à connaître la vérité qui le
passionnait dans cet exemplaire unique, éphémère et précieux, d’une matière
translucide, si chaude et si belle. Et puis l’avantage qu’il se sentait — qu’il
avait tant besoin de se sentir — sur eux, était peut-être moins de savoir, que
de pouvoir leur montrer qu’il savait. Il se haussa sur la pointe des pieds. Il
frappa. On n’avait pas entendu, il refrappa plus fort, la conversation s’arrêta.
Une voix d’homme dont il chercha à distinguer auquel de ceux des amis d’Odette
qu’il connaissait elle pouvait appartenir, demanda:

—«Qui est là?»

Il n’était pas sûr de la reconnaître. Il frappa encore une fois. On ouvrit la
fenêtre, puis les volets. Maintenant, il n’y avait plus moyen de reculer, et,
puisqu’elle allait tout savoir, pour ne pas avoir l’air trop malheureux, trop
jaloux et curieux, il se contenta de crier d’un air négligent et gai:

—«Ne vous dérangez pas, je passais par là, j’ai vu de la lumière, j’ai voulu
savoir si vous n’étiez plus souffrante.»

Il regarda. Devant lui, deux vieux messieurs étaient à la fenêtre, l’un tenant
une lampe, et alors, il vit la chambre, une chambre inconnue. Ayant l’habitude,
quand il venait chez Odette très tard, de reconnaître sa fenêtre à ce que
c’était la seule éclairée entre les fenêtres toutes pareilles, il s’était trompé
et avait frappé à la fenêtre suivante qui appartenait à la maison voisine. Il
s’éloigna en s’excusant et rentra chez lui, heureux que la satisfaction de sa
curiosité eût laissé leur amour intact et qu’après avoir simulé depuis si
longtemps vis-à-vis d’Odette une sorte d’indifférence, il ne lui eût pas donné,
par sa jalousie, cette preuve qu’il l’aimait trop, qui, entre deux amants,
dispense, à tout jamais, d’aimer assez, celui qui la reçoit. Il ne lui parla pas
de cette mésaventure, lui-même n’y songeait plus. Mais, par moments, un
mouvement de sa pensée venait en rencontrer le souvenir qu’elle n’avait pas
aperçu, le heurtait, l’enfonçait plus avant et Swann avait ressenti une douleur
brusque et profonde. Comme si ç’avait été une douleur physique, les pensées de
Swann ne pouvaient pas l’amoindrir; mais du moins la douleur physique, parce
qu’elle est indépendante de la pensée, la pensée peut s’arrêter sur elle,
constater qu’elle a diminué, qu’elle a momentanément cessé! Mais cette
douleur-là, la pensée, rien qu’en se la rappelant, la recréait. Vouloir n’y pas
penser, c’était y penser encore, en souffrir encore. Et quand, causant avec des
amis, il oubliait son mal, tout d’un coup un mot qu’on lui disait le faisait
changer de visage, comme un blessé dont un maladroit vient de toucher sans
précaution le membre douloureux. Quand il quittait Odette, il était heureux, il
se sentait calme, il se rappelait les sourires qu’elle avait eus, railleurs, en
parlant de tel ou tel autre, et tendres pour lui, la lourdeur de sa tête qu’elle
avait détachée de son axe pour l’incliner, la laisser tomber, presque malgré
elle, sur ses lèvres, comme elle avait fait la première fois en voiture, les
regards mourants qu’elle lui avait jetés pendant qu’elle était dans ses bras,
tout en contractant frileusement contre l’épaule sa tête inclinée.

Mais aussitôt sa jalousie, comme si elle était l’ombre de son amour, se
complétait du double de ce nouveau sourire qu’elle lui avait adressé le soir
même — et qui, inverse maintenant, raillait Swann et se chargeait d’amour pour
un autre — de cette inclinaison de sa tête mais renversée vers d’autres lèvres,
et, données à un autre, de toutes les marques de tendresse qu’elle avait eues
pour lui. Et tous les souvenirs voluptueux qu’il emportait de chez elle, étaient
comme autant d’esquisses, de «projets» pareils à ceux que vous soumet un
décorateur, et qui permettaient à Swann de se faire une idée des attitudes
ardentes ou pâmées qu’elle pouvait avoir avec d’autres. De sorte qu’il en
arrivait à regretter chaque plaisir qu’il goûtait près d’elle, chaque caresse
inventée et dont il avait eu l’imprudence de lui signaler la douceur, chaque
grâce qu’il lui découvrait, car il savait qu’un instant après, elles allaient
enrichir d’instruments nouveaux son supplice.

Celui-ci était rendu plus cruel encore quand revenait à Swann le souvenir d’un
bref regard qu’il avait surpris, il y avait quelques jours, et pour la première
fois, dans les yeux d’Odette. C’était après dîner, chez les Verdurin. Soit que
Forcheville sentant que Saniette, son beau-frère, n’était pas en faveur chez
eux, eût voulu le prendre comme tête de Turc et briller devant eux à ses dépens,
soit qu’il eût été irrité par un mot maladroit que celui-ci venait de lui dire
et qui, d’ailleurs, passa inaperçu pour les assistants qui ne savaient pas
quelle allusion désobligeante il pouvait renfermer, bien contre le gré de celui
qui le prononçait sans malice aucune, soit enfin qu’il cherchât depuis quelque
temps une occasion de faire sortir de la maison quelqu’un qui le connaissait
trop bien et qu’il savait trop délicat pour qu’il ne se sentît pas gêné à
certains moments rien que de sa présence, Forcheville répondit à ce propos
maladroit de Saniette avec une telle grossièreté, se mettant à l’insulter,
s’enhardissant, au fur et à mesure qu’il vociférait, de l’effroi, de la douleur,
des supplications de l’autre, que le malheureux, après avoir demandé à Mme
Verdurin s’il devait rester, et n’ayant pas reçu de réponse, s’était retiré en
balbutiant, les larmes aux yeux. Odette avait assisté impassible à cette scène,
mais quand la porte se fut refermée sur Saniette, faisant descendre en quelque
sorte de plusieurs crans l’expression habituelle de son visage, pour pouvoir se
trouver dans la bassesse, de plain-pied avec Forcheville, elle avait brillanté
ses prunelles d’un sourire sournois de félicitations pour l’audace qu’il avait
eue, d’ironie pour celui qui en avait été victime; elle lui avait jeté un regard
de complicité dans le mal, qui voulait si bien dire: «voilà une exécution, ou je
ne m’y connais pas. Avez-vous vu son air penaud, il en pleurait», que
Forcheville, quand ses yeux rencontrèrent ce regard, dégrisé soudain de la
colère ou de la simulation de colère dont il était encore chaud, sourit et
répondit:

—«Il n’avait qu’à être aimable, il serait encore ici, une bonne correction peut
être utile à tout âge.»

Un jour que Swann était sorti au milieu de l’après-midi pour faire une visite,
n’ayant pas trouvé la personne qu’il voulait rencontrer, il eut l’idée d’entrer
chez Odette à cette heure où il n’allait jamais chez elle, mais où il savait
qu’elle était toujours à la maison à faire sa sieste ou à écrire des lettres
avant l’heure du thé, et où il aurait plaisir à la voir un peu sans la déranger.
Le concierge lui dit qu’il croyait qu’elle était là; il sonna, crut entendre du
bruit, entendre marcher, mais on n’ouvrit pas. Anxieux, irrité, il alla dans la
petite rue où donnait l’autre face de l’hôtel, se mit devant la fenêtre de la
chambre d’Odette; les rideaux l’empêchaient de rien voir, il frappa avec force
aux carreaux, appela; personne n’ouvrit. Il vit que des voisins le regardaient.
Il partit, pensant qu’après tout, il s’était peut-être trompé en croyant
entendre des pas; mais il en resta si préoccupé qu’il ne pouvait penser à autre
chose. Une heure après, il revint. Il la trouva; elle lui dit qu’elle était chez
elle tantôt quand il avait sonné, mais dormait; la sonnette l’avait éveillée,
elle avait deviné que c’était Swann, elle avait couru après lui, mais il était
déjà parti. Elle avait bien entendu frapper aux carreaux. Swann reconnut tout de
suite dans ce dire un de ces fragments d’un fait exact que les menteurs pris de
court se consolent de faire entrer dans la composition du fait faux qu’ils
inventent, croyant y faire sa part et y dérober sa ressemblance à la Vérité.
Certes quand Odette venait de faire quelque chose qu’elle ne voulait pas
révéler, elle le cachait bien au fond d’elle-même. Mais dès qu’elle se trouvait
en présence de celui à qui elle voulait mentir, un trouble la prenait, toutes
ses idées s’effondraient, ses facultés d’invention et de raisonnement étaient
paralysées, elle ne trouvait plus dans sa tête que le vide, il fallait pourtant
dire quelque chose et elle rencontrait à sa portée précisément la chose qu’elle
avait voulu dissimuler et qui étant vraie, était restée là. Elle en détachait un
petit morceau, sans importance par lui-même, se disant qu’après tout c’était
mieux ainsi puisque c’était un détail véritable qui n’offrait pas les mêmes
dangers qu’un détail faux. «Ça du moins, c’est vrai, se disait-elle, c’est
toujours autant de gagné, il peut s’informer, il reconnaîtra que c’est vrai, ce
n’est toujours pas ça qui me trahira.» Elle se trompait, c’était cela qui la
trahissait, elle ne se rendait pas compte que ce détail vrai avait des angles
qui ne pouvaient s’emboîter que dans les détails contigus du fait vrai dont elle
l’avait arbitrairement détaché et qui, quels que fussent les détails inventés
entre lesquels elle le placerait, révéleraient toujours par la matière excédante
et les vides non remplis, que ce n’était pas d’entre ceux-là qu’il venait. «Elle
avoue qu’elle m’avait entendu sonner, puis frapper, et qu’elle avait cru que
c’était moi, qu’elle avait envie de me voir, se disait Swann. Mais cela ne
s’arrange pas avec le fait qu’elle n’ait pas fait ouvrir.»

Mais il ne lui fit pas remarquer cette contradiction, car il pensait que, livrée
à elle-même, Odette produirait peut-être quelque mensonge qui serait un faible
indice de la vérité; elle parlait; il ne l’interrompait pas, il recueillait avec
une piété avide et douloureuse ces mots qu’elle lui disait et qu’il sentait
(justement, parce qu’elle la cachait derrière eux tout en lui parlant) garder
vaguement, comme le voile sacré, l’empreinte, dessiner l’incertain modelé, de
cette réalité infiniment précieuse et hélas introuvable:— ce qu’elle faisait
tantôt à trois heures, quand il était venu — de laquelle il ne posséderait
jamais que ces mensonges, illisibles et divins vestiges, et qui n’existait plus
que dans le souvenir receleur de cet être qui la contemplait sans savoir
l’apprécier, mais ne la lui livrerait pas. Certes il se doutait bien par moments
qu’en elles-mêmes les actions quotidiennes d’Odette n’étaient pas passionnément
intéressantes, et que les relations qu’elle pouvait avoir avec d’autres hommes
n’exhalaient pas naturellement d’une façon universelle et pour tout être
pensant, une tristesse morbide, capable de donner la fièvre du suicide. Il se
rendait compte alors que cet intérêt, cette tristesse n’existaient qu’en lui
comme une maladie, et que quand celle-ci serait guérie, les actes d’Odette, les
baisers qu’elle aurait pu donner redeviendraient inoffensifs comme ceux de tant
d’autres femmes. Mais que la curiosité douloureuse que Swann y portait
maintenant n’eût sa cause qu’en lui, n’était pas pour lui faire trouver
déraisonnable de considérer cette curiosité comme importante et de mettre tout
en œuvre pour lui donner satisfaction. C’est que Swann arrivait à un âge dont la
philosophie — favorisée par celle de l’époque, par celle aussi du milieu où
Swann avait beaucoup vécu, de cette coterie de la princesse des Laumes où il
était convenu qu’on est intelligent dans la mesure où on doute de tout et où on
ne trouvait de réel et d’incontestable que les goûts de chacun — n’est déjà plus
celle de la jeunesse, mais une philosophie positive, presque médicale, d’hommes
qui au lieu d’extérioriser les objets de leurs aspirations, essayent de dégager
de leurs années déjà écoulées un résidu fixe d’habitudes, de passions qu’ils
puissent considérer en eux comme caractéristiques et permanentes et auxquelles,
délibérément, ils veilleront d’abord que le genre d’existence qu’ils adoptent
puisse donner satisfaction. Swann trouvait sage de faire dans sa vie la part de
la souffrance qu’il éprouvait à ignorer ce qu’avait fait Odette, aussi bien que
la part de la recrudescence qu’un climat humide causait à son eczéma; de prévoir
dans son budget une disponibilité importante pour obtenir sur l’emploi des
journées d’Odette des renseignements sans lesquels il se sentirait malheureux,
aussi bien qu’il en réservait pour d’autres goûts dont il savait qu’il pouvait
attendre du plaisir, au moins avant qu’il fût amoureux, comme celui des
collections et de la bonne cuisine.

Quand il voulut dire adieu à Odette pour rentrer, elle lui demanda de rester
encore et le retint même vivement, en lui prenant le bras, au moment où il
allait ouvrir là porte pour sortir. Mais il n’y prit pas garde, car, dans la
multitude des gestes, des propos, des petits incidents qui remplissent une
conversation, il est inévitable que nous passions, sans y rien remarquer qui
éveille notre attention, près de ceux qui cachent une vérité que nos soupçons
cherchent au hasard, et que nous nous arrêtions au contraire à ceux sous
lesquels il n’y a rien. Elle lui redisait tout le temps: «Quel malheur que toi,
qui ne viens jamais l’après-midi, pour une fois que cela t’arrive, je ne t’aie
pas vu.» Il savait bien qu’elle n’était pas assez amoureuse de lui pour avoir un
regret si vif d’avoir manqué sa visite, mais comme elle était bonne, désireuse
de lui faire plaisir, et souvent triste quand elle l’avait contrarié, il trouva
tout naturel qu’elle le fût cette fois de l’avoir privé de ce plaisir de passer
une heure ensemble qui était très grand, non pour elle, mais pour lui. C’était
pourtant une chose assez peu importante pour que l’air douloureux qu’elle
continuait d’avoir finît par l’étonner. Elle rappelait ainsi plus encore qu’il
ne le trouvait d’habitude, les figures de femmes du peintre de la Primavera.
Elle avait en ce moment leur visage abattu et navré qui semble succomber sous le
poids d’une douleur trop lourde pour elles, simplement quand elles laissent
l’enfant Jésus jouer avec une grenade ou regardent Moïse verser de l’eau dans
une auge. Il lui avait déjà vu une fois une telle tristesse, mais ne savait plus
quand. Et tout d’un coup, il se rappela: c’était quand Odette avait menti en
parlant à Mme Verdurin le lendemain de ce dîner où elle n’était pas venue sous
prétexte qu’elle était malade et en réalité pour rester avec Swann. Certes,
eût-elle été la plus scrupuleuse des femmes qu’elle n’aurait pu avoir de remords
d’un mensonge aussi innocent. Mais ceux que faisait couramment Odette l’étaient
moins et servaient à empêcher des découvertes qui auraient pu lui créer avec les
uns ou avec les autres, de terribles difficultés. Aussi quand elle mentait,
prise de peur, se sentant peu armée pour se défendre, incertaine du succès, elle
avait envie de pleurer, par fatigue, comme certains enfants qui n’ont pas dormi.
Puis elle savait que son mensonge lésait d’ordinaire gravement l’homme à qui
elle le faisait, et à la merci duquel elle allait peut-être tomber si elle
mentait mal. Alors elle se sentait à la fois humble et coupable devant lui. Et
quand elle avait à faire un mensonge insignifiant et mondain, par association de
sensations et de souvenirs, elle éprouvait le malaise d’un surmenage et le
regret d’une méchanceté.

Quel mensonge déprimant était-elle en train de faire à Swann pour qu’elle eût ce
regard douloureux, cette voix plaintive qui semblaient fléchir sous l’effort
qu’elle s’imposait, et demander grâce? Il eut l’idée que ce n’était pas
seulement la vérité sur l’incident de l’après-midi qu’elle s’efforçait de lui
cacher, mais quelque chose de plus actuel, peut-être de non encore survenu et de
tout prochain, et qui pourrait l’éclairer sur cette vérité. A ce moment, il
entendit un coup de sonnette. Odette ne cessa plus de parler, mais ses paroles
n’étaient qu’un gémissement: son regret de ne pas avoir vu Swann dans
l’après-midi, de ne pas lui avoir ouvert, était devenu un véritable désespoir.

On entendit la porte d’entrée se refermer et le bruit d’une voiture, comme si
repartait une personne — celle probablement que Swann ne devait pas rencontrer
—à qui on avait dit qu’Odette était sortie. Alors en songeant que rien qu’en
venant à une heure où il n’en avait pas l’habitude, il s’était trouvé déranger
tant de choses qu’elle ne voulait pas qu’il sût, il éprouva un sentiment de
découragement, presque de détresse. Mais comme il aimait Odette, comme il avait
l’habitude de tourner vers elle toutes ses pensées, la pitié qu’il eût pu
s’inspirer à lui-même ce fut pour elle qu’il la ressentit, et il murmura:
«Pauvre chérie!» Quand il la quitta, elle prit plusieurs lettres qu’elle avait
sur sa table et lui demanda s’il ne pourrait pas les mettre à la poste. Il les
emporta et, une fois rentré, s’aperçut qu’il avait gardé les lettres sur lui. Il
retourna jusqu’à la poste, les tira de sa poche et avant de les jeter dans la
boîte regarda les adresses. Elles étaient toutes pour des fournisseurs, sauf une
pour Forcheville. Il la tenait dans sa main. Il se disait: «Si je voyais ce
qu’il y a dedans, je saurais comment elle l’appelle, comment elle lui parle,
s’il y a quelque chose entre eux. Peut-être même qu’en ne la regardant pas, je
commets une indélicatesse à l’égard d’Odette, car c’est la seule manière de me
délivrer d’un soupçon peut-être calomnieux pour elle, destiné en tous cas à la
faire souffrir et que rien ne pourrait plus détruire, une fois la lettre
partie.»

Il rentra chez lui en quittant la poste, mais il avait gardé sur lui cette
dernière lettre. Il alluma une bougie et en approcha l’enveloppe qu’il n’avait
pas osé ouvrir. D’abord il ne put rien lire, mais l’enveloppe était mince, et en
la faisant adhérer à la carte dure qui y était incluse, il put à travers sa
transparence, lire les derniers mots. C’était une formule finale très froide.
Si, au lieu que ce fût lui qui regardât une lettre adressée à Forcheville, c’eût
été Forcheville qui eût lu une lettre adressée à Swann, il aurait pu voir des
mots autrement tendres! Il maintint immobile la carte qui dansait dans
l’enveloppe plus grande qu’elle, puis, la faisant glisser avec le pouce, en
amena successivement les différentes lignes sous la partie de l’enveloppe qui
n’était pas doublée, la seule à travers laquelle on pouvait lire.

Malgré cela il ne distinguait pas bien. D’ailleurs cela ne faisait rien car il
en avait assez vu pour se rendre compte qu’il s’agissait d’un petit événement
sans importance et qui ne touchait nullement à des relations amoureuses, c’était
quelque chose qui se rapportait à un oncle d’Odette. Swann avait bien lu au
commencement de la ligne: «J’ai eu raison», mais ne comprenait pas ce qu’Odette
avait eu raison de faire, quand soudain, un mot qu’il n’avait pas pu déchiffrer
d’abord, apparut et éclaira le sens de la phrase tout entière: «J’ai eu raison
d’ouvrir, c’était mon oncle.» D’ouvrir! alors Forcheville était là tantôt quand
Swann avait sonné et elle l’avait fait partir, d’où le bruit qu’il avait
entendu.

Alors il lut toute la lettre; à la fin elle s’excusait d’avoir agi aussi sans
façon avec lui et lui disait qu’il avait oublié ses cigarettes chez elle, la
même phrase qu’elle avait écrite à Swann une des premières fois qu’il était
venu. Mais pour Swann elle avait ajouté: puissiez-vous y avoir laissé votre
cœur, je ne vous aurais pas laissé le reprendre. Pour Forcheville rien de tel:
aucune allusion qui pût faire supposer une intrigue entre eux. A vrai dire
d’ailleurs, Forcheville était en tout ceci plus trompé que lui puisque Odette
lui écrivait pour lui faire croire que le visiteur était son oncle. En somme,
c’était lui, Swann, l’homme à qui elle attachait de l’importance et pour qui
elle avait congédié l’autre. Et pourtant, s’il n’y avait rien entre Odette et
Forcheville, pourquoi n’avoir pas ouvert tout de suite, pourquoi avoir dit:
«J’ai bien fait d’ouvrir, c’était mon oncle»; si elle ne faisait rien de mal à
ce moment-là, comment Forcheville pourrait-il même s’expliquer qu’elle eût pu ne
pas ouvrir? Swann restait là, désolé, confus et pourtant heureux, devant cette
enveloppe qu’Odette lui avait remise sans crainte, tant était absolue la
confiance qu’elle avait en sa délicatesse, mais à travers le vitrage transparent
de laquelle se dévoilait à lui, avec le secret d’un incident qu’il n’aurait
jamais cru possible de connaître, un peu de la vie d’Odette, comme dans une
étroite section lumineuse pratiquée à même l’inconnu. Puis sa jalousie s’en
réjouissait, comme si cette jalousie eût eu une vitalité indépendante, égoïste,
vorace de tout ce qui la nourrirait, fût-ce aux dépens de lui-même. Maintenant
elle avait un aliment et Swann allait pouvoir commencer à s’inquiéter chaque
jour des visites qu’Odette avait reçues vers cinq heures, à chercher à apprendre
où se trouvait Forcheville à cette heure-là. Car la tendresse de Swann
continuait à garder le même caractère que lui avait imprimé dès le début à la
fois l’ignorance où il était de l’emploi des journées d’Odette et la paresse
cérébrale qui l’empêchait de suppléer à l’ignorance par l’imagination. Il ne fut
pas jaloux d’abord de toute la vie d’Odette, mais des seuls moments où une
circonstance, peut-être mal interprétée, l’avait amené à supposer qu’Odette
avait pu le tromper. Sa jalousie, comme une pieuvre qui jette une première, puis
une seconde, puis une troisième amarre, s’attacha solidement à ce moment de cinq
heures du soir, puis à un autre, puis à un autre encore. Mais Swann ne savait
pas inventer ses souffrances. Elles n’étaient que le souvenir, la perpétuation
d’une souffrance qui lui était venue du dehors.

Mais là tout lui en apportait. Il voulut éloigner Odette de Forcheville,
l’emmener quelques jours dans le Midi. Mais il croyait qu’elle était désirée par
tous les hommes qui se trouvaient dans l’hôtel et qu’elle-même les désirait.
Aussi lui qui jadis en voyage recherchait les gens nouveaux, les assemblées
nombreuses, on le voyait sauvage, fuyant la société des hommes comme si elle
l’eût cruellement blessé. Et comment n’aurait-il pas été misanthrope quand dans
tout homme il voyait un amant possible pour Odette? Et ainsi sa jalousie plus
encore que n’avait fait le goût voluptueux et riant qu’il avait d’abord pour
Odette, altérait le caractère de Swann et changeait du tout au tout, aux yeux
des autres, l’aspect même des signes extérieurs par lesquels ce caractère se
manifestait.

Un mois après le jour où il avait lu la lettre adressée par Odette à
Forcheville, Swann alla à un dîner que les Verdurin donnaient au Bois. Au moment
où on se préparait à partir, il remarqua des conciliabules entre Mme Verdurin et
plusieurs des invités et crut comprendre qu’on rappelait au pianiste de venir le
lendemain à une partie à Chatou; or, lui, Swann, n’y était pas invité.

Les Verdurin n’avaient parlé qu’à demi-voix et en termes vagues, mais le
peintre, distrait sans doute, s’écria:

—«Il ne faudra aucune lumière et qu’il joue la sonate Clair de lune dans
l’obscurité pour mieux voir s’éclairer les choses.»

Mme Verdurin, voyant que Swann était à deux pas, prit cette expression où le
désir de faire taire celui qui parle et de garder un air innocent aux yeux de
celui qui entend, se neutralise en une nullité intense du regard, où l’immobile
signe d’intelligence du complice se dissimule sous les sourires de l’ingénu et
qui enfin, commune à tous ceux qui s’aperçoivent d’une gaffe, la révèle
instantanément sinon à ceux qui la font, du moins à celui qui en est l’objet.
Odette eut soudain l’air d’une désespérée qui renonce à lutter contre les
difficultés écrasantes de la vie, et Swann comptait anxieusement les minutes qui
le séparaient du moment où, après avoir quitté ce restaurant, pendant le retour
avec elle, il allait pouvoir lui demander des explications, obtenir qu’elle
n’allât pas le lendemain à Chatou ou qu’elle l’y fit inviter et apaiser dans ses
bras l’angoisse qu’il ressentait. Enfin on demanda leurs voitures. Mme Verdurin
dit à Swann:

— Alors, adieu, à bientôt, n’est-ce pas? tâchant par l’amabilité du regard et la
contrainte du sourire de l’empêcher de penser qu’elle ne lui disait pas, comme
elle eût toujours fait jusqu’ici:

«A demain à Chatou, à après-demain chez moi.»

M. et Mme Verdurin firent monter avec eux Forcheville, la voiture de Swann
s’était rangée derrière la leur dont il attendait le départ pour faire monter
Odette dans la sienne.

—«Odette, nous vous ramenons, dit Mme Verdurin, nous avons une petite place pour
vous à côté de M. de Forcheville.

—«Oui, Madame», répondit Odette.

—«Comment, mais je croyais que je vous reconduisais», s’écria Swann, disant sans
dissimulation, les mots nécessaires, car la portière était ouverte, les secondes
étaient comptées, et il ne pouvait rentrer sans elle dans l’état où il était.

—«Mais Mme Verdurin m’a demandé . . . »

—«Voyons, vous pouvez bien revenir seul, nous vous l’avons laissée assez de
fois, dit Mme Verdurin.»

— Mais c’est que j’avais une chose importante à dire à Madame.

— Eh bien! vous la lui écrirez . . .

— Adieu, lui dit Odette en lui tendant la main.

Il essaya de sourire mais il avait l’air atterré.

— As-tu vu les façons que Swann se permet maintenant avec nous? dit Mme Verdurin
à son mari quand ils furent rentrés. J’ai cru qu’il allait me manger, parce que
nous ramenions Odette. C’est d’une inconvenance, vraiment! Alors, qu’il dise
tout de suite que nous tenons une maison de rendez-vous! Je ne comprends pas
qu’Odette supporte des manières pareilles. Il a absolument l’air de dire: vous
m’appartenez. Je dirai ma manière de penser à Odette, j’espère qu’elle
comprendra.»

Et elle ajouta encore un instant après, avec colère:

— Non, mais voyez-vous, cette sale bête! employant sans s’en rendre compte, et
peut-être en obéissant au même besoin obscur de se justifier — comme Françoise à
Combray quand le poulet ne voulait pas mourir — les mots qu’arrachent les
derniers sursauts d’un animal inoffensif qui agonise, au paysan qui est en train
de l’écraser.

Et quand la voiture de Mme Verdurin fut partie et que celle de Swann s’avança,
son cocher le regardant lui demanda s’il n’était pas malade ou s’il n’était pas
arrivé de malheur.

Swann le renvoya, il voulait marcher et ce fut à pied, par le Bois, qu’il
rentra. Il parlait seul, à haute voix, et sur le même ton un peu factice qu’il
avait pris jusqu’ici quand il détaillait les charmes du petit noyau et exaltait
la magnanimité des Verdurin. Mais de même que les propos, les sourires, les
baisers d’Odette lui devenaient aussi odieux qu’il les avait trouvés doux, s’ils
étaient adressés à d’autres que lui, de même, le salon des Verdurin, qui tout à
l’heure encore lui semblait amusant, respirant un goût vrai pour l’art et même
une sorte de noblesse morale, maintenant que c’était un autre que lui qu’Odette
allait y rencontrer, y aimer librement, lui exhibait ses ridicules, sa sottise,
son ignominie.

Il se représentait avec dégoût la soirée du lendemain à Chatou. «D’abord cette
idée d’aller à Chatou! Comme des merciers qui viennent de fermer leur boutique!
vraiment ces gens sont sublimes de bourgeoisisme, ils ne doivent pas exister
réellement, ils doivent sortir du théâtre de Labiche!»

Il y aurait là les Cottard, peut-être Brichot. «Est-ce assez grotesque cette vie
de petites gens qui vivent les uns sur les autres, qui se croiraient perdus, ma
parole, s’ils ne se retrouvaient pas tous demain à Chatou!» Hélas! il y aurait
aussi le peintre, le peintre qui aimait à «faire des mariages», qui inviterait
Forcheville à venir avec Odette à son atelier. Il voyait Odette avec une
toilette trop habillée pour cette partie de campagne, «car elle est si vulgaire
et surtout, la pauvre petite, elle est tellement bête!!!»

Il entendit les plaisanteries que ferait Mme Verdurin après dîner, les
plaisanteries qui, quel que fût l’ennuyeux qu’elles eussent pour cible,
l’avaient toujours amusé parce qu’il voyait Odette en rire, en rire avec lui,
presque en lui. Maintenant il sentait que c’était peut-être de lui qu’on allait
faire rire Odette. «Quelle gaieté fétide! disait-il en donnant à sa bouche une
expression de dégoût si forte qu’il avait lui-même la sensation musculaire de sa
grimace jusque dans son cou révulsé contre le col de sa chemise. Et comment une
créature dont le visage est fait à l’image de Dieu peut-elle trouver matière à
rire dans ces plaisanteries nauséabondes? Toute narine un peu délicate se
détournerait avec horreur pour ne pas se laisser offusquer par de tels relents.
C’est vraiment incroyable de penser qu’un être humain peut ne pas comprendre
qu’en se permettant un sourire à l’égard d’un semblable qui lui a tendu
loyalement la main, il se dégrade jusqu’à une fange d’où il ne sera plus
possible à la meilleure volonté du monde de jamais le relever. J’habite à trop
de milliers de mètres d’altitude au-dessus des bas-fonds où clapotent et
clabaudent de tels sales papotages, pour que je puisse être éclaboussé par les
plaisanteries d’une Verdurin, s’écria-t-il, en relevant la tête, en redressant
fièrement son corps en arrière. Dieu m’est témoin que j’ai sincèrement voulu
tirer Odette de là, et l’élever dans une atmosphère plus noble et plus pure.
Mais la patience humaine a des bornes, et la mienne est à bout, se dit-il, comme
si cette mission d’arracher Odette à une atmosphère de sarcasmes datait de plus
longtemps que de quelques minutes, et comme s’il ne se l’était pas donnée
seulement depuis qu’il pensait que ces sarcasmes l’avaient peut-être lui-même
pour objet et tentaient de détacher Odette de lui.

Il voyait le pianiste prêt à jouer la sonate Clair de lune et les mines de Mme
Verdurin s’effrayant du mal que la musique de Beethoven allait faire à ses
nerfs: «Idiote, menteuse! s’écria-t-il, et ça croit aimer l’Art!». Elle dirait à
Odette, après lui avoir insinué adroitement quelques mots louangeurs pour
Forcheville, comme elle avait fait si souvent pour lui: «Vous allez faire une
petite place à côté de vous à M. de Forcheville.» «Dans l’obscurité! maquerelle,
entremetteuse!». «Entremetteuse», c’était le nom qu’il donnait aussi à la
musique qui les convierait à se taire, à rêver ensemble, à se regarder, à se
prendre la main. Il trouvait du bon à la sévérité contre les arts, de Platon, de
Bossuet, et de la vieille éducation française.

En somme la vie qu’on menait chez les Verdurin et qu’il avait appelée si souvent
«la vraie vie», lui semblait la pire de toutes, et leur petit noyau le dernier
des milieux. «C’est vraiment, disait-il, ce qu’il y a de plus bas dans l’échelle
sociale, le dernier cercle de Dante. Nul doute que le texte auguste ne se réfère
aux Verdurin! Au fond, comme les gens du monde dont on peut médire, mais qui
tout de même sont autre chose que ces bandes de voyous, montrent leur profonde
sagesse en refusant de les connaître, d’y salir même le bout de leurs doigts.
Quelle divination dans ce «Noli me tangere» du faubourg Saint-Germain.» Il avait
quitté depuis bien longtemps les allées du Bois, il était presque arrivé chez
lui, que, pas encore dégrisé de sa douleur et de la verve d’insincérité dont les
intonations menteuses, la sonorité artificielle de sa propre voix lui versaient
d’instant en instant plus abondamment l’ivresse, il continuait encore à pérorer
tout haut dans le silence de la nuit: «Les gens du monde ont leurs défauts que
personne ne reconnaît mieux que moi, mais enfin ce sont tout de même des gens
avec qui certaines choses sont impossibles. Telle femme élégante que j’ai connue
était loin d’être parfaite, mais enfin il y avait tout de même chez elle un fond
de délicatesse, une loyauté dans les procédés qui l’auraient rendue, quoi qu’il
arrivât, incapable d’une félonie et qui suffisent à mettre des abîmes entre elle
et une mégère comme la Verdurin. Verdurin! quel nom! Ah! on peut dire qu’ils
sont complets, qu’ils sont beaux dans leur genre! Dieu merci, il n’était que
temps de ne plus condescendre à la promiscuité avec cette infamie, avec ces
ordures.»

Mais, comme les vertus qu’il attribuait tantôt encore aux Verdurin, n’auraient
pas suffi, même s’ils les avaient vraiment possédées, mais s’ils n’avaient pas
favorisé et protégé son amour, à provoquer chez Swann cette ivresse où il
s’attendrissait sur leur magnanimité et qui, même propagée à travers d’autres
personnes, ne pouvait lui venir que d’Odette — de même, l’immoralité, eût-elle
été réelle, qu’il trouvait aujourd’hui aux Verdurin aurait été impuissante,
s’ils n’avaient pas invité Odette avec Forcheville et sans lui, à déchaîner son
indignation et à lui faire flétrir «leur infamie». Et sans doute la voix de
Swann était plus clairvoyante que lui-même, quand elle se refusait à prononcer
ces mots pleins de dégoût pour le milieu Verdurin et de la joie d’en avoir fini
avec lui, autrement que sur un ton factice et comme s’ils étaient choisis plutôt
pour assouvir sa colère que pour exprimer sa pensée. Celle-ci, en effet, pendant
qu’il se livrait à ces invectives, était probablement, sans qu’il s’en aperçût,
occupée d’un objet tout à fait différent, car une fois arrivé chez lui, à peine
eut-il refermé la porte cochère, que brusquement il se frappa le front, et, la
faisant rouvrir, ressortit en s’écriant d’une voix naturelle cette fois: «Je
crois que j’ai trouvé le moyen de me faire inviter demain au dîner de Chatou!»
Mais le moyen devait être mauvais, car Swann ne fut pas invité: le docteur
Cottard qui, appelé en province pour un cas grave, n’avait pas vu les Verdurin
depuis plusieurs jours et n’avait pu aller à Chatou, dit, le lendemain de ce
dîner, en se mettant à table chez eux:

—«Mais, est-ce que nous ne venons pas M. Swann, ce soir? Il est bien ce qu’on
appelle un ami personnel du . . . »

—«Mais j’espère bien que non! s’écria Mme Verdurin, Dieu nous en préserve, il
est assommant, bête et mal élevé.»

Cottard à ces mots manifesta en même temps son étonnement et sa soumission,
comme devant une vérité contraire à tout ce qu’il avait cru jusque-là, mais
d’une évidence irrésistible; et, baissant d’un air ému et peureux son nez dans
son assiette, il se contenta de répondre: «Ah!-ah!-ah!-ah!-ah!» en traversant à
reculons, dans sa retraite repliée en bon ordre jusqu’au fond de lui-même, le
long d’une gamme descendante, tout le registre de sa voix. Et il ne fut plus
question de Swann chez les Verdurin.

Alors ce salon qui avait réuni Swann et Odette devint un obstacle à leurs
rendez-vous. Elle ne lui disait plus comme au premier temps de leur amour: «Nous
nous venons en tous cas demain soir, il y a un souper chez les Verdurin.» Mais:
«Nous ne pourrons pas nous voir demain soir, il y a un souper chez les
Verdurin.» Ou bien les Verdurin devaient l’emmener à l’Opéra-Comique voir «Une
nuit de Cléopâtre» et Swann lisait dans les yeux d’Odette cet effroi qu’il lui
demandât de n’y pas aller, que naguère il n’aurait pu se retenir de baiser au
passage sur le visage de sa maîtresse, et qui maintenant l’exaspérait. «Ce n’est
pas de la colère, pourtant, se disait-il à lui-même, que j’éprouve en voyant
l’envie qu’elle a d’aller picorer dans cette musique stercoraire. C’est du
chagrin, non pas certes pour moi, mais pour elle; du chagrin de voir qu’après
avoir vécu plus de six mois en contact quotidien avec moi, elle n’a pas su
devenir assez une autre pour éliminer spontanément Victor Massé! Surtout pour ne
pas être arrivée à comprendre qu’il y a des soirs où un être d’une essence un
peu délicate doit savoir renoncer à un plaisir, quand on le lui demande. Elle
devrait savoir dire «je n’irai pas», ne fût-ce que par intelligence, puisque
c’est sur sa réponse qu’on classera une fois pour toutes sa qualité d’âme. «Et
s’étant persuadé à lui-même que c’était seulement en effet pour pouvoir porter
un jugement plus favorable sur la valeur spirituelle d’Odette qu’il désirait que
ce soir-là elle restât avec lui au lieu d’aller à l’Opéra-Comique, il lui tenait
le même raisonnement, au même degré d’insincérité qu’à soi-même, et même, à un
degré de plus, car alors il obéissait aussi au désir de la prendre par
l’amour-propre.

— Je te jure, lui disait-il, quelques instants avant qu’elle partît pour le
théâtre, qu’en te demandant de ne pas sortir, tous mes souhaits, si j’étais
égoïste, seraient pour que tu me refuses, car j’ai mille choses à faire ce soir
et je me trouverai moi-même pris au piège et bien ennuyé si contre toute attente
tu me réponds que tu n’iras pas. Mais mes occupations, mes plaisirs, ne sont pas
tout, je dois penser à toi. Il peut venir un jour où me voyant à jamais détaché
de toi tu auras le droit de me reprocher de ne pas t’avoir avertie dans les
minutes décisives où je sentais que j’allais porter sur toi un de ces jugements
sévères auxquels l’amour ne résiste pas longtemps. Vois-tu, «Une nuit de
Cléopâtre» (quel titre!) n’est rien dans la circonstance. Ce qu’il faut savoir
c’est si vraiment tu es cet être qui est au dernier rang de l’esprit, et même du
charme, l’être méprisable qui n’est pas capable de renoncer à un plaisir. Alors,
si tu es cela, comment pourrait-on t’aimer, car tu n’es même pas une personne,
une créature définie, imparfaite, mais du moins perfectible? Tu es une eau
informe qui coule selon la pente qu’on lui offre, un poisson sans mémoire et
sans réflexion qui tant qu’il vivra dans son aquarium se heurtera cent fois par
jour contre le vitrage qu’il continuera à prendre pour de l’eau. Comprends-tu
que ta réponse, je ne dis pas aura pour effet que je cesserai de t’aimer
immédiatement, bien entendu, mais te rendra moins séduisante à mes yeux quand je
comprendrai que tu n’es pas une personne, que tu es au-dessous de toutes les
choses et ne sais te placer au-dessus d’aucune? Évidemment j’aurais mieux aimé
te demander comme une chose sans importance, de renoncer à «Une nuit de
Cléopâtre» (puisque tu m’obliges à me souiller les lèvres de ce nom abject) dans
l’espoir que tu irais cependant. Mais, décidé à tenir un tel compte, à tirer de
telles conséquences de ta réponse, j’ai trouvé plus loyal de t’en prévenir.»

Odette depuis un moment donnait des signes d’émotion et d’incertitude. A défaut
du sens de ce discours, elle comprenait qu’il pouvait rentrer dans le genre
commun des «laïus», et scènes de reproches ou de supplications dont l’habitude
qu’elle avait des hommes lui permettait sans s’attacher aux détails des mots, de
conclure qu’ils ne les prononceraient pas s’ils n’étaient pas amoureux, que du
moment qu’ils étaient amoureux, il était inutile de leur obéir, qu’ils ne le
seraient que plus après. Aussi aurait-elle écouté Swann avec le plus grand calme
si elle n’avait vu que l’heure passait et que pour peu qu’il parlât encore
quelque temps, elle allait, comme elle le lui dit avec un sourire tendre,
obstiné et confus, «finir par manquer l’Ouverture!»

D’autres fois il lui disait que ce qui plus que tout ferait qu’il cesserait de
l’aimer, c’est qu’elle ne voulût pas renoncer à mentir. «Même au simple point de
vue de la coquetterie, lui disait-il, ne comprends-tu donc pas combien tu perds
de ta séduction en t’abaissant à mentir? Par un aveu! combien de fautes tu
pourrais racheter! Vraiment tu es bien moins intelligente que je ne croyais!»
Mais c’est en vain que Swann lui exposait ainsi toutes les raisons qu’elle avait
de ne pas mentir; elles auraient pu ruiner chez Odette un système général du
mensonge; mais Odette n’en possédait pas; elle se contentait seulement, dans
chaque cas où elle voulait que Swann ignorât quelque chose qu’elle avait fait,
de ne pas le lui dire. Ainsi le mensonge était pour elle un expédient d’ordre
particulier; et ce qui seul pouvait décider si elle devait s’en servir ou avouer
la vérité, c’était une raison d’ordre particulier aussi, la chance plus ou moins
grande qu’il y avait pour que Swann pût découvrir qu’elle n’avait pas dit la
vérité.

Physiquement, elle traversait une mauvaise phase: elle épaississait; et le
charme expressif et dolent, les regards étonnés et rêveurs qu’elle avait
autrefois semblaient avoir disparu avec sa première jeunesse. De sorte qu’elle
était devenue si chère à Swann au moment pour ainsi dire où il la trouvait
précisément bien moins jolie. Il la regardait longuement pour tâcher de
ressaisir le charme qu’il lui avait connu, et ne le retrouvait pas. Mais savoir
que sous cette chrysalide nouvelle, c’était toujours Odette qui vivait, toujours
la même volonté fugace, insaisissable et sournoise, suffisait à Swann pour qu’il
continuât de mettre la même passion à chercher à la capter. Puis il regardait
des photographies d’il y avait deux ans, il se rappelait comme elle avait été
délicieuse. Et cela le consolait un peu de se donner tant de mal pour elle.

Quand les Verdurin l’emmenaient à Saint-Germain, à Chatou, à Meulan, souvent, si
c’était dans la belle saison, ils proposaient, sur place, de rester à coucher et
de ne revenir que le lendemain. Mme Verdurin cherchait à apaiser les scrupules
du pianiste dont la tante était restée à Paris.

— Elle sera enchantée d’être débarrassée de vous pour un jour. Et comment
s’inquiéterait-elle, elle vous sait avec nous? d’ailleurs je prends tout sous
mon bonnet.

Mais si elle n’y réussissait pas, M. Verdurin partait en campagne, trouvait un
bureau de télégraphe ou un messager et s’informait de ceux des fidèles qui
avaient quelqu’un à faire prévenir. Mais Odette le remerciait et disait qu’elle
n’avait de dépêche à faire pour personne, car elle avait dit à Swann une fois
pour toutes qu’en lui en envoyant une aux yeux de tous, elle se compromettrait.
Parfois c’était pour plusieurs jours qu’elle s’absentait, les Verdurin
l’emmenaient voir les tombeaux de Dreux, ou à Compiègne admirer, sur le conseil
du peintre, des couchers de soleil en forêt et on poussait jusqu’au château de
Pierrefonds.

—«Penser qu’elle pourrait visiter de vrais monuments avec moi qui ai étudié
l’architecture pendant dix ans et qui suis tout le temps supplié de mener à
Beauvais ou à Saint-Loup-de-Naud des gens de la plus haute valeur et ne le
ferais que pour elle, et qu’à la place elle va avec les dernières des brutes
s’extasier successivement devant les déjections de Louis-Philippe et devant
celles de Viollet-le-Duc! Il me semble qu’il n’y a pas besoin d’être artiste
pour cela et que, même sans flair particulièrement fin, on ne choisit pas
d’aller villégiaturer dans des latrines pour être plus à portée de respirer des
excréments.»

Mais quand elle était partie pour Dreux ou pour Pierrefonds — hélas, sans lui
permettre d’y aller, comme par hasard, de son côté, car «cela ferait un effet
déplorable», disait-elle — il se plongeait dans le plus enivrant des romans
d’amour, l’indicateur des chemins de fer, qui lui apprenait les moyens de la
rejoindre, l’après-midi, le soir, ce matin même! Le moyen? presque davantage:
l’autorisation. Car enfin l’indicateur et les trains eux-mêmes n’étaient pas
faits pour des chiens. Si on faisait savoir au public, par voie d’imprimés, qu’à
huit heures du matin partait un train qui arrivait à Pierrefonds à dix heures,
c’est donc qu’aller à Pierrefonds était un acte licite, pour lequel la
permission d’Odette était superflue; et c’était aussi un acte qui pouvait avoir
un tout autre motif que le désir de rencontrer Odette, puisque des gens qui ne
la connaissaient pas l’accomplissaient chaque jour, en assez grand nombre pour
que cela valût la peine de faire chauffer des locomotives.

En somme elle ne pouvait tout de même pas l’empêcher d’aller à Pierrefonds s’il
en avait envie! Or, justement, il sentait qu’il en avait envie, et que s’il
n’avait pas connu Odette, certainement il y serait allé. Il y avait longtemps
qu’il voulait se faire une idée plus précise des travaux de restauration de
Viollet-le-Duc. Et par le temps qu’il faisait, il éprouvait l’impérieux désir
d’une promenade dans la forêt de Compiègne.

Ce n’était vraiment pas de chance qu’elle lui défendît le seul endroit qui le
tentait aujourd’hui. Aujourd’hui! S’il y allait, malgré son interdiction, il
pourrait la voir aujourd’hui même! Mais, alors que, si elle eût retrouvé à
Pierrefonds quelque indifférent, elle lui eût dit joyeusement: «Tiens, vous
ici!», et lui aurait demandé d’aller la voir à l’hôtel où elle était descendue
avec les Verdurin, au contraire si elle l’y rencontrait, lui, Swann, elle serait
froissée, elle se dirait qu’elle était suivie, elle l’aimerait moins, peut-être
se détournerait-elle avec colère en l’apercevant. «Alors, je n’ai plus le droit
de voyager!», lui dirait-elle au retour, tandis qu’en somme c’était lui qui
n’avait plus le droit de voyager!

Il avait eu un moment l’idée, pour pouvoir aller à Compiègne et à Pierrefonds
sans avoir l’air que ce fût pour rencontrer Odette, de s’y faire emmener par un
de ses amis, le marquis de Forestelle, qui avait un château dans le voisinage.
Celui-ci, à qui il avait fait part de son projet sans lui en dire le motif, ne
se sentait pas de joie et s’émerveillait que Swann, pour la première fois depuis
quinze ans, consentît enfin à venir voir sa propriété et, quoiqu’il ne voulait
pas s’y arrêter, lui avait-il dit, lui promît du moins de faire ensemble des
promenades et des excursions pendant plusieurs jours. Swann s’imaginait déjà
là-bas avec M. de Forestelle. Même avant d’y voir Odette, même s’il ne
réussissait pas à l’y voir, quel bonheur il aurait à mettre le pied sur cette
terre où ne sachant pas l’endroit exact, à tel moment, de sa présence, il
sentirait palpiter partout la possibilité de sa brusque apparition: dans la cour
du château, devenu beau pour lui parce que c’était à cause d’elle qu’il était
allé le voir; dans toutes les rues de la ville, qui lui semblait romanesque; sur
chaque route de la forêt, rosée par un couchant profond et tendre; — asiles
innombrables et alternatifs, où venait simultanément se réfugier, dans
l’incertaine ubiquité de ses espérances, son cœur heureux, vagabond et
multiplié. «Surtout, dirait-il à M. de Forestelle, prenons garde de ne pas
tomber sur Odette et les Verdurin; je viens d’apprendre qu’ils sont justement
aujourd’hui à Pierrefonds. On a assez le temps de se voir à Paris, ce ne serait
pas la peine de le quitter pour ne pas pouvoir faire un pas les uns sans les
autres.» Et son ami ne comprendrait pas pourquoi une fois là-bas il changerait
vingt fois de projets, inspecterait les salles à manger de tous les hôtels de
Compiègne sans se décider à s’asseoir dans aucune de celles où pourtant on
n’avait pas vu trace de Verdurin, ayant l’air de rechercher ce qu’il disait
vouloir fuir et du reste le fuyant dès qu’il l’aurait trouvé, car s’il avait
rencontré le petit groupe, il s’en serait écarté avec affectation, content
d’avoir vu Odette et qu’elle l’eût vu, surtout qu’elle l’eût vu ne se souciant
pas d’elle. Mais non, elle devinerait bien que c’était pour elle qu’il était là.
Et quand M. de Forestelle venait le chercher pour partir, il lui disait: «Hélas!
non, je ne peux pas aller aujourd’hui à Pierrefonds, Odette y est justement.» Et
Swann était heureux malgré tout de sentir que, si seul de tous les mortels il
n’avait pas le droit en ce jour d’aller à Pierrefonds, c’était parce qu’il était
en effet pour Odette quelqu’un de différent des autres, son amant, et que cette
restriction apportée pour lui au droit universel de libre circulation, n’était
qu’une des formes de cet esclavage, de cet amour qui lui était si cher.
Décidément il valait mieux ne pas risquer de se brouiller avec elle, patienter,
attendre son retour. Il passait ses journées penché sur une carte de la forêt de
Compiègne comme si ç’avait été la carte du Tendre, s’entourait de photographies
du château de Pierrefonds. Dés que venait le jour où il était possible qu’elle
revînt, il rouvrait l’indicateur, calculait quel train elle avait dû prendre, et
si elle s’était attardée, ceux qui lui restaient encore. Il ne sortait pas de
peur de manquer une dépêche, ne se couchait pas, pour le cas où, revenue par le
dernier train, elle aurait voulu lui faire la surprise de venir le voir au
milieu de la nuit. Justement il entendait sonner à la porte cochère, il lui
semblait qu’on tardait à ouvrir, il voulait éveiller le concierge, se mettait à
la fenêtre pour appeler Odette si c’était elle, car malgré les recommandations
qu’il était descendu faire plus de dix fois lui-même, on était capable de lui
dire qu’il n’était pas là. C’était un domestique qui rentrait. Il remarquait le
vol incessant des voitures qui passaient, auquel il n’avait jamais fait
attention autrefois. Il écoutait chacune venir au loin, s’approcher, dépasser sa
porte sans s’être arrêtée et porter plus loin un message qui n’était pas pour
lui. Il attendait toute la nuit, bien inutilement, car les Verdurin ayant avancé
leur retour, Odette était à Paris depuis midi; elle n’avait pas eu l’idée de
l’en prévenir; ne sachant que faire elle avait été passer sa soirée seule au
théâtre et il y avait longtemps qu’elle était rentrée se coucher et dormait.

C’est qu’elle n’avait même pas pensé à lui. Et de tels moments où elle oubliait
jusqu’à l’existence de Swann étaient plus utiles à Odette, servaient mieux à lui
attacher Swann, que toute sa coquetterie. Car ainsi Swann vivait dans cette
agitation douloureuse qui avait déjà été assez puissante pour faire éclore son
amour le soir où il n’avait pas trouvé Odette chez les Verdurin et l’avait
cherchée toute la soirée. Et il n’avait pas, comme j’eus à Combray dans mon
enfance, des journées heureuses pendant lesquelles s’oublient les souffrances
qui renaîtront le soir. Les journées, Swann les passait sans Odette; et par
moments il se disait que laisser une aussi jolie femme sortir ainsi seule dans
Paris était aussi imprudent que de poser un écrin plein de bijoux au milieu de
la rue. Alors il s’indignait contre tous les passants comme contre autant de
voleurs. Mais leur visage collectif et informe échappant à son imagination ne
nourrissait pas sa jalousie. Il fatiguait la pensée de Swann, lequel, se passant
la main sur les yeux, s’écriait: «A la grâce de Dieu», comme ceux qui après
s’être acharnés à étreindre le problème de la réalité du monde extérieur ou de
l’immortalité de l’âme accordent la détente d’un acte de foi à leur cerveau
lassé. Mais toujours la pensée de l’absente était indissolublement mêlée aux
actes les plus simples de la vie de Swann — déjeuner, recevoir son courrier,
sortir, se coucher — par la tristesse même qu’il avait à les accomplir sans
elle, comme ces initiales de Philibert le Beau que dans l’église de Brou, à
cause du regret qu’elle avait de lui, Marguerite d’Autriche entrelaça partout
aux siennes. Certains jours, au lieu de rester chez lui, il allait prendre son
déjeuner dans un restaurant assez voisin dont il avait apprécié autrefois la
bonne cuisine et où maintenant il n’allait plus que pour une de ces raisons, à
la fois mystiques et saugrenues, qu’on appelle romanesques; c’est que ce
restaurant (lequel existe encore) portait le même nom que la rue habitée par
Odette: Lapérouse. Quelquefois, quand elle avait fait un court déplacement ce
n’est qu’après plusieurs jours qu’elle songeait à lui faire savoir qu’elle était
revenue à Paris. Et elle lui disait tout simplement, sans plus prendre comme
autrefois la précaution de se couvrir à tout hasard d’un petit morceau emprunté
à la vérité, qu’elle venait d’y rentrer à l’instant même par le train du matin.
Ces paroles étaient mensongères; du moins pour Odette elles étaient mensongères,
inconsistantes, n’ayant pas, comme si elles avaient été vraies, un point d’appui
dans le souvenir de son arrivée à la gare; même elle était empêchée de se les
représenter au moment où elle les prononçait, par l’image contradictoire de ce
qu’elle avait fait de tout différent au moment où elle prétendait être descendue
du train. Mais dans l’esprit de Swann au contraire ces paroles qui ne
rencontraient aucun obstacle venaient s’incruster et prendre l’inamovibilité
d’une vérité si indubitable que si un ami lui disait être venu par ce train et
ne pas avoir vu Odette il était persuadé que c’était l’ami qui se trompait de
jour ou d’heure puisque son dire ne se conciliait pas avec les paroles d’Odette.
Celles-ci ne lui eussent paru mensongères que s’il s’était d’abord défié
qu’elles le fussent. Pour qu’il crût qu’elle mentait, un soupçon préalable était
une condition nécessaire. C’était d’ailleurs aussi une condition suffisante.
Alors tout ce que disait Odette lui paraissait suspect. L’entendait-il citer un
nom, c’était certainement celui d’un de ses amants; une fois cette supposition
forgée, il passait des semaines à se désoler; il s’aboucha même une fois avec
une agence de renseignements pour savoir l’adresse, l’emploi du temps de
l’inconnu qui ne le laisserait respirer que quand il serait parti en voyage, et
dont il finit par apprendre que c’était un oncle d’Odette mort depuis vingt ans.

Bien qu’elle ne lui permît pas en général de la rejoindre dans des lieux publics
disant que cela ferait jaser, il arrivait que dans une soirée où il était invité
comme elle — chez Forcheville, chez le peintre, ou à un bal de charité dans un
ministère — il se trouvât en même temps qu’elle. Il la voyait mais n’osait pas
rester de peur de l’irriter en ayant l’air d’épier les plaisirs qu’elle prenait
avec d’autres et qui — tandis qu’il rentrait solitaire, qu’il allait se coucher
anxieux comme je devais l’être moi-même quelques années plus tard les soirs où
il viendrait dîner à la maison, à Combray — lui semblaient illimités parce qu’il
n’en avait pas vu la fin. Et une fois ou deux il connut par de tels soirs de ces
joies qu’on serait tenté, si elles ne subissaient avec tant de violence le choc
en retour de l’inquiétude brusquement arrêtée, d’appeler des joies calmes, parce
qu’elles consistent en un apaisement: il était allé passer un instant à un raout
chez le peintre et s’apprêtait à le quitter; il y laissait Odette muée en une
brillante étrangère, au milieu d’hommes à qui ses regards et sa gaieté qui
n’étaient pas pour lui, semblaient parler de quelque volupté, qui serait goûtée
là ou ailleurs (peut-être au «Bal des Incohérents» où il tremblait qu’elle
n’allât ensuite) et qui causait à Swann plus de jalousie que l’union charnelle
même parce qu’il l’imaginait plus difficilement; il était déjà prêt à passer la
porte de l’atelier quand il s’entendait rappeler par ces mots (qui en
retranchant de la fête cette fin qui l’épouvantait, la lui rendaient
rétrospectivement innocente, faisaient du retour d’Odette une chose non plus
inconcevable et terrible, mais douce et connue et qui tiendrait à côté de lui,
pareille à un peu de sa vie de tous les jours, dans sa voiture, et dépouillait
Odette elle-même de son apparence trop brillante et gaie, montraient que ce
n’était qu’un déguisement qu’elle avait revêtu un moment, pour lui-même, non en
vue de mystérieux plaisirs, et duquel elle était déjà lasse), par ces mots
qu’Odette lui jetait, comme il était déjà sur le seuil: «Vous ne voudriez pas
m’attendre cinq minutes, je vais partir, nous reviendrions ensemble, vous me
ramèneriez chez moi.»

Il est vrai qu’un jour Forcheville avait demandé à être ramené en même temps,
mais comme, arrivé devant la porte d’Odette il avait sollicité la permission
d’entrer aussi, Odette lui avait répondu en montrant Swann: «Ah! cela dépend de
ce monsieur-là, demandez-lui. Enfin, entrez un moment si vous voulez, mais pas
longtemps parce que je vous préviens qu’il aime causer tranquillement avec moi,
et qu’il n’aime pas beaucoup qu’il y ait des visites quand il vient. Ah! si vous
connaissiez cet être-là autant que je le connais; n’est-ce pas, my love, il n’y
a que moi qui vous connaisse bien?»

Et Swann était peut-être encore plus touché de la voir ainsi lui adresser en
présence de Forcheville, non seulement ces paroles de tendresse, de
prédilection, mais encore certaines critiques comme: «Je suis sûre que vous
n’avez pas encore répondu à vos amis pour votre dîner de dimanche. N’y allez pas
si vous ne voulez pas, mais soyez au moins poli», ou: «Avez-vous laissé
seulement ici votre essai sur Ver Meer pour pouvoir l’avancer un peu demain?
Quel paresseux! Je vous ferai travailler, moi!» qui prouvaient qu’Odette se
tenait au courant de ses invitations dans le monde et de ses études d’art,
qu’ils avaient bien une vie à eux deux. Et en disant cela elle lui adressait un
sourire au fond duquel il la sentait toute à lui.

Alors à ces moments-là, pendant qu’elle leur faisait de l’orangeade, tout d’un
coup, comme quand un réflecteur mal réglé d’abord promène autour d’un objet, sur
la muraille, de grandes ombres fantastiques qui viennent ensuite se replier et
s’anéantir en lui, toutes les idées terribles et mouvantes qu’il se faisait
d’Odette s’évanouissaient, rejoignaient le corps charmant que Swann avait devant
lui. Il avait le brusque soupçon que cette heure passée chez Odette, sous la
lampe, n’était peut-être pas une heure factice, à son usage à lui (destinée à
masquer cette chose effrayante et délicieuse à laquelle il pensait sans cesse
sans pouvoir bien se la représenter, une heure de la vraie vie d’Odette, de la
vie d’Odette quand lui n’était pas là), avec des accessoires de théâtre et des
fruits de carton, mais était peut-être une heure pour de bon de la vie d’Odette,
que s’il n’avait pas été là elle eût avancé à Forcheville le même fauteuil et
lui eût versé non un breuvage inconnu, mais précisément cette orangeade; que le
monde habité par Odette n’était pas cet autre monde effroyable et surnaturel où
il passait son temps à la situer et qui n’existait peut-être que dans son
imagination, mais l’univers réel, ne dégageant aucune tristesse spéciale,
comprenant cette table où il allait pouvoir écrire et cette boisson à laquelle
il lui serait permis de goûter, tous ces objets qu’il contemplait avec autant de
curiosité et d’admiration que de gratitude, car si en absorbant ses rêves ils
l’en avaient délivré, eux en revanche, s’en étaient enrichis, ils lui en
montraient la réalisation palpable, et ils intéressaient son esprit, ils
prenaient du relief devant ses regards, en même temps qu’ils tranquillisaient
son cœur. Ah! si le destin avait permis qu’il pût n’avoir qu’une seule demeure
avec Odette et que chez elle il fût chez lui, si en demandant au domestique ce
qu’il y avait à déjeuner c’eût été le menu d’Odette qu’il avait appris en
réponse, si quand Odette voulait aller le matin se promener avenue du
Bois-de-Boulogne, son devoir de bon mari l’avait obligé, n’eût-il pas envie de
sortir, à l’accompagner, portant son manteau quand elle avait trop chaud, et le
soir après le dîner si elle avait envie de rester chez elle en déshabillé, s’il
avait été forcé de rester là près d’elle, à faire ce qu’elle voudrait; alors
combien tous les riens de la vie de Swann qui lui semblaient si tristes, au
contraire parce qu’ils auraient en même temps fait partie de la vie d’Odette
auraient pris, même les plus familiers — et comme cette lampe, cette orangeade,
ce fauteuil qui contenaient tant de rêve, qui matérialisaient tant de désir —
une sorte de douceur surabondante et de densité mystérieuse.

Pourtant il se doutait bien que ce qu’il regrettait ainsi c’était un calme, une
paix qui n’auraient pas été pour son amour une atmosphère favorable. Quand
Odette cesserait d’être pour lui une créature toujours absente, regrettée,
imaginaire, quand le sentiment qu’il aurait pour elle ne serait plus ce même
trouble mystérieux que lui causait la phrase de la sonate, mais de l’affection,
de la reconnaissance quand s’établiraient entre eux des rapports normaux qui
mettraient fin à sa folie et à sa tristesse, alors sans doute les actes de la
vie d’Odette lui paraîtraient peu intéressants en eux-mêmes — comme il avait
déjà eu plusieurs fois le soupçon qu’ils étaient, par exemple le jour où il
avait lu à travers l’enveloppe la lettre adressée à Forcheville. Considérant son
mal avec autant de sagacité que s’il se l’était inoculé pour en faire l’étude,
il se disait que, quand il serait guéri, ce que pourrait faire Odette lui serait
indifférent. Mais du sein de son état morbide, à vrai dire, il redoutait à
l’égal de la mort une telle guérison, qui eût été en effet la mort de tout ce
qu’il était actuellement.

Après ces tranquilles soirées, les soupçons de Swann étaient calmés; il
bénissait Odette et le lendemain, dès le matin, il faisait envoyer chez elle les
plus beaux bijoux, parce que ces bontés de la veille avaient excité ou sa
gratitude, ou le désir de les voir se renouveler, ou un paroxysme d’amour qui
avait besoin de se dépenser.

Mais, à d’autres moments, sa douleur le reprenait, il s’imaginait qu’Odette
était la maîtresse de Forcheville et que quand tous deux l’avaient vu, du fond
du landau des Verdurin, au Bois, la veille de la fête de Chatou où il n’avait
pas été invité, la prier vainement, avec cet air de désespoir qu’avait remarqué
jusqu’à son cocher, de revenir avec lui, puis s’en retourner de son côté, seul
et vaincu, elle avait dû avoir pour le désigner à Forcheville et lui dire:
«Hein! ce qu’il rage!» les mêmes regards, brillants, malicieux, abaissés et
sournois, que le jour où celui-ci avait chassé Saniette de chez les Verdurin.

Alors Swann la détestait. «Mais aussi, je suis trop bête, se disait-il, je paie
avec mon argent le plaisir des autres. Elle fera tout de même bien de faire
attention et de ne pas trop tirer sur la corde, car je pourrais bien ne plus
rien donner du tout. En tous cas, renonçons provisoirement aux gentillesses
supplémentaires! Penser que pas plus tard qu’hier, comme elle disait avoir envie
d’assister à la saison de Bayreuth, j’ai eu la bêtise de lui proposer de louer
un des jolis châteaux du roi de Bavière pour nous deux dans les environs. Et
d’ailleurs elle n’a pas paru plus ravie que cela, elle n’a encore dit ni oui ni
non; espérons qu’elle refusera, grand Dieu! Entendre du Wagner pendant quinze
jours avec elle qui s’en soucie comme un poisson d’une pomme, ce serait gai!» Et
sa haine, tout comme son amour, ayant besoin de se manifester et d’agir, il se
plaisait à pousser de plus en plus loin ses imaginations mauvaises, parce que,
grâce aux perfidies qu’il prêtait à Odette, il la détestait davantage et
pourrait si — ce qu’il cherchait à se figurer — elles se trouvaient être vraies,
avoir une occasion de la punir et d’assouvir sur elle sa rage grandissante. Il
alla ainsi jusqu’à supposer qu’il allait recevoir une lettre d’elle où elle lui
demanderait de l’argent pour louer ce château près de Bayreuth, mais en le
prévenant qu’il n’y pourrait pas venir, parce qu’elle avait promis à Forcheville
et aux Verdurin de les inviter. Ah! comme il eût aimé qu’elle pût avoir cette
audace. Quelle joie il aurait à refuser, à rédiger la réponse vengeresse dont il
se complaisait à choisir, à énoncer tout haut les termes, comme s’il avait reçu
la lettre en réalité.

Or, c’est ce qui arriva le lendemain même. Elle lui écrivit que les Verdurin et
leurs amis avaient manifesté le désir d’assister à ces représentations de Wagner
et que, s’il voulait bien lui envoyer cet argent, elle aurait enfin, après avoir
été si souvent reçue chez eux, le plaisir de les inviter à son tour. De lui,
elle ne disait pas un mot, il était sous-entendu que leur présence excluait la
sienne.

Alors cette terrible réponse dont il avait arrêté chaque mot la veille sans oser
espérer qu’elle pourrait servir jamais il avait la joie de la lui faire porter.
Hélas! il sentait bien qu’avec l’argent qu’elle avait, ou qu’elle trouverait
facilement, elle pourrait tout de même louer à Bayreuth puisqu’elle en avait
envie, elle qui n’était pas capable de faire de différence entre Bach et
Clapisson. Mais elle y vivrait malgré tout plus chichement. Pas moyen comme s’il
lui eût envoyé cette fois quelques billets de mille francs, d’organiser chaque
soir, dans un château, de ces soupers fins après lesquels elle se serait
peut-être passé la fantaisie — qu’il était possible qu’elle n’eût jamais eue
encore — de tomber dans les bras de Forcheville. Et puis du moins, ce voyage
détesté, ce n’était pas lui, Swann, qui le paierait! — Ah! s’il avait pu
l’empêcher, si elle avait pu se fouler le pied avant de partir, si le cocher de
la voiture qui l’emmènerait à la gare avait consenti, à n’importe quel prix, à
la conduire dans un lieu où elle fût restée quelque temps séquestrée, cette
femme perfide, aux yeux émaillés par un sourire de complicité adressé à
Forcheville, qu’Odette était pour Swann depuis quarante-huit heures.

Mais elle ne l’était jamais pour très longtemps; au bout de quelques jours le
regard luisant et fourbe perdait de son éclat et de sa duplicité, cette image
d’une Odette exécrée disant à Forcheville: «Ce qu’il rage!» commençait à pâlir,
à s’effacer. Alors, progressivement reparaissait et s’élevait en brillant
doucement, le visage de l’autre Odette, de celle qui adressait aussi un sourire
à Forcheville, mais un sourire où il n’y avait pour Swann que de la tendresse,
quand elle disait: «Ne restez pas longtemps, car ce monsieur-là n’aime pas
beaucoup que j’aie des visites quand il a envie d’être auprès de moi. Ah! si
vous connaissiez cet être-là autant que je le connais!», ce même sourire qu’elle
avait pour remercier Swann de quelque trait de sa délicatesse qu’elle prisait si
fort, de quelque conseil qu’elle lui avait demandé dans une de ces circonstances
graves où elle n’avait confiance qu’en lui.

Alors, à cette Odette-là, il se demandait comment il avait pu écrire cette
lettre outrageante dont sans doute jusqu’ici elle ne l’eût pas cru capable, et
qui avait dû le faire descendre du rang élevé, unique, que par sa bonté, sa
loyauté, il avait conquis dans son estime. Il allait lui devenir moins cher, car
c’était pour ces qualités-là, qu’elle ne trouvait ni à Forcheville ni à aucun
autre, qu’elle l’aimait. C’était à cause d’elles qu’Odette lui témoignait si
souvent une gentillesse qu’il comptait pour rien au moment où il était jaloux,
parce qu’elle n’était pas une marque de désir, et prouvait même plutôt de
l’affection que de l’amour, mais dont il recommençait à sentir l’importance au
fur et à mesure que la détente spontanée de ses soupçons, souvent accentuée par
la distraction que lui apportait une lecture d’art ou la conversation d’un ami,
rendait sa passion moins exigeante de réciprocités.

Maintenant qu’après cette oscillation, Odette était naturellement revenue à la
place d’où la jalousie de Swann l’avait un moment écartée, dans l’angle où il la
trouvait charmante, il se la figurait pleine de tendresse, avec un regard de
consentement, si jolie ainsi, qu’il ne pouvait s’empêcher d’avancer les lèvres
vers elle comme si elle avait été là et qu’il eût pu l’embrasser; et il lui
gardait de ce regard enchanteur et bon autant de reconnaissance que si elle
venait de l’avoir réellement et si cela n’eût pas été seulement son imagination
qui venait de le peindre pour donner satisfaction à son désir.

Comme il avait dû lui faire de la peine! Certes il trouvait des raisons valables
à son ressentiment contre elle, mais elles n’auraient pas suffi à le lui faire
éprouver s’il ne l’avait pas autant aimée. N’avait-il pas eu des griefs aussi
graves contre d’autres femmes, auxquelles il eût néanmoins volontiers rendu
service aujourd’hui, étant contre elles sans colère parce qu’il ne les aimait
plus. S’il devait jamais un jour se trouver dans le même état d’indifférence
vis-à-vis d’Odette, il comprendrait que c’était sa jalousie seule qui lui avait
fait trouver quelque chose d’atroce, d’impardonnable, à ce désir, au fond si
naturel, provenant d’un peu d’enfantillage et aussi d’une certaine délicatesse
d’âme, de pouvoir à son tour, puisqu’une occasion s’en présentait, rendre des
politesses aux Verdurin, jouer à la maîtresse de maison.

Il revenait à ce point de vue — opposé à celui de son amour et de sa jalousie et
auquel il se plaçait quelquefois par une sorte d’équité intellectuelle et pour
faire la part des diverses probabilités — d’où il essayait de juger Odette comme
s’il ne l’avait pas aimée, comme si elle était pour lui une femme comme les
autres, comme si la vie d’Odette n’avait pas été, dès qu’il n’était plus là,
différente, tramée en cachette de lui, ourdie contre lui.

Pourquoi croire qu’elle goûterait là-bas avec Forcheville ou avec d’autres des
plaisirs enivrants qu’elle n’avait pas connus auprès de lui et que seule sa
jalousie forgeait de toutes pièces? A Bayreuth comme à Paris, s’il arrivait que
Forcheville pensât à lui ce n’eût pu être que comme à quelqu’un qui comptait
beaucoup dans la vie d’Odette, à qui il était obligé de céder la place, quand
ils se rencontraient chez elle. Si Forcheville et elle triomphaient d’être
là-bas malgré lui, c’est lui qui l’aurait voulu en cherchant inutilement à
l’empêcher d’y aller, tandis que s’il avait approuvé son projet, d’ailleurs
défendable, elle aurait eu l’air d’être là-bas d’après son avis, elle s’y serait
sentie envoyée, logée par lui, et le plaisir qu’elle aurait éprouvé à recevoir
ces gens qui l’avaient tant reçue, c’est à Swann qu’elle en aurait su gré.

Et — au lieu qu’elle allait partir brouillée avec lui, sans l’avoir revu — s’il
lui envoyait cet argent, s’il l’encourageait à ce voyage et s’occupait de le lui
rendre agréable, elle allait accourir, heureuse, reconnaissante, et il aurait
cette joie de la voir qu’il n’avait pas goûtée depuis près d’une semaine et que
rien ne pouvait lui remplacer. Car sitôt que Swann pouvait se la représenter
sans horreur, qu’il revoyait de la bonté dans son sourire, et que le désir de
l’enlever à tout autre, n’était plus ajouté par la jalousie à son amour, cet
amour redevenait surtout un goût pour les sensations que lui donnait la personne
d’Odette, pour le plaisir qu’il avait à admirer comme un spectacle ou à
interroger comme un phénomène, le lever d’un de ses regards, la formation d’un
de ses sourires, l’émission d’une intonation de sa voix. Et ce plaisir différent
de tous les autres, avait fini par créer en lui un besoin d’elle et qu’elle
seule pouvait assouvir par sa présence ou ses lettres, presque aussi
désintéressé, presque aussi artistique, aussi pervers, qu’un autre besoin qui
caractérisait cette période nouvelle de la vie de Swann où à la sécheresse, à la
dépression des années antérieures avait succédé une sorte de trop-plein
spirituel, sans qu’il sût davantage à quoi il devait cet enrichissement inespéré
de sa vie intérieure qu’une personne de santé délicate qui à partir d’un certain
moment se fortifie, engraisse, et semble pendant quelque temps s’acheminer vers
une complète guérison — cet autre besoin qui se développait aussi en dehors du
monde réel, c’était celui d’entendre, de connaître de la musique.

Ainsi, par le chimisme même de son mal, après qu’il avait fait de la jalousie
avec son amour, il recommençait à fabriquer de la tendresse, de la pitié pour
Odette. Elle était redevenue l’Odette charmante et bonne. Il avait des remords
d’avoir été dur pour elle. Il voulait qu’elle vînt près de lui et, auparavant,
il voulait lui avoir procuré quelque plaisir, pour voir la reconnaissance pétrir
son visage et modeler son sourire.

Aussi Odette, sûre de le voir venir après quelques jours, aussi tendre et soumis
qu’avant, lui demander une réconciliation, prenait-elle l’habitude de ne plus
craindre de lui déplaire et même de l’irriter et lui refusait-elle, quand cela
lui était commode, les faveurs auxquelles il tenait le plus.

Peut-être ne savait-elle pas combien il avait été sincère vis-à-vis d’elle
pendant la brouille, quand il lui avait dit qu’il ne lui enverrait pas d’argent
et chercherait à lui faire du mal. Peut-être ne savait-elle pas davantage
combien il l’était, vis-à-vis sinon d’elle, du moins de lui-même, en d’autres
cas où dans l’intérêt de l’avenir de leur liaison, pour montrer à Odette qu’il
était capable de se passer d’elle, qu’une rupture restait toujours possible, il
décidait de rester quelque temps sans aller chez elle.

Parfois c’était après quelques jours où elle ne lui avait pas causé de souci
nouveau; et comme, des visites prochaines qu’il lui ferait, il savait qu’il ne
pouvait tirer nulle bien grande joie mais plus probablement quelque chagrin qui
mettrait fin au calme où il se trouvait, il lui écrivait qu’étant très occupé il
ne pourrait la voir aucun des jours qu’il lui avait dit. Or une lettre d’elle,
se croisant avec la sienne, le priait précisément de déplacer un rendez-vous. Il
se demandait pourquoi; ses soupçons, sa douleur le reprenaient. Il ne pouvait
plus tenir, dans l’état nouveau d’agitation où il se trouvait, l’engagement
qu’il avait pris dans l’état antérieur de calme relatif, il courait chez elle et
exigeait de la voir tous les jours suivants. Et même si elle ne lui avait pas
écrit la première, si elle répondait seulement, cela suffisait pour qu’il ne pût
plus rester sans la voir. Car, contrairement au calcul de Swann, le consentement
d’Odette avait tout changé en lui. Comme tous ceux qui possèdent une chose, pour
savoir ce qui arriverait s’il cessait un moment de la posséder, il avait ôté
cette chose de son esprit, en y laissant tout le reste dans le même état que
quand elle était là. Or l’absence d’une chose, ce n’est pas que cela, ce n’est
pas un simple manque partiel, c’est un bouleversement de tout le reste, c’est un
état nouveau qu’on ne peut prévoir dans l’ancien.

Mais d’autres fois au contraire — Odette était sur le point de partir en voyage
— c’était après quelque petite querelle dont il choisissait le prétexte, qu’il
se résolvait à ne pas lui écrire et à ne pas la revoir avant son retour, donnant
ainsi les apparences, et demandant le bénéfice d’une grande brouille, qu’elle
croirait peut-être définitive, à une séparation dont la plus longue part était
inévitable du fait du voyage et qu’il faisait commencer seulement un peu plus
tôt. Déjà il se figurait Odette inquiète, affligée, de n’avoir reçu ni visite ni
lettre et cette image, en calmant sa jalousie, lui rendait facile de se
déshabituer de la voir. Sans doute, par moments, tout au bout de son esprit où
sa résolution la refoulait grâce à toute la longueur interposée des trois
semaines de séparation acceptée, c’était avec plaisir qu’il considérait l’idée
qu’il reverrait Odette à son retour: mais c’était aussi avec si peu d’impatience
qu’il commençait à se demander s’il ne doublerait pas volontierement la durée
d’une abstinence si facile. Elle ne datait encore que de trois jours, temps
beaucoup moins long que celui qu’il avait souvent passé en ne voyant pas Odette,
et sans l’avoir comme maintenant prémédité. Et pourtant voici qu’une légère
contrariété ou un malaise physique — en l’incitant à considérer le moment
présent comme un moment exceptionnel, en dehors de la règle, où la sagesse même
admettrait d’accueillir l’apaisement qu’apporte un plaisir et de donner congé,
jusqu’à la reprise utile de l’effort, à la volonté— suspendait l’action de
celle-ci qui cessait d’exercer sa compression; ou, moins que cela, le souvenir
d’un renseignement qu’il avait oublié de demander à Odette, si elle avait décidé
la couleur dont elle voulait faire repeindre sa voiture, ou pour une certaine
valeur de bourse, si c’était des actions ordinaires ou privilégiées qu’elle
désirait acquérir (c’était très joli de lui montrer qu’il pouvait rester sans la
voir, mais si après ça la peinture était à refaire ou si les actions ne
donnaient pas de dividende, il serait bien avancé), voici que comme un
caoutchouc tendu qu’on lâche ou comme l’air dans une machine pneumatique qu’on
entr’ouvre, l’idée de la revoir, des lointains où elle était maintenue, revenait
d’un bond dans le champ du présent et des possibilités immédiates.

Elle y revenait sans plus trouver de résistance, et d’ailleurs si irrésistible
que Swann avait eu bien moins de peine à sentir s’approcher un à un les quinze
jours qu’il devait rester séparé d’Odette, qu’il n’en avait à attendre les dix
minutes que son cocher mettait pour atteler la voiture qui allait l’emmener chez
elle et qu’il passait dans des transports d’impatience et de joie où il
ressaisissait mille fois pour lui prodiguer sa tendresse cette idée de la
retrouver qui, par un retour si brusque, au moment où il la croyait si loin,
était de nouveau près de lui dans sa plus proche conscience. C’est qu’elle ne
trouvait plus pour lui faire obstacle le désir de chercher sans plus tarder à
lui résister qui n’existait plus chez Swann depuis que s’étant prouvé à lui-même
— il le croyait du moins — qu’il en était si aisément capable, il ne voyait plus
aucun inconvénient à ajourner un essai de séparation qu’il était certain
maintenant de mettre à exécution dès qu’il le voudrait. C’est aussi que cette
idée de la revoir revenait parée pour lui d’une nouveauté, d’une séduction,
douée d’une virulence que l’habitude avait émoussées, mais qui s’étaient
retrempées dans cette privation non de trois jours mais de quinze (car la durée
d’un renoncement doit se calculer, par anticipation, sur le terme assigné), et
de ce qui jusque-là eût été un plaisir attendu qu’on sacrifie aisément, avait
fait un bonheur inespéré contre lequel on est sans force. C’est enfin qu’elle y
revenait embellie par l’ignorance où était Swann de ce qu’Odette avait pu
penser, faire peut-être en voyant qu’il ne lui avait pas donné signe de vie, si
bien que ce qu’il allait trouver c’était la révélation passionnante d’une Odette
presque inconnue.

Mais elle, de même qu’elle avait cru que son refus d’argent n’était qu’une
feinte, ne voyait qu’un prétexte dans le renseignement que Swann venait lui
demander, sur la voiture à repeindre, ou la valeur à acheter. Car elle ne
reconstituait pas les diverses phases de ces crises qu’il traversait et dans
l’idée qu’elle s’en faisait, elle omettait d’en comprendre le mécanisme, ne
croyant qu’à ce qu’elle connaissait d’avance, à la nécessaire, à l’infaillible
et toujours identique terminaison. Idée incomplète — d’autant plus profonde
peut-être — si on la jugeait du point de vue de Swann qui eût sans doute trouvé
qu’il était incompris d’Odette, comme un morphinomane ou un tuberculeux,
persuadés qu’ils ont été arrêtés, l’un par un événement extérieur au moment où
il allait se délivrer de son habitude invétérée, l’autre par une indisposition
accidentelle au moment où il allait être enfin rétabli, se sentent incompris du
médecin qui n’attache pas la même importance qu’eux à ces prétendues
contingences, simples déguisements, selon lui, revêtus, pour redevenir sensibles
à ses malades, par le vice et l’état morbide qui, en réalité, n’ont pas cessé de
peser incurablement sur eux tandis qu’ils berçaient des rêves de sagesse ou de
guérison. Et de fait, l’amour de Swann en était arrivé à ce degré où le médecin
et, dans certaines affections, le chirurgien le plus audacieux, se demandent si
priver un malade de son vice ou lui ôter son mal, est encore raisonnable ou même
possible.

Certes l’étendue de cet amour, Swann n’en avait pas une conscience directe.
Quand il cherchait à le mesurer, il lui arrivait parfois qu’il semblât diminué,
presque réduit à rien; par exemple, le peu de goût, presque le dégoût que lui
avaient inspiré, avant qu’il aimât Odette, ses traits expressifs, son teint sans
fraîcheur, lui revenait à certains jours. «Vraiment il y a progrès sensible, se
disait-il le lendemain; à voir exactement les choses, je n’avais presque aucun
plaisir hier à être dans son lit, c’est curieux je la trouvais même laide.» Et
certes, il était sincère, mais son amour s’étendait bien au-delà des régions du
désir physique. La personne même d’Odette n’y tenait plus une grande place.
Quand du regard il rencontrait sur sa table la photographie d’Odette, ou quand
elle venait le voir, il avait peine à identifier la figure de chair ou de
bristol avec le trouble douloureux et constant qui habitait en lui. Il se disait
presque avec étonnement: «C’est elle» comme si tout d’un coup on nous montrait
extériorisée devant nous une de nos maladies et que nous ne la trouvions pas
ressemblante à ce que nous souffrons. «Elle», il essayait de se demander ce que
c’était; car c’est une ressemblance de l’amour et de la mort, plutôt que celles
si vagues, que l’on redit toujours, de nous faire interroger plus avant, dans la
peur que sa réalité se dérobe, le mystère de la personnalité. Et cette maladie
qu’était l’amour de Swann avait tellement multiplié, il était si étroitement
mêlé à toutes les habitudes de Swann, à tous ses actes, à sa pensée, à sa santé,
à son sommeil, à sa vie, même à ce qu’il désirait pour après sa mort, il ne
faisait tellement plus qu’un avec lui, qu’on n’aurait pas pu l’arracher de lui
sans le détruire lui-même à peu près tout entier: comme on dit en chirurgie, son
amour n’était plus opérable.

Par cet amour Swann avait été tellement détaché de tous les intérêts, que quand
par hasard il retournait dans le monde en se disant que ses relations comme une
monture élégante qu’elle n’aurait pas d’ailleurs su estimer très exactement,
pouvaient lui rendre à lui-même un peu de prix aux yeux d’Odette (et ç’aurait
peut-être été vrai en effet si elles n’avaient été avilies par cet amour même,
qui pour Odette dépréciait toutes les choses qu’il touchait par le fait qu’il
semblait les proclamer moins précieuses), il y éprouvait, à côté de la détresse
d’être dans des lieux, au milieu de gens qu’elle ne connaissait pas, le plaisir
désintéressé qu’il aurait pris à un roman ou à un tableau où sont peints les
divertissements d’une classe oisive, comme, chez lui, il se complaisait à
considérer le fonctionnement de sa vie domestique, l’élégance de sa garde-robe
et de sa livrée, le bon placement de ses valeurs, de la même façon qu’à lire
dans Saint-Simon, qui était un de ses auteurs favoris, la mécanique des
journées, le menu des repas de Mme de Maintenon, ou l’avarice avisée et le grand
train de Lulli. Et dans la faible mesure où ce détachement n’était pas absolu,
la raison de ce plaisir nouveau que goûtait Swann, c’était de pouvoir émigrer un
moment dans les rares parties de lui-même restées presque étrangères à son
amour, à son chagrin. A cet égard cette personnalité, que lui attribuait ma
grand’tante, de «fils Swann», distincte de sa personnalité plus individuelle de
Charles Swann, était celle où il se plaisait maintenant le mieux. Un jour que,
pour l’anniversaire de la princesse de Parme (et parce qu’elle pouvait souvent
être indirectement agréable à Odette en lui faisant avoir des places pour des
galas, des jubilés), il avait voulu lui envoyer des fruits, ne sachant pas trop
comment les commander, il en avait chargé une cousine de sa mère qui, ravie de
faire une commission pour lui, lui avait écrit, en lui rendant compte qu’elle
n’avait pas pris tous les fruits au même endroit, mais les raisins chez Crapote
dont c’est la spécialité, les fraises chez Jauret, les poires chez Chevet où
elles étaient plus belles, etc., «chaque fruit visité et examiné un par un par
moi». Et en effet, par les remerciements de la princesse, il avait pu juger du
parfum des fraises et du moelleux des poires. Mais surtout le «chaque fruit
visité et examiné un par un par moi» avait été un apaisement à sa souffrance, en
emmenant sa conscience dans une région où il se rendait rarement, bien qu’elle
lui appartînt comme héritier d’une famille de riche et bonne bourgeoisie où
s’étaient conservés héréditairement, tout prêts à être mis à son service dès
qu’il le souhaitait, la connaissance des «bonnes adresses» et l’art de savoir
bien faire une commande.

Certes, il avait trop longtemps oublié qu’il était le «fils Swann» pour ne pas
ressentir quand il le redevenait un moment, un plaisir plus vif que ceux qu’il
eût pu éprouver le reste du temps et sur lesquels il était blasé; et si
l’amabilité des bourgeois, pour lesquels il restait surtout cela, était moins
vive que celle de l’aristocratie (mais plus flatteuse d’ailleurs, car chez eux
du moins elle ne se sépare jamais de la considération), une lettre d’altesse,
quelques divertissements princiers qu’elle lui proposât, ne pouvait lui être
aussi agréable que celle qui lui demandait d’être témoin, ou seulement
d’assister à un mariage dans la famille de vieux amis de ses parents dont les
uns avaient continué à le voir — comme mon grand-père qui, l’année précédente,
l’avait invité au mariage de ma mère — et dont certains autres le connaissaient
personnellement à peine mais se croyaient des devoirs de politesse envers le
fils, envers le digne successeur de feu M. Swann.

Mais, par les intimités déjà anciennes qu’il avait parmi eux, les gens du monde,
dans une certaine mesure, faisaient aussi partie de sa maison, de son domestique
et de sa famille. Il se sentait, à considérer ses brillantes amitiés, le même
appui hors de lui-même, le même confort, qu’à regarder les belles terres, la
belle argenterie, le beau linge de table, qui lui venaient des siens. Et la
pensée que s’il tombait chez lui frappé d’une attaque ce serait tout
naturellement le duc de Chartres, le prince de Reuss, le duc de Luxembourg et le
baron de Charlus, que son valet de chambre courrait chercher, lui apportait la
même consolation qu’à notre vieille Françoise de savoir qu’elle serait ensevelie
dans des draps fins à elle, marqués, non reprisés (ou si finement que cela ne
donnait qu’une plus haute idée du soin de l’ouvrière), linceul de l’image
fréquente duquel elle tirait une certaine satisfaction, sinon de bien-être, au
moins d’amour-propre. Mais surtout, comme dans toutes celles de ses actions, et
de ses pensées qui se rapportaient à Odette, Swann était constamment dominé et
dirigé par le sentiment inavoué qu’il lui était peut-être pas moins cher, mais
moins agréable à voir que quiconque, que le plus ennuyeux fidèle des Verdurin,
quand il se reportait à un monde pour qui il était l’homme exquis par
excellence, qu’on faisait tout pour attirer, qu’on se désolait de ne pas voir,
il recommençait à croire à l’existence d’une vie plus heureuse, presque à en
éprouver l’appétit, comme il arrive à un malade alité depuis des mois, à la
diète, et qui aperçoit dans un journal le menu d’un déjeuner officiel ou
l’annonce d’une croisière en Sicile.

S’il était obligé de donner des excuses aux gens du monde pour ne pas leur faire
de visites, c’était de lui en faire qu’il cherchait à s’excuser auprès d’Odette.
Encore les payait-il (se demandant à la fin du mois, pour peu qu’il eût un peu
abusé de sa patience et fût allé souvent la voir, si c’était assez de lui
envoyer quatre mille francs), et pour chacune trouvait un prétexte, un présent à
lui apporter, un renseignement dont elle avait besoin, M. de Charlus qu’elle
avait rencontré allant chez elle, et qui avait exigé qu’il l’accompagnât. Et à
défaut d’aucun, il priait M. de Charlus de courir chez elle, de lui dire comme
spontanément, au cours de la conversation, qu’il se rappelait avoir à parler à
Swann, qu’elle voulût bien lui faire demander de passer tout de suite chez elle;
mais le plus souvent Swann attendait en vain et M. de Charlus lui disait le soir
que son moyen n’avait pas réussi. De sorte que si elle faisait maintenant de
fréquentes absences, même à Paris, quand elle y restait, elle le voyait peu, et
elle qui, quand elle l’aimait, lui disait: «Je suis toujours libre» et
«Qu’est-ce que l’opinion des autres peut me faire?», maintenant, chaque fois
qu’il voulait la voir, elle invoquait les convenances ou prétextait des
occupations. Quand il parlait d’aller à une fête de charité, à un vernissage, à
une première, où elle serait, elle lui disait qu’il voulait afficher leur
liaison, qu’il la traitait comme une fille. C’est au point que pour tâcher de
n’être pas partout privé de la rencontrer, Swann qui savait qu’elle connaissait
et affectionnait beaucoup mon grand-oncle Adolphe dont il avait été lui-même
l’ami, alla le voir un jour dans son petit appartement de la rue de Bellechasse
afin de lui demander d’user de son influence sur Odette. Comme elle prenait
toujours, quand elle parlait à Swann, de mon oncle, des airs poétiques, disant:
«Ah! lui, ce n’est pas comme toi, c’est une si belle chose, si grande, si jolie,
que son amitié pour moi. Ce n’est pas lui qui me considérerait assez peu pour
vouloir se montrer avec moi dans tous les lieux publics», Swann fut embarrassé
et ne savait pas à quel ton il devait se hausser pour parler d’elle à mon oncle.
Il posa d’abord l’excellence a priori d’Odette, l’axiome de sa supra-humanité
séraphique, la révélation de ses vertus indémontrables et dont la notion ne
pouvait dériver de l’expérience. «Je veux parler avec vous. Vous, vous savez
quelle femme au-dessus de toutes les femmes, quel être adorable, quel ange est
Odette. Mais vous savez ce que c’est que la vie de Paris. Tout le monde ne
connaît pas Odette sous le jour où nous la connaissons vous et moi. Alors il y a
des gens qui trouvent que je joue un rôle un peu ridicule; elle ne peut même pas
admettre que je la rencontre dehors, au théâtre. Vous, en qui elle a tant de
confiance, ne pourriez-vous lui dire quelques mots pour moi, lui assurer qu’elle
s’exagère le tort qu’un salut de moi lui cause?»

Mon oncle conseilla à Swann de rester un peu sans voir Odette qui ne l’en
aimerait que plus, et à Odette de laisser Swann la retrouver partout où cela lui
plairait. Quelques jours après, Odette disait à Swann qu’elle venait d’avoir une
déception en voyant que mon oncle était pareil à tous les hommes: il venait
d’essayer de la prendre de force. Elle calma Swann qui au premier moment voulait
aller provoquer mon oncle, mais il refusa de lui serrer la main quand il le
rencontra. Il regretta d’autant plus cette brouille avec mon oncle Adolphe qu’il
avait espéré, s’il l’avait revu quelquefois et avait pu causer en toute
confiance avec lui, tâcher de tirer au clair certains bruits relatifs à la vie
qu’Odette avait menée autrefois à Nice. Or mon oncle Adolphe y passait l’hiver.
Et Swann pensait que c’était même peut-être là qu’il avait connu Odette. Le peu
qui avait échappé à quelqu’un devant lui, relativement à un homme qui aurait été
l’amant d’Odette avait bouleversé Swann. Mais les choses qu’il aurait avant de
les connaître, trouvé le plus affreux d’apprendre et le plus impossible de
croire, une fois qu’il les savait, elles étaient incorporées à tout jamais à sa
tristesse, il les admettait, il n’aurait plus pu comprendre qu’elles n’eussent
pas été. Seulement chacune opérait sur l’idée qu’il se faisait de sa maîtresse
une retouche ineffaçable. Il crut même comprendre, une fois, que cette légèreté
des mœurs d’Odette qu’il n’eût pas soupçonnée, était assez connue, et qu’à Bade
et à Nice, quand elle y passait jadis plusieurs mois, elle avait eu une sorte de
notoriété galante. Il chercha, pour les interroger, à se rapprocher de certains
viveurs; mais ceux-ci savaient qu’il connaissait Odette; et puis il avait peur
de les faire penser de nouveau à elle, de les mettre sur ses traces. Mais lui à
qui jusque-là rien n’aurait pu paraître aussi fastidieux que tout ce qui se
rapportait à la vie cosmopolite de Bade ou de Nice, apprenant qu’Odette avait
peut-être fait autrefois la fête dans ces villes de plaisir, sans qu’il dût
jamais arriver à savoir si c’était seulement pour satisfaire à des besoins
d’argent que grâce à lui elle n’avait plus, ou à des caprices qui pouvaient
renaître, maintenant il se penchait avec une angoisse impuissante, aveugle et
vertigineuse vers l’abîme sans fond où étaient allées s’engloutir ces années du
début du Septennat pendant lesquelles on passait l’hiver sur la promenade des
Anglais, l’été sous les tilleuls de Bade, et il leur trouvait une profondeur
douloureuse mais magnifique comme celle que leur eût prêtée un poète; et il eût
mis à reconstituer les petits faits de la chronique de la Côte d’Azur d’alors,
si elle avait pu l’aider à comprendre quelque chose du sourire ou des regards —
pourtant si honnêtes et si simples — d’Odette, plus de passion que l’esthéticien
qui interroge les documents subsistant de la Florence du XVe siècle pour tâcher
d’entrer plus avant dans l’âme de la Primavera, de la bella Vanna, ou de la
Vénus, de Botticelli. Souvent sans lui rien dire il la regardait, il songeait;
elle lui disait: «Comme tu as l’air triste!» Il n’y avait pas bien longtemps
encore, de l’idée qu’elle était une créature bonne, analogue aux meilleures
qu’il eût connues, il avait passé à l’idée qu’elle était une femme entretenue;
inversement il lui était arrivé depuis de revenir de l’Odette de Crécy,
peut-être trop connue des fêtards, des hommes à femmes, à ce visage d’une
expression parfois si douce, à cette nature si humaine. Il se disait: «Qu’est-ce
que cela veut dire qu’à Nice tout le monde sache qui est Odette de Crécy? Ces
réputations-là, même vraies, sont faites avec les idées des autres»; il pensait
que cette légende — fût-elle authentique —était extérieure à Odette, n’était pas
en elle comme une personnalité irréductible et malfaisante; que la créature qui
avait pu être amenée à mal faire, c’était une femme aux bons yeux, au cœur plein
de pitié pour la souffrance, au corps docile qu’il avait tenu, qu’il avait serré
dans ses bras et manié, une femme qu’il pourrait arriver un jour à posséder
toute, s’il réussissait à se rendre indispensable à elle. Elle était là, souvent
fatiguée, le visage vidé pour un instant de la préoccupation fébrile et joyeuse
des choses inconnues qui faisaient souffrir Swann; elle écartait ses cheveux
avec ses mains; son front, sa figure paraissaient plus larges; alors, tout d’un
coup, quelque pensée simplement humaine, quelque bon sentiment comme il en
existe dans toutes les créatures, quand dans un moment de repos ou de repliement
elles sont livrées à elles-mêmes, jaillissait dans ses yeux comme un rayon
jaune. Et aussitôt tout son visage s’éclairait comme une campagne grise,
couverte de nuages qui soudain s’écartent, pour sa transfiguration, au moment du
soleil couchant. La vie qui était en Odette à ce moment-là, l’avenir même
qu’elle semblait rêveusement regarder, Swann aurait pu les partager avec elle;
aucune agitation mauvaise ne semblait y avoir laissé de résidu. Si rares qu’ils
devinssent, ces moments-là ne furent pas inutiles. Par le souvenir Swann reliait
ces parcelles, abolissait les intervalles, coulait comme en or une Odette de
bonté et de calme pour laquelle il fit plus tard (comme on le verra dans la
deuxième partie de cet ouvrage) des sacrifices que l’autre Odette n’eût pas
obtenus. Mais que ces moments étaient rares, et que maintenant il la voyait peu!
Même pour leur rendez-vous du soir, elle ne lui disait qu’à la dernière minute
si elle pourrait le lui accorder car, comptant qu’elle le trouverait toujours
libre, elle voulait d’abord être certaine que personne d’autre ne lui
proposerait de venir. Elle alléguait qu’elle était obligée d’attendre une
réponse de la plus haute importance pour elle, et même si après qu’elle avait
fait venir Swann des amis demandaient à Odette, quand la soirée était déjà
commencée, de les rejoindre au théâtre ou à souper, elle faisait un bond joyeux
et s’habillait à la hâte. Au fur et à mesure qu’elle avançait dans sa toilette,
chaque mouvement qu’elle faisait rapprochait Swann du moment où il faudrait la
quitter, où elle s’enfuirait d’un élan irrésistible; et quand, enfin prête,
plongeant une dernière fois dans son miroir ses regards tendus et éclairés par
l’attention, elle remettait un peu de rouge à ses lèvres, fixait une mèche sur
son front et demandait son manteau de soirée bleu ciel avec des glands d’or,
Swann avait l’air si triste qu’elle ne pouvait réprimer un geste d’impatience et
disait: «Voilà comme tu me remercies de t’avoir gardé jusqu’à la dernière
minute. Moi qui croyais avoir fait quelque chose de gentil. C’est bon à savoir
pour une autre fois!» Parfois, au risque de la fâcher, il se promettait de
chercher à savoir où elle était allée, il rêvait d’une alliance avec Forcheville
qui peut-être aurait pu le renseigner. D’ailleurs quand il savait avec qui elle
passait la soirée, il était bien rare qu’il ne pût pas découvrir dans toutes ses
relations à lui quelqu’un qui connaissait fût-ce indirectement l’homme avec qui
elle était sortie et pouvait facilement en obtenir tel ou tel renseignement. Et
tandis qu’il écrivait à un de ses amis pour lui demander de chercher à éclaircir
tel ou tel point, il éprouvait le repos de cesser de se poser ses questions sans
réponses et de transférer à un autre la fatigue d’interroger. Il est vrai que
Swann n’était guère plus avancé quand il avait certains renseignements. Savoir
ne permet pas toujours d’empêcher, mais du moins les choses que nous savons,
nous les tenons, sinon entre nos mains, du moins dans notre pensée où nous les
disposons à notre gré, ce qui nous donne l’illusion d’une sorte de pouvoir sur
elles. Il était heureux toutes les fois où M. de Charlus était avec Odette.
Entre M. de Charlus et elle, Swann savait qu’il ne pouvait rien se passer, que
quand M. de Charlus sortait avec elle c’était par amitié pour lui et qu’il ne
ferait pas difficulté à lui raconter ce qu’elle avait fait. Quelquefois elle
avait déclaré si catégoriquement à Swann qu’il lui était impossible de le voir
un certain soir, elle avait l’air de tenir tant à une sortie, que Swann
attachait une véritable importance à ce que M. de Charlus fût libre de
l’accompagner. Le lendemain, sans oser poser beaucoup de questions à M. de
Charlus, il le contraignait, en ayant l’air de ne pas bien comprendre ses
premières réponses, à lui en donner de nouvelles, après chacune desquelles il se
sentait plus soulagé, car il apprenait bien vite qu’Odette avait occupé sa
soirée aux plaisirs les plus innocents. «Mais comment, mon petit Mémé, je ne
comprends pas bien . . ., ce n’est pas en sortant de chez elle que vous êtes
allés au musée Grévin? Vous étiez allés ailleurs d’abord. Non? Oh! que c’est
drôle! Vous ne savez pas comme vous m’amusez, mon petit Mémé. Mais quelle drôle
d’idée elle a eue d’aller ensuite au Chat Noir, c’est bien une idée d’elle . . .
Non? c’est vous. C’est curieux. Après tout ce n’est pas une mauvaise idée, elle
devait y connaître beaucoup de monde? Non? elle n’a parlé à personne? C’est
extraordinaire. Alors vous êtes restés là comme cela tous les deux tous seuls?
Je vois d’ici cette scène. Vous êtes gentil, mon petit Mémé, je vous aime bien.»
Swann se sentait soulagé. Pour lui, à qui il était arrivé en causant avec des
indifférents qu’il écoutait à peine, d’entendre quelquefois certaines phrases
(celle-ci par exemple: «J’ai vu hier Mme de Crécy, elle était avec un monsieur
que je ne connais pas»), phrases qui aussitôt dans le cœur de Swann passaient à
l’état solide, s’y durcissaient comme une incrustation, le déchiraient, n’en
bougeaient plus, qu’ils étaient doux au contraire ces mots: «Elle ne connaissait
personne, elle n’a parlé à personne», comme ils circulaient aisément en lui,
qu’ils étaient fluides, faciles, respirables! Et pourtant au bout d’un instant
il se disait qu’Odette devait le trouver bien ennuyeux pour que ce fussent là
les plaisirs qu’elle préférait à sa compagnie. Et leur insignifiance, si elle le
rassurait, lui faisait pourtant de la peine comme une trahison.

Même quand il ne pouvait savoir où elle était allée, il lui aurait suffi pour
calmer l’angoisse qu’il éprouvait alors, et contre laquelle la présence
d’Odette, la douceur d’être auprès d’elle était le seul spécifique (un
spécifique qui à la longue aggravait le mal avec bien des remèdes, mais du moins
calmait momentanément la souffrance), il lui aurait suffi, si Odette l’avait
seulement permis, de rester chez elle tant qu’elle ne serait pas là, de
l’attendre jusqu’à cette heure du retour dans l’apaisement de laquelle seraient
venues se confondre les heures qu’un prestige, un maléfice lui avaient fait
croire différentes des autres. Mais elle ne le voulait pas; il revenait chez
lui; il se forçait en chemin à former divers projets, il cessait de songer à
Odette; même il arrivait, tout en se déshabillant, à rouler en lui des pensées
assez joyeuses; c’est le cœur plein de l’espoir d’aller le lendemain voir
quelque chef-d’œuvre qu’il se mettait au lit et éteignait sa lumière; mais, dès
que, pour se préparer à dormir, il cessait d’exercer sur lui-même une contrainte
dont il n’avait même pas conscience tant elle était devenue habituelle, au même
instant un frisson glacé refluait en lui et il se mettait à sangloter. Il ne
voulait même pas savoir pourquoi, s’essuyait les yeux, se disait en riant:
«C’est charmant, je deviens névropathe.» Puis il ne pouvait penser sans une
grande lassitude que le lendemain il faudrait recommencer de chercher à savoir
ce qu’Odette avait fait, à mettre en jeu des influences pour tâcher de la voir.
Cette nécessité d’une activité sans trêve, sans variété, sans résultats, lui
était si cruelle qu’un jour apercevant une grosseur sur son ventre, il ressentit
une véritable joie à la pensée qu’il avait peut-être une tumeur mortelle, qu’il
n’allait plus avoir à s’occuper de rien, que c’était la maladie qui allait le
gouverner, faire de lui son jouet, jusqu’à la fin prochaine. Et en effet si, à
cette époque, il lui arriva souvent sans se l’avouer de désirer la mort, c’était
pour échapper moins à l’acuité de ses souffrances qu’à la monotonie de son
effort.

Et pourtant il aurait voulu vivre jusqu’à l’époque où il ne l’aimerait plus, où
elle n’aurait aucune raison de lui mentir et où il pourrait enfin apprendre
d’elle si le jour où il était allé la voir dans l’après-midi, elle était ou non
couchée avec Forcheville. Souvent pendant quelques jours, le soupçon qu’elle
aimait quelqu’un d’autre le détournait de se poser cette question relative à
Forcheville, la lui rendait presque indifférente, comme ces formes nouvelles
d’un même état maladif qui semblent momentanément nous avoir délivrés des
précédentes. Même il y avait des jours où il n’était tourmenté par aucun
soupçon. Il se croyait guéri. Mais le lendemain matin, au réveil, il sentait à
la même place la même douleur dont, la veille pendant la journée, il avait comme
dilué la sensation dans le torrent des impressions différentes. Mais elle
n’avait pas bougé de place. Et même, c’était l’acuité de cette douleur qui avait
réveillé Swann.

Comme Odette ne lui donnait aucun renseignement sur ces choses si importantes
qui l’occupaient tant chaque jour (bien qu’il eût assez vécu pour savoir qu’il
n’y en a jamais d’autres que les plaisirs), il ne pouvait pas chercher longtemps
de suite à les imaginer, son cerveau fonctionnait à vide; alors il passait son
doigt sur ses paupières fatiguées comme il aurait essuyé le verre de son
lorgnon, et cessait entièrement de penser. Il surnageait pourtant à cet inconnu
certaines occupations qui réapparaissaient de temps en temps, vaguement
rattachées par elle à quelque obligation envers des parents éloignés ou des amis
d’autrefois, qui, parce qu’ils étaient les seuls qu’elle lui citait souvent
comme l’empêchant de le voir, paraissaient à Swann former le cadre fixe,
nécessaire, de la vie d’Odette. A cause du ton dont elle lui disait de temps à
autre «le jour où je vais avec mon amie à l’Hippodrome», si, s’étant senti
malade et ayant pensé: «peut-être Odette voudrait bien passer chez moi», il se
rappelait brusquement que c’était justement ce jour-là, il se disait: «Ah! non,
ce n’est pas la peine de lui demander de venir, j’aurais dû y penser plus tôt,
c’est le jour où elle va avec son amie à l’Hippodrome. Réservons-nous pour ce
qui est possible; c’est inutile de s’user à proposer des choses inacceptables et
refusées d’avance.» Et ce devoir qui incombait à Odette d’aller à l’Hippodrome
et devant lequel Swann s’inclinait ainsi ne lui paraissait pas seulement
inéluctable; mais ce caractère de nécessité dont il était empreint semblait
rendre plausible et légitime tout ce qui de près ou de loin se rapportait à lui.
Si Odette dans la rue ayant reçu d’un passant un salut qui avait éveillé la
jalousie de Swann, elle répondait aux questions de celui-ci en rattachant
l’existence de l’inconnu à un des deux ou trois grands devoirs dont elle lui
parlait, si, par exemple, elle disait: «C’est un monsieur qui était dans la loge
de mon amie avec qui je vais à l’Hippodrome», cette explication calmait les
soupçons de Swann, qui en effet trouvait inévitable que l’amie eût d’autre
invités qu’Odette dans sa loge à l’Hippodrome, mais n’avait jamais cherché ou
réussi à se les figurer. Ah! comme il eût aimé la connaître, l’amie qui allait à
l’Hippodrome, et qu’elle l’y emmenât avec Odette! Comme il aurait donné toutes
ses relations pour n’importe quelle personne qu’avait l’habitude de voir Odette,
fût-ce une manucure ou une demoiselle de magasin. Il eût fait pour elles plus de
frais que pour des reines. Ne lui auraient-elles pas fourni, dans ce qu’elles
contenaient de la vie d’Odette, le seul calmant efficace pour ses souffrances?
Comme il aurait couru avec joie passer les journées chez telle de ces petites
gens avec lesquelles Odette gardait des relations, soit par intérêt, soit par
simplicité véritable. Comme il eût volontiers élu domicile à jamais au cinquième
étage de telle maison sordide et enviée où Odette ne l’emmenait pas, et où, s’il
y avait habité avec la petite couturière retirée dont il eût volontiers fait
semblant d’être l’amant, il aurait presque chaque jour reçu sa visite. Dans ces
quartiers presque populaires, quelle existence modeste, abjecte, mais douce,
mais nourrie de calme et de bonheur, il eût accepté de vivre indéfiniment.

Il arrivait encore parfois, quand, ayant rencontré Swann, elle voyait
s’approcher d’elle quelqu’un qu’il ne connaissait pas, qu’il pût remarquer sur
le visage d’Odette cette tristesse qu’elle avait eue le jour où il était venu
pour la voir pendant que Forcheville était là. Mais c’était rare; car les jours
où malgré tout ce qu’elle avait à faire et la crainte de ce que penserait le
monde, elle arrivait à voir Swann, ce qui dominait maintenant dans son attitude
était l’assurance: grand contraste, peut-être revanche inconsciente ou réaction
naturelle de l’émotion craintive qu’aux premiers temps où elle l’avait connu,
elle éprouvait auprès de lui, et même loin de lui, quand elle commençait une
lettre par ces mots: «Mon ami, ma main tremble si fort que je peux à peine
écrire» (elle le prétendait du moins et un peu de cet émoi devait être sincère
pour qu’elle désirât d’en feindre davantage). Swann lui plaisait alors. On ne
tremble jamais que pour soi, que pour ceux qu’on aime. Quand notre bonheur n’est
plus dans leurs mains, de quel calme, de quelle aisance, de quelle hardiesse on
jouit auprès d’eux! En lui parlant, en lui écrivant, elle n’avait plus de ces
mots par lesquels elle cherchait à se donner l’illusion qu’il lui appartenait,
faisant naître les occasions de dire «mon», «mien», quand il s’agissait de lui:
«Vous êtes mon bien, c’est le parfum de notre amitié, je le garde», de lui
parler de l’avenir, de la mort même, comme d’une seule chose pour eux deux. Dans
ce temps-là, à tout de qu’il disait, elle répondait avec admiration: «Vous, vous
ne serez jamais comme tout le monde»; elle regardait sa longue tête un peu
chauve, dont les gens qui connaissaient les succès de Swann pensaient: «Il n’est
pas régulièrement beau si vous voulez, mais il est chic: ce toupet, ce monocle,
ce sourire!», et, plus curieuse peut-être de connaître ce qu’il était que
désireuse d’être sa maîtresse, elle disait:

—«Si je pouvais savoir ce qu’il y a dans cette tête là!»

Maintenant, à toutes les paroles de Swann elle répondait d’un ton parfois
irrité, parfois indulgent:

—«Ah! tu ne seras donc jamais comme tout le monde!»

Elle regardait cette tête qui n’était qu’un peu plus vieillie par le souci (mais
dont maintenant tous pensaient, en vertu de cette même aptitude qui permet de
découvrir les intentions d’un morceau symphonique dont on a lu le programme, et
les ressemblances d’un enfant quand on connaît sa parenté: «Il n’est pas
positivement laid si vous voulez, mais il est ridicule: ce monocle, ce toupet,
ce sourire!», réalisant dans leur imagination suggestionnée la démarcation
immatérielle qui sépare à quelques mois de distance une tête d’amant de cœur et
une tête de cocu), elle disait:

—«Ah! si je pouvais changer, rendre raisonnable ce qu’il y a dans cette
tête-là.»

Toujours prêt à croire ce qu’il souhaitait si seulement les manières d’être
d’Odette avec lui laissaient place au doute, il se jetait avidement sur cette
parole:

—«Tu le peux si tu le veux, lui disait-il.»

Et il tâchait de lui montrer que l’apaiser, le diriger, le faire travailler,
serait une noble tâche à laquelle ne demandaient qu’à se vouer d’autres femmes
qu’elle, entre les mains desquelles il est vrai d’ajouter que la noble tâche ne
lui eût paru plus qu’une indiscrète et insupportable usurpation de sa liberté.
«Si elle ne m’aimait pas un peu, se disait-il, elle ne souhaiterait pas de me
transformer. Pour me transformer, il faudra qu’elle me voie davantage.» Ainsi
trouvait-il dans ce reproche qu’elle lui faisait, comme une preuve d’intérêt,
d’amour peut-être; et en effet, elle lui en donnait maintenant si peu qu’il
était obligé de considérer comme telles les défenses qu’elle lui faisait d’une
chose ou d’une autre. Un jour, elle lui déclara qu’elle n’aimait pas son cocher,
qu’il lui montait peut-être la tête contre elle, qu’en tous cas il n’était pas
avec lui de l’exactitude et de la déférence qu’elle voulait. Elle sentait qu’il
désirait lui entendre dire: «Ne le prends plus pour venir chez moi», comme il
aurait désiré un baiser. Comme elle était de bonne humeur, elle le lui dit; il
fut attendri. Le soir, causant avec M. de Charlus avec qui il avait la douceur
de pouvoir parler d’elle ouvertement (car les moindres propos qu’il tenait, même
aux personnes qui ne la connaissaient pas, se rapportaient en quelque manière à
elle), il lui dit:

— Je crois pourtant qu’elle m’aime; elle est si gentille pour moi, ce que je
fais ne lui est certainement pas indifférent.

Et si, au moment d’aller chez elle, montant dans sa voiture avec un ami qu’il
devait laisser en route, l’autre lui disait:

—«Tiens, ce n’est pas Lorédan qui est sur le siège?», avec quelle joie
mélancolique Swann lui répondait:

—«Oh! sapristi non! je te dirai, je ne peux pas prendre Lorédan quand je vais
rue La Pérouse. Odette n’aime pas que je prenne Lorédan, elle ne le trouve pas
bien pour moi; enfin que veux-tu, les femmes, tu sais! je sais que ça lui
déplairait beaucoup. Ah bien oui! je n’aurais eu qu’à prendre Rémi! j’en aurais
eu une histoire!»

Ces nouvelles façons indifférentes, distraites, irritables, qui étaient
maintenant celles d’Odette avec lui, certes Swann en souffrait; mais il ne
connaissait pas sa souffrance; comme c’était progressivement, jour par jour,
qu’Odette s’était refroidie à son égard, ce n’est qu’en mettant en regard de ce
qu’elle était aujourd’hui ce qu’elle avait été au début, qu’il aurait pu sonder
la profondeur du changement qui s’était accompli. Or ce changement c’était sa
profonde, sa secrète blessure, qui lui faisait mal jour et nuit, et dès qu’il
sentait que ses pensées allaient un peu trop près d’elle, vivement il les
dirigeait d’un autre côté de peur de trop souffrir. Il se disait bien d’une
façon abstraite: «Il fut un temps où Odette m’aimait davantage», mais jamais il
ne revoyait ce temps. De même qu’il y avait dans son cabinet une commode qu’il
s’arrangeait à ne pas regarder, qu’il faisait un crochet pour éviter en entrant
et en sortant, parce que dans un tiroir étaient serrés le chrysanthème qu’elle
lui avait donné le premier soir où il l’avait reconduite, les lettres où elle
disait: «Que n’y avez-vous oublié aussi votre cœur, je ne vous aurais pas laissé
le reprendre» et: «A quelque heure du jour et de la nuit que vous ayez besoin de
moi, faites-moi signe et disposez de ma vie», de même il y avait en lui une
place dont il ne laissait jamais approcher son esprit, lui faisant faire s’il le
fallait le détour d’un long raisonnement pour qu’il n’eût pas à passer devant
elle: c’était celle où vivait le souvenir des jours heureux.

Mais sa si précautionneuse prudence fut déjouée un soir qu’il était allé dans le
monde.

C’était chez la marquise de Saint-Euverte, à la dernière, pour cette année-là,
des soirées où elle faisait entendre des artistes qui lui servaient ensuite pour
ses concerts de charité. Swann, qui avait voulu successivement aller à toutes
les précédentes et n’avait pu s’y résoudre, avait reçu, tandis qu’il s’habillait
pour se rendre à celle-ci, la visite du baron de Charlus qui venait lui offrir
de retourner avec lui chez la marquise, si sa compagnie devait l’aider à s’y
ennuyer un peu moins, à s’y trouver moins triste. Mais Swann lui avait répondu:

—«Vous ne doutez pas du plaisir que j’aurais à être avec vous. Mais le plus
grand plaisir que vous puissiez me faire c’est d’aller plutôt voir Odette. Vous
savez l’excellente influence que vous avez sur elle. Je crois qu’elle ne sort
pas ce soir avant d’aller chez son ancienne couturière où du reste elle sera
sûrement contente que vous l’accompagniez. En tous cas vous la trouveriez chez
elle avant. Tâchez de la distraire et aussi de lui parler raison. Si vous
pouviez arranger quelque chose pour demain qui lui plaise et que nous pourrions
faire tous les trois ensemble. Tâchez aussi de poser des jalons pour cet été, si
elle avait envie de quelque chose, d’une croisière que nous ferions tous les
trois, que sais-je? Quant à ce soir, je ne compte pas la voir; maintenant si
elle le désirait ou si vous trouviez un joint, vous n’avez qu’à m’envoyer un mot
chez Mme de Saint-Euverte jusqu’à minuit, et après chez moi. Merci de tout ce
que vous faites pour moi, vous savez comme je vous aime.»

Le baron lui promit d’aller faire la visite qu’il désirait après qu’il l’aurait
conduit jusqu’à la porte de l’hôtel Saint-Euverte, où Swann arriva tranquillisé
par la pensée que M. de Charlus passerait la soirée rue La Pérouse, mais dans un
état de mélancolique indifférence à toutes les choses qui ne touchaient pas
Odette, et en particulier aux choses mondaines, qui leur donnait le charme de ce
qui, n’étant plus un but pour notre volonté, nous apparaît en soi-même. Dès sa
descente de voiture, au premier plan de ce résumé fictif de leur vie domestique
que les maîtresses de maison prétendent offrir à leurs invités les jours de
cérémonie et où elles cherchent à respecter la vérité du costume et celle du
décor, Swann prit plaisir à voir les héritiers des «tigres» de Balzac, les
grooms, suivants ordinaires de la promenade, qui, chapeautés et bottés,
restaient dehors devant l’hôtel sur le sol de l’avenue, ou devant les écuries,
comme des jardiniers auraient été rangés à l’entrée de leurs parterres. La
disposition particulière qu’il avait toujours eue à chercher des analogies entre
les êtres vivants et les portraits des musées s’exerçait encore mais d’une façon
plus constante et plus générale; c’est la vie mondaine tout entière, maintenant
qu’il en était détaché, qui se présentait à lui comme une suite de tableaux.
Dans le vestibule où, autrefois, quand il était un mondain, il entrait enveloppé
dans son pardessus pour en sortir en frac, mais sans savoir ce qui s’y était
passé, étant par la pensée, pendant les quelques instants qu’il y séjournait, ou
bien encore dans la fête qu’il venait de quitter, ou bien déjà dans la fête où
on allait l’introduire, pour la première fois il remarqua, réveillée par
l’arrivée inopinée d’un invité aussi tardif, la meute éparse, magnifique et
désœuvrée de grands valets de pied qui dormaient çà et là sur des banquettes et
des coffres et qui, soulevant leurs nobles profils aigus de lévriers, se
dressèrent et, rassemblés, formèrent le cercle autour de lui.

L’un d’eux, d’aspect particulièrement féroce et assez semblable à l’exécuteur
dans certains tableaux de la Renaissance qui figurent des supplices, s’avança
vers lui d’un air implacable pour lui prendre ses affaires. Mais la dureté de
son regard d’acier était compensée par la douceur de ses gants de fil, si bien
qu’en approchant de Swann il semblait témoigner du mépris pour sa personne et
des égards pour son chapeau. Il le prit avec un soin auquel l’exactitude de sa
pointure donnait quelque chose de méticuleux et une délicatesse que rendait
presque touchante l’appareil de sa force. Puis il le passa à un de ses aides,
nouveau, et timide, qui exprimait l’effroi qu’il ressentait en roulant en tous
sens des regards furieux et montrait l’agitation d’une bête captive dans les
premières heures de sa domesticité.

A quelques pas, un grand gaillard en livrée rêvait, immobile, sculptural,
inutile, comme ce guerrier purement décoratif qu’on voit dans les tableaux les
plus tumultueux de Mantegna, songer, appuyé sur son bouclier, tandis qu’on se
précipite et qu’on s’égorge à côté de lui; détaché du groupe de ses camarades
qui s’empressaient autour de Swann, il semblait aussi résolu à se désintéresser
de cette scène, qu’il suivait vaguement de ses yeux glauques et cruels, que si
ç’eût été le massacre des Innocents ou le martyre de saint Jacques. Il semblait
précisément appartenir à cette race disparue — ou qui peut-être n’exista jamais
que dans le retable de San Zeno et les fresques des Eremitani où Swann l’avait
approchée et où elle rêve encore — issue de la fécondation d’une statue antique
par quelque modèle padouan du Maître ou quelque saxon d’Albert Dürer. Et les
mèches de ses cheveux roux crespelés par la nature, mais collés par la
brillantine, étaient largement traitées comme elles sont dans la sculpture
grecque qu’étudiait sans cesse le peintre de Mantoue, et qui, si dans la
création elle ne figure que l’homme, sait du moins tirer de ses simples formes
des richesses si variées et comme empruntées à toute la nature vivante, qu’une
chevelure, par l’enroulement lisse et les becs aigus de ses boucles, ou dans la
superposition du triple et fleurissant diadème de ses tresses, a l’air à la fois
d’un paquet d’algues, d’une nichée de colombes, d’un bandeau de jacinthes et
d’une torsade de serpent.

D’autres encore, colossaux aussi, se tenaient sur les degrés d’un escalier
monumental que leur présence décorative et leur immobilité marmoréenne auraient
pu faire nommer comme celui du Palais Ducal: «l’Escalier des Géants» et dans
lequel Swann s’engagea avec la tristesse de penser qu’Odette ne l’avait jamais
gravi. Ah! avec quelle joie au contraire il eût grimpé les étages noirs, mal
odorants et casse-cou de la petite couturière retirée, dans le «cinquième» de
laquelle il aurait été si heureux de payer plus cher qu’une avant-scène
hebdomadaire à l’Opéra le droit de passer la soirée quand Odette y venait et
même les autres jours pour pouvoir parler d’elle, vivre avec les gens qu’elle
avait l’habitude de voir quand il n’était pas là et qui à cause de cela lui
paraissaient recéler, de la vie de sa maîtresse, quelque chose de plus réel, de
plus inaccessible et de plus mystérieux. Tandis que dans cet escalier
pestilentiel et désiré de l’ancienne couturière, comme il n’y en avait pas un
second pour le service, on voyait le soir devant chaque porte une boîte au lait
vide et sale préparée sur le paillasson, dans l’escalier magnifique et dédaigné
que Swann montait à ce moment, d’un côté et de l’autre, à des hauteurs
différentes, devant chaque anfractuosité que faisait dans le mur la fenêtre de
la loge, ou la porte d’un appartement, représentant le service intérieur qu’ils
dirigeaient et en faisant hommage aux invités, un concierge, un majordome, un
argentier (braves gens qui vivaient le reste de la semaine un peu indépendants
dans leur domaine, y dînaient chez eux comme de petits boutiquiers et seraient
peut-être demain au service bourgeois d’un médecin ou d’un industriel) attentifs
à ne pas manquer aux recommandations qu’on leur avait faites avant de leur
laisser endosser la livrée éclatante qu’ils ne revêtaient qu’à de rares
intervalles et dans laquelle ils ne se sentaient pas très à leur aise, se
tenaient sous l’arcature de leur portail avec un éclat pompeux tempéré de
bonhomie populaire, comme des saints dans leur niche; et un énorme suisse,
habillé comme à l’église, frappait les dalles de sa canne au passage de chaque
arrivant. Parvenu en haut de l’escalier le long duquel l’avait suivi un
domestique à face blême, avec une petite queue de cheveux, noués d’un catogan,
derrière la tête, comme un sacristain de Goya ou un tabellion du répertoire,
Swann passa devant un bureau où des valets, assis comme des notaires devant de
grands registres, se levèrent et inscrivirent son nom. Il traversa alors un
petit vestibule qui — tel que certaines pièces aménagées par leur propriétaire
pour servir de cadre à une seule œuvre d’art, dont elles tirent leur nom, et
d’une nudité voulue, ne contiennent rien d’autre — exhibait à son entrée, comme
quelque précieuse effigie de Benvenuto Cellini représentant un homme de guet, un
jeune valet de pied, le corps légèrement fléchi en avant, dressant sur son
hausse-col rouge une figure plus rouge encore d’où s’échappaient des torrents de
feu, de timidité et de zèle, et qui, perçant les tapisseries d’Aubusson tendues
devant le salon où on écoutait la musique, de son regard impétueux, vigilant,
éperdu, avait l’air, avec une impassibilité militaire ou une foi surnaturelle —
allégorie de l’alarme, incarnation de l’attente, commémoration du branle-bas —
d’épier, ange ou vigie, d’une tour de donjon ou de cathédrale, l’apparition de
l’ennemi ou l’heure du Jugement. Il ne restait plus à Swann qu’à pénétrer dans
la salle du concert dont un huissier chargé de chaînes lui ouvrit les portes, en
s’inclinant, comme il lui aurait remis les clefs d’une ville. Mais il pensait à
la maison où il aurait pu se trouver en ce moment même, si Odette l’avait
permis, et le souvenir entrevu d’une boîte au lait vide sur un paillasson lui
serra le cœur.

Swann retrouva rapidement le sentiment de la laideur masculine, quand, au delà
de la tenture de tapisserie, au spectacle des domestiques succéda celui des
invités. Mais cette laideur même de visages qu’il connaissait pourtant si bien,
lui semblait neuve depuis que leurs traits — au lieu d’être pour lui des signes
pratiquement utilisables à l’identification de telle personne qui lui avait
représenté jusque-là un faisceau de plaisirs à poursuivre, d’ennuis à éviter, ou
de politesses à rendre — reposaient, coordonnés seulement par des rapports
esthétiques, dans l’autonomie de leurs lignes. Et en ces hommes, au milieu
desquels Swann se trouva enserré, il n’était pas jusqu’aux monocles que beaucoup
portaient (et qui, autrefois, auraient tout au plus permis à Swann de dire
qu’ils portaient un monocle), qui, déliés maintenant de signifier une habitude,
la même pour tous, ne lui apparussent chacun avec une sorte d’individualité.
Peut-être parce qu’il ne regarda le général de Froberville et le marquis de
Bréauté qui causaient dans l’entrée que comme deux personnages dans un tableau,
alors qu’ils avaient été longtemps pour lui les amis utiles qui l’avaient
présenté au Jockey et assisté dans des duels, le monocle du général, resté entre
ses paupières comme un éclat d’obus dans sa figure vulgaire, balafrée et
triomphale, au milieu du front qu’il éborgnait comme l’œil unique du cyclope,
apparut à Swann comme une blessure monstrueuse qu’il pouvait être glorieux
d’avoir reçue, mais qu’il était indécent d’exhiber; tandis que celui que M. de
Bréauté ajoutait, en signe de festivité, aux gants gris perle, au «gibus», à la
cravate blanche et substituait au binocle familier (comme faisait Swann
lui-même) pour aller dans le monde, portait collé à son revers, comme une
préparation d’histoire naturelle sous un microscope, un regard infinitésimal et
grouillant d’amabilité, qui ne cessait de sourire à la hauteur des plafonds, à
la beauté des fêtes, à l’intérêt des programmes et à la qualité des
rafraîchissements.

— Tiens, vous voilà, mais il y a des éternités qu’on ne vous a vu, dit à Swann
le général qui, remarquant ses traits tirés et en concluant que c’était
peut-être une maladie grave qui l’éloignait du monde, ajouta: «Vous avez bonne
mine, vous savez!» pendant que M. de Bréauté demandait:

—«Comment, vous, mon cher, qu’est-ce que vous pouvez bien faire ici?» à un
romancier mondain qui venait d’installer au coin de son œil un monocle, son seul
organe d’investigation psychologique et d’impitoyable analyse, et répondit d’un
air important et mystérieux, en roulant l’r:

—«J’observe.»

Le monocle du marquis de Forestelle était minuscule, n’avait aucune bordure et
obligeant à une crispation incessante et douloureuse l’œil où il s’incrustait
comme un cartilage superflu dont la présence est inexplicable et la matière
recherchée, il donnait au visage du marquis une délicatesse mélancolique, et le
faisait juger par les femmes comme capable de grands chagrins d’amour. Mais
celui de M. de Saint-Candé, entouré d’un gigantesque anneau, comme Saturne,
était le centre de gravité d’une figure qui s’ordonnait à tout moment par
rapport à lui, dont le nez frémissant et rouge et la bouche lippue et
sarcastique tâchaient par leurs grimaces d’être à la hauteur des feux roulants
d’esprit dont étincelait le disque de verre, et se voyait préférer aux plus
beaux regards du monde par des jeunes femmes snobs et dépravées qu’il faisait
rêver de charmes artificiels et d’un raffinement de volupté; et cependant,
derrière le sien, M. de Palancy qui avec sa grosse tête de carpe aux yeux ronds,
se déplaçait lentement au milieu des fêtes, en desserrant d’instant en instant
ses mandibules comme pour chercher son orientation, avait l’air de transporter
seulement avec lui un fragment accidentel, et peut-être purement symbolique, du
vitrage de son aquarium, partie destinée à figurer le tout qui rappela à Swann,
grand admirateur des Vices et des Vertus de Giotto à Padoue, cet Injuste à côté
duquel un rameau feuillu évoque les forêts où se cache son repaire.

Swann s’était avancé, sur l’insistance de Mme de Saint-Euverte et pour entendre
un air d’Orphée qu’exécutait un flûtiste, s’était mis dans un coin où il avait
malheureusement comme seule perspective deux dames déjà mûres assises l’une à
côté de l’autre, la marquise de Cambremer et la vicomtesse de Franquetot,
lesquelles, parce qu’elles étaient cousines, passaient leur temps dans les
soirées, portant leurs sacs et suivies de leurs filles, à se chercher comme dans
une gare et n’étaient tranquilles que quand elles avaient marqué, par leur
éventail ou leur mouchoir, deux places voisines: Mme de Cambremer, comme elle
avait très peu de relations, étant d’autant plus heureuse d’avoir une compagne,
Mme de Franquetot, qui était au contraire très lancée, trouvait quelque chose
d’élégant, d’original, à montrer à toutes ses belles connaissances qu’elle leur
préférait une dame obscure avec qui elle avait en commun des souvenirs de
jeunesse. Plein d’une mélancolique ironie, Swann les regardait écouter
l’intermède de piano («Saint François parlant aux oiseaux», de Liszt) qui avait
succédé à l’air de flûte, et suivre le jeu vertigineux du virtuose. Mme de
Franquetot anxieusement, les yeux éperdus comme si les touches sur lesquelles il
courait avec agilité avaient été une suite de trapèzes d’où il pouvait tomber
d’une hauteur de quatre-vingts mètres, et non sans lancer à sa voisine des
regards d’étonnement, de dénégation qui signifiaient: «Ce n’est pas croyable, je
n’aurais jamais pensé qu’un homme pût faire cela», Mme de Cambremer, en femme
qui a reçu une forte éducation musicale, battant la mesure avec sa tête
transformée en balancier de métronome dont l’amplitude et la rapidité
d’oscillations d’une épaule à l’autre étaient devenues telles (avec cette espèce
d’égarement et d’abandon du regard qu’ont les douleurs qui ne se connaissent
plus ni ne cherchent à se maîtriser et disent: «Que voulez-vous!») qu’à tout
moment elle accrochait avec ses solitaires les pattes de son corsage et était
obligée de redresser les raisins noirs qu’elle avait dans les cheveux, sans
cesser pour cela d’accélérer le mouvement. De l’autre côté de Mme de Franquetot,
mais un peu en avant, était la marquise de Gallardon, occupée à sa pensée
favorite, l’alliance qu’elle avait avec les Guermantes et d’où elle tirait pour
le monde et pour elle-même beaucoup de gloire avec quelque honte, les plus
brillants d’entre eux la tenant un peu à l’écart, peut-être parce qu’elle était
ennuyeuse, ou parce qu’elle était méchante, ou parce qu’elle était d’une branche
inférieure, ou peut-être sans aucune raison. Quand elle se trouvait auprès de
quelqu’un qu’elle ne connaissait pas, comme en ce moment auprès de Mme de
Franquetot, elle souffrait que la conscience qu’elle avait de sa parenté avec
les Guermantes ne pût se manifester extérieurement en caractères visibles comme
ceux qui, dans les mosaïques des églises byzantines, placés les uns au-dessous
des autres, inscrivent en une colonne verticale, à côté d’un Saint Personnage
les mots qu’il est censé prononcer. Elle songeait en ce moment qu’elle n’avait
jamais reçu une invitation ni une visite de sa jeune cousine la princesse des
Laumes, depuis six ans que celle-ci était mariée. Cette pensée la remplissait de
colère, mais aussi de fierté; car à force de dire aux personnes qui s’étonnaient
de ne pas la voir chez Mme des Laumes, que c’est parce qu’elle aurait été
exposée à y rencontrer la princesse Mathilde — ce que sa famille
ultra-légitimiste ne lui aurait jamais pardonné, elle avait fini par croire que
c’était en effet la raison pour laquelle elle n’allait pas chez sa jeune
cousine. Elle se rappelait pourtant qu’elle avait demandé plusieurs fois à Mme
des Laumes comment elle pourrait faire pour la rencontrer, mais ne se le
rappelait que confusément et d’ailleurs neutralisait et au delà ce souvenir un
peu humiliant en murmurant: «Ce n’est tout de même pas à moi à faire les
premiers pas, j’ai vingt ans de plus qu’elle.» Grâce à la vertu de ces paroles
intérieures, elle rejetait fièrement en arrière ses épaules détachées de son
buste et sur lesquelles sa tête posée presque horizontalement faisait penser à
la tête «rapportée» d’un orgueilleux faisan qu’on sert sur une table avec toutes
ses plumes. Ce n’est pas qu’elle ne fût par nature courtaude, hommasse et
boulotte; mais les camouflets l’avaient redressée comme ces arbres qui, nés dans
une mauvaise position au bord d’un précipice, sont forcés de croître en arrière
pour garder leur équilibre. Obligée, pour se consoler de ne pas être tout à fait
l’égale des autres Guermantes, de se dire sans cesse que c’était par
intransigeance de principes et fierté qu’elle les voyait peu, cette pensée avait
fini par modeler son corps et par lui enfanter une sorte de prestance qui
passait aux yeux des bourgeoises pour un signe de race et troublait quelquefois
d’un désir fugitif le regard fatigué des hommes de cercle. Si on avait fait
subir à la conversation de Mme de Gallardon ces analyses qui en relevant la
fréquence plus ou moins grande de chaque terme permettent de découvrir la clef
d’un langage chiffré, on se fût rendu compte qu’aucune expression, même la plus
usuelle, n’y revenait aussi souvent que «chez mes cousins de Guermantes», «chez
ma tante de Guermantes», «la santé d’Elzéar de Guermantes», «la baignoire de ma
cousine de Guermantes». Quand on lui parlait d’un personnage illustre, elle
répondait que, sans le connaître personnellement, elle l’avait rencontré mille
fois chez sa tante de Guermantes, mais elle répondait cela d’un ton si glacial
et d’une voix si sourde qu’il était clair que si elle ne le connaissait pas
personnellement c’était en vertu de tous les principes indéracinables et entêtés
auxquels ses épaules touchaient en arrière, comme à ces échelles sur lesquelles
les professeurs de gymnastique vous font étendre pour vous développer le thorax.

Or, la princesse des Laumes qu’on ne se serait pas attendu à voir chez Mme de
Saint-Euverte, venait précisément d’arriver. Pour montrer qu’elle ne cherchait
pas à faire sentir dans un salon où elle ne venait que par condescendance, la
supériorité de son rang, elle était entrée en effaçant les épaules là même où il
n’y avait aucune foule à fendre et personne à laisser passer, restant exprès
dans le fond, de l’air d’y être à sa place, comme un roi qui fait la queue à la
porte d’un théâtre tant que les autorités n’ont pas été prévenues qu’il est là;
et, bornant simplement son regard — pour ne pas avoir l’air de signaler sa
présence et de réclamer des égards —à la considération d’un dessin du tapis ou
de sa propre jupe, elle se tenait debout à l’endroit qui lui avait paru le plus
modeste (et d’où elle savait bien qu’une exclamation ravie de Mme de
Saint-Euverte allait la tirer dès que celle-ci l’aurait aperçue), à côté de Mme
de Cambremer qui lui était inconnue. Elle observait la mimique de sa voisine
mélomane, mais ne l’imitait pas. Ce n’est pas que, pour une fois qu’elle venait
passer cinq minutes chez Mme de Saint-Euverte, la princesse des Laumes n’eût
souhaité, pour que la politesse qu’elle lui faisait comptât double, se montrer
le plus aimable possible. Mais par nature, elle avait horreur de ce qu’elle
appelait «les exagérations» et tenait à montrer qu’elle «n’avait pas à» se
livrer à des manifestations qui n’allaient pas avec le «genre» de la coterie où
elle vivait, mais qui pourtant d’autre part ne laissaient pas de
l’impressionner, à la faveur de cet esprit d’imitation voisin de la timidité que
développe chez les gens les plus sûrs d’eux-mêmes l’ambiance d’un milieu
nouveau, fût-il inférieur. Elle commençait à se demander si cette gesticulation
n’était pas rendue nécessaire par le morceau qu’on jouait et qui ne rentrait
peut-être pas dans le cadre de la musique qu’elle avait entendue jusqu’à ce
jour, si s’abstenir n’était pas faire preuve d’incompréhension à l’égard de
l’œuvre et d’inconvenance vis-à-vis de la maîtresse de la maison: de sorte que
pour exprimer par une «cote mal taillée» ses sentiments contradictoires, tantôt
elle se contentait de remonter la bride de ses épaulettes ou d’assurer dans ses
cheveux blonds les petites boules de corail ou d’émail rose, givrées de diamant,
qui lui faisaient une coiffure simple et charmante, en examinant avec une froide
curiosité sa fougueuse voisine, tantôt de son éventail elle battait pendant un
instant la mesure, mais, pour ne pas abdiquer son indépendance, à contretemps.
Le pianiste ayant terminé le morceau de Liszt et ayant commencé un prélude de
Chopin, Mme de Cambremer lança à Mme de Franquetot un sourire attendri de
satisfaction compétente et d’allusion au passé. Elle avait appris dans sa
jeunesse à caresser les phrases, au long col sinueux et démesuré, de Chopin, si
libres, si flexibles, si tactiles, qui commencent par chercher et essayer leur
place en dehors et bien loin de la direction de leur départ, bien loin du point
où on avait pu espérer qu’atteindrait leur attouchement, et qui ne se jouent
dans cet écart de fantaisie que pour revenir plus délibérément — d’un retour
plus prémédité, avec plus de précision, comme sur un cristal qui résonnerait
jusqu’à faire crier — vous frapper au cœur.

Vivant dans une famille provinciale qui avait peu de relations, n’allant guère
au bal, elle s’était grisée dans la solitude de son manoir, à ralentir, à
précipiter la danse de tous ces couples imaginaires, à les égrener comme des
fleurs, à quitter un moment le bal pour entendre le vent souffler dans les
sapins, au bord du lac, et à y voir tout d’un coup s’avancer, plus différent de
tout ce qu’on a jamais rêvé que ne sont les amants de la terre, un mince jeune
homme à la voix un peu chantante, étrangère et fausse, en gants blancs. Mais
aujourd’hui la beauté démodée de cette musique semblait défraîchie. Privée
depuis quelques années de l’estime des connaisseurs, elle avait perdu son
honneur et son charme et ceux mêmes dont le goût est mauvais n’y trouvaient plus
qu’un plaisir inavoué et médiocre. Mme de Cambremer jeta un regard furtif
derrière elle. Elle savait que sa jeune bru (pleine de respect pour sa nouvelle
famille, sauf en ce qui touchait les choses de l’esprit sur lesquelles, sachant
jusqu’à l’harmonie et jusqu’au grec, elle avait des lumières spéciales)
méprisait Chopin et souffrait quand elle en entendait jouer. Mais loin de la
surveillance de cette wagnérienne qui était plus loin avec un groupe de
personnes de son âge, Mme de Cambremer se laissait aller à des impressions
délicieuses. La princesse des Laumes les éprouvait aussi. Sans être par nature
douée pour la musique, elle avait reçu il y a quinze ans les leçons qu’un
professeur de piano du faubourg Saint-Germain, femme de génie qui avait été à la
fin de sa vie réduite à la misère, avait recommencé, à l’âge de soixante-dix
ans, à donner aux filles et aux petites-filles de ses anciennes élèves. Elle
était morte aujourd’hui. Mais sa méthode, son beau son, renaissaient parfois
sous les doigts de ses élèves, même de celles qui étaient devenues pour le reste
des personnes médiocres, avaient abandonné la musique et n’ouvraient presque
plus jamais un piano. Aussi Mme des Laumes put-elle secouer la tête, en pleine
connaissance de cause, avec une appréciation juste de la façon dont le pianiste
jouait ce prélude qu’elle savait par cœur. La fin de la phrase commencée chanta
d’elle-même sur ses lèvres. Et elle murmura «C’est toujours charmant», avec un
double ch au commencement du mot qui était une marque de délicatesse et dont
elle sentait ses lèvres si romanesquement froissées comme une belle fleur,
qu’elle harmonisa instinctivement son regard avec elles en lui donnant à ce
moment-là une sorte de sentimentalité et de vague. Cependant Mme de Gallardon
était en train de se dire qu’il était fâcheux qu’elle n’eût que bien rarement
l’occasion de rencontrer la princesse des Laumes, car elle souhaitait lui donner
une leçon en ne répondant pas à son salut. Elle ne savait pas que sa cousine fût
là. Un mouvement de tête de Mme de Franquetot la lui découvrit. Aussitôt elle se
précipita vers elle en dérangeant tout le monde; mais désireuse de garder un air
hautain et glacial qui rappelât à tous qu’elle ne désirait pas avoir de
relations avec une personne chez qui on pouvait se trouver nez à nez avec la
princesse Mathilde, et au-devant de qui elle n’avait pas à aller car elle
n’était pas «sa contemporaine», elle voulut pourtant compenser cet air de
hauteur et de réserve par quelque propos qui justifiât sa démarche et forçât la
princesse à engager la conversation; aussi une fois arrivée près de sa cousine,
Mme de Gallardon, avec un visage dur, une main tendue comme une carte forcée,
lui dit: «Comment va ton mari?» de la même voix soucieuse que si le prince avait
été gravement malade. La princesse éclatant d’un rire qui lui était particulier
et qui était destiné à la fois à montrer aux autres qu’elle se moquait de
quelqu’un et aussi à se faire paraître plus jolie en concentrant les traits de
son visage autour de sa bouche animée et de son regard brillant, lui répondit:

— Mais le mieux du monde!

Et elle rit encore. Cependant tout en redressant sa taille et refroidissant sa
mine, inquiète encore pourtant de l’état du prince, Mme de Gallardon dit à sa
cousine:

— Oriane (ici Mme des Laumes regarda d’un air étonné et rieur un tiers invisible
vis-à-vis duquel elle semblait tenir à attester qu’elle n’avait jamais autorisé
Mme de Gallardon à l’appeler par son prénom), je tiendrais beaucoup à ce que tu
viennes un moment demain soir chez moi entendre un quintette avec clarinette de
Mozart. Je voudrais avoir ton appréciation.

Elle semblait non pas adresser une invitation, mais demander un service, et
avoir besoin de l’avis de la princesse sur le quintette de Mozart comme si
ç’avait été un plat de la composition d’une nouvelle cuisinière sur les talents
de laquelle il lui eût été précieux de recueillir l’opinion d’un gourmet.

— Mais je connais ce quintette, je peux te dire tout de suite . . . que je
l’aime!

— Tu sais, mon mari n’est pas bien, son foie . . ., cela lui ferait grand
plaisir de te voir, reprit Mme de Gallardon, faisant maintenant à la princesse
une obligation de charité de paraître à sa soirée.

La princesse n’aimait pas à dire aux gens qu’elle ne voulait pas aller chez eux.
Tous les jours elle écrivait son regret d’avoir été privée — par une visite
inopinée de sa belle-mère, par une invitation de son beau-frère, par l’Opéra,
par une partie de campagne — d’une soirée à laquelle elle n’aurait jamais songé
à se rendre. Elle donnait ainsi à beaucoup de gens la joie de croire qu’elle
était de leurs relations, qu’elle eût été volontiers chez eux, qu’elle n’avait
été empêchée de le faire que par les contretemps princiers qu’ils étaient
flattés de voir entrer en concurrence avec leur soirée. Puis, faisant partie de
cette spirituelle coterie des Guermantes où survivait quelque chose de l’esprit
alerte, dépouillé de lieux communs et de sentiments convenus, qui descend de
Mérimée — et a trouvé sa dernière expression dans le théâtre de Meilhac et
Halévy — elle l’adaptait même aux rapports sociaux, le transposait jusque dans
sa politesse qui s’efforçait d’être positive, précise, de se rapprocher de
l’humble vérité. Elle ne développait pas longuement à une maîtresse de maison
l’expression du désir qu’elle avait d’aller à sa soirée; elle trouvait plus
aimable de lui exposer quelques petits faits d’où dépendrait qu’il lui fût ou
non possible de s’y rendre.

— Ecoute, je vais te dire, dit-elle à Mme de Gallardon, il faut demain soir que
j’aille chez une amie qui m’a demandé mon jour depuis longtemps. Si elle nous
emmène au théâtre, il n’y aura pas, avec la meilleure volonté, possibilité que
j’aille chez toi; mais si nous restons chez elle, comme je sais que nous serons
seuls, je pourrai la quitter.

— Tiens, tu as vu ton ami M. Swann?

— Mais non, cet amour de Charles, je ne savais pas qu’il fût là, je vais tâcher
qu’il me voie.

— C’est drôle qu’il aille même chez la mère Saint-Euverte, dit Mme de Gallardon.
Oh! je sais qu’il est intelligent, ajouta-t-elle en voulant dire par là
intrigant, mais cela ne fait rien, un juif chez la sœur et la belle-sœur de deux
archevêques!

— J’avoue à ma honte que je n’en suis pas choquée, dit la princesse des Laumes.

— Je sais qu’il est converti, et même déjà ses parents et ses grands-parents.
Mais on dit que les convertis restent plus attachés à leur religion que les
autres, que c’est une frime, est-ce vrai?

— Je suis sans lumières à ce sujet.

Le pianiste qui avait à jouer deux morceaux de Chopin, après avoir terminé le
prélude avait attaqué aussitôt une polonaise. Mais depuis que Mme de Gallardon
avait signalé à sa cousine la présence de Swann, Chopin ressuscité aurait pu
venir jouer lui-même toutes ses œuvres sans que Mme des Laumes pût y faire
attention. Elle faisait partie d’une de ces deux moitiés de l’humanité chez qui
la curiosité qu’a l’autre moitié pour les êtres qu’elle ne connaît pas est
remplacée par l’intérêt pour les êtres qu’elle connaît. Comme beaucoup de femmes
du faubourg Saint-Germain la présence dans un endroit où elle se trouvait de
quelqu’un de sa coterie, et auquel d’ailleurs elle n’avait rien de particulier à
dire, accaparait exclusivement son attention aux dépens de tout le reste. A
partir de ce moment, dans l’espoir que Swann la remarquerait, la princesse ne
fit plus, comme une souris blanche apprivoisée à qui on tend puis on retire un
morceau de sucre, que tourner sa figure, remplie de mille signes de connivence
dénués de rapports avec le sentiment de la polonaise de Chopin, dans la
direction où était Swann et si celui-ci changeait de place, elle déplaçait
parallèlement son sourire aimanté.

— Oriane, ne te fâche pas, reprit Mme de Gallardon qui ne pouvait jamais
s’empêcher de sacrifier ses plus grandes espérances sociales et d’éblouir un
jour le monde, au plaisir obscur, immédiat et privé, de dire quelque chose de
désagréable, il y a des gens qui prétendent que ce M. Swann, c’est quelqu’un
qu’on ne peut pas recevoir chez soi, est-ce vrai?

— Mais . . . tu dois bien savoir que c’est vrai, répondit la princesse des
Laumes, puisque tu l’as invité cinquante fois et qu’il n’est jamais venu.

Et quittant sa cousine mortifiée, elle éclata de nouveau d’un rire qui
scandalisa les personnes qui écoutaient la musique, mais attira l’attention de
Mme de Saint-Euverte, restée par politesse près du piano et qui aperçut
seulement alors la princesse. Mme de Saint-Euverte était d’autant plus ravie de
voir Mme des Laumes qu’elle la croyait encore à Guermantes en train de soigner
son beau-père malade.

— Mais comment, princesse, vous étiez là?

— Oui, je m’étais mise dans un petit coin, j’ai entendu de belles choses.

— Comment, vous êtes là depuis déjà un long moment!

— Mais oui, un très long moment qui m’a semblé très court, long seulement parce
que je ne vous voyais pas.

Mme de Saint-Euverte voulut donner son fauteuil à la princesse qui répondit:

— Mais pas du tout! Pourquoi? Je suis bien n’importe où!

Et, avisant avec intention, pour mieux manifester sa simplicité de grande dame,
un petit siège sans dossier:

— Tenez, ce pouf, c’est tout ce qu’il me faut. Cela me fera tenir droite. Oh!
mon Dieu, je fais encore du bruit, je vais me faire conspuer.

Cependant le pianiste redoublant de vitesse, l’émotion musicale était à son
comble, un domestique passait des rafraîchissements sur un plateau et faisait
tinter des cuillers et, comme chaque semaine, Mme de Saint-Euverte lui faisait,
sans qu’il la vît, des signes de s’en aller. Une nouvelle mariée, à qui on avait
appris qu’une jeune femme ne doit pas avoir l’air blasé, souriait de plaisir, et
cherchait des yeux la maîtresse de maison pour lui témoigner par son regard sa
reconnaissance d’avoir «pensé à elle» pour un pareil régal. Pourtant, quoique
avec plus de calme que Mme de Franquetot, ce n’est pas sans inquiétude qu’elle
suivait le morceau; mais la sienne avait pour objet, au lieu du pianiste, le
piano sur lequel une bougie tressautant à chaque fortissimo, risquait, sinon de
mettre le feu à l’abat-jour, du moins de faire des taches sur le palissandre. A
la fin elle n’y tint plus et, escaladant les deux marches de l’estrade, sur
laquelle était placé le piano, se précipita pour enlever la bobèche. Mais à
peine ses mains allaient-elles la toucher que sur un dernier accord, le morceau
finit et le pianiste se leva. Néanmoins l’initiative hardie de cette jeune
femme, la courte promiscuité qui en résulta entre elle et l’instrumentiste,
produisirent une impression généralement favorable.

— Vous avez remarqué ce qu’a fait cette personne, princesse, dit le général de
Froberville à la princesse des Laumes qu’il était venu saluer et que Mme de
Saint-Euverte quitta un instant. C’est curieux. Est-ce donc une artiste?

— Non, c’est une petite Mme de Cambremer, répondit étourdiment la princesse et
elle ajouta vivement: Je vous répète ce que j’ai entendu dire, je n’ai aucune
espèce de notion de qui c’est, on a dit derrière moi que c’étaient des voisins
de campagne de Mme de Saint-Euverte, mais je ne crois pas que personne les
connaisse. Ça doit être des «gens de la campagne»! Du reste, je ne sais pas si
vous êtes très répandu dans la brillante société qui se trouve ici, mais je n’ai
pas idée du nom de toutes ces étonnantes personnes. A quoi pensez-vous qu’ils
passent leur vie en dehors des soirées de Mme de Saint-Euverte? Elle a dû les
faire venir avec les musiciens, les chaises et les rafraîchissements. Avouez que
ces «invités de chez Belloir» sont magnifiques. Est-ce que vraiment elle a le
courage de louer ces figurants toutes les semaines. Ce n’est pas possible!

— Ah! Mais Cambremer, c’est un nom authentique et ancien, dit le général.

— Je ne vois aucun mal à ce que ce soit ancien, répondit sèchement la princesse,
mais en tous cas ce n’est-ce pas euphonique, ajouta-t-elle en détachant le mot
euphonique comme s’il était entre guillemets, petite affectation de dépit qui
était particulière à la coterie Guermantes.

— Vous trouvez? Elle est jolie à croquer, dit le général qui ne perdait pas Mme
de Cambremer de vue. Ce n’est pas votre avis, princesse?

— Elle se met trop en avant, je trouve que chez une si jeune femme, ce n’est pas
agréable, car je ne crois pas qu’elle soit ma contemporaine, répondit Mme des
Laumes (cette expression étant commune aux Gallardon et aux Guermantes).

Mais la princesse voyant que M. de Froberville continuait à regarder Mme de
Cambremer, ajouta moitié par méchanceté pour celle-ci, moitié par amabilité pour
le général: «Pas agréable . . . pour son mari! Je regrette de ne pas la
connaître puisqu’elle vous tient à cœur, je vous aurais présenté,» dit la
princesse qui probablement n’en aurait rien fait si elle avait connu la jeune
femme. «Je vais être obligée de vous dire bonsoir, parce que c’est la fête d’une
amie à qui je dois aller la souhaiter, dit-elle d’un ton modeste et vrai,
réduisant la réunion mondaine à laquelle elle se rendait à la simplicité d’une
cérémonie ennuyeuse mais où il était obligatoire et touchant d’aller. D’ailleurs
je dois y retrouver Basin qui, pendant que j’étais ici, est allé voir ses amis
que vous connaissez, je crois, qui ont un nom de pont, les Iéna.»

—«Ç’a été d’abord un nom de victoire, princesse, dit le général. Qu’est-ce que
vous voulez, pour un vieux briscard comme moi, ajouta-t-il en ôtant son monocle
pour l’essuyer, comme il aurait changé un pansement, tandis que la princesse
détournait instinctivement les yeux, cette noblesse d’Empire, c’est autre chose
bien entendu, mais enfin, pour ce que c’est, c’est très beau dans son genre, ce
sont des gens qui en somme se sont battus en héros.»

— Mais je suis pleine de respect pour les héros, dit la princesse, sur un ton
légèrement ironique: si je ne vais pas avec Basin chez cette princesse d’Iéna,
ce n’est pas du tout pour ça, c’est tout simplement parce que je ne les connais
pas. Basin les connaît, les chérit. Oh! non, ce n’est pas ce que vous pouvez
penser, ce n’est pas un flirt, je n’ai pas à m’y opposer! Du reste, pour ce que
cela sert quand je veux m’y opposer! ajouta-t-elle d’une voix mélancolique, car
tout le monde savait que dès le lendemain du jour où le prince des Laumes avait
épousé sa ravissante cousine, il n’avait pas cessé de la tromper. Mais enfin ce
n’est pas le cas, ce sont des gens qu’il a connus autrefois, il en fait ses
choux gras, je trouve cela très bien. D’abord je vous dirai que rien que ce
qu’il m’a dit de leur maison . . . Pensez que tous leurs meubles sont «Empire!»

— Mais, princesse, naturellement, c’est parce que c’est le mobilier de leurs
grands-parents.

— Mais je ne vous dis pas, mais ça n’est pas moins laid pour ça. Je comprends
très bien qu’on ne puisse pas avoir de jolies choses, mais au moins qu’on n’ait
pas de choses ridicules. Qu’est-ce que vous voulez? je ne connais rien de plus
pompier, de plus bourgeois que cet horrible style avec ces commodes qui ont des
têtes de cygnes comme des baignoires.

— Mais je crois même qu’ils ont de belles choses, ils doivent avoir la fameuse
table de mosaïque sur laquelle a été signé le traité de . . .

— Ah! Mais qu’ils aient des choses intéressantes au point de vue de l’histoire,
je ne vous dis pas. Mais ça ne peut pas être beau . . . puisque c’est horrible!
Moi j’ai aussi des choses comme ça que Basin a héritées des Montesquiou.
Seulement elles sont dans les greniers de Guermantes où personne ne les voit.
Enfin, du reste, ce n’est pas la question, je me précipiterais chez eux avec
Basin, j’irais les voir même au milieu de leurs sphinx et de leur cuivre si je
les connaissais, mais . . . je ne les connais pas! Moi, on m’a toujours dit
quand j’étais petite que ce n’était pas poli d’aller chez les gens qu’on ne
connaissait pas, dit-elle en prenant un ton puéril. Alors, je fais ce qu’on m’a
appris. Voyez-vous ces braves gens s’ils voyaient entrer une personne qu’ils ne
connaissent pas? Ils me recevraient peut-être très mal! dit la princesse.

Et par coquetterie elle embellit le sourire que cette supposition lui arrachait,
en donnant à son regard fixé sur le général une expression rêveuse et douce.

—«Ah! princesse, vous savez bien qu’ils ne se tiendraient pas de joie . . . »

—«Mais non, pourquoi?» lui demanda-t-elle avec une extrême vivacité, soit pour
ne pas avoir l’air de savoir que c’est parce qu’elle était une des plus grandes
dames de France, soit pour avoir le plaisir de l’entendre dire au général.
«Pourquoi? Qu’en savez-vous? Cela leur serait peut-être tout ce qu’il y a de
plus désagréable. Moi je ne sais pas, mais si j’en juge par moi, cela m’ennuie
déjà tant de voir les personnes que je connais, je crois que s’il fallait voir
des gens que je ne connais pas, «même héroïques», je deviendrais folle.
D’ailleurs, voyons, sauf lorsqu’il s’agit de vieux amis comme vous qu’on connaît
sans cela, je ne sais pas si l’héroïsme serait d’un format très portatif dans le
monde. Ça m’ennuie déjà souvent de donner des dîners, mais s’il fallait offrir
le bras à Spartacus pour aller à table . . . Non vraiment, ce ne serait jamais à
Vercingétorix que je ferais signe comme quatorzième. Je sens que je le
réserverais pour les grandes soirées. Et comme je n’en donne pas . . . »

— Ah! princesse, vous n’êtes pas Guermantes pour des prunes. Le possédez-vous
assez, l’esprit des Guermantes!

— Mais on dit toujours l’esprit des Guermantes, je n’ai jamais pu comprendre
pourquoi. Vous en connaissez donc d’autres qui en aient, ajouta-t-elle dans un
éclat de rire écumant et joyeux, les traits de son visage concentrés, accouplés
dans le réseau de son animation, les yeux étincelants, enflammés d’un
ensoleillement radieux de gaîté que seuls avaient le pouvoir de faire rayonner
ainsi les propos, fussent-ils tenus par la princesse elle-même, qui étaient une
louange de son esprit ou de sa beauté. Tenez, voilà Swann qui a l’air de saluer
votre Cambremer; là . . . il est à côté de la mère Saint-Euverte, vous ne voyez
pas! Demandez-lui de vous présenter. Mais dépêchez-vous, il cherche à s’en
aller!

— Avez-vous remarqué quelle affreuse mine il a? dit le général.

— Mon petit Charles! Ah! enfin il vient, je commençais à supposer qu’il ne
voulait pas me voir!

Swann aimait beaucoup la princesse des Laumes, puis sa vue lui rappelait
Guermantes, terre voisine de Combray, tout ce pays qu’il aimait tant et où il ne
retournait plus pour ne pas s’éloigner d’Odette. Usant des formes mi-artistes,
mi-galantes, par lesquelles il savait plaire à la princesse et qu’il retrouvait
tout naturellement quand il se retrempait un instant dans son ancien milieu — et
voulant d’autre part pour lui-même exprimer la nostalgie qu’il avait de la
campagne:

— Ah! dit-il à la cantonade, pour être entendu à la fois de Mme de Saint-Euverte
à qui il parlait et de Mme des Laumes pour qui il parlait, voici la charmante
princesse! Voyez, elle est venue tout exprès de Guermantes pour entendre le
Saint-François d’Assise de Liszt et elle n’a eu le temps, comme une jolie
mésange, que d’aller piquer pour les mettre sur sa tête quelques petits fruits
de prunier des oiseaux et d’aubépine; il y a même encore de petites gouttes de
rosée, un peu de la gelée blanche qui doit faire gémir la duchesse. C’est très
joli, ma chère princesse.

— Comment la princesse est venue exprès de Guermantes? Mais c’est trop! Je ne
savais pas, je suis confuse, s’écrie naïvement Mme de Saint-Euverte qui était
peu habituée au tour d’esprit de Swann. Et examinant la coiffure de la
princesse: Mais c’est vrai, cela imite . . . comment dirais-je, pas les
châtaignes, non, oh! c’est une idée ravissante, mais comment la princesse
pouvait-elle connaître mon programme. Les musiciens ne me l’ont même pas
communiqué à moi.

Swann, habitué quand il était auprès d’une femme avec qui il avait gardé des
habitudes galantes de langage, de dire des choses délicates que beaucoup de gens
du monde ne comprenaient pas, ne daigna pas expliquer à Mme de Saint-Euverte
qu’il n’avait parlé que par métaphore. Quant à la princesse, elle se mit à rire
aux éclats, parce que l’esprit de Swann était extrêmement apprécié dans sa
coterie et aussi parce qu’elle ne pouvait entendre un compliment s’adressant à
elle sans lui trouver les grâces les plus fines et une irrésistible drôlerie.

— Hé bien! je suis ravie, Charles, si mes petits fruits d’aubépine vous
plaisent. Pourquoi est-ce que vous saluez cette Cambremer, est-ce que vous êtes
aussi son voisin de campagne?

Mme de Saint-Euverte voyant que la princesse avait l’air content de causer avec
Swann s’était éloignée.

— Mais vous l’êtes vous-même, princesse.

— Moi, mais ils ont donc des campagnes partout, ces gens! Mais comme j’aimerais
être à leur place!

— Ce ne sont pas les Cambremer, c’étaient ses parents à elle; elle est une
demoiselle Legrandin qui venait à Combray. Je ne sais pas si vous savez que vous
êtes la comtesse de Combray et que le chapitre vous doit une redevance.

— Je ne sais pas ce que me doit le chapitre mais je sais que je suis tapée de
cent francs tous les ans par le curé, ce dont je me passerais. Enfin ces
Cambremer ont un nom bien étonnant. Il finit juste à temps, mais il finit mal!
dit-elle en riant.

— Il ne commence pas mieux, répondit Swann.

— En effet cette double abréviation! . . .

— C’est quelqu’un de très en colère et de très convenable qui n’a pas osé aller
jusqu’au bout du premier mot.

— Mais puisqu’il ne devait pas pouvoir s’empêcher de commencer le second, il
aurait mieux fait d’achever le premier pour en finir une bonne fois. Nous sommes
en train de faire des plaisanteries d’un goût charmant, mon petit Charles, mais
comme c’est ennuyeux de ne plus vous voir, ajouta-t-elle d’un ton câlin, j’aime
tant causer avec vous. Pensez que je n’aurais même pas pu faire comprendre à cet
idiot de Froberville que le nom de Cambremer était étonnant. Avouez que la vie
est une chose affreuse. Il n’y a que quand je vous vois que je cesse de
m’ennuyer.

Et sans doute cela n’était pas vrai. Mais Swann et la princesse avaient une même
manière de juger les petites choses qui avait pour effet —à moins que ce ne fût
pour cause — une grande analogie dans la façon de s’exprimer et jusque dans la
prononciation. Cette ressemblance ne frappait pas parce que rien n’était plus
différent que leurs deux voix. Mais si on parvenait par la pensée à ôter aux
propos de Swann la sonorité qui les enveloppait, les moustaches d’entre
lesquelles ils sortaient, on se rendait compte que c’étaient les mêmes phrases,
les mêmes inflexions, le tour de la coterie Guermantes. Pour les choses
importantes, Swann et la princesse n’avaient les mêmes idées sur rien. Mais
depuis que Swann était si triste, ressentant toujours cette espèce de frisson
qui précède le moment où l’on va pleurer, il avait le même besoin de parler du
chagrin qu’un assassin a de parler de son crime. En entendant la princesse lui
dire que la vie était une chose affreuse, il éprouva la même douceur que si elle
lui avait parlé d’Odette.

— Oh! oui, la vie est une chose affreuse. Il faut que nous nous voyions, ma
chère amie. Ce qu’il y a de gentil avec vous, c’est que vous n’êtes pas gaie. On
pourrait passer une soirée ensemble.

— Mais je crois bien, pourquoi ne viendriez-vous pas à Guermantes, ma belle-mère
serait folle de joie. Cela passe pour très laid, mais je vous dirai que ce pays
ne me déplaît pas, j’ai horreur des pays «pittoresques».

— Je crois bien, c’est admirable, répondit Swann, c’est presque trop beau, trop
vivant pour moi, en ce moment; c’est un pays pour être heureux. C’est peut-être
parce que j’y ai vécu, mais les choses m’y parlent tellement. Dès qu’il se lève
un souffle d’air, que les blés commencent à remuer, il me semble qu’il y a
quelqu’un qui va arriver, que je vais recevoir une nouvelle; et ces petites
maisons au bord de l’eau . . . je serais bien malheureux!

— Oh! mon petit Charles, prenez garde, voilà l’affreuse Rampillon qui m’a vue,
cachez-moi, rappelez-moi donc ce qui lui est arrivé, je confonds, elle a marié
sa fille ou son amant, je ne sais plus; peut-être les deux . . . et ensemble! .
. . Ah! non, je me rappelle, elle a été répudiée par son prince . . . ayez l’air
de me parler pour que cette Bérénice ne vienne pas m’inviter à dîner. Du reste,
je me sauve. Ecoutez, mon petit Charles, pour une fois que je vous vois, vous ne
voulez pas vous laisser enlever et que je vous emmène chez la princesse de Parme
qui serait tellement contente, et Basin aussi qui doit m’y rejoindre. Si on
n’avait pas de vos nouvelles par Mémé . . . Pensez que je ne vous vois plus
jamais!

Swann refusa; ayant prévenu M. de Charlus qu’en quittant de chez Mme de
Saint-Euverte il rentrerait directement chez lui, il ne se souciait pas en
allant chez la princesse de Parme de risquer de manquer un mot qu’il avait tout
le temps espéré se voir remettre par un domestique pendant la soirée, et que
peut-être il allait trouver chez son concierge. «Ce pauvre Swann, dit ce soir-là
Mme des Laumes à son mari, il est toujours gentil, mais il a l’air bien
malheureux. Vous le verrez, car il a promis de venir dîner un de ces jours. Je
trouve ridicule au fond qu’un homme de son intelligence souffre pour une
personne de ce genre et qui n’est même pas intéressante, car on la dit idiote»,
ajouta-t-elle avec la sagesse des gens non amoureux qui trouvent qu’un homme
d’esprit ne devrait être malheureux que pour une personne qui en valût la peine;
c’est à peu près comme s’étonner qu’on daigne souffrir du choléra par le fait
d’un être aussi petit que le bacille virgule.

Swann voulait partir, mais au moment où il allait enfin s’échapper, le général
de Froberville lui demanda à connaître Mme de Cambremer et il fut obligé de
rentrer avec lui dans le salon pour la chercher.

— Dites donc, Swann, j’aimerais mieux être le mari de cette femme-là que d’être
massacré par les sauvages, qu’en dites-vous?

Ces mots «massacré par les sauvages» percèrent douloureusement le cœur de Swann;
aussitôt il éprouva le besoin de continuer la conversation avec le général:

—«Ah! lui dit-il, il y a eu de bien belles vies qui ont fini de cette façon . .
. Ainsi vous savez . . . ce navigateur dont Dumont d’Urville ramena les cendres,
La Pérouse . . . (et Swann était déjà heureux comme s’il avait parlé d’Odette.)
«C’est un beau caractère et qui m’intéresse beaucoup que celui de La Pérouse,
ajouta-t-il d’un air mélancolique.»

— Ah! parfaitement, La Pérouse, dit le général. C’est un nom connu. Il a sa rue.
— Vous connaissez quelqu’un rue La Pérouse? demanda Swann d’un air agité.

— Je ne connais que Mme de Chanlivault, la sœur de ce brave Chaussepierre. Elle
nous a donné une jolie soirée de comédie l’autre jour. C’est un salon qui sera
un jour très élégant, vous verrez!

— Ah! elle demeure rue La Pérouse. C’est sympathique, c’est une jolie rue, si
triste.

— Mais non; c’est que vous n’y êtes pas allé depuis quelque temps; ce n’est plus
triste, cela commence à se construire, tout ce quartier-là.

Quand enfin Swann présenta M. de Froberville à la jeune Mme de Cambremer, comme
c’était la première fois qu’elle entendait le nom du général, elle esquissa le
sourire de joie et de surprise qu’elle aurait eu si on n’en avait jamais
prononcé devant elle d’autre que celui-là, car ne connaissant pas les amis de sa
nouvelle famille, à chaque personne qu’on lui amenait, elle croyait que c’était
l’un d’eux, et pensant qu’elle faisait preuve de tact en ayant l’air d’en avoir
tant entendu parler depuis qu’elle était mariée, elle tendait la main d’un air
hésitant destiné à prouver la réserve apprise qu’elle avait à vaincre et la
sympathie spontanée qui réussissait à en triompher. Aussi ses beaux-parents,
qu’elle croyait encore les gens les plus brillants de France, déclaraient-ils
qu’elle était un ange; d’autant plus qu’ils préféraient paraître, en la faisant
épouser à leur fils, avoir cédé à l’attrait plutôt de ses qualités que de sa
grande fortune.

— On voit que vous êtes musicienne dans l’âme, madame, lui dit le général en
faisant inconsciemment allusion à l’incident de la bobèche.

Mais le concert recommença et Swann comprit qu’il ne pourrait pas s’en aller
avant la fin de ce nouveau numéro du programme. Il souffrait de rester enfermé
au milieu de ces gens dont la bêtise et les ridicules le frappaient d’autant
plus douloureusement qu’ignorant son amour, incapables, s’ils l’avaient connu,
de s’y intéresser et de faire autre chose que d’en sourire comme d’un
enfantillage ou de le déplorer comme une folie, ils le lui faisaient apparaître
sous l’aspect d’un état subjectif qui n’existait que pour lui, dont rien
d’extérieur ne lui affirmait la réalité; il souffrait surtout, et au point que
même le son des instruments lui donnait envie de crier, de prolonger son exil
dans ce lieu où Odette ne viendrait jamais, où personne, où rien ne la
connaissait, d’où elle était entièrement absente.

Mais tout à coup ce fut comme si elle était entrée, et cette apparition lui fut
une si déchirante souffrance qu’il dut porter la main à son cœur. C’est que le
violon était monté à des notes hautes où il restait comme pour une attente, une
attente qui se prolongeait sans qu’il cessât de les tenir, dans l’exaltation où
il était d’apercevoir déjà l’objet de son attente qui s’approchait, et avec un
effort désespéré pour tâcher de durer jusqu’à son arrivée, de l’accueillir avant
d’expirer, de lui maintenir encore un moment de toutes ses dernières forces le
chemin ouvert pour qu’il pût passer, comme on soutient une porte qui sans cela
retomberait. Et avant que Swann eût eu le temps de comprendre, et de se dire:
«C’est la petite phrase de la sonate de Vinteuil, n’écoutons pas!» tous ses
souvenirs du temps où Odette était éprise de lui, et qu’il avait réussi jusqu’à
ce jour à maintenir invisibles dans les profondeurs de son être, trompés par ce
brusque rayon du temps d’amour qu’ils crurent revenu, s’étaient réveillés, et à
tire d’aile, étaient remontés lui chanter éperdument, sans pitié pour son
infortune présente, les refrains oubliés du bonheur.

Au lieu des expressions abstraites «temps où j’étais heureux», «temps où j’étais
aimé», qu’il avait souvent prononcées jusque-là et sans trop souffrir, car son
intelligence n’y avait enfermé du passé que de prétendus extraits qui n’en
conservaient rien, il retrouva tout ce qui de ce bonheur perdu avait fixé à
jamais la spécifique et volatile essence; il revit tout, les pétales neigeux et
frisés du chrysanthème qu’elle lui avait jeté dans sa voiture, qu’il avait gardé
contre ses lèvres — l’adresse en relief de la «Maison Dorée» sur la lettre où il
avait lu: «Ma main tremble si fort en vous écrivant»— le rapprochement de ses
sourcils quand elle lui avait dit d’un air suppliant: «Ce n’est pas dans trop
longtemps que vous me ferez signe?», il sentit l’odeur du fer du coiffeur par
lequel il se faisait relever sa «brosse» pendant que Lorédan allait chercher la
petite ouvrière, les pluies d’orage qui tombèrent si souvent ce printemps-là, le
retour glacial dans sa victoria, au clair de lune, toutes les mailles
d’habitudes mentales, d’impressions saisonnières, de créations cutanées, qui
avaient étendu sur une suite de semaines un réseau uniforme dans lequel son
corps se trouvait repris. A ce moment-là, il satisfaisait une curiosité
voluptueuse en connaissant les plaisirs des gens qui vivent par l’amour. Il
avait cru qu’il pourrait s’en tenir là, qu’il ne serait pas obligé d’en
apprendre les douleurs; comme maintenant le charme d’Odette lui était peu de
chose auprès de cette formidable terreur qui le prolongeait comme un trouble
halo, cette immense angoisse de ne pas savoir à tous moments ce qu’elle avait
fait, de ne pas la posséder partout et toujours! Hélas, il se rappela l’accent
dont elle s’était écriée: «Mais je pourrai toujours vous voir, je suis toujours
libre!» elle qui ne l’était plus jamais! l’intérêt, la curiosité qu’elle avait
eus pour sa vie à lui, le désir passionné qu’il lui fit la faveur — redoutée au
contraire par lui en ce temps-là comme une cause d’ennuyeux dérangements — de
l’y laisser pénétrer; comme elle avait été obligée de le prier pour qu’il se
laissât mener chez les Verdurin; et, quand il la faisait venir chez lui une fois
par mois, comme il avait fallu, avant qu’il se laissât fléchir, qu’elle lui
répétât le délice que serait cette habitude de se voir tous les jours dont elle
rêvait alors qu’elle ne lui semblait à lui qu’un fastidieux tracas, puis qu’elle
avait prise en dégoût et définitivement rompue, pendant qu’elle était devenue
pour lui un si invincible et si douloureux besoin. Il ne savait pas dire si vrai
quand, à la troisième fois qu’il l’avait vue, comme elle lui répétait: «Mais
pourquoi ne me laissez-vous pas venir plus souvent», il lui avait dit en riant,
avec galanterie: «par peur de souffrir». Maintenant, hélas! il arrivait encore
parfois qu’elle lui écrivît d’un restaurant ou d’un hôtel sur du papier qui en
portait le nom imprimé; mais c’était comme des lettres de feu qui le brûlaient.
«C’est écrit de l’hôtel Vouillemont? Qu’y peut-elle être allée faire! avec qui?
que s’y est-il passé?» Il se rappela les becs de gaz qu’on éteignait boulevard
des Italiens quand il l’avait rencontrée contre tout espoir parmi les ombres
errantes dans cette nuit qui lui avait semblé presque surnaturelle et qui en
effet — nuit d’un temps où il n’avait même pas à se demander s’il ne la
contrarierait pas en la cherchant, en la retrouvant, tant il était sûr qu’elle
n’avait pas de plus grande joie que de le voir et de rentrer avec lui —
appartenait bien à un monde mystérieux où on ne peut jamais revenir quand les
portes s’en sont refermées, Et Swann aperçut, immobile en face de ce bonheur
revécu, un malheureux qui lui fit pitié parce qu’il ne le reconnut pas tout de
suite, si bien qu’il dut baisser les yeux pour qu’on ne vît pas qu’ils étaient
pleins de larmes. C’était lui-même.

Quand il l’eut compris, sa pitié cessa, mais il fut jaloux de l’autre lui-même
qu’elle avait aimé, il fut jaloux de ceux dont il s’était dit souvent sans trop
souffrir, «elle les aime peut-être», maintenant qu’il avait échangé l’idée vague
d’aimer, dans laquelle il n’y a pas d’amour, contre les pétales du chrysanthème
et l’«en tête» de la Maison d’Or, qui, eux en étaient pleins. Puis sa souffrance
devenant trop vive, il passa sa main sur son front, laissa tomber son monocle,
en essuya le verre. Et sans doute s’il s’était vu à ce moment-là, il eut ajouté
à la collection de ceux qu’il avait distingués le monocle qu’il déplaçait comme
une pensée importune et sur la face embuée duquel, avec un mouchoir, il
cherchait à effacer des soucis.

Il y a dans le violon — si ne voyant pas l’instrument, on ne peut pas rapporter
ce qu’on entend à son image laquelle modifie la sonorité— des accents qui lui
sont si communs avec certaines voix de contralto, qu’on a l’illusion qu’une
chanteuse s’est ajoutée au concert. On lève les yeux, on ne voit que les étuis,
précieux comme des boîtes chinoises, mais, par moment, on est encore trompé par
l’appel décevant de la sirène; parfois aussi on croit entendre un génie captif
qui se débat au fond de la docte boîte, ensorcelée et frémissante, comme un
diable dans un bénitier; parfois enfin, c’est, dans l’air, comme un être
surnaturel et pur qui passe en déroulant son message invisible.

Comme si les instrumentistes, beaucoup moins jouaient la petite phrase qu’ils
n’exécutaient les rites exigés d’elle pour qu’elle apparût, et procédaient aux
incantations nécessaires pour obtenir et prolonger quelques instants le prodige
de son évocation, Swann, qui ne pouvait pas plus la voir que si elle avait
appartenu à un monde ultra-violet, et qui goûtait comme le rafraîchissement
d’une métamorphose dans la cécité momentanée dont il était frappé en approchant
d’elle, Swann la sentait présente, comme une déesse protectrice et confidente de
son amour, et qui pour pouvoir arriver jusqu’à lui devant la foule et l’emmener
à l’écart pour lui parler, avait revêtu le déguisement de cette apparence
sonore. Et tandis qu’elle passait, légère, apaisante et murmurée comme un
parfum, lui disant ce qu’elle avait à lui dire et dont il scrutait tous les
mots, regrettant de les voir s’envoler si vite, il faisait involontairement avec
ses lèvres le mouvement de baiser au passage le corps harmonieux et fuyant. Il
ne se sentait plus exilé et seul puisque, elle, qui s’adressait à lui, lui
parlait à mi-voix d’Odette. Car il n’avait plus comme autrefois l’impression
qu’Odette et lui n’étaient pas connus de la petite phrase. C’est que si souvent
elle avait été témoin de leurs joies! Il est vrai que souvent aussi elle l’avait
averti de leur fragilité. Et même, alors que dans ce temps-là il devinait de la
souffrance dans son sourire, dans son intonation limpide et désenchantée,
aujourd’hui il y trouvait plutôt la grâce d’une résignation presque gaie. De ces
chagrins dont elle lui parlait autrefois et qu’il la voyait, sans qu’il fût
atteint par eux, entraîner en souriant dans son cours sinueux et rapide, de ces
chagrins qui maintenant étaient devenus les siens sans qu’il eût l’espérance
d’en être jamais délivré, elle semblait lui dire comme jadis de son bonheur:
«Qu’est-ce, cela? tout cela n’est rien.» Et la pensée de Swann se porta pour la
première fois dans un élan de pitié et de tendresse vers ce Vinteuil, vers ce
frère inconnu et sublime qui lui aussi avait dû tant souffrir; qu’avait pu être
sa vie? au fond de quelles douleurs avait-il puisé cette force de dieu, cette
puissance illimitée de créer? Quand c’était la petite phrase qui lui parlait de
la vanité de ses souffrances, Swann trouvait de la douceur à cette même sagesse
qui tout à l’heure pourtant lui avait paru intolérable, quand il croyait la lire
dans les visages des indifférents qui considéraient son amour comme une
divagation sans importance. C’est que la petite phrase au contraire, quelque
opinion qu’elle pût avoir sur la brève durée de ces états de l’âme, y voyait
quelque chose, non pas comme faisaient tous ces gens, de moins sérieux que la
vie positive, mais au contraire de si supérieur à elle que seul il valait la
peine d’être exprimé. Ces charmes d’une tristesse intime, c’était eux qu’elle
essayait d’imiter, de recréer, et jusqu’à leur essence qui est pourtant d’être
incommunicables et de sembler frivoles à tout autre qu’à celui qui les éprouve,
la petite phrase l’avait captée, rendue visible. Si bien qu’elle faisait
confesser leur prix et goûter leur douceur divine, par tous ces mêmes assistants
— si seulement ils étaient un peu musiciens — qui ensuite les méconnaîtraient
dans la vie, en chaque amour particulier qu’ils verraient naître près d’eux.
Sans doute la forme sous laquelle elle les avait codifiés ne pouvait pas se
résoudre en raisonnements. Mais depuis plus d’une année que lui révélant à
lui-même bien des richesses de son âme, l’amour de la musique était pour quelque
temps au moins né en lui, Swann tenait les motifs musicaux pour de véritables
idées, d’un autre monde, d’un autre ordre, idées voilées de ténèbres, inconnues,
impénétrables à l’intelligence, mais qui n’en sont pas moins parfaitement
distinctes les unes des autres, inégales entre elles de valeur et de
signification. Quand après la soirée Verdurin, se faisant rejouer la petite
phrase, il avait cherché à démêler comment à la façon d’un parfum, d’une
caresse, elle le circonvenait, elle l’enveloppait, il s’était rendu compte que
c’était au faible écart entre les cinq notes qui la composaient et au rappel
constant de deux d’entre elles qu’était due cette impression de douceur
rétractée et frileuse; mais en réalité il savait qu’il raisonnait ainsi non sur
la phrase elle-même mais sur de simples valeurs, substituées pour la commodité
de son intelligence à la mystérieuse entité qu’il avait perçue, avant de
connaître les Verdurin, à cette soirée où il avait entendu pour la première fois
la sonate. Il savait que le souvenir même du piano faussait encore le plan dans
lequel il voyait les choses de la musique, que le champ ouvert au musicien n’est
pas un clavier mesquin de sept notes, mais un clavier incommensurable, encore
presque tout entier inconnu, où seulement çà et là, séparées par d’épaisses
ténèbres inexplorées, quelques-unes des millions de touches de tendresse, de
passion, de courage, de sérénité, qui le composent, chacune aussi différente des
autres qu’un univers d’un autre univers, ont été découvertes par quelques grands
artistes qui nous rendent le service, en éveillant en nous le correspondant du
thème qu’ils ont trouvé, de nous montrer quelle richesse, quelle variété, cache
à notre insu cette grande nuit impénétrée et décourageante de notre âme que nous
prenons pour du vide et pour du néant. Vinteuil avait été l’un de ces musiciens.
En sa petite phrase, quoiqu’elle présentât à la raison une surface obscure, on
sentait un contenu si consistant, si explicite, auquel elle donnait une force si
nouvelle, si originale, que ceux qui l’avaient entendue la conservaient en eux
de plain-pied avec les idées de l’intelligence. Swann s’y reportait comme à une
conception de l’amour et du bonheur dont immédiatement il savait aussi bien en
quoi elle était particulière, qu’il le savait pour la «Princesse de Clèves», ou
pour «René», quand leur nom se présentait à sa mémoire. Même quand il ne pensait
pas à la petite phrase, elle existait latente dans son esprit au même titre que
certaines autres notions sans équivalent, comme les notions de la lumière, du
son, du relief, de la volupté physique, qui sont les riches possessions dont se
diversifie et se pare notre domaine intérieur. Peut-être les perdrons-nous,
peut-être s’effaceront-elles, si nous retournons au néant. Mais tant que nous
vivons nous ne pouvons pas plus faire que nous ne les ayons connues que nous ne
le pouvons pour quelque objet réel, que nous ne pouvons, par exemple, douter de
la lumière de la lampe qu’on allume devant les objets métamorphosés de notre
chambre d’où s’est échappé jusqu’au souvenir de l’obscurité. Par là, la phrase
de Vinteuil avait, comme tel thème de Tristan par exemple, qui nous représente
aussi une certaine acquisition sentimentale, épousé notre condition mortelle,
pris quelque chose d’humain qui était assez touchant. Son sort était lié à
l’avenir, à la réalité de notre âme dont elle était un des ornements les plus
particuliers, les mieux différenciés. Peut-être est-ce le néant qui est le vrai
et tout notre rêve est-il inexistant, mais alors nous sentons qu’il faudra que
ces phrases musicales, ces notions qui existent par rapport à lui, ne soient
rien non plus. Nous périrons mais nous avons pour otages ces captives divines
qui suivront notre chance. Et la mort avec elles a quelque chose de moins amer,
de moins inglorieux, peut-être de moins probable.

Swann n’avait donc pas tort de croire que la phrase de la sonate existât
réellement. Certes, humaine à ce point de vue, elle appartenait pourtant à un
ordre de créatures surnaturelles et que nous n’avons jamais vues, mais que
malgré cela nous reconnaissons avec ravissement quand quelque explorateur de
l’invisible arrive à en capter une, à l’amener, du monde divin où il a accès,
briller quelques instants au-dessus du nôtre. C’est ce que Vinteuil avait fait
pour la petite phrase. Swann sentait que le compositeur s’était contenté, avec
ses instruments de musique, de la dévoiler, de la rendre visible, d’en suivre et
d’en respecter le dessin d’une main si tendre, si prudente, si délicate et si
sûre que le son s’altérait à tout moment, s’estompant pour indiquer une ombre,
revivifié quand il lui fallait suivre à la piste un plus hardi contour. Et une
preuve que Swann ne se trompait pas quand il croyait à l’existence réelle de
cette phrase, c’est que tout amateur un peu fin se fût tout de suite aperçu de
l’imposture, si Vinteuil ayant eu moins de puissance pour en voir et en rendre
les formes, avait cherché à dissimuler, en ajoutant çà et là des traits de son
cru, les lacunes de sa vision ou les défaillances de sa main.

Elle avait disparu. Swann savait qu’elle reparaîtrait à la fin du dernier
mouvement, après tout un long morceau que le pianiste de Mme Verdurin sautait
toujours. Il y avait là d’admirables idées que Swann n’avait pas distinguées à
la première audition et qu’il percevait maintenant, comme si elles se fussent,
dans le vestiaire de sa mémoire, débarrassées du déguisement uniforme de la
nouveauté. Swann écoutait tous les thèmes épars qui entreraient dans la
composition de la phrase, comme les prémisses dans la conclusion nécessaire, il
assistait à sa genèse. «O audace aussi géniale peut-être, se disait-il, que
celle d’un Lavoisier, d’un Ampère, l’audace d’un Vinteuil expérimentant,
découvrant les lois secrètes d’une force inconnue, menant à travers l’inexploré,
vers le seul but possible, l’attelage invisible auquel il se fie et qu’il
n’apercevra jamais.» Le beau dialogue que Swann entendit entre le piano et le
violon au commencement du dernier morceau! La suppression des mots humains, loin
d’y laisser régner la fantaisie, comme on aurait pu croire, l’en avait éliminée;
jamais le langage parlé ne fut si inflexiblement nécessité, ne connut à ce point
la pertinence des questions, l’évidence des réponses. D’abord le piano solitaire
se plaignit, comme un oiseau abandonné de sa compagne; le violon l’entendit, lui
répondit comme d’un arbre voisin. C’était comme au commencement du monde, comme
s’il n’y avait encore eu qu’eux deux sur la terre, ou plutôt dans ce monde fermé
à tout le reste, construit par la logique d’un créateur et où ils ne seraient
jamais que tous les deux: cette sonate. Est-ce un oiseau, est-ce l’âme
incomplète encore de la petite phrase, est-ce une fée, invisible et gémissant
dont le piano ensuite redisait tendrement la plainte? Ses cris étaient si
soudains que le violoniste devait se précipiter sur son archet pour les
recueillir. Merveilleux oiseau! le violoniste semblait vouloir le charmer,
l’apprivoiser, le capter. Déjà il avait passé dans son âme, déjà la petite
phrase évoquée agitait comme celui d’un médium le corps vraiment possédé du
violoniste. Swann savait qu’elle allait parler encore une fois. Et il s’était si
bien dédoublé que l’attente de l’instant imminent où il allait se retrouver en
face d’elle le secoua d’un de ces sanglots qu’un beau vers ou une triste
nouvelle provoquent en nous, non pas quand nous sommes seuls, mais si nous les
apprenons à des amis en qui nous nous apercevons comme un autre dont l’émotion
probable les attendrit. Elle reparut, mais cette fois pour se suspendre dans
l’air et se jouer un instant seulement, comme immobile, et pour expirer après.
Aussi Swann ne perdait-il rien du temps si court où elle se prorogeait. Elle
était encore là comme une bulle irisée qui se soutient. Tel un arc-en-ciel, dont
l’éclat faiblit, s’abaisse, puis se relève et avant de s’éteindre, s’exalte un
moment comme il n’avait pas encore fait: aux deux couleurs qu’elle avait
jusque-là laissé paraître, elle ajouta d’autres cordes diaprées, toutes celles
du prisme, et les fit chanter. Swann n’osait pas bouger et aurait voulu faire
tenir tranquilles aussi les autres personnes, comme si le moindre mouvement
avait pu compromettre le prestige surnaturel, délicieux et fragile qui était si
près de s’évanouir. Personne, à dire vrai, ne songeait à parler. La parole
ineffable d’un seul absent, peut-être d’un mort (Swann ne savait pas si Vinteuil
vivait encore) s’exhalant au-dessus des rites de ces officiants, suffisait à
tenir en échec l’attention de trois cents personnes, et faisait de cette estrade
où une âme était ainsi évoquée un des plus nobles autels où pût s’accomplir une
cérémonie surnaturelle. De sorte que quand la phrase se fut enfin défaite
flottant en lambeaux dans les motifs suivants qui déjà avaient pris sa place, si
Swann au premier instant fut irrité de voir la comtesse de Monteriender, célèbre
par ses naïvetés, se pencher vers lui pour lui confier ses impressions avant
même que la sonate fût finie, il ne put s’empêcher de sourire, et peut-être de
trouver aussi un sens profond qu’elle n’y voyait pas, dans les mots dont elle se
servit. Émerveillée par la virtuosité des exécutants, la comtesse s’écria en
s’adressant à Swann: «C’est prodigieux, je n’ai jamais rien vu d’aussi fort . .
. » Mais un scrupule d’exactitude lui faisant corriger cette première assertion,
elle ajouta cette réserve: «rien d’aussi fort . . . depuis les tables
tournantes!»

A partir de cette soirée, Swann comprit que le sentiment qu’Odette avait eu pour
lui ne renaîtrait jamais, que ses espérances de bonheur ne se réaliseraient
plus. Et les jours où par hasard elle avait encore été gentille et tendre avec
lui, si elle avait eu quelque attention, il notait ces signes apparents et
menteurs d’un léger retour vers lui, avec cette sollicitude attendrie et
sceptique, cette joie désespérée de ceux qui, soignant un ami arrivé aux
derniers jours d’une maladie incurable, relatent comme des faits précieux «hier,
il a fait ses comptes lui-même et c’est lui qui a relevé une erreur d’addition
que nous avions faite; il a mangé un œuf avec plaisir, s’il le digère bien on
essaiera demain d’une côtelette», quoiqu’ils les sachent dénués de signification
à la veille d’une mort inévitable. Sans doute Swann était certain que s’il avait
vécu maintenant loin d’Odette, elle aurait fini par lui devenir indifférente, de
sorte qu’il aurait été content qu’elle quittât Paris pour toujours; il aurait eu
le courage de rester; mais il n’avait pas celui de partir.

Il en avait eu souvent la pensée. Maintenant qu’il s’était remis à son étude sur
Ver Meer il aurait eu besoin de retourner au moins quelques jours à la Haye, à
Dresde, à Brunswick. Il était persuadé qu’une «Toilette de Diane» qui avait été
achetée par le Mauritshuis à la vente Goldschmidt comme un Nicolas Maes était en
réalité de Ver Meer. Et il aurait voulu pouvoir étudier le tableau sur place
pour étayer sa conviction. Mais quitter Paris pendant qu’Odette y était et même
quand elle était absente — car dans des lieux nouveaux où les sensations ne sont
pas amorties par l’habitude, on retrempe, on ranime une douleur — c’était pour
lui un projet si cruel, qu’il ne se sentait capable d’y penser sans cesse que
parce qu’il se savait résolu à ne l’exécuter jamais. Mais il arrivait qu’en
dormant, l’intention du voyage renaissait en lui — sans qu’il se rappelât que ce
voyage était impossible — et elle s’y réalisait. Un jour il rêva qu’il partait
pour un an; penché à la portière du wagon vers un jeune homme qui sur le quai
lui disait adieu en pleurant, Swann cherchait à le convaincre de partir avec
lui. Le train s’ébranlant, l’anxiété le réveilla, il se rappela qu’il ne partait
pas, qu’il verrait Odette ce soir-là, le lendemain et presque chaque jour. Alors
encore tout ému de son rêve, il bénit les circonstances particulières qui le
rendaient indépendant, grâce auxquelles il pouvait rester près d’Odette, et
aussi réussir à ce qu’elle lui permît de la voir quelquefois; et, récapitulant
tous ces avantages: sa situation — sa fortune, dont elle avait souvent trop
besoin pour ne pas reculer devant une rupture (ayant même, disait-on, une
arrière-pensée de se faire épouser par lui) — cette amitié de M. de Charlus, qui
à vrai dire ne lui avait jamais fait obtenir grand’chose d’Odette, mais lui
donnait la douceur de sentir qu’elle entendait parler de lui d’une manière
flatteuse par cet ami commun pour qui elle avait une si grande estime — et
jusqu’à son intelligence enfin, qu’il employait tout entière à combiner chaque
jour une intrigue nouvelle qui rendît sa présence sinon agréable, du moins
nécessaire à Odette — il songea à ce qu’il serait devenu si tout cela lui avait
manqué, il songea que s’il avait été, comme tant d’autres, pauvre, humble,
dénué, obligé d’accepter toute besogne, ou lié à des parents, à une épouse, il
aurait pu être obligé de quitter Odette, que ce rêve dont l’effroi était encore
si proche aurait pu être vrai, et il se dit: «On ne connaît pas son bonheur. On
n’est jamais aussi malheureux qu’on croit.» Mais il compta que cette existence
durait déjà depuis plusieurs années, que tout ce qu’il pouvait espérer c’est
qu’elle durât toujours, qu’il sacrifierait ses travaux, ses plaisirs, ses amis,
finalement toute sa vie à l’attente quotidienne d’un rendez-vous qui ne pouvait
rien lui apporter d’heureux, et il se demanda s’il ne se trompait pas, si ce qui
avait favorisé sa liaison et en avait empêché la rupture n’avait pas desservi sa
destinée, si l’événement désirable, ce n’aurait pas été celui dont il se
réjouissait tant qu’il n’eût eu lieu qu’en rêve: son départ; il se dit qu’on ne
connaît pas son malheur, qu’on n’est jamais si heureux qu’on croit.

Quelquefois il espérait qu’elle mourrait sans souffrances dans un accident, elle
qui était dehors, dans les rues, sur les routes, du matin au soir. Et comme elle
revenait saine et sauve, il admirait que le corps humain fût si souple et si
fort, qu’il pût continuellement tenir en échec, déjouer tous les périls qui
l’environnent (et que Swann trouvait innombrables depuis que son secret désir
les avait supputés), et permît ainsi aux êtres de se livrer chaque jour et à peu
près impunément à leur œuvre de mensonge, à la poursuite du plaisir. Et Swann
sentait bien près de son cœur ce Mahomet II dont il aimait le portrait par
Bellini et qui, ayant senti qu’il était devenu amoureux fou d’une de ses femmes
la poignarda afin, dit naïvement son biographe vénitien, de retrouver sa liberté
d’esprit. Puis il s’indignait de ne penser ainsi qu’à soi, et les souffrances
qu’il avait éprouvées lui semblaient ne mériter aucune pitié puisque lui-même
faisait si bon marché de la vie d’Odette.

Ne pouvant se séparer d’elle sans retour, du moins, s’il l’avait vue sans
séparations, sa douleur aurait fini par s’apaiser et peut-être son amour par
s’éteindre. Et du moment qu’elle ne voulait pas quitter Paris à jamais, il eût
souhaité qu’elle ne le quittât jamais. Du moins comme il savait que la seule
grande absence qu’elle faisait était tous les ans celle d’août et septembre, il
avait le loisir plusieurs mois d’avance d’en dissoudre l’idée amère dans tout le
Temps à venir qu’il portait en lui par anticipation et qui, composé de jours
homogènes aux jours actuels, circulait transparent et froid en son esprit où il
entretenait la tristesse, mais sans lui causer de trop vives souffrances. Mais
cet avenir intérieur, ce fleuve, incolore, et libre, voici qu’une seule parole
d’Odette venait l’atteindre jusqu’en Swann et, comme un morceau de glace,
l’immobilisait, durcissait sa fluidité, le faisait geler tout entier; et Swann
s’était senti soudain rempli d’une masse énorme et infrangible qui pesait sur
les parois intérieures de son être jusqu’à le faire éclater: c’est qu’Odette lui
avait dit, avec un regard souriant et sournois qui l’observait: «Forcheville va
faire un beau voyage, à la Pentecôte. Il va en Égypte», et Swann avait aussitôt
compris que cela signifiait: «Je vais aller en Égypte à la Pentecôte avec
Forcheville.» Et en effet, si quelques jours après, Swann lui disait: «Voyons, à
propos de ce voyage que tu m’as dit que tu ferais avec Forcheville», elle
répondait étourdiment: «Oui, mon petit, nous partons le 19, on t’enverra une vue
des Pyramides.» Alors il voulait apprendre si elle était la maîtresse de
Forcheville, le lui demander à elle-même. Il savait que, superstitieuse comme
elle était, il y avait certains parjures qu’elle ne ferait pas et puis la
crainte, qui l’avait retenu jusqu’ici, d’irriter Odette en l’interrogeant, de se
faire détester d’elle, n’existait plus maintenant qu’il avait perdu tout espoir
d’en être jamais aimé.

Un jour il reçut une lettre anonyme, qui lui disait qu’Odette avait été la
maîtresse d’innombrables hommes (dont on lui citait quelques-uns parmi lesquels
Forcheville, M. de Bréauté et le peintre), de femmes, et qu’elle fréquentait les
maisons de passe. Il fut tourmenté de penser qu’il y avait parmi ses amis un
être capable de lui avoir adressé cette lettre (car par certains détails elle
révélait chez celui qui l’avait écrite une connaissance familière de la vie de
Swann). Il chercha qui cela pouvait être. Mais il n’avait jamais eu aucun
soupçon des actions inconnues des êtres, de celles qui sont sans liens visibles
avec leurs propos. Et quand il voulut savoir si c’était plutôt sous le caractère
apparent de M. de Charlus, de M. des Laumes, de M. d’Orsan, qu’il devait situer
la région inconnue où cet acte ignoble avait dû naître, comme aucun de ces
hommes n’avait jamais approuvé devant lui les lettres anonymes et que tout ce
qu’ils lui avaient dit impliquait qu’ils les réprouvaient, il ne vit pas de
raisons pour relier cette infamie plutôt à la nature de l’un que de l’autre.
Celle de M. de Charlus était un peu d’un détraqué mais foncièrement bonne et
tendre; celle de M. des Laumes un peu sèche mais saine et droite. Quant à M.
d’Orsan, Swann, n’avait jamais rencontré personne qui dans les circonstances
même les plus tristes vînt à lui avec une parole plus sentie, un geste plus
discret et plus juste. C’était au point qu’il ne pouvait comprendre le rôle peu
délicat qu’on prêtait à M. d’Orsan dans la liaison qu’il avait avec une femme
riche, et que chaque fois que Swann pensait à lui il était obligé de laisser de
côté cette mauvaise réputation inconciliable avec tant de témoignages certains
de délicatesse. Un instant Swann sentit que son esprit s’obscurcissait et il
pensa à autre chose pour retrouver un peu de lumière. Puis il eut le courage de
revenir vers ces réflexions. Mais alors après n’avoir pu soupçonner personne, il
lui fallut soupçonner tout le monde. Après tout M. de Charlus l’aimait, avait
bon cœur. Mais c’était un névropathe, peut-être demain pleurerait-il de le
savoir malade, et aujourd’hui par jalousie, par colère, sur quelque idée subite
qui s’était emparée de lui, avait-il désiré lui faire du mal. Au fond, cette
race d’hommes est la pire de toutes. Certes, le prince des Laumes était bien
loin d’aimer Swann autant que M. de Charlus. Mais à cause de cela même il
n’avait pas avec lui les mêmes susceptibilités; et puis c’était une nature
froide sans doute, mais aussi incapable de vilenies que de grandes actions.
Swann se repentait de ne s’être pas attaché, dans la vie, qu’à de tels êtres.
Puis il songeait que ce qui empêche les hommes de faire du mal à leur prochain,
c’est la bonté, qu’il ne pouvait au fond répondre que de natures analogues à la
sienne, comme était, à l’égard du cœur, celle de M. de Charlus. La seule pensée
de faire cette peine à Swann eût révolté celui-ci. Mais avec un homme
insensible, d’une autre humanité, comme était le prince des Laumes, comment
prévoir à quels actes pouvaient le conduire des mobiles d’une essence
différente. Avoir du cœur c’est tout, et M. de Charlus en avait. M. d’Orsan n’en
manquait pas non plus et ses relations cordiales mais peu intimes avec Swann,
nées de l’agrément que, pensant de même sur tout, ils avaient à causer ensemble,
étaient de plus de repos que l’affection exaltée de M. de Charlus, capable de se
porter à des actes de passion, bons ou mauvais. S’il y avait quelqu’un par qui
Swann s’était toujours senti compris et délicatement aimé, c’était par M.
d’Orsan. Oui, mais cette vie peu honorable qu’il menait? Swann regrettait de
n’en avoir pas tenu compte, d’avoir souvent avoué en plaisantant qu’il n’avait
jamais éprouvé si vivement des sentiments de sympathie et d’estime que dans la
société d’une canaille. Ce n’est pas pour rien, se disait-il maintenant, que
depuis que les hommes jugent leur prochain, c’est sur ses actes. Il n’y a que
cela qui signifie quelque chose, et nullement ce que nous disons, ce que nous
pensons. Charlus et des Laumes peuvent avoir tels ou tels défauts, ce sont
d’honnêtes gens. Orsan n’en a peut-être pas, mais ce n’est pas un honnête homme.
Il a pu mal agir une fois de plus. Puis Swann soupçonna Rémi, qui il est vrai
n’aurait pu qu’inspirer la lettre, mais cette piste lui parut un instant la
bonne. D’abord Lorédan avait des raisons d’en vouloir à Odette. Et puis comment
ne pas supposer que nos domestiques, vivant dans une situation inférieure à la
nôtre, ajoutant à notre fortune et à nos défauts des richesses et des vices
imaginaires pour lesquels ils nous envient et nous méprisent, se trouveront
fatalement amenés à agir autrement que des gens de notre monde. Il soupçonna
aussi mon grand-père. Chaque fois que Swann lui avait demandé un service, ne le
lui avait-il pas toujours refusé? puis avec ses idées bourgeoises il avait pu
croire agir pour le bien de Swann. Celui-ci soupçonna encore Bergotte, le
peintre, les Verdurin, admira une fois de plus au passage la sagesse des gens du
monde de ne pas vouloir frayer avec ces milieux artistes où de telles choses
sont possibles, peut-être même avouées sous le nom de bonnes farces; mais il se
rappelait des traits de droiture de ces bohèmes, et les rapprocha de la vie
d’expédients, presque d’escroqueries, où le manque d’argent, le besoin de luxe,
la corruption des plaisirs conduisent souvent l’aristocratie. Bref cette lettre
anonyme prouvait qu’il connaissait un être capable de scélératesse, mais il ne
voyait pas plus de raison pour que cette scélératesse fût cachée dans le tuf —
inexploré d’autrui — du caractère de l’homme tendre que de l’homme froid, de
l’artiste que du bourgeois, du grand seigneur que du valet. Quel critérium
adopter pour juger les hommes? au fond il n’y avait pas une seule des personnes
qu’il connaissait qui ne pût être capable d’une infamie. Fallait-il cesser de
les voir toutes? Son esprit se voila; il passa deux ou trois fois ses mains sur
son front, essuya les verres de son lorgnon avec son mouchoir, et, songeant
qu’après tout, des gens qui le valaient fréquentaient M. de Charlus, le prince
des Laumes, et les autres, il se dit que cela signifiait sinon qu’ils fussent
incapables d’infamie, du moins, que c’est une nécessité de la vie à laquelle
chacun se soumet de fréquenter des gens qui n’en sont peut-être pas incapables.
Et il continua à serrer la main à tous ces amis qu’il avait soupçonnés, avec
cette réserve de pur style qu’ils avaient peut-être cherché à le désespérer.
Quant au fond même de la lettre, il ne s’en inquiéta pas, car pas une des
accusations formulées contre Odette n’avait l’ombre de vraisemblance. Swann
comme beaucoup de gens avait l’esprit paresseux et manquait d’invention. Il
savait bien comme une vérité générale que la vie des êtres est pleine de
contrastes, mais pour chaque être en particulier il imaginait toute la partie de
sa vie qu’il ne connaissait pas comme identique à la partie qu’il connaissait.
Il imaginait ce qu’on lui taisait à l’aide de ce qu’on lui disait. Dans les
moments où Odette était auprès de lui, s’ils parlaient ensemble d’une action
indélicate commise, ou d’un sentiment indélicat éprouvé, par un autre, elle les
flétrissait en vertu des mêmes principes que Swann avait toujours entendu
professer par ses parents et auxquels il était resté fidèle; et puis elle
arrangeait ses fleurs, elle buvait une tasse de thé, elle s’inquiétait des
travaux de Swann. Donc Swann étendait ces habitudes au reste de la vie d’Odette,
il répétait ces gestes quand il voulait se représenter les moments où elle était
loin de lui. Si on la lui avait dépeinte telle qu’elle était, ou plutôt qu’elle
avait été si longtemps avec lui, mais auprès d’un autre homme, il eût souffert,
car cette image lui eût paru vraisemblable. Mais qu’elle allât chez des
maquerelles, se livrât à des orgies avec des femmes, qu’elle menât la vie
crapuleuse de créatures abjectes, quelle divagation insensée à la réalisation de
laquelle, Dieu merci, les chrysanthèmes imaginés, les thés successifs, les
indignations vertueuses ne laissaient aucune place. Seulement de temps à autre,
il laissait entendre à Odette que par méchanceté, on lui racontait tout ce
qu’elle faisait; et, se servant à propos, d’un détail insignifiant mais vrai,
qu’il avait appris par hasard, comme s’il était le seul petit bout qu’il laissât
passer malgré lui, entre tant d’autres, d’une reconstitution complète de la vie
d’Odette qu’il tenait cachée en lui, il l’amenait à supposer qu’il était
renseigné sur des choses qu’en réalité il ne savait ni même ne soupçonnait, car
si bien souvent il adjurait Odette de ne pas altérer la vérité, c’était
seulement, qu’il s’en rendît compte ou non, pour qu’Odette lui dît tout ce
qu’elle faisait. Sans doute, comme il le disait à Odette, il aimait la
sincérité, mais il l’aimait comme une proxénète pouvant le tenir au courant de
la vie de sa maîtresse. Aussi son amour de la sincérité n’étant pas
désintéressé, ne l’avait pas rendu meilleur. La vérité qu’il chérissait c’était
celle que lui dirait Odette; mais lui-même, pour obtenir cette vérité, ne
craignait pas de recourir au mensonge, le mensonge qu’il ne cessait de peindre à
Odette comme conduisant à la dégradation toute créature humaine. En somme il
mentait autant qu’Odette parce que plus malheureux qu’elle, il n’était pas moins
égoïste. Et elle, entendant Swann lui raconter ainsi à elle-même des choses
qu’elle avait faites, le regardait d’un air méfiant, et, à toute aventure,
fâché, pour ne pas avoir l’air de s’humilier et de rougir de ses actes.

Un jour, étant dans la période de calme la plus longue qu’il eût encore pu
traverser sans être repris d’accès de jalousie, il avait accepté d’aller le soir
au théâtre avec la princesse des Laumes. Ayant ouvert le journal, pour chercher
ce qu’on jouait, la vue du titre: Les Filles de Marbre de Théodore Barrière le
frappa si cruellement qu’il eut un mouvement de recul et détourna la tête.
Éclairé comme par la lumière de la rampe, à la place nouvelle où il figurait, ce
mot de «marbre» qu’il avait perdu la faculté de distinguer tant il avait
l’habitude de l’avoir souvent sous les yeux, lui était soudain redevenu visible
et l’avait aussitôt fait souvenir de cette histoire qu’Odette lui avait racontée
autrefois, d’une visite qu’elle avait faite au Salon du Palais de l’Industrie
avec Mme Verdurin et où celle-ci lui avait dit: «Prends garde, je saurai bien te
dégeler, tu n’es pas de marbre.» Odette lui avait affirmé que ce n’était qu’une
plaisanterie, et il n’y avait attaché aucune importance. Mais il avait alors
plus de confiance en elle qu’aujourd’hui. Et justement la lettre anonyme parlait
d’amour de ce genre. Sans oser lever les yeux vers le journal, il le déplia,
tourna une feuille pour ne plus voir ce mot: «Les Filles de Marbre» et commença
à lire machinalement les nouvelles des départements. Il y avait eu une tempête
dans la Manche, on signalait des dégâts à Dieppe, à Cabourg, à Beuzeval.
Aussitôt il fit un nouveau mouvement en arrière.

Le nom de Beuzeval l’avait fait penser à celui d’une autre localité de cette
région, Beuzeville, qui porte uni à celui-là par un trait d’union, un autre nom,
celui de Bréauté, qu’il avait vu souvent sur les cartes, mais dont pour la
première fois il remarquait que c’était le même que celui de son ami M. de
Bréauté dont la lettre anonyme disait qu’il avait été l’amant d’Odette. Après
tout, pour M. de Bréauté, l’accusation n’était pas invraisemblable; mais en ce
qui concernait Mme Verdurin, il y avait impossibilité. De ce qu’Odette mentait
quelquefois, on ne pouvait conclure qu’elle ne disait jamais la vérité et dans
ces propos qu’elle avait échangés avec Mme Verdurin et qu’elle avait racontés
elle-même à Swann, il avait reconnu ces plaisanteries inutiles et dangereuses
que, par inexpérience de la vie et ignorance du vice, tiennent des femmes dont
ils révèlent l’innocence, et qui — comme par exemple Odette — sont plus
éloignées qu’aucune d’éprouver une tendresse exaltée pour une autre femme.
Tandis qu’au contraire, l’indignation avec laquelle elle avait repoussé les
soupçons qu’elle avait involontairement fait naître un instant en lui par son
récit, cadrait avec tout ce qu’il savait des goûts, du tempérament de sa
maîtresse. Mais à ce moment, par une de ces inspirations de jaloux, analogues à
celle qui apporte au poète ou au savant, qui n’a encore qu’une rime ou qu’une
observation, l’idée ou la loi qui leur donnera toute leur puissance, Swann se
rappela pour la première fois une phrase qu’Odette lui avait dite il y avait
déjà deux ans: «Oh! Mme Verdurin, en ce moment il n’y en a que pour moi, je suis
un amour, elle m’embrasse, elle veut que je fasse des courses avec elle, elle
veut que je la tutoie.» Loin de voir alors dans cette phrase un rapport
quelconque avec les absurdes propos destinés à simuler le vice que lui avait
racontés Odette, il l’avait accueillie comme la preuve d’une chaleureuse amitié.
Maintenant voilà que le souvenir de cette tendresse de Mme Verdurin était venu
brusquement rejoindre le souvenir de sa conversation de mauvais goût. Il ne
pouvait plus les séparer dans son esprit, et les vit mêlées aussi dans la
réalité, la tendresse donnant quelque chose de sérieux et d’important à ces
plaisanteries qui en retour lui faisaient perdre de son innocence. Il alla chez
Odette. Il s’assit loin d’elle. Il n’osait l’embrasser, ne sachant si en elle,
si en lui, c’était l’affection ou la colère qu’un baiser réveillerait. Il se
taisait, il regardait mourir leur amour. Tout à coup il prit une résolution.

— Odette, lui dit-il, mon chéri, je sais bien que je suis odieux, mais il faut
que je te demande des choses. Tu te souviens de l’idée que j’avais eue à propos
de toi et de Mme Verdurin? Dis-moi si c’était vrai, avec elle ou avec une autre.

Elle secoua la tête en fronçant la bouche, signe fréquemment employé par les
gens pour répondre qu’ils n’iront pas, que cela les ennuie a quelqu’un qui leur
a demandé: «Viendrez-vous voir passer la cavalcade, assisterez-vous à la Revue?»
Mais ce hochement de tête affecté ainsi d’habitude à un événement à venir mêle à
cause de cela de quelque incertitude la dénégation d’un événement passé. De plus
il n’évoque que des raisons de convenance personnelle plutôt que la réprobation,
qu’une impossibilité morale. En voyant Odette lui faire ainsi le signe que
c’était faux, Swann comprit que c’était peut-être vrai.

— Je te l’ai dit, tu le sais bien, ajouta-t-elle d’un air irrité et malheureux.

— Oui, je sais, mais en es-tu sûre? Ne me dis pas: «Tu le sais bien», dis-moi:
«Je n’ai jamais fait ce genre de choses avec aucune femme.»

Elle répéta comme une leçon, sur un ton ironique et comme si elle voulait se
débarrasser de lui:

— Je n’ai jamais fait ce genre de choses avec aucune femme.

— Peux-tu me le jurer sur ta médaille de Notre-Dame de Laghet?

Swann savait qu’Odette ne se parjurerait pas sur cette médaille-là.

—«Oh! que tu me rends malheureuse, s’écria-t-elle en se dérobant par un sursaut
à l’étreinte de sa question. Mais as-tu bientôt fini? Qu’est-ce que tu as
aujourd’hui? Tu as donc décidé qu’il fallait que je te déteste, que je t’exècre?
Voilà, je voulais reprendre avec toi le bon temps comme autrefois et voilà ton
remerciement!»

Mais, ne la lâchant pas, comme un chirurgien attend la fin du spasme qui
interrompt son intervention mais ne l’y fait pas renoncer:

— Tu as bien tort de te figurer que je t’en voudrais le moins du monde, Odette,
lui dit-il avec une douceur persuasive et menteuse. Je ne te parle jamais que de
ce que je sais, et j’en sais toujours bien plus long que je ne dis. Mais toi
seule peux adoucir par ton aveu ce qui me fait te haïr tant que cela ne m’a été
dénoncé que par d’autres. Ma colère contre toi ne vient pas de tes actions, je
te pardonne tout puisque je t’aime, mais de ta fausseté, de ta fausseté absurde
qui te fait persévérer à nier des choses que je sais. Mais comment veux-tu que
je puisse continuer à t’aimer, quand je te vois me soutenir, me jurer une chose
que je sais fausse. Odette, ne prolonge pas cet instant qui est une torture pour
nous deux. Si tu le veux ce sera fini dans une seconde, tu seras pour toujours
délivrée. Dis-moi sur ta médaille, si oui ou non, tu as jamais fais ces choses.

— Mais je n’en sais rien, moi, s’écria-t-elle avec colère, peut-être il y a très
longtemps, sans me rendre compte de ce que je faisais, peut-être deux ou trois
fois.

Swann avait envisagé toutes les possibilités. La réalité est donc quelque chose
qui n’a aucun rapport avec les possibilités, pas plus qu’un coup de couteau que
nous recevons avec les légers mouvements des nuages au-dessus de notre tête,
puisque ces mots: «deux ou trois fois» marquèrent à vif une sorte de croix dans
son cœur. Chose étrange que ces mots «deux ou trois fois», rien que des mots,
des mots prononcés dans l’air, à distance, puissent ainsi déchirer le cœur comme
s’ils le touchaient véritablement, puissent rendre malade, comme un poison qu’on
absorberait. Involontairement Swann pensa à ce mot qu’il avait entendu chez Mme
de Saint-Euverte: «C’est ce que j’ai vu de plus fort depuis les tables
tournantes.» Cette souffrance qu’il ressentait ne ressemblait à rien de ce qu’il
avait cru. Non pas seulement parce que dans ses heures de plus entière méfiance
il avait rarement imaginé si loin dans le mal, mais parce que même quand il
imaginait cette chose, elle restait vague, incertaine, dénuée de cette horreur
particulière qui s’était échappée des mots «peut-être deux ou trois fois»,
dépourvue de cette cruauté spécifique aussi différente de tout ce qu’il avait
connu qu’une maladie dont on est atteint pour la première fois. Et pourtant
cette Odette d’où lui venait tout ce mal, ne lui était pas moins chère, bien au
contraire plus précieuse, comme si au fur et à mesure que grandissait la
souffrance, grandissait en même temps le prix du calmant, du contrepoison que
seule cette femme possédait. Il voulait lui donner plus de soins comme à une
maladie qu’on découvre soudain plus grave. Il voulait que la chose affreuse
qu’elle lui avait dit avoir faite «deux ou trois fois» ne pût pas se renouveler.
Pour cela il lui fallait veiller sur Odette. On dit souvent qu’en dénonçant à un
ami les fautes de sa maîtresse, on ne réussit qu’à le rapprocher d’elle parce
qu’il ne leur ajoute pas foi, mais combien davantage s’il leur ajoute foi. Mais,
se disait Swann, comment réussir à la protéger? Il pouvait peut-être la
préserver d’une certaine femme mais il y en avait des centaines d’autres et il
comprit quelle folie avait passé sur lui quand il avait le soir où il n’avait
pas trouvé Odette chez les Verdurin, commencé de désirer la possession, toujours
impossible, d’un autre être. Heureusement pour Swann, sous les souffrances
nouvelles qui venaient d’entrer dans son âme comme des hordes d’envahisseurs, il
existait un fond de nature plus ancien, plus doux et silencieusement laborieux,
comme les cellules d’un organe blessé qui se mettent aussitôt en mesure de
refaire les tissus lésés, comme les muscles d’un membre paralysé qui tendent à
reprendre leurs mouvements. Ces plus anciens, plus autochtones habitants de son
âme, employèrent un instant toutes les forces de Swann à ce travail obscurément
réparateur qui donne l’illusion du repos à un convalescent, à un opéré. Cette
fois-ci ce fut moins comme d’habitude dans le cerveau de Swann que se produisit
cette détente par épuisement, ce fut plutôt dans son cœur. Mais toutes les
choses de la vie qui ont existé une fois tendent à se récréer, et comme un
animal expirant qu’agite de nouveau le sursaut d’une convulsion qui semblait
finie, sur le cœur, un instant épargné, de Swann, d’elle-même la même souffrance
vint retracer la même croix. Il se rappela ces soirs de clair de lune, où
allongé dans sa victoria qui le menait rue La Pérouse, il cultivait
voluptueusement en lui les émotions de l’homme amoureux, sans savoir le fruit
empoisonné qu’elles produiraient nécessairement. Mais toutes ces pensées ne
durèrent que l’espace d’une seconde, le temps qu’il portât la main à son cœur,
reprit sa respiration et parvint à sourire pour dissimuler sa torture. Déjà il
recommençait à poser ses questions. Car sa jalousie qui avait pris une peine
qu’un ennemi ne se serait pas donnée pour arriver à lui faire asséner ce coup, à
lui faire faire la connaissance de la douleur la plus cruelle qu’il eût encore
jamais connue, sa jalousie ne trouvait pas qu’il eut assez souffert et cherchait
à lui faire recevoir une blessure plus profonde encore. Telle comme une divinité
méchante, sa jalousie inspirait Swann et le poussait à sa perte. Ce ne fut pas
sa faute, mais celle d’Odette seulement si d’abord son supplice ne s’aggrava
pas.

— Ma chérie, lui dit-il, c’est fini, était-ce avec une personne que je connais?

— Mais non je te jure, d’ailleurs je crois que j’ai exagéré, que je n’ai pas été
jusque-là.

Il sourit et reprit:

— Que veux-tu? cela ne fait rien, mais c’est malheureux que tu ne puisses pas me
dire le nom. De pouvoir me représenter la personne, cela m’empêcherait de plus
jamais y penser. Je le dis pour toi parce que je ne t’ennuierais plus. C’est si
calmant de se représenter les choses. Ce qui est affreux c’est ce qu’on ne peut
pas imaginer. Mais tu as déjà été si gentille, je ne veux pas te fatiguer. Je te
remercie de tout mon cœur de tout le bien que tu m’as fait. C’est fini.
Seulement ce mot: «Il y a combien de temps?»

— Oh! Charles, mais tu ne vois pas que tu me tues, c’est tout ce qu’il y a de
plus ancien. Je n’y avais jamais repensé, on dirait que tu veux absolument me
redonner ces idées-là. Tu seras bien avancé, dit-elle, avec une sottise
inconsciente et une méchanceté voulue.

— Oh! je voulais seulement savoir si c’est depuis que je te connais. Mais ce
serait si naturel, est-ce que ça se passait ici; tu ne peux pas me dire un
certain soir, que je me représente ce que je faisais ce soir-là; tu comprends
bien qu’il n’est pas possible que tu ne te rappelles pas avec qui, Odette, mon
amour.

— Mais je ne sais pas, moi, je crois que c’était au Bois un soir où tu es venu
nous retrouver dans l’île. Tu avais dîné chez la princesse des Laumes, dit-elle,
heureuse de fournir un détail précis qui attestait sa véracité. A une table
voisine il y avait une femme que je n’avais pas vue depuis très longtemps. Elle
m’a dit: «Venez donc derrière le petit rocher voir l’effet du clair de lune sur
l’eau.» D’abord j’ai bâillé et j’ai répondu: «Non, je suis fatiguée et je suis
bien ici.» Elle a assuré qu’il n’y avait jamais eu un clair de lune pareil. Je
lui ai dit «cette blague!» je savais bien où elle voulait en venir.

Odette racontait cela presque en riant, soit que cela lui parût tout naturel, ou
parce qu’elle croyait en atténuer ainsi l’importance, ou pour ne pas avoir l’air
humilié. En voyant le visage de Swann, elle changea de ton:

— Tu es un misérable, tu te plais à me torturer, à me faire faire des mensonges
que je dis afin que tu me laisses tranquille.

Ce second coup porté à Swann était plus atroce encore que le premier. Jamais il
n’avait supposé que ce fût une chose aussi récente, cachée à ses yeux qui
n’avaient pas su la découvrir, non dans un passé qu’il n’avait pas connu, mais
dans des soirs qu’il se rappelait si bien, qu’il avait vécus avec Odette, qu’il
avait cru connus si bien par lui et qui maintenant prenaient rétrospectivement
quelque chose de fourbe et d’atroce; au milieu d’eux tout d’un coup se creusait
cette ouverture béante, ce moment dans l’Ile du Bois. Odette sans être
intelligente avait le charme du naturel. Elle avait raconté, elle avait mimé
cette scène avec tant de simplicité que Swann haletant voyait tout; le
bâillement d’Odette, le petit rocher. Il l’entendait répondre — gaiement,
hélas!: «Cette blague»!!! Il sentait qu’elle ne dirait rien de plus ce soir,
qu’il n’y avait aucune révélation nouvelle à attendre en ce moment; il se
taisait; il lui dit:

— Mon pauvre chéri, pardonne-moi, je sens que je te fais de la peine, c’est
fini, je n’y pense plus.

Mais elle vit que ses yeux restaient fixés sur les choses qu’il ne savait pas et
sur ce passé de leur amour, monotone et doux dans sa mémoire parce qu’il était
vague, et que déchirait maintenant comme une blessure cette minute dans l’île du
Bois, au clair de lune, après le dîner chez la princesse des Laumes. Mais il
avait tellement pris l’habitude de trouver la vie intéressante — d’admirer les
curieuses découvertes qu’on peut y faire — que tout en souffrant au point de
croire qu’il ne pourrait pas supporter longtemps une pareille douleur, il se
disait: «La vie est vraiment étonnante et réserve de belles surprises; en somme
le vice est quelque chose de plus répandu qu’on ne croit. Voilà une femme en qui
j’avais confiance, qui a l’air si simple, si honnête, en tous cas, si même elle
était légère, qui semblait bien normale et saine dans ses goûts: sur une
dénonciation invraisemblable, je l’interroge et le peu qu’elle m’avoue révèle
bien plus que ce qu’on eût pu soupçonner.» Mais il ne pouvait pas se borner à
ces remarques désintéressées. Il cherchait à apprécier exactement la valeur de
ce qu’elle lui avait raconté, afin de savoir s’il devait conclure que ces
choses, elle les avait faites souvent, qu’elles se renouvelleraient. Il se
répétait ces mots qu’elle avait dits: «Je voyais bien où elle voulait en venir»,
«Deux ou trois fois», «Cette blague!» mais ils ne reparaissaient pas désarmés
dans la mémoire de Swann, chacun d’eux tenait son couteau et lui en portait un
nouveau coup. Pendant bien longtemps, comme un malade ne peut s’empêcher
d’essayer à toute minute de faire le mouvement qui lui est douloureux, il se
redisait ces mots: «Je suis bien ici», «Cette blague!», mais la souffrance était
si forte qu’il était obligé de s’arrêter. Il s’émerveillait que des actes que
toujours il avait jugés si légèrement, si gaiement, maintenant fussent devenus
pour lui graves comme une maladie dont on peut mourir. Il connaissait bien des
femmes à qui il eût pu demander de surveiller Odette. Mais comment espérer
qu’elles se placeraient au même point de vue que lui et ne resteraient pas à
celui qui avait été si longtemps le sien, qui avait toujours guidé sa vie
voluptueuse, ne lui diraient pas en riant: «Vilain jaloux qui veut priver les
autres d’un plaisir.» Par quelle trappe soudainement abaissée (lui qui n’avait
eu autrefois de son amour pour Odette que des plaisirs délicats) avait-il été
brusquement précipité dans ce nouveau cercle de l’enfer d’où il n’apercevait pas
comment il pourrait jamais sortir. Pauvre Odette! il ne lui en voulait pas. Elle
n’était qu’à demi coupable. Ne disait-on pas que c’était par sa propre mère
qu’elle avait été livrée, presque enfant, à Nice, à un riche Anglais. Mais
quelle vérité douloureuse prenait pour lui ces lignes du Journal d’un Poète
d’Alfred de Vigny qu’il avait lues avec indifférence autrefois: «Quand on se
sent pris d’amour pour une femme, on devrait se dire: Comment est-elle entourée?
Quelle a été sa vie? Tout le bonheur de la vie est appuyé là-dessus.» Swann
s’étonnait que de simples phrases épelées par sa pensée, comme «Cette blague!»,
«Je voyais bien où elle voulait en venir» pussent lui faire si mal. Mais il
comprenait que ce qu’il croyait de simples phrases n’était que les pièces de
l’armature entre lesquelles tenait, pouvait lui être rendue, la souffrance qu’il
avait éprouvée pendant le récit d’Odette. Car c’était bien cette souffrance-là
qu’il éprouvait de nouveau. Il avait beau savoir maintenant — même, il eut beau,
le temps passant, avoir un peu oublié, avoir pardonné — au moment où il se
redisait ses mots, la souffrance ancienne le refaisait tel qu’il était avant
qu’Odette ne parlât: ignorant, confiant; sa cruelle jalousie le replaçait pour
le faire frapper par l’aveu d’Odette dans la position de quelqu’un qui ne sait
pas encore, et au bout de plusieurs mois cette vieille histoire le bouleversait
toujours comme une révélation. Il admirait la terrible puissance recréatrice de
sa mémoire. Ce n’est que de l’affaiblissement de cette génératrice dont la
fécondité diminue avec l’âge qu’il pouvait espérer un apaisement à sa torture.
Mais quand paraissait un peu épuisé le pouvoir qu’avait de le faire souffrir un
des mots prononcés par Odette, alors un de ceux sur lesquels l’esprit de Swann
s’était moins arrêté jusque-là, un mot presque nouveau venait relayer les autres
et le frappait avec une vigueur intacte. La mémoire du soir où il avait dîné
chez la princesse des Laumes lui était douloureuse, mais ce n’était que le
centre de son mal. Celui-ci irradiait confusément à l’entour dans tous les jours
avoisinants. Et à quelque point d’elle qu’il voulût toucher dans ses souvenirs,
c’est la saison tout entière où les Verdurin avaient si souvent dîné dans l’île
du Bois qui lui faisait mal. Si mal que peu à peu les curiosités qu’excitait en
lui sa jalousie furent neutralisées par la peur des tortures nouvelles qu’il
s’infligerait en les satisfaisant. Il se rendait compte que toute la période de
la vie d’Odette écoulée avant qu’elle ne le rencontrât, période qu’il n’avait
jamais cherché à se représenter, n’était pas l’étendue abstraite qu’il voyait
vaguement, mais avait été faite d’années particulières, remplie d’incidents
concrets. Mais en les apprenant, il craignait que ce passé incolore, fluide et
supportable, ne prît un corps tangible et immonde, un visage individuel et
diabolique. Et il continuait à ne pas chercher à le concevoir non plus par
paresse de penser, mais par peur de souffrir. Il espérait qu’un jour il finirait
par pouvoir entendre le nom de l’île du Bois, de la princesse des Laumes, sans
ressentir le déchirement ancien, et trouvait imprudent de provoquer Odette à lui
fournir de nouvelles paroles, le nom d’endroits, de circonstances différentes
qui, son mal à peine calmé, le feraient renaître sous une autre forme.

Mais souvent les choses qu’il ne connaissait pas, qu’il redoutait maintenant de
connaître, c’est Odette elle-même qui les lui révélait spontanément, et sans
s’en rendre compte; en effet l’écart que le vice mettait entre la vie réelle
d’Odette et la vie relativement innocente que Swann avait cru, et bien souvent
croyait encore, que menait sa maîtresse, cet écart Odette en ignorait l’étendue:
un être vicieux, affectant toujours la même vertu devant les êtres de qui il ne
veut pas que soient soupçonnés ses vices, n’a pas de contrôle pour se rendre
compte combien ceux-ci, dont la croissance continue est insensible pour lui-même
l’entraînent peu à peu loin des façons de vivre normales. Dans leur
cohabitation, au sein de l’esprit d’Odette, avec le souvenir des actions qu’elle
cachait à Swann, d’autres peu à peu en recevaient le reflet, étaient
contagionnées par elles, sans qu’elle pût leur trouver rien d’étrange, sans
qu’elles détonassent dans le milieu particulier où elle les faisait vivre en
elle; mais si elle les racontait à Swann, il était épouvanté par la révélation
de l’ambiance qu’elles trahissaient. Un jour il cherchait, sans blesser Odette,
à lui demander si elle n’avait jamais été chez des entremetteuses. A vrai dire
il était convaincu que non; la lecture de la lettre anonyme en avait introduit
la supposition dans son intelligence, mais d’une façon mécanique; elle n’y avait
rencontré aucune créance, mais en fait y était restée, et Swann, pour être
débarrassé de la présence purement matérielle mais pourtant gênante du soupçon,
souhaitait qu’Odette l’extirpât. «Oh! non! Ce n’est pas que je ne sois pas
persécutée pour cela, ajouta-t-elle, en dévoilant dans un sourire une
satisfaction de vanité qu’elle ne s’apercevait plus ne pas pouvoir paraître
légitime à Swann. Il y en a une qui est encore restée plus de deux heures hier à
m’attendre, elle me proposait n’importe quel prix. Il paraît qu’il y a un
ambassadeur qui lui a dit: «Je me tue si vous ne me l’amenez pas.» On lui a dit
que j’étais sortie, j’ai fini par aller moi-même lui parler pour qu’elle s’en
aille. J’aurais voulu que tu voies comme je l’ai reçue, ma femme de chambre qui
m’entendait de la pièce voisine m’a dit que je criais à tue-tête: «Mais puisque
je vous dis que je ne veux pas! C’est une idée comme ça, ça ne me plaît pas. Je
pense que je suis libre de faire ce que je veux tout de même! Si j’avais besoin
d’argent, je comprends . . . » Le concierge a ordre de ne plus la laisser
entrer, il dira que je suis à la campagne. Ah! j’aurais voulu que tu sois caché
quelque part. Je crois que tu aurais été content, mon chéri. Elle a du bon, tout
de même, tu vois, ta petite Odette, quoiqu’on la trouve si détestable.»

D’ailleurs ses aveux même, quand elle lui en faisait, de fautes qu’elle le
supposait avoir découvertes, servaient plutôt pour Swann de point de départ à de
nouveaux doutes qu’ils ne mettaient un terme aux anciens. Car ils n’étaient
jamais exactement proportionnés à ceux-ci. Odette avait eu beau retrancher de sa
confession tout l’essentiel, il restait dans l’accessoire quelque chose que
Swann n’avait jamais imaginé, qui l’accablait de sa nouveauté et allait lui
permettre de changer les termes du problème de sa jalousie. Et ces aveux il ne
pouvait plus les oublier. Son âme les charriait, les rejetait, les berçait,
comme des cadavres. Et elle en était empoisonnée.

Une fois elle lui parla d’une visite que Forcheville lui avait faite le jour de
la Fête de Paris-Murcie. «Comment, tu le connaissais déjà? Ah! oui, c’est vrai,
dit-il en se reprenant pour ne pas paraître l’avoir ignoré.» Et tout d’un coup
il se mit à trembler à la pensée que le jour de cette fête de Paris-Murcie où il
avait reçu d’elle la lettre qu’il avait si précieusement gardée, elle déjeunait
peut-être avec Forcheville à la Maison d’Or. Elle lui jura que non. «Pourtant la
Maison d’Or me rappelle je ne sais quoi que j’ai su ne pas être vrai, lui dit-il
pour l’effrayer.»—«Oui, que je n’y étais pas allée le soir où je t’ai dit que
j’en sortais quand tu m’avais cherchée chez Prévost», lui répondit-elle (croyant
à son air qu’il le savait), avec une décision où il y avait, beaucoup plus que
du cynisme, de la timidité, une peur de contrarier Swann et que par amour-propre
elle voulait cacher, puis le désir de lui montrer qu’elle pouvait être franche.
Aussi frappa-t-elle avec une netteté et une vigueur de bourreau et qui étaient
exemptes de cruauté car Odette n’avait pas conscience du mal qu’elle faisait à
Swann; et même elle se mit à rire, peut-être il est vrai, surtout pour ne pas
avoir l’air humilié, confus. «C’est vrai que je n’avais pas été à la Maison
Dorée, que je sortais de chez Forcheville. J’avais vraiment été chez Prévost, ça
c’était pas de la blague, il m’y avait rencontrée et m’avait demandé d’entrer
regarder ses gravures. Mais il était venu quelqu’un pour le voir. Je t’ai dit
que je venais de la Maison d’Or parce que j’avais peur que cela ne t’ennuie. Tu
vois, c’était plutôt gentil de ma part. Mettons que j’aie eu tort, au moins je
te le dis carrément. Quel intérêt aurais-je à ne pas te dire aussi bien que
j’avais déjeuné avec lui le jour de la Fête Paris-Murcie, si c’était vrai?
D’autant plus qu’à ce moment-là on ne se connaissait pas encore beaucoup tous
les deux, dis, chéri.» Il lui sourit avec la lâcheté soudaine de l’être sans
forces qu’avaient fait de lui ces accablantes paroles. Ainsi, même dans les mois
auxquels il n’avait jamais plus osé repenser parce qu’ils avaient été trop
heureux, dans ces mois où elle l’avait aimé, elle lui mentait déjà! Aussi bien
que ce moment (le premier soir qu’ils avaient «fait catleya») où elle lui avait
dit sortir de la Maison Dorée, combien devait-il y en avoir eu d’autres,
recéleurs eux aussi d’un mensonge que Swann n’avait pas soupçonné. Il se rappela
qu’elle lui avait dit un jour: «Je n’aurais qu’à dire à Mme Verdurin que ma robe
n’a pas été prête, que mon cab est venu en retard. Il y a toujours moyen de
s’arranger.» A lui aussi probablement, bien des fois où elle lui avait glissé de
ces mots qui expliquent un retard, justifient un changement d’heure dans un
rendezvous, ils avaient dû cacher sans qu’il s’en fût douté alors, quelque chose
qu’elle avait à faire avec un autre à qui elle avait dit: «Je n’aurai qu’à dire
à Swann que ma robe n’a pas été prête, que mon cab est arrivé en retard, il y a
toujours moyen de s’arranger.» Et sous tous les souvenirs les plus doux de
Swann, sous les paroles les plus simples que lui avait dites autrefois Odette,
qu’il avait crues comme paroles d’évangile, sous les actions quotidiennes
qu’elle lui avait racontées, sous les lieux les plus accoutumés, la maison de sa
couturière, l’avenue du Bois, l’Hippodrome, il sentait (dissimulée à la faveur
de cet excédent de temps qui dans les journées les plus détaillées laisse encore
du jeu, de la place, et peut servir de cachette à certaines actions), il sentait
s’insinuer la présence possible et souterraine de mensonges qui lui rendaient
ignoble tout ce qui lui était resté le plus cher, ses meilleurs soirs, la rue La
Pérouse elle-même, qu’Odette avait toujours dû quitter à d’autres heures que
celles qu’elle lui avait dites, faisant circuler partout un peu de la ténébreuse
horreur qu’il avait ressentie en entendant l’aveu relatif à la Maison Dorée, et,
comme les bêtes immondes dans la Désolation de Ninive, ébranlant pierre à pierre
tout son passé. Si maintenant il se détournait chaque fois que sa mémoire lui
disait le nom cruel de la Maison Dorée, ce n’était plus comme tout récemment
encore à la soirée de Mme de Saint-Euverte, parce qu’il lui rappelait un bonheur
qu’il avait perdu depuis longtemps, mais un malheur qu’il venait seulement
d’apprendre. Puis il en fut du nom de la Maison Dorée comme de celui de l’Ile du
Bois, il cessa peu à peu de faire souffrir Swann. Car ce que nous croyons notre
amour, notre jalousie, n’est pas une même passion continue, indivisible. Ils se
composent d’une infinité d’amours successifs, de jalousies différentes et qui
sont éphémères, mais par leur multitude ininterrompue donnent l’impression de la
continuité, l’illusion de l’unité. La vie de l’amour de Swann, la fidélité de sa
jalousie, étaient faites de la mort, de l’infidélité, d’innombrables désirs,
d’innombrables doutes, qui avaient tous Odette pour objet. S’il était resté
longtemps sans la voir, ceux qui mouraient n’auraient pas été remplacés par
d’autres. Mais la présence d’Odette continuait d’ensemencer le cœur de Swann de
tendresse et de soupçons alternés.

Certains soirs elle redevenait tout d’un coup avec lui d’une gentillesse dont
elle l’avertissait durement qu’il devait profiter tout de suite, sous peine de
ne pas la voir se renouveler avant des années; il fallait rentrer immédiatement
chez elle «faire catleya» et ce désir qu’elle prétendait avoir de lui était si
soudain, si inexplicable, si impérieux, les caresses qu’elle lui prodiguait
ensuite si démonstratives et si insolites, que cette tendresse brutale et sans
vraisemblance faisait autant de chagrin à Swann qu’un mensonge et qu’une
méchanceté. Un soir qu’il était ainsi, sur l’ordre qu’elle lui en avait donné,
rentré avec elle, et qu’elle entremêlait ses baisers de paroles passionnées qui
contrastaient avec sa sécheresse ordinaire, il crut tout d’un coup entendre du
bruit; il se leva, chercha partout, ne trouva personne, mais n’eut pas le
courage de reprendre sa place auprès d’elle qui alors, au comble de la rage,
brisa un vase et dit à Swann: «On ne peut jamais rien faire avec toi!» Et il
resta incertain si elle n’avait pas caché quelqu’un dont elle avait voulu faire
souffrir la jalousie ou allumer les sens.

Quelquefois il allait dans des maisons de rendezvous, espérant apprendre quelque
chose d’elle, sans oser la nommer cependant. «J’ai une petite qui va vous
plaire», disait l’entremetteuse.» Et il restait une heure à causer tristement
avec quelque pauvre fille étonnée qu’il ne fit rien de plus. Une toute jeune et
ravissante lui dit un jour: «Ce que je voudrais, c’est trouver un ami, alors il
pourrait être sûr, je n’irais plus jamais avec personne.»—«Vraiment, crois-tu
que ce soit possible qu’une femme soit touchée qu’on l’aime, ne vous trompe
jamais?» lui demanda Swann anxieusement. «Pour sûr! ça dépend des caractères!»
Swann ne pouvait s’empêcher de dire à ces filles les mêmes choses qui auraient
plu à la princesse des Laumes. A celle qui cherchait un ami, il dit en souriant:
«C’est gentil, tu as mis des yeux bleus de la couleur de ta ceinture.»—«Vous
aussi, vous avez des manchettes bleues.»—«Comme nous avons une belle
conversation, pour un endroit de ce genre! Je ne t’ennuie pas, tu as peut-être à
faire?»—«Non, j’ai tout mon temps. Si vous m’aviez ennuyée, je vous l’aurais
dit. Au contraire j’aime bien vous entendre causer.»—«Je suis très flatté.
N’est-ce pas que nous causons gentiment?» dit-il à l’entremetteuse qui venait
d’entrer. —«Mais oui, c’est justement ce que je me disais. Comme ils sont sages!
Voilà! on vient maintenant pour causer chez moi. Le Prince le disait, l’autre
jour, c’est bien mieux ici que chez sa femme. Il paraît que maintenant dans le
monde elles ont toutes un genre, c’est un vrai scandale! Je vous quitte, je suis
discrète.» Et elle laissa Swann avec la fille qui avait les yeux bleus. Mais
bientôt il se leva et lui dit adieu, elle lui était indifférente, elle ne
connaissait pas Odette.

Le peintre ayant été malade, le docteur Cottard lui conseilla un voyage en mer;
plusieurs fidèles parlèrent de partir avec lui; les Verdurin ne purent se
résoudre à rester seuls, louèrent un yacht, puis s’en rendirent acquéreurs et
ainsi Odette fit de fréquentes croisières. Chaque fois qu’elle était partie
depuis un peu de temps, Swann sentait qu’il commençait à se détacher d’elle,
mais comme si cette distance morale était proportionnée à la distance
matérielle, dès qu’il savait Odette de retour, il ne pouvait pas rester sans la
voir. Une fois, partis pour un mois seulement, croyaient-ils, soit qu’ils
eussent été tentés en route, soit que M. Verdurin eût sournoisement arrangé les
choses d’avance pour faire plaisir à sa femme et n’eût averti les fidèles qu’au
fur et à mesure, d’Alger ils allèrent à Tunis, puis en Italie, puis en Grèce, à
Constantinople, en Asie Mineure. Le voyage durait depuis près d’un an. Swann se
sentait absolument tranquille, presque heureux. Bien que M. Verdurin eût cherché
à persuader au pianiste et au docteur Cottard que la tante de l’un et les
malades de l’autre n’avaient aucun besoin d’eux, et, qu’en tous cas, il était
imprudent de laisser Mme Cottard rentrer à Paris que Mme Verdurin assurait être
en révolution, il fut obligé de leur rendre leur liberté à Constantinople. Et le
peintre partit avec eux. Un jour, peu après le retour de ces trois voyageurs,
Swann voyant passer un omnibus pour le Luxembourg où il avait à faire, avait
sauté dedans, et s’y était trouvé assis en face de Mme Cottard qui faisait sa
tournée de visites «de jours» en grande tenue, plumet au chapeau, robe de soie,
manchon, en-tout-cas, porte-cartes et gants blancs nettoyés. Revêtue de ces
insignes, quand il faisait sec, elle allait à pied d’une maison à l’autre, dans
un même quartier, mais pour passer ensuite dans un quartier différent usait de
l’omnibus avec correspondance. Pendant les premiers instants, avant que la
gentillesse native de la femme eût pu percer l’empesé de la petite bourgeoise,
et ne sachant trop d’ailleurs si elle devait parler des Verdurin à Swann, elle
tint tout naturellement, de sa voix lente, gauche et douce que par moments
l’omnibus couvrait complètement de son tonnerre, des propos choisis parmi ceux
qu’elle entendait et répétait dans les vingt-cinq maisons dont elle montait les
étages dans une journée:

—«Je ne vous demande pas, monsieur, si un homme dans le mouvement comme vous, a
vu, aux Mirlitons, le portrait de Machard qui fait courir tout Paris. Eh bien!
qu’en dites-vous? Etes-vous dans le camp de ceux qui approuvent ou dans le camp
de ceux qui blâment? Dans tous les salons on ne parle que du portrait de
Machard, on n’est pas chic, on n’est pas pur, on n’est pas dans le train, si on
ne donne pas son opinion sur le portrait de Machard.»

Swann ayant répondu qu’il n’avait pas vu ce portrait, Mme Cottard eut peur de
l’avoir blessé en l’obligeant à le confesser.

—«Ah! c’est très bien, au moins vous l’avouez franchement, vous ne vous croyez
pas déshonoré parce que vous n’avez pas vu le portrait de Machard. Je trouve
cela très beau de votre part. Hé bien, moi je l’ai vu, les avis sont partagés,
il y en a qui trouvent que c’est un peu léché, un peu crème fouettée, moi, je le
trouve idéal. Évidemment elle ne ressemble pas aux femmes bleues et jaunes de
notre ami Biche. Mais je dois vous l’avouer franchement, vous ne me trouverez
pas très fin de siècle, mais je le dis comme je le pense, je ne comprends pas.
Mon Dieu je reconnais les qualités qu’il y a dans le portrait de mon mari, c’est
moins étrange que ce qu’il fait d’habitude mais il a fallu qu’il lui fasse des
moustaches bleues. Tandis que Machard! Tenez justement le mari de l’amie chez
qui je vais en ce moment (ce qui me donne le très grand plaisir de faire route
avec vous) lui a promis s’il est nommé à l’Académie (c’est un des collègues du
docteur) de lui faire faire son portrait par Machard. Évidemment c’est un beau
rêve! j’ai une autre amie qui prétend qu’elle aime mieux Leloir. Je ne suis
qu’une pauvre profane et Leloir est peut-être encore supérieur comme science.
Mais je trouve que la première qualité d’un portrait, surtout quand il coûte
10.000 francs, est d’être ressemblant et d’une ressemblance agréable.»

Ayant tenu ces propos que lui inspiraient la hauteur de son aigrette, le chiffre
de son porte-cartes, le petit numéro tracé à l’encre dans ses gants par le
teinturier, et l’embarras de parler à Swann des Verdurin, Mme Cottard, voyant
qu’on était encore loin du coin de la rue Bonaparte où le conducteur devait
l’arrêter, écouta son cœur qui lui conseillait d’autres paroles.

— Les oreilles ont dû vous tinter, monsieur, lui dit-elle, pendant le voyage que
nous avons fait avec Mme Verdurin. On ne parlait que de vous.

Swann fut bien étonné, il supposait que son nom n’était jamais proféré devant
les Verdurin.

— D’ailleurs, ajouta Mme Cottard, Mme de Crécy était là et c’est tout dire.
Quand Odette est quelque part elle ne peut jamais rester bien longtemps sans
parler de vous. Et vous pensez que ce n’est pas en mal. Comment! vous en doutez,
dit-elle, en voyant un geste sceptique de Swann?

Et emportée par la sincérité de sa conviction, ne mettant d’ailleurs aucune
mauvaise pensée sous ce mot qu’elle prenait seulement dans le sens où on
l’emploie pour parler de l’affection qui unit des amis:

— Mais elle vous adore! Ah! je crois qu’il ne faudrait pas dire ça de vous
devant elle! On serait bien arrangé! A propos de tout, si on voyait un tableau
par exemple elle disait: «Ah! s’il était là, c’est lui qui saurait vous dire si
c’est authentique ou non. Il n’y a personne comme lui pour ça.» Et à tout moment
elle demandait: «Qu’est-ce qu’il peut faire en ce moment? Si seulement il
travaillait un peu! C’est malheureux, un garçon si doué, qu’il soit si
paresseux. (Vous me pardonnez, n’est-ce pas?)» En ce moment je le vois, il pense
à nous, il se demande où nous sommes.» Elle a même eu un mot que j’ai trouvé
bien joli; M. Verdurin lui disait: «Mais comment pouvez-vous voir ce qu’il fait
en ce moment puisque vous êtes à huit cents lieues de lui?» Alors Odette lui a
répondu: «Rien n’est impossible à l’œil d’une amie.» Non je vous jure, je ne
vous dis pas cela pour vous flatter, vous avez là une vraie amie comme on n’en a
pas beaucoup. Je vous dirai du reste que si vous ne le savez pas, vous êtes le
seul. Mme Verdurin me le disait encore le dernier jour (vous savez les veilles
de départ on cause mieux): «Je ne dis pas qu’Odette ne nous aime pas, mais tout
ce que nous lui disons ne pèserait pas lourd auprès de ce que lui dirait M.
Swann.» Oh! mon Dieu, voilà que le conducteur m’arrête, en bavardant avec vous
j’allais laisser passer la rue Bonaparte . . . me rendriez-vous le service de me
dire si mon aigrette est droite?»

Et Mme Cottard sortit de son manchon pour la tendre à Swann sa main gantée de
blanc d’où s’échappa, avec une correspondance, une vision de haute vie qui
remplit l’omnibus, mêlée à l’odeur du teinturier. Et Swann se sentit déborder de
tendresse pour elle, autant que pour Mme Verdurin (et presque autant que pour
Odette, car le sentiment qu’il éprouvait pour cette dernière n’étant plus mêlé
de douleur, n’était plus guère de l’amour), tandis que de la plate-forme il la
suivait de ses yeux attendris, qui enfilait courageusement la rue Bonaparte,
l’aigrette haute, d’une main relevant sa jupe, de l’autre tenant son en-tout-cas
et son porte-cartes dont elle laissait voir le chiffre, laissant baller devant
elle son manchon.

Pour faire concurrence aux sentiments maladifs que Swann avait pour Odette, Mme
Cottard, meilleur thérapeute que n’eût été son mari, avait greffé à côté d’eux
d’autres sentiments, normaux ceux-là, de gratitude, d’amitié, des sentiments qui
dans l’esprit de Swann rendraient Odette plus humaine (plus semblable aux autres
femmes, parce que d’autres femmes aussi pouvaient les lui inspirer), hâteraient
sa transformation définitive en cette Odette aimée d’affection paisible, qui
l’avait ramené un soir après une fête chez le peintre boire un verre d’orangeade
avec Forcheville et près de qui Swann avait entrevu qu’il pourrait vivre
heureux.

Jadis ayant souvent pensé avec terreur qu’un jour il cesserait d’être épris
d’Odette, il s’était promis d’être vigilant, et dès qu’il sentirait que son
amour commencerait à le quitter, de s’accrocher à lui, de le retenir. Mais voici
qu’à l’affaiblissement de son amour correspondait simultanément un
affaiblissement du désir de rester amoureux. Car on ne peut pas changer,
c’est-à-dire devenir une autre personne, tout en continuant à obéir aux
sentiments de celle qu’on n’est plus. Parfois le nom aperçu dans un journal,
d’un des hommes qu’il supposait avoir pu être les amants d’Odette, lui redonnait
de la jalousie. Mais elle était bien légère et comme elle lui prouvait qu’il
n’était pas encore complètement sorti de ce temps où il avait tant souffert —
mais aussi où il avait connu une manière de sentir si voluptueuse — et que les
hasards de la route lui permettraient peut-être d’en apercevoir encore
furtivement et de loin les beautés, cette jalousie lui procurait plutôt une
excitation agréable comme au morne Parisien qui quitte Venise pour retrouver la
France, un dernier moustique prouve que l’Italie et l’été ne sont pas encore
bien loin. Mais le plus souvent le temps si particulier de sa vie d’où il
sortait, quand il faisait effort sinon pour y rester, du moins pour en avoir une
vision claire pendant qu’il le pouvait encore, il s’apercevait qu’il ne le
pouvait déjà plus; il aurait voulu apercevoir comme un paysage qui allait
disparaître cet amour qu’il venait de quitter; mais il est si difficile d’être
double et de se donner le spectacle véridique d’un sentiment qu’on a cessé de
posséder, que bientôt l’obscurité se faisant dans son cerveau, il ne voyait plus
rien, renonçait à regarder, retirait son lorgnon, en essuyait les verres; et il
se disait qu’il valait mieux se reposer un peu, qu’il serait encore temps tout à
l’heure, et se rencognait, avec l’incuriosité, dans l’engourdissement, du
voyageur ensommeillé qui rabat son chapeau sur ses yeux pour dormir dans le
wagon qu’il sent l’entraîner de plus en plus vite, loin du pays, où il a si
longtemps vécu et qu’il s’était promis de ne pas laisser fuir sans lui donner un
dernier adieu. Même, comme ce voyageur s’il se réveille seulement en France,
quand Swann ramassa par hasard près de lui la preuve que Forcheville avait été
l’amant d’Odette, il s’aperçut qu’il n’en ressentait aucune douleur, que l’amour
était loin maintenant et regretta de n’avoir pas été averti du moment où il le
quittait pour toujours. Et de même qu’avant d’embrasser Odette pour la première
fois il avait cherché à imprimer dans sa mémoire le visage qu’elle avait eu si
longtemps pour lui et qu’allait transformer le souvenir de ce baiser, de même il
eût voulu, en pensée au moins, avoir pu faire ses adieux, pendant qu’elle
existait encore, à cette Odette lui inspirant de l’amour, de la jalousie, à
cette Odette lui causant des souffrances et que maintenant il ne reverrait
jamais. Il se trompait. Il devait la revoir une fois encore, quelques semaines
plus tard. Ce fut en dormant, dans le crépuscule d’un rêve. Il se promenait avec
Mme Verdurin, le docteur Cottard, un jeune homme en fez qu’il ne pouvait
identifier, le peintre, Odette, Napoléon III et mon grand-père, sur un chemin
qui suivait la mer et la surplombait à pic tantôt de très haut, tantôt de
quelques mètres seulement, de sorte qu’on montait et redescendait constamment;
ceux des promeneurs qui redescendaient déjà n’étaient plus visibles à ceux qui
montaient encore, le peu de jour qui restât faiblissait et il semblait alors
qu’une nuit noire allait s’étendre immédiatement. Par moment les vagues
sautaient jusqu’au bord et Swann sentait sur sa joue des éclaboussures glacées.
Odette lui disait de les essuyer, il ne pouvait pas et en était confus vis-à-vis
d’elle, ainsi que d’être en chemise de nuit. Il espérait qu’à cause de
l’obscurité on ne s’en rendait pas compte, mais cependant Mme Verdurin le fixa
d’un regard étonné durant un long moment pendant lequel il vit sa figure se
déformer, son nez s’allonger et qu’elle avait de grandes moustaches. Il se
détourna pour regarder Odette, ses joues étaient pâles, avec des petits points
rouges, ses traits tirés, cernés, mais elle le regardait avec des yeux pleins de
tendresse prêts à se détacher comme des larmes pour tomber sur lui et il se
sentait l’aimer tellement qu’il aurait voulu l’emmener tout de suite. Tout d’un
coup Odette tourna son poignet, regarda une petite montre et dit: «Il faut que
je m’en aille», elle prenait congé de tout le monde, de la même façon, sans
prendre à part à Swann, sans lui dire où elle le reverrait le soir ou un autre
jour. Il n’osa pas le lui demander, il aurait voulu la suivre et était obligé,
sans se retourner vers elle, de répondre en souriant à une question de Mme
Verdurin, mais son cœur battait horriblement, il éprouvait de la haine pour
Odette, il aurait voulu crever ses yeux qu’il aimait tant tout à l’heure,
écraser ses joues sans fraîcheur. Il continuait à monter avec Mme Verdurin,
c’est-à-dire à s’éloigner à chaque pas d’Odette, qui descendait en sens inverse.
Au bout d’une seconde il y eut beaucoup d’heures qu’elle était partie. Le
peintre fit remarquer à Swann que Napoléon III s’était éclipsé un instant après
elle. «C’était certainement entendu entre eux, ajouta-t-il, ils ont dû se
rejoindre en bas de la côte mais n’ont pas voulu dire adieu ensemble à cause des
convenances. Elle est sa maîtresse.» Le jeune homme inconnu se mit à pleurer.
Swann essaya de le consoler. «Après tout elle a raison, lui dit-il en lui
essuyant les yeux et en lui ôtant son fez pour qu’il fût plus à son aise. Je le
lui ai conseillé dix fois. Pourquoi en être triste? C’était bien l’homme qui
pouvait la comprendre.» Ainsi Swann se parlait-il à lui-même, car le jeune homme
qu’il n’avait pu identifier d’abord était aussi lui; comme certains romanciers,
il avait distribué sa personnalité à deux personnages, celui qui faisait le
rêve, et un qu’il voyait devant lui coiffé d’un fez.

Quant à Napoléon III, c’est à Forcheville que quelque vague association d’idées,
puis une certaine modification dans la physionomie habituelle du baron, enfin le
grand cordon de la Légion d’honneur en sautoir, lui avaient fait donner ce nom;
mais en réalité, et pour tout ce que le personnage présent dans le rêve lui
représentait et lui rappelait, c’était bien Forcheville. Car, d’images
incomplètes et changeantes Swann endormi tirait des déductions fausses, ayant
d’ailleurs momentanément un tel pouvoir créateur qu’il se reproduisait par
simple division comme certains organismes inférieurs; avec la chaleur sentie de
sa propre paume il modelait le creux d’une main étrangère qu’il croyait serrer
et, de sentiments et d’impressions dont il n’avait pas conscience encore faisait
naître comme des péripéties qui, par leur enchaînement logique amèneraient à
point nommé dans le sommeil de Swann le personnage nécessaire pour recevoir son
amour ou provoquer son réveil. Une nuit noire se fit tout d’un coup, un tocsin
sonna, des habitants passèrent en courant, se sauvant des maisons en flammes;
Swann entendait le bruit des vagues qui sautaient et son cœur qui, avec la même
violence, battait d’anxiété dans sa poitrine. Tout d’un coup ses palpitations de
cœur redoublèrent de vitesse, il éprouva une souffrance, une nausée
inexplicables; un paysan couvert de brûlures lui jetait en passant: «Venez
demander à Charlus où Odette est allée finir la soirée avec son camarade, il a
été avec elle autrefois et elle lui dit tout. C’est eux qui ont mis le feu.»
C’était son valet de chambre qui venait l’éveiller et lui disait:

— Monsieur, il est huit heures et le coiffeur est là, je lui ai dit de repasser
dans une heure.

Mais ces paroles en pénétrant dans les ondes du sommeil où Swann était plongé,
n’étaient arrivées jusqu’à sa conscience qu’en subissant cette déviation qui
fait qu’au fond de l’eau un rayon paraît un soleil, de même qu’un moment
auparavant le bruit de la sonnette prenant au fond de ces abîmes une sonorité de
tocsin avait enfanté l’épisode de l’incendie. Cependant le décor qu’il avait
sous les yeux vola en poussière, il ouvrit les yeux, entendit une dernière fois
le bruit d’une des vagues de la mer qui s’éloignait. Il toucha sa joue. Elle
était sèche. Et pourtant il se rappelait la sensation de l’eau froide et le goût
du sel. Il se leva, s’habilla. Il avait fait venir le coiffeur de bonne heure
parce qu’il avait écrit la veille à mon grand-père qu’il irait dans l’après-midi
à Combray, ayant appris que Mme de Cambremer — Mlle Legrandin — devait y passer
quelques jours. Associant dans son souvenir au charme de ce jeune visage celui
d’une campagne où il n’était pas allé depuis si longtemps, ils lui offraient
ensemble un attrait qui l’avait décidé à quitter enfin Paris pour quelques
jours. Comme les différents hasards qui nous mettent en présence de certaines
personnes ne coïncident pas avec le temps où nous les aimons, mais, le
dépassant, peuvent se produire avant qu’il commence et se répéter après qu’il a
fini, les premières apparitions que fait dans notre vie un être destiné plus
tard à nous plaire, prennent rétrospectivement à nos yeux une valeur
d’avertissement, de présage. C’est de cette façon que Swann s’était souvent
reporté à l’image d’Odette rencontrée au théâtre, ce premier soir où il ne
songeait pas à la revoir jamais — et qu’il se rappelait maintenant la soirée de
Mme de Saint-Euverte où il avait présenté le général de Froberville à Mme de
Cambremer. Les intérêts de notre vie sont si multiples qu’il n’est pas rare que
dans une même circonstance les jalons d’un bonheur qui n’existe pas encore
soient posés à côté de l’aggravation d’un chagrin dont nous souffrons. Et sans
doute cela aurait pu arriver à Swann ailleurs que chez Mme de Saint-Euverte. Qui
sait même, dans le cas où, ce soir-là, il se fût trouvé ailleurs, si d’autres
bonheurs, d’autres chagrins ne lui seraient pas arrivés, et qui ensuite lui
eussent paru avoir été inévitables? Mais ce qui lui semblait l’avoir été,
c’était ce qui avait eu lieu, et il n’était pas loin de voir quelque chose de
providentiel dans ce qu’il se fût décidé à aller à la soirée de Mme de
Saint-Euverte, parce que son esprit désireux d’admirer la richesse d’invention
de la vie et incapable de se poser longtemps une question difficile, comme de
savoir ce qui eût été le plus à souhaiter, considérait dans les souffrances
qu’il avait éprouvées ce soir-là et les plaisirs encore insoupçonnés qui
germaient déjà — et entre lesquels la balance était trop difficile à établir —
une sorte d’enchaînement nécessaire.

Mais tandis que, une heure après son réveil, il donnait des indications au
coiffeur pour que sa brosse ne se dérangeât pas en wagon, il repensa à son rêve,
il revit comme il les avait sentis tout près de lui, le teint pâle d’Odette, les
joues trop maigres, les traits tirés, les yeux battus, tout ce que — au cours
des tendresses successives qui avaient fait de son durable amour pour Odette un
long oubli de l’image première qu’il avait reçue d’elle — il avait cessé de
remarquer depuis les premiers temps de leur liaison dans lesquels sans doute,
pendant qu’il dormait, sa mémoire en avait été chercher la sensation exacte. Et
avec cette muflerie intermittente qui reparaissait chez lui dès qu’il n’était
plus malheureux et que baissait du même coup le niveau de sa moralité, il
s’écria en lui-même: «Dire que j’ai gâché des années de ma vie, que j’ai voulu
mourir, que j’ai eu mon plus grand amour, pour une femme qui ne me plaisait pas,
qui n’était pas mon genre!»


%%%%%%%%%%%%%%%%%%%%%%%%%%%%%%%%%%%%%%%%%%%%%%%%%%%
\part{Noms De Pays: Le Nom}


Parmi les chambres dont j’évoquais le plus souvent l’image dans mes nuits
d’insomnie, aucune ne ressemblait moins aux chambres de Combray, saupoudrées
d’une atmosphère grenue, pollinisée, comestible et dévote, que celle du
Grand-Hôtel de la Plage, à Balbec, dont les murs passés au ripolin contenaient
comme les parois polies d’une piscine où l’eau bleuit, un air pur, azuré et
salin. Le tapissier bavarois qui avait été chargé de l’aménagement de cet hôtel
avait varié la décoration des pièces et sur trois côtés, fait courir le long des
murs, dans celle que je me trouvai habiter, des bibliothèques basses, à vitrines
en glace, dans lesquelles selon la place qu’elles occupaient, et par un effet
qu’il n’avait pas prévu, telle ou telle partie du tableau changeant de la mer se
reflétait, déroulant une frise de claires marines, qu’interrompaient seuls les
pleins de l’acajou. Si bien que toute la pièce avait l’air d’un de ces dortoirs
modèles qu’on présente dans les expositions «modern style» du mobilier où ils
sont ornés d’œuvres d’art qu’on a supposées capables de réjouir les yeux de
celui qui couchera là et auxquelles on a donné des sujets en rapport avec le
genre de site où l’habitation doit se trouver.

Mais rien ne ressemblait moins non plus à ce Balbec réel que celui dont j’avais
souvent rêvé, les jours de tempête, quand le vent était si fort que Françoise en
me menant aux Champs-Élysées me recommandait de ne pas marcher trop près des
murs pour ne pas recevoir de tuiles sur la tête et parlait en gémissant des
grands sinistres et naufrages annoncés par les journaux. Je n’avais pas de plus
grand désir que de voir une tempête sur la mer, moins comme un beau spectacle
que comme un moment dévoilé de la vie réelle de la nature; ou plutôt il n’y
avait pour moi de beaux spectacles que ceux que je savais qui n’étaient pas
artificiellement combinés pour mon plaisir, mais étaient nécessaires,
inchangeables — les beautés des paysages ou du grand art. Je n’étais curieux, je
n’étais avide de connaître que ce que je croyais plus vrai que moi-même, ce qui
avait pour moi le prix de me montrer un peu de la pensée d’un grand génie, ou de
la force ou de la grâce de la nature telle qu’elle se manifeste livrée à
elle-même, sans l’intervention des hommes. De même que le beau son de sa voix,
isolément reproduit par le phonographe, ne nous consolerait pas d’avoir perdu
notre mère, de même une tempête mécaniquement imitée m’aurait laissé aussi
indifférent que les fontaines lumineuses de l’Exposition. Je voulais aussi pour
que la tempête fût absolument vraie, que le rivage lui-même fût un rivage
naturel, non une digue récemment créée par une municipalité. D’ailleurs la
nature par tous les sentiments qu’elle éveillait en moi, me semblait ce qu’il y
avait de plus opposé aux productions mécaniques des hommes. Moins elle portait
leur empreinte et plus elle offrait d’espace à l’expansion de mon cœur. Or
j’avais retenu le nom de Balbec que nous avait cité Legrandin, comme d’une plage
toute proche de «ces côtes funèbres, fameuses par tant de naufrages
qu’enveloppent six mois de l’année le linceul des brumes et l’écume des vagues».

«On y sent encore sous ses pas, disait-il, bien plus qu’au Finistère lui-même
(et quand bien même des hôtels s’y superposeraient maintenant sans pouvoir y
modifier la plus antique ossature de la terre), on y sent la véritable fin de la
terre française, européenne, de la Terre antique. Et c’est le dernier campement
de pêcheurs, pareils à tous les pêcheurs qui ont vécu depuis le commencement du
monde, en face du royaume éternel des brouillards de la mer et des ombres.» Un
jour qu’à Combray j’avais parlé de cette plage de Balbec devant M. Swann afin
d’apprendre de lui si c’était le point le mieux choisi pour voir les plus fortes
tempêtes, il m’avait répondu: «Je crois bien que je connais Balbec! L’église de
Balbec, du XIIe et XIIIe siècle, encore à moitié romane, est peut-être le plus
curieux échantillon du gothique normand, et si singulière, on dirait de l’art
persan.» Et ces lieux qui jusque-là ne m’avaient semblé que de la nature
immémoriale, restée contemporaine des grands phénomènes géologiques — et tout
aussi en dehors de l’histoire humaine que l’Océan ou la grande Ourse, avec ces
sauvages pêcheurs pour qui, pas plus que pour les baleines, il n’y eut de moyen
âge — ç’avait été un grand charme pour moi de les voir tout d’un coup entrés
dans la série des siècles, ayant connu l’époque romane, et de savoir que le
trèfle gothique était venu nervurer aussi ces rochers sauvages à l’heure voulue,
comme ces plantes frêles mais vivaces qui, quand c’est le printemps, étoilent çà
et là la neige des pôles. Et si le gothique apportait à ces lieux et à ces
hommes une détermination qui leur manquait, eux aussi lui en conféraient une en
retour. J’essayais de me représenter comment ces pêcheurs avaient vécu, le
timide et insoupçonné essai de rapports sociaux qu’ils avaient tenté là, pendant
le moyen âge, ramassés sur un point des côtes d’Enfer, aux pieds des falaises de
la mort; et le gothique me semblait plus vivant maintenant que, séparé des
villes où je l’avais toujours imaginé jusque-là, je pouvais voir comment, dans
un cas particulier, sur des rochers sauvages, il avait germé et fleuri en un fin
clocher. On me mena voir des reproductions des plus célèbres statues de Balbec —
les apôtres moutonnants et camus, la Vierge du porche, et de joie ma respiration
s’arrêtait dans ma poitrine quand je pensais que je pourrais les voir se modeler
en relief sur le brouillard éternel et salé. Alors, par les soirs orageux et
doux de février, le vent — soufflant dans mon cœur, qu’il ne faisait pas
trembler moins fort que la cheminée de ma chambre, le projet d’un voyage à
Balbec — mêlait en moi le désir de l’architecture gothique avec celui d’une
tempête sur la mer.

J’aurais voulu prendre dès le lendemain le beau train généreux d’une heure
vingt-deux dont je ne pouvais jamais sans que mon cœur palpitât lire, dans les
réclames des Compagnies de chemin de fer, dans les annonces de voyages
circulaires, l’heure de départ: elle me semblait inciser à un point précis de
l’après-midi une savoureuse entaille, une marque mystérieuse à partir de
laquelle les heures déviées conduisaient bien encore au soir, au matin du
lendemain, mais qu’on verrait, au lieu de Paris, dans l’une de ces villes par où
le train passe et entre lesquelles il nous permettait de choisir; car il
s’arrêtait à Bayeux, à Coutances, à Vitré, à Questambert, à Pontorson, à Balbec,
à Lannion, à Lamballe, à Benodet, à Pont-Aven, à Quimperlé, et s’avançait
magnifiquement surchargé de noms qu’il m’offrait et entre lesquels je ne savais
lequel j’aurais préféré, par impossibilité d’en sacrifier aucun. Mais sans même
l’attendre, j’aurais pu en m’habillant à la hâte partir le soir même, si mes
parents me l’avaient permis, et arriver à Balbec quand le petit jour se lèverait
sur la mer furieuse, contre les écumes envolées de laquelle j’irais me réfugier
dans l’église de style persan. Mais à l’approche des vacances de Pâques, quand
mes parents m’eurent promis de me les faire passer une fois dans le nord de
l’Italie, voilà qu’à ces rêves de tempête dont j’avais été rempli tout entier,
ne souhaitant voir que des vagues accourant de partout, toujours plus haut, sur
la côte la plus sauvage, près d’églises escarpées et rugueuses comme des
falaises et dans les tours desquelles crieraient les oiseaux de mer, voilà que
tout à coup les effaçant, leur ôtant tout charme, les excluant parce qu’ils lui
étaient opposés et n’auraient pu que l’affaiblir, se substituaient en moi le
rêve contraire du printemps le plus diapré, non pas le printemps de Combray qui
piquait encore aigrement avec toutes les aiguilles du givre, mais celui qui
couvrait déjà de lys et d’anémones les champs de Fiésole et éblouissait Florence
de fonds d’or pareils à ceux de l’Angelico. Dès lors, seuls les rayons, les
parfums, les couleurs me semblaient avoir du prix; car l’alternance des images
avait amené en moi un changement de front du désir, et — aussi brusque que ceux
qu’il y a parfois en musique, un complet changement de ton dans ma sensibilité.
Puis il arriva qu’une simple variation atmosphérique suffit à provoquer en moi
cette modulation sans qu’il y eût besoin d’attendre le retour d’une saison. Car
souvent dans l’une, on trouve égaré un jour d’une autre, qui nous y fait vivre,
en évoque aussitôt, en fait désirer les plaisirs particuliers et interrompt les
rêves que nous étions en train de faire, en plaçant, plus tôt ou plus tard qu’à
son tour, ce feuillet détaché d’un autre chapitre, dans le calendrier interpolé
du Bonheur. Mais bientôt comme ces phénomènes naturels dont notre confort ou
notre santé ne peuvent tirer qu’un bénéfice accidentel et assez mince jusqu’au
jour où la science s’empare d’eux, et les produisant à volonté, remet en nos
mains la possibilité de leur apparition, soustraite à la tutelle et dispensée de
l’agrément du hasard, de même la production de ces rêves d’Atlantique et
d’Italie cessa d’être soumise uniquement aux changements des saisons et du
temps. Je n’eus besoin pour les faire renaître que de prononcer ces noms:
Balbec, Venise, Florence, dans l’intérieur desquels avait fini par s’accumuler
le désir que m’avaient inspiré les lieux qu’ils désignaient. Même au printemps,
trouver dans un livre le nom de Balbec suffisait à réveiller en moi le désir des
tempêtes et du gothique normand; même par un jour de tempête le nom de Florence
ou de Venise me donnait le désir du soleil, des lys, du palais des Doges et de
Sainte-Marie-des-Fleurs.

Mais si ces noms absorbèrent à tout jamais l’image que j’avais de ces villes, ce
ne fut qu’en la transformant, qu’en soumettant sa réapparition en moi à leurs
lois propres; ils eurent ainsi pour conséquence de la rendre plus belle, mais
aussi plus différente de ce que les villes de Normandie ou de Toscane pouvaient
être en réalité, et, en accroissant les joies arbitraires de mon imagination,
d’aggraver la déception future de mes voyages. Ils exaltèrent l’idée que je me
faisais de certains lieux de la terre, en les faisant plus particuliers, par
conséquent plus réels. Je ne me représentais pas alors les villes, les paysages,
les monuments, comme des tableaux plus ou moins agréables, découpés çà et là
dans une même matière, mais chacun d’eux comme un inconnu, essentiellement
différent des autres, dont mon âme avait soif et qu’elle aurait profit à
connaître. Combien ils prirent quelque chose de plus individuel encore, d’être
désignés par des noms, des noms qui n’étaient que pour eux, des noms comme en
ont les personnes. Les mots nous présentent des choses une petite image claire
et usuelle comme celles que l’on suspend aux murs des écoles pour donner aux
enfants l’exemple de ce qu’est un établi, un oiseau, une fourmilière, choses
conçues comme pareilles à toutes celles de même sorte. Mais les noms présentent
des personnes — et des villes qu’ils nous habituent à croire individuelles,
uniques comme des personnes — une image confuse qui tire d’eux, de leur sonorité
éclatante ou sombre, la couleur dont elle est peinte uniformément comme une de
ces affiches, entièrement bleues ou entièrement rouges, dans lesquelles, à cause
des limites du procédé employé ou par un caprice du décorateur, sont bleus ou
rouges, non seulement le ciel et la mer, mais les barques, l’église, les
passants. Le nom de Parme, une des villes où je désirais le plus aller, depuis
que j’avais lu la Chartreuse, m’apparaissant compact, lisse, mauve et doux; si
on me parlait d’une maison quelconque de Parme dans laquelle je serais reçu, on
me causait le plaisir de penser que j’habiterais une demeure lisse, compacte,
mauve et douce, qui n’avait de rapport avec les demeures d’aucune ville d’Italie
puisque je l’imaginais seulement à l’aide de cette syllabe lourde du nom de
Parme, où ne circule aucun air, et de tout ce que je lui avais fait absorber de
douceur stendhalienne et du reflet des violettes. Et quand je pensais à
Florence, c’était comme à une ville miraculeusement embaumée et semblable à une
corolle, parce qu’elle s’appelait la cité des lys et sa cathédrale,
Sainte-Marie-des-Fleurs. Quant à Balbec, c’était un de ces noms où comme sur une
vieille poterie normande qui garde la couleur de la terre d’où elle fut tirée,
on voit se peindre encore la représentation de quelque usage aboli, de quelque
droit féodal, d’un état ancien de lieux, d’une manière désuète de prononcer qui
en avait formé les syllabes hétéroclites et que je ne doutais pas de retrouver
jusque chez l’aubergiste qui me servirait du café au lait à mon arrivée, me
menant voir la mer déchaînée devant l’église et auquel je prêtais l’aspect
disputeur, solennel et médiéval d’un personnage de fabliau.

Si ma santé s’affermissait et que mes parents me permissent, sinon d’aller
séjourner à Balbec, du moins de prendre une fois, pour faire connaissance avec
l’architecture et les paysages de la Normandie ou de la Bretagne, ce train d’une
heure vingt-deux dans lequel j’étais monté tant de fois en imagination, j’aurais
voulu m’arrêter de préférence dans les villes les plus belles; mais j’avais beau
les comparer, comment choisir plus qu’entre des êtres individuels, qui ne sont
pas interchangeables, entre Bayeux si haute dans sa noble dentelle rougeâtre et
dont le faîte était illuminé par le vieil or de sa dernière syllabe; Vitré dont
l’accent aigu losangeait de bois noir le vitrage ancien; le doux Lamballe qui,
dans son blanc, va du jaune coquille d’œuf au gris perle; Coutances, cathédrale
normande, que sa diphtongue finale, grasse et jaunissante couronne par une tour
de beurre; Lannion avec le bruit, dans son silence villageois, du coche suivi de
la mouche; Questambert, Pontorson, risibles et naïfs, plumes blanches et becs
jaunes éparpillés sur la route de ces lieux fluviatiles et poétiques; Benodet,
nom à peine amarré que semble vouloir entraîner la rivière au milieu de ses
algues, Pont-Aven, envolée blanche et rose de l’aile d’une coiffe légère qui se
reflète en tremblant dans une eau verdie de canal; Quimperlé, lui, mieux attaché
et, depuis le moyen âge, entre les ruisseaux dont il gazouille et s’emperle en
une grisaille pareille à celle que dessinent, à travers les toiles d’araignées
d’une verrière, les rayons de soleil changés en pointes émoussées d’argent
bruni?

Ces images étaient fausses pour une autre raison encore; c’est qu’elles étaient
forcément très simplifiées; sans doute ce à quoi aspirait mon imagination et que
mes sens ne percevaient qu’incomplètement et sans plaisir dans le présent, je
l’avais enfermé dans le refuge des noms; sans doute, parce que j’y avais
accumulé du rêve, ils aimantaient maintenant mes désirs; mais les noms ne sont
pas très vastes; c’est tout au plus si je pouvais y faire entrer deux ou trois
des «curiosités» principales de la ville et elles s’y juxtaposaient sans
intermédiaires; dans le nom de Balbec, comme dans le verre grossissant de ces
porte-plume qu’on achète aux bains de mer, j’apercevais des vagues soulevées
autour d’une église de style persan. Peut-être même la simplification de ces
images fut-elle une des causes de l’empire qu’elles prirent sur moi. Quand mon
père eut décidé, une année, que nous irions passer les vacances de Pâques à
Florence et à Venise, n’ayant pas la place de faire entrer dans le nom de
Florence les éléments qui composent d’habitude les villes, je fus contraint à
faire sortir une cité surnaturelle de la fécondation, par certains parfums
printaniers, de ce que je croyais être, en son essence, le génie de Giotto. Tout
au plus — et parce qu’on ne peut pas faire tenir dans un nom beaucoup plus de
durée que d’espace — comme certains tableaux de Giotto eux-mêmes qui montrent à
deux moments différents de l’action un même personnage, ici couché dans son lit,
là s’apprêtant à monter à cheval, le nom de Florence était-il divisé en deux
compartiments. Dans l’un, sous un dais architectural, je contemplais une fresque
à laquelle était partiellement superposé un rideau de soleil matinal, poudreux,
oblique et progressif; dans l’autre (car ne pensant pas aux noms comme à un
idéal inaccessible mais comme à une ambiance réelle dans laquelle j’irais me
plonger, la vie non vécue encore, la vie intacte et pure que j’y enfermais
donnait aux plaisirs les plus matériels, aux scènes les plus simples, cet
attrait qu’ils ont dans les œuvres des primitifs), je traversais rapidement —
pour trouver plus vite le déjeuner qui m’attendait avec des fruits et du vin de
Chianti — le Ponte-Vecchio encombré de jonquilles, de narcisses et d’anémones.
Voilà (bien que je fusse à Paris) ce que je voyais et non ce qui était autour de
moi. Même à un simple point de vue réaliste, les pays que nous désirons tiennent
à chaque moment beaucoup plus de place dans notre vie véritable, que le pays où
nous nous trouvons effectivement. Sans doute si alors j’avais fait moi-même plus
attention à ce qu’il y avait dans ma pensée quand je prononçais les mots «aller
à Florence, à Parme, à Pise, à Venise», je me serais rendu compte que ce que je
voyais n’était nullement une ville, mais quelque chose d’aussi différent de tout
ce que je connaissais, d’aussi délicieux, que pourrait être pour une humanité
dont la vie se serait toujours écoulée dans des fins d’après-midi d’hiver, cette
merveille inconnue: une matinée de printemps. Ces images irréelles, fixes,
toujours pareilles, remplissant mes nuits et mes jours, différencièrent cette
époque de ma vie de celles qui l’avaient précédée (et qui auraient pu se
confondre avec elle aux yeux d’un observateur qui ne voit les choses que du
dehors, c’est-à-dire qui ne voit rien), comme dans un opéra un motif mélodique
introduit une nouveauté qu’on ne pourrait pas soupçonner si on ne faisait que
lire le livret, moins encore si on restait en dehors du théâtre à compter
seulement les quarts d’heure qui s’écoulent. Et encore, même à ce point de vue
de simple quantité, dans notre vie les jours ne sont pas égaux. Pour parcourir
les jours, les natures un peu nerveuses, comme était la mienne, disposent, comme
les voitures automobiles, de «vitesses» différentes. Il y a des jours montueux
et malaisés qu’on met un temps infini à gravir et des jours en pente qui se
laissent descendre à fond de train en chantant. Pendant ce mois — où je
ressassai comme une mélodie, sans pouvoir m’en rassasier, ces images de
Florence, de Venise et de Pise desquelles le désir qu’elles excitaient en moi
gardait quelque chose d’aussi profondément individuel que si ç’avait été un
amour, un amour pour une personne — je ne cessai pas de croire qu’elles
correspondaient à une réalité indépendante de moi, et elles me firent connaître
une aussi belle espérance que pouvait en nourrir un chrétien des premiers âges à
la veille d’entrer dans le paradis. Aussi sans que je me souciasse de la
contradiction qu’il y avait à vouloir regarder et toucher avec les organes des
sens, ce qui avait été élaboré par la rêverie et non perçu par eux — et d’autant
plus tentant pour eux, plus différent de ce qu’ils connaissaient — c’est ce qui
me rappelait la réalité de ces images, qui enflammait le plus mon désir, parce
que c’était comme une promesse qu’il serait contenté. Et, bien que mon
exaltation eût pour motif un désir de jouissances artistiques, les guides
l’entretenaient encore plus que les livres d’esthétiques et, plus que les
guides, l’indicateur des chemins de fer. Ce qui m’émouvait c’était de penser que
cette Florence que je voyais proche mais inaccessible dans mon imagination, si
le trajet qui la séparait de moi, en moi-même, n’était pas viable, je pourrais
l’atteindre par un biais, par un détour, en prenant la «voie de terre». Certes,
quand je me répétais, donnant ainsi tant de valeur à ce que j’allais voir, que
Venise était «l’école de Giorgione, la demeure du Titien, le plus complet musée
de l’architecture domestique au moyen âge», je me sentais heureux. Je l’étais
pourtant davantage quand, sorti pour une course, marchant vite à cause du temps
qui, après quelques jours de printemps précoce était redevenu un temps d’hiver
(comme celui que nous trouvions d’habitude à Combray, la Semaine Sainte) —
voyant sur les boulevards les marronniers qui, plongés dans un air glacial et
liquide comme de l’eau, n’en commençaient pas moins, invités exacts, déjà en
tenue, et qui ne se sont pas laissé décourager, à arrondir et à ciseler en leurs
blocs congelés, l’irrésistible verdure dont la puissance abortive du froid
contrariait mais ne parvenait pas à réfréner la progressive poussée — je pensais
que déjà le Ponte-Vecchio était jonché à foison de jacinthes et d’anémones et
que le soleil du printemps teignait déjà les flots du Grand Canal d’un si sombre
azur et de si nobles émeraudes qu’en venant se briser aux pieds des peintures du
Titien, ils pouvaient rivaliser de riche coloris avec elles. Je ne pus plus
contenir ma joie quand mon père, tout en consultant le baromètre et en déplorant
le froid, commença à chercher quels seraient les meilleurs trains, et quand je
compris qu’en pénétrant après le déjeuner dans le laboratoire charbonneux, dans
la chambre magique qui se chargeait d’opérer la transmutation tout autour
d’elle, on pouvait s’éveiller le lendemain dans la cité de marbre et d’or
«rehaussée de jaspe et pavée d’émeraudes». Ainsi elle et la Cité des lys
n’étaient pas seulement des tableaux fictifs qu’on mettait à volonté devant son
imagination, mais existaient à une certaine distance de Paris qu’il fallait
absolument franchir si l’on voulait les voir, à une certaine place déterminée de
la terre, et à aucune autre, en un mot étaient bien réelles. Elles le devinrent
encore plus pour moi, quand mon père en disant: «En somme, vous pourriez rester
à Venise du 20 avril au 29 et arriver à Florence dès le matin de Pâques», les
fit sortir toutes deux non plus seulement de l’Espace abstrait, mais de ce Temps
imaginaire où nous situons non pas un seul voyage à la fois, mais d’autres,
simultanés et sans trop d’émotion puisqu’ils ne sont que possibles — ce Temps
qui se refabrique si bien qu’on peut encore le passer dans une ville après qu’on
l’a passé dans une autre — et leur consacra de ces jours particuliers qui sont
le certificat d’authenticité des objets auxquels on les emploie, car ces jours
uniques, ils se consument par l’usage, ils ne reviennent pas, on ne peut plus
les vivre ici quand on les a vécus là; je sentis que c’était vers la semaine qui
commençait le lundi où la blanchisseuse devait rapporter le gilet blanc que
j’avais couvert d’encre, que se dirigeaient pour s’y absorber au sortir du temps
idéal où elles n’existaient pas encore, les deux Cités Reines dont j’allais
avoir, par la plus émouvante des géométries, à inscrire les dômes et les tours
dans le plan de ma propre vie. Mais je n’étais encore qu’en chemin vers le
dernier degré de l’allégresse; je l’atteignis enfin (ayant seulement alors la
révélation que sur les rues clapotantes, rougies du reflet des fresques de
Giorgione, ce n’était pas, comme j’avais, malgré tant d’avertissements, continué
à l’imaginer, les hommes «majestueux et terribles comme la mer, portant leur
armure aux reflets de bronze sous les plis de leur manteau sanglant» qui se
promèneraient dans Venise la semaine prochaine, la veille de Pâques, mais que ce
pourrait être moi le personnage minuscule que, dans une grande photographie de
Saint-Marc qu’on m’avait prêtée, l’illustrateur avait représenté, en chapeau
melon, devant les proches), quand j’entendis mon père me dire: «Il doit faire
encore froid sur le Grand Canal, tu ferais bien de mettre à tout hasard dans ta
malle ton pardessus d’hiver et ton gros veston.» A ces mots je m’élevai à une
sorte d’extase; ce que j’avais cru jusque-là impossible, je me sentis vraiment
pénétrer entre ces «rochers d’améthyste pareils à un récif de la mer des Indes»;
par une gymnastique suprême et au-dessus de mes forces, me dévêtant comme d’une
carapace sans objet de l’air de ma chambre qui m’entourait, je le remplaçai par
des parties égales d’air vénitien, cette atmosphère marine, indicible et
particulière comme celle des rêves que mon imagination avait enfermée dans le
nom de Venise, je sentis s’opérer en moi une miraculeuse désincarnation; elle se
doubla aussitôt de la vague envie de vomir qu’on éprouve quand on vient de
prendre un gros mal de gorge, et on dut me mettre au lit avec une fièvre si
tenace, que le docteur déclara qu’il fallait renoncer non seulement à me laisser
partir maintenant à Florence et à Venise mais, même quand je serais entièrement
rétabli, m’éviter d’ici au moins un an, tout projet de voyage et toute cause
d’agitation.

Et hélas, il défendit aussi d’une façon absolue qu’on me laissât aller au
théâtre entendre la Berma; l’artiste sublime, à laquelle Bergotte trouvait du
génie, m’aurait en me faisant connaître quelque chose qui était peut-être aussi
important et aussi beau, consolé de n’avoir pas été à Florence et à Venise, de
n’aller pas à Balbec. On devait se contenter de m’envoyer chaque jour aux
Champs-Elysées, sous la surveillance d’une personne qui m’empêcherait de me
fatiguer et qui fut Françoise, entrée à notre service après la mort de ma tante
Léonie. Aller aux Champs-Élysées me fut insupportable. Si seulement Bergotte les
eût décrits dans un de ses livres, sans doute j’aurais désiré de les connaître,
comme toutes les choses dont on avait commencé par mettre le «double» dans mon
imagination. Elle les réchauffait, les faisait vivre, leur donnait une
personnalité, et je voulais les retrouver dans la réalité; mais dans ce jardin
public rien ne se rattachait à mes rêves.

Un jour, comme je m’ennuyais à notre place familière, à côté des chevaux de
bois, Françoise m’avait emmené en excursion — au delà de la frontière que
gardent à intervalles égaux les petits bastions des marchandes de sucre d’orge —
dans ces régions voisines mais étrangères où les visages sont inconnus, où passe
la voiture aux chèvres; puis elle était revenue prendre ses affaires sur sa
chaise adossée à un massif de lauriers; en l’attendant je foulais la grande
pelouse chétive et rase, jaunie par le soleil, au bout de laquelle le bassin est
dominé par une statue quand, de l’allée, s’adressant à une fillette à cheveux
roux qui jouait au volant devant la vasque, une autre, en train de mettre son
manteau et de serrer sa raquette, lui cria, d’une voix brève: «Adieu, Gilberte,
je rentre, n’oublie pas que nous venons ce soir chez toi après dîner.» Ce nom de
Gilberte passa près de moi, évoquant d’autant plus l’existence de celle qu’il
désignait qu’il ne la nommait pas seulement comme un absent dont on parle, mais
l’interpellait; il passa ainsi près de moi, en action pour ainsi dire, avec une
puissance qu’accroissait la courbe de son jet et l’approche de son but; —
transportant à son bord, je le sentais, la connaissance, les notions qu’avait de
celle à qui il était adressé, non pas moi, mais l’amie qui l’appelait, tout ce
que, tandis qu’elle le prononçait, elle revoyait ou du moins, possédait en sa
mémoire, de leur intimité quotidienne, des visites qu’elles se faisaient l’une
chez l’autre, de tout cet inconnu encore plus inaccessible et plus douloureux
pour moi d’être au contraire si familier et si maniable pour cette fille
heureuse qui m’en frôlait sans que j’y puisse pénétrer et le jetait en plein air
dans un cri; — laissant déjà flotter dans l’air l’émanation délicieuse qu’il
avait fait se dégager, en les touchant avec précision, de quelques points
invisibles de la vie de Mlle Swann, du soir qui allait venir, tel qu’il serait,
après dîner, chez elle — formant, passager céleste au milieu des enfants et des
bonnes, un petit nuage d’une couleur précieuse, pareil à celui qui, bombé
au-dessus d’un beau jardin du Poussin, reflète minutieusement comme un nuage
d’opéra, plein de chevaux et de chars, quelque apparition de la vie des dieux; —
jetant enfin, sur cette herbe pelée, à l’endroit où elle était un morceau à la
fois de pelouse flétrie et un moment de l’après-midi de la blonde joueuse de
volant (qui ne s’arrêta de le lancer et de le rattraper que quand une
institutrice à plumet bleu l’eut appelée), une petite bande merveilleuse et
couleur d’héliotrope impalpable comme un reflet et superposée comme un tapis sur
lequel je ne pus me lasser de promener mes pas attardés, nostalgiques et
profanateurs, tandis que Françoise me criait: «Allons, aboutonnez voir votre
paletot et filons» et que je remarquais pour la première fois avec irritation
qu’elle avait un langage vulgaire, et hélas, pas de plumet bleu à son chapeau.

Retournerait-elle seulement aux Champs-Élysées? Le lendemain elle n’y était pas;
mais je l’y vis les jours suivants; je tournais tout le temps autour de
l’endroit où elle jouait avec ses amies, si bien qu’une fois où elles ne se
trouvèrent pas en nombre pour leur partie de barres, elle me fit demander si je
voulais compléter leur camp, et je jouai désormais avec elle chaque fois qu’elle
était là. Mais ce n’était pas tous les jours; il y en avait où elle était
empêchée de venir par ses cours, le catéchisme, un goûter, toute cette vie
séparée de la mienne que par deux fois, condensée dans le nom de Gilberte,
j’avais senti passer si douloureusement près de moi, dans le raidillon de
Combray et sur la pelouse des Champs-Élysées. Ces jours-là, elle annonçait
d’avance qu’on ne la verrait pas; si c’était à cause de ses études, elle disait:
«C’est rasant, je ne pourrai pas venir demain; vous allez tous vous amuser sans
moi», d’un air chagrin qui me consolait un peu; mais en revanche quand elle
était invitée à une matinée, et que, ne le sachant pas je lui demandais si elle
viendrait jouer, elle me répondait: «J’espère bien que non! J’espère bien que
maman me laissera aller chez mon amie.» Du moins ces jours-là, je savais que je
ne la verrais pas, tandis que d’autres fois, c’était à l’improviste que sa mère
l’emmenait faire des courses avec elle, et le lendemain elle disait: «Ah! oui,
je suis sortie avec maman», comme une chose naturelle, et qui n’eût pas été pour
quelqu’un le plus grand malheur possible. Il y avait aussi les jours de mauvais
temps où son institutrice, qui pour elle-même craignait la pluie, ne voulait pas
l’emmener aux Champs-Élysées.

Aussi si le ciel était douteux, dès le matin je ne cessais de l’interroger et je
tenais compte de tous les présages. Si je voyais la dame d’en face qui, près de
la fenêtre, mettait son chapeau, je me disais: «Cette dame va sortir; donc il
fait un temps où l’on peut sortir: pourquoi Gilberte ne ferait-elle pas comme
cette dame?» Mais le temps s’assombrissait, ma mère disait qu’il pouvait se
lever encore, qu’il suffirait pour cela d’un rayon de soleil, mais que plus
probablement il pleuvrait; et s’il pleuvait à quoi bon aller aux Champs Élysées?
Aussi depuis le déjeuner mes regards anxieux ne quittaient plus le ciel
incertain et nuageux. Il restait sombre. Devant la fenêtre, le balcon était
gris. Tout d’un coup, sur sa pierre maussade je ne voyais pas une couleur moins
terne, mais je sentais comme un effort vers une couleur moins terne, la
pulsation d’un rayon hésitant qui voudrait libérer sa lumière. Un instant après,
le balcon était pâle et réfléchissant comme une eau matinale, et mille reflets
de la ferronnerie de son treillage étaient venus s’y poser. Un souffle de vent
les dispersait, la pierre s’était de nouveau assombrie, mais, comme apprivoisés,
ils revenaient; elle recommençait imperceptiblement à blanchir et par un de ces
crescendos continus comme ceux qui, en musique, à la fin d’une Ouverture, mènent
une seule note jusqu’au fortissimo suprême en la faisant passer rapidement par
tous les degrés intermédiaires, je la voyais atteindre à cet or inaltérable et
fixe des beaux jours, sur lequel l’ombre découpée de l’appui ouvragé de la
balustrade se détachait en noir comme une végétation capricieuse, avec une
ténuité dans la délinéation des moindres détails qui semblait trahir une
conscience appliquée, une satisfaction d’artiste, et avec un tel relief, un tel
velours dans le repos de ses masses sombres et heureuses qu’en vérité ces
reflets larges et feuillus qui reposaient sur ce lac de soleil semblaient savoir
qu’ils étaient des gages de calme et de bonheur.

Lierre instantané, flore pariétaire et fugitive! la plus incolore, la plus
triste, au gré de beaucoup, de celles qui peuvent ramper sur le mur ou décorer
la croisée; pour moi, de toutes la plus chère depuis le jour où elle était
apparue sur notre balcon, comme l’ombre même de la présence de Gilberte qui
était peut-être déjà aux Champs-Elysées, et dès que j’y arriverais, me dirait:
«Commençons tout de suite à jouer aux barres, vous êtes dans mon camp»; fragile,
emportée par un souffle, mais aussi en rapport non pas avec la saison, mais avec
l’heure; promesse du bonheur immédiat que la journée refuse ou accomplira, et
par là du bonheur immédiat par excellence, le bonheur de l’amour; plus douce,
plus chaude sur la pierre que n’est la mousse même; vivace, à qui il suffit d’un
rayon pour naître et faire éclore de la joie, même au cœur de l’hiver.

Et jusque dans ces jours où toute autre végétation a disparu, où le beau cuir
vert qui enveloppe le tronc des vieux arbres est caché sous la neige, quand
celle-ci cessait de tomber, mais que le temps restait trop couvert pour espérer
que Gilberte sortît, alors tout d’un coup, faisant dire à ma mère: «Tiens voilà
justement qu’il fait beau, vous pourriez peut-être essayer tout de même d’aller
aux Champs-Élysées», sur le manteau de neige qui couvrait le balcon, le soleil
apparu entrelaçait des fils d’or et brodait des reflets noirs. Ce jour-là nous
ne trouvions personne ou une seule fillette prête à partir qui m’assurait que
Gilberte ne viendrait pas. Les chaises désertées par l’assemblée imposante mais
frileuse des institutrices étaient vides. Seule, près de la pelouse, était
assise une dame d’un certain âge qui venait par tous les temps, toujours
hanarchée d’une toilette identique, magnifique et sombre, et pour faire la
connaissance de laquelle j’aurais à cette époque sacrifié, si l’échange m’avait
été permis, tous les plus grands avantages futurs de ma vie. Car Gilberte allait
tous les jours la saluer; elle demandait à Gilberte des nouvelles de «son amour
de mère»; et il me semblait que si je l’avais connue, j’avais été pour Gilberte
quelqu’un de tout autre, quelqu’un qui connaissait les relations de ses parents.
Pendant que ses petits-enfants jouaient plus loin, elle lisait toujours les
Débats qu’elle appelait «mes vieux Débats» et, par genre aristocratique, disait
en parlant du sergent de ville ou de la loueuse de chaises: «Mon vieil ami le
sergent de ville», «la loueuse de chaises et moi qui sommes de vieux amis».

Françoise avait trop froid pour rester immobile, nous allâmes jusqu’au pont de
la Concorde voir la Seine prise, dont chacun et même les enfants s’approchaient
sans peur comme d’une immense baleine échouée, sans défense, et qu’on allait
dépecer. Nous revenions aux Champs-Élysées; je languissais de douleur entre les
chevaux de bois immobiles et la pelouse blanche prise dans le réseau noir des
allées dont on avait enlevé la neige et sur laquelle la statue avait à la main
un jet de glace ajouté qui semblait l’explication de son geste. La vieille dame
elle-même ayant plié ses Débats, demanda l’heure à une bonne d’enfants qui
passait et qu’elle remercia en lui disant: «Comme vous êtes aimable!» puis,
priant le cantonnier de dire à ses petits enfants de revenir, qu’elle avait
froid, ajouta: «Vous serez mille fois bon. Vous savez que je suis confuse!» Tout
à coup l’air se déchira: entre le guignol et le cirque, à l’horizon embelli, sur
le ciel entr’ouvert, je venais d’apercevoir, comme un signe fabuleux, le plumet
bleu de Mademoiselle. Et déjà Gilberte courait à toute vitesse dans ma
direction, étincelante et rouge sous un bonnet carré de fourrure, animée par le
froid, le retard et le désir du jeu; un peu avant d’arriver à moi, elle se
laissa glisser sur la glace et, soit pour mieux garder son équilibre, soit parce
qu’elle trouvait cela plus gracieux, ou par affectation du maintien d’une
patineuse, c’est les bras grands ouverts qu’elle avançait en souriant, comme si
elle avait voulu m’y recevoir. «Brava! Brava! ça c’est très bien, je dirais
comme vous que c’est chic, que c’est crâne, si je n’étais pas d’un autre temps,
du temps de l’ancien régime, s’écria la vieille dame prenant la parole au nom
des Champs-Élysées silencieux pour remercier Gilberte d’être venue sans se
laisser intimider par le temps. Vous êtes comme moi, fidèle quand même à nos
vieux Champs-Élysées; nous sommes deux intrépides. Si je vous disais que je les
aime, même ainsi. Cette neige, vous allez rire de moi, ça me fait penser à de
l’hermine!» Et la vieille dame se mit à rire.

Le premier de ces jours — auxquels la neige, image des puissances qui pouvaient
me priver de voir Gilberte, donnait la tristesse d’un jour de séparation et
jusqu’à l’aspect d’un jour de départ parce qu’il changeait la figure et
empêchait presque l’usage du lieu habituel de nos seules entrevues maintenant
changé, tout enveloppé de housses — ce jour fit pourtant faire un progrès à mon
amour, car il fut comme un premier chagrin qu’elle eût partagé avec moi. Il n’y
avait que nous deux de notre bande, et être ainsi le seul qui fût avec elle,
c’était non seulement comme un commencement d’intimité, mais aussi de sa part —
comme si elle ne fût venue rien que pour moi par un temps pareil — cela me
semblait aussi touchant que si un de ces jours où elle était invitée à une
matinée, elle y avait renoncé pour venir me retrouver aux Champs-Élysées; je
prenais plus de confiance en la vitalité et en l’avenir de notre amitié qui
restait vivace au milieu de l’engourdissement, de la solitude et de la ruine des
choses environnantes; et tandis qu’elle me mettait des boules de neige dans le
cou, je souriais avec attendrissement à ce qui me semblait à la fois une
prédilection qu’elle me marquait en me tolérant comme compagnon de voyage dans
ce pays hivernal et nouveau, et une sorte de fidélité qu’elle me gardait au
milieu du malheur. Bientôt l’une après l’autre, comme des moineaux hésitants,
ses amies arrivèrent toutes noires sur la neige. Nous commençâmes à jouer et
comme ce jour si tristement commencé devait finir dans la joie, comme je
m’approchais, avant de jouer aux barres, de l’amie à la voix brève que j’avais
entendue le premier jour crier le nom de Gilberte, elle me dit: «Non, non, on
sait bien que vous aimez mieux être dans le camp de Gilberte, d’ailleurs vous
voyez elle vous fait signe.» Elle m’appelait en effet pour que je vinsse sur la
pelouse de neige, dans son camp, dont le soleil en lui donnant les reflets
roses, l’usure métallique des brocarts anciens, faisait un camp du drap d’or.

Ce jour que j’avais tant redouté fut au contraire un des seuls où je ne fus pas
trop malheureux.

Car, moi qui ne pensais plus qu’à ne jamais rester un jour sans voir Gilberte
(au point qu’une fois ma grand’mère n’étant pas rentrée pour l’heure du dîner,
je ne pus m’empêcher de me dire tout de suite que si elle avait été écrasée par
une voiture, je ne pourrais pas aller de quelque temps aux Champs-Élysées; on
n’aime plus personne dès qu’on aime) pourtant ces moments où j’étais auprès
d’elle et que depuis la veille j’avais si impatiemment attendus, pour lesquels
j’avais tremblé, auxquels j’aurais sacrifié tout le reste, n’étaient nullement
des moments heureux; et je le savais bien car c’était les seuls moments de ma
vie sur lesquels je concentrasse une attention méticuleuse, acharnée, et elle ne
découvrait pas en eux un atome de plaisir.

Tout le temps que j’étais loin de Gilberte, j’avais besoin de la voir, parce que
cherchant sans cesse à me représenter son image, je finissais par ne plus y
réussir, et par ne plus savoir exactement à quoi correspondait mon amour. Puis,
elle ne m’avait encore jamais dit qu’elle m’aimait. Bien au contraire, elle
avait souvent prétendu qu’elle avait des amis qu’elle me préférait, que j’étais
un bon camarade avec qui elle jouait volontiers quoique trop distrait, pas assez
au jeu; enfin elle m’avait donné souvent des marques apparentes de froideur qui
auraient pu ébranler ma croyance que j’étais pour elle un être différent des
autres, si cette croyance avait pris sa source dans un amour que Gilberte aurait
eu pour moi, et non pas, comme cela était, dans l’amour que j’avais pour elle,
ce qui la rendait autrement résistante, puisque cela la faisait dépendre de la
manière même dont j’étais obligé, par une nécessité intérieure, de penser à
Gilberte. Mais les sentiments que je ressentais pour elle, moi-même je ne les
lui avais pas encore déclarés. Certes, à toutes les pages de mes cahiers,
j’écrivais indéfiniment son nom et son adresse, mais à la vue de ces vagues
lignes que je traçais sans qu’elle pensât pour cela à moi, qui lui faisaient
prendre autour de moi tant de place apparente sans qu’elle fût mêlée davantage à
ma vie, je me sentais découragé parce qu’elles ne me parlaient pas de Gilberte
qui ne les verrait même pas, mais de mon propre désir qu’elles semblaient me
montrer comme quelque chose de purement personnel, d’irréel, de fastidieux et
d’impuissant. Le plus pressé était que nous nous vissions Gilberte et moi, et
que nous puissions nous faire l’aveu réciproque de notre amour, qui jusque-là
n’aurait pour ainsi dire pas commencé. Sans doute les diverses raisons qui me
rendaient si impatient de la voir auraient été moins impérieuses pour un homme
mûr. Plus tard, il arrive que devenus habiles dans la culture de nos plaisirs,
nous nous contentions de celui que nous avons à penser à une femme comme je
pensais à Gilberte, sans être inquiets de savoir si cette image correspond à la
réalité, et aussi de celui de l’aimer sans avoir besoin d’être certain qu’elle
nous aime; ou encore que nous renoncions au plaisir de lui avouer notre
inclination pour elle, afin d’entretenir plus vivace l’inclination qu’elle a
pour nous, imitant ces jardiniers japonais qui pour obtenir une plus belle
fleur, en sacrifient plusieurs autres. Mais à l’époque où j’aimais Gilberte, je
croyais encore que l’Amour existait réellement en dehors de nous; que, en
permettant tout au plus que nous écartions les obstacles, il offrait ses
bonheurs dans un ordre auquel on n’était pas libre de rien changer; il me
semblait que si j’avais, de mon chef, substitué à la douceur de l’aveu la
simulation de l’indifférence, je ne me serais pas seulement privé d’une des
joies dont j’avais le plus rêvé mais que je me serais fabriqué à ma guise un
amour factice et sans valeur, sans communication avec le vrai, dont j’aurais
renoncé à suivre les chemins mystérieux et préexistants.

Mais quand j’arrivais aux Champs-Élysées — et que d’abord j’allais pouvoir
confronter mon amour pour lui faire subir les rectifications nécessaires à sa
cause vivante, indépendante de moi — dès que j’étais en présence de cette
Gilberte Swann sur la vue de laquelle j’avais compté pour rafraîchir les images
que ma mémoire fatiguée ne retrouvait plus, de cette Gilberte Swann avec qui
j’avais joué hier, et que venait de me faire saluer et reconnaître un instinct
aveugle comme celui qui dans la marche nous met un pied devant l’autre avant que
nous ayons eu le temps de penser, aussitôt tout se passait comme si elle et la
fillette qui était l’objet de mes rêves avaient été deux êtres différents. Par
exemple si depuis la veille je portais dans ma mémoire deux yeux de feu dans des
joues pleines et brillantes, la figure de Gilberte m’offrait maintenant avec
insistance quelque chose que précisément je ne m’étais pas rappelé, un certain
effilement aigu du nez qui, s’associant instantanément à d’autres traits,
prenait l’importance de ces caractères qui en histoire naturelle définissent une
espèce, et la transmuait en une fillette du genre de celles à museau pointu.
Tandis que je m’apprêtais à profiter de cet instant désiré pour me livrer, sur
l’image de Gilberte que j’avais préparée avant de venir et que je ne retrouvais
plus dans ma tête, à la mise au point qui me permettrait dans les longues heures
où j’étais seul d’être sûr que c’était bien elle que je me rappelais, que
c’était bien mon amour pour elle que j’accroissais peu à peu comme un ouvrage
qu’on compose, elle me passait une balle; et comme le philosophe idéaliste dont
le corps tient compte du monde extérieur à la réalité duquel son intelligence ne
croit pas, le même moi qui m’avait fait la saluer avant que je l’eusse
identifiée, s’empressait de me faire saisir la balle qu’elle me tendait (comme
si elle était une camarade avec qui j’étais venu jouer, et non une âme sœur que
j’étais venu rejoindre), me faisait lui tenir par bienséance jusqu’à l’heure où
elle s’en allait, mille propos aimables et insignifiants et m’empêchait ainsi,
ou de garder le silence pendant lequel j’aurais pu enfin remettre la main sur
l’image urgente et égarée, ou de lui dire les paroles qui pouvaient faire faire
à notre amour les progrès décisifs sur lesquels j’étais chaque fois obligé de ne
plus compter que pour l’après-midi suivante. Il en faisait pourtant
quelques-uns. Un jour que nous étions allés avec Gilberte jusqu’à la baraque de
notre marchande qui était particulièrement aimable pour nous — car c’était chez
elle que M. Swann faisait acheter son pain d’épices, et par hygiène, il en
consommait beaucoup, souffrant d’un eczéma ethnique et de la constipation des
Prophètes — Gilberte me montrait en riant deux petits garçons qui étaient comme
le petit coloriste et le petit naturaliste des livres d’enfants. Car l’un ne
voulait pas d’un sucre d’orge rouge parce qu’il préférait le violet et l’autre,
les larmes aux yeux, refusait une prune que voulait lui acheter sa bonne, parce
que, finit-il par dire d’une voix passionnée: «J’aime mieux l’autre prune, parce
qu’elle a un ver!» J’achetai deux billes d’un sou. Je regardais avec admiration,
lumineuses et captives dans une sébile isolée, les billes d’agate qui me
semblaient précieuses parce qu’elles étaient souriantes et blondes comme des
jeunes filles et parce qu’elles coûtaient cinquante centimes pièce. Gilberte à
qui on donnait beaucoup plus d’argent qu’à moi me demanda laquelle je trouvais
la plus belle. Elles avaient la transparence et le fondu de la vie. Je n’aurais
voulu lui en faire sacrifier aucune. J’aurais aimé qu’elle pût les acheter, les
délivrer toutes. Pourtant je lui en désignai une qui avait la couleur de ses
yeux. Gilberte la prit, chercha son rayon doré, la caressa, paya sa rançon, mais
aussitôt me remit sa captive en me disant: «Tenez, elle est à vous, je vous la
donne, gardez-la comme souvenir.»

Une autre fois, toujours préoccupé du désir d’entendre la Berma dans une pièce
classique, je lui avais demandé si elle ne possédait pas une brochure où
Bergotte parlait de Racine, et qui ne se trouvait plus dans le commerce. Elle
m’avait prié de lui en rappeler le titre exact, et le soir je lui avais adressé
un petit télégramme en écrivant sur l’enveloppe ce nom de Gilberte Swann que
j’avais tant de fois tracé sur mes cahiers. Le lendemain elle m’apporta dans un
paquet noué de faveurs mauves et scellé de cire blanche, la brochure qu’elle
avait fait chercher. «Vous voyez que c’est bien ce que vous m’avez demandé, me
dit-elle, tirant de son manchon le télégramme que je lui avais envoyé.» Mais
dans l’adresse de ce pneumatique — qui, hier encore n’était rien, n’était qu’un
petit bleu que j’avais écrit, et qui depuis qu’un télégraphiste l’avait remis au
concierge de Gilberte et qu’un domestique l’avait porté jusqu’à sa chambre,
était devenu cette chose sans prix, un des petits bleus qu’elle avait reçus ce
jour-là — j’eus peine à reconnaître les lignes vaines et solitaires de mon
écriture sous les cercles imprimés qu’y avait apposés la poste, sous les
inscriptions qu’y avait ajoutées au crayon un des facteurs, signes de
réalisation effective, cachets du monde extérieur, violettes ceintures
symboliques de la vie, qui pour la première fois venaient épouser, maintenir,
relever, réjouir mon rêve.

Et il y eut un jour aussi où elle me dit: «Vous savez, vous pouvez m’appeler
Gilberte, en tous cas moi, je vous appellerai par votre nom de baptême. C’est
trop gênant.» Pourtant elle continua encore un moment à se contenter de me dire
«vous» et comme je le lui faisais remarquer, elle sourit, et composant,
construisant une phrase comme celles qui dans les grammaires étrangères n’ont
d’autre but que de nous faire employer un mot nouveau, elle la termina par mon
petit nom. Et me souvenant plus tard de ce que j’avais senti alors, j’y ai
démêlé l’impression d’avoir été tenu un instant dans sa bouche, moi-même, nu,
sans plus aucune des modalités sociales qui appartenaient aussi, soit à ses
autres camarades, soit, quand elle disait mon nom de famille, à mes parents, et
dont ses lèvres — en l’effort qu’elle faisait, un peu comme son père, pour
articuler les mots qu’elle voulait mettre en valeur — eurent l’air de me
dépouiller, de me dévêtir, comme de sa peau un fruit dont on ne peut avaler que
la pulpe, tandis que son regard, se mettant au même degré nouveau d’intimité que
prenait sa parole, m’atteignait aussi plus directement, non sans témoigner la
conscience, le plaisir et jusque la gratitude qu’il en avait, en se faisant
accompagner d’un sourire.

Mais au moment même, je ne pouvais apprécier la valeur de ces plaisirs nouveaux.
Ils n’étaient pas donnés par la fillette que j’aimais, au moi qui l’aimait, mais
par l’autre, par celle avec qui je jouais, à cet autre moi qui ne possédait ni
le souvenir de la vraie Gilberte, ni le cœur indisponible qui seul aurait pu
savoir le prix d’un bonheur, parce que seul il l’avait désiré. Même après être
rentré à la maison je ne les goûtais pas, car, chaque jour, la nécessité qui me
faisait espérer que le lendemain j’aurais la contemplation exacte, calme,
heureuse de Gilberte, qu’elle m’avouerait enfin son amour, en m’expliquant pour
quelles raisons elle avait dû me le cacher jusqu’ici, cette même nécessité me
forçait à tenir le passé pour rien, à ne jamais regarder que devant moi, à
considérer les petits avantages qu’elle m’avait donnés non pas en eux-mêmes et
comme s’ils se suffisaient, mais comme des échelons nouveaux où poser le pied,
qui allaient me permettre de faire un pas de plus en avant et d’atteindre enfin
le bonheur que je n’avais pas encore rencontré.

Si elle me donnait parfois de ces marques d’amitié, elle me faisait aussi de la
peine en ayant l’air de ne pas avoir de plaisir à me voir, et cela arrivait
souvent les jours mêmes sur lesquels j’avais le plus compté pour réaliser mes
espérances. J’étais sûr que Gilberte viendrait aux Champs-Élysées et j’éprouvais
une allégresse qui me paraissait seulement la vague anticipation d’un grand
bonheur quand — entrant dès le matin au salon pour embrasser maman déjà toute
prête, la tour de ses cheveux noirs entièrement construite, et ses belles mains
blanches et potelées sentant encore le savon — j’avais appris, en voyant une
colonne de poussière se tenir debout toute seule au-dessus du piano, et en
entendant un orgue de Barbarie jouer sous la fenêtre: «En revenant de la revue»,
que l’hiver recevait jusqu’au soir la visite inopinée et radieuse d’une journée
de printemps. Pendant que nous déjeunions, en ouvrant sa croisée, la dame d’en
face avait fait décamper en un clin d’œil, d’à côté de ma chaise — rayant d’un
seul bond toute la largeur de notre salle à manger — un rayon qui y avait
commencé sa sieste et était déjà revenu la continuer l’instant d’après. Au
collège, à la classe d’une heure, le soleil me faisait languir d’impatience et
d’ennui en laissant traîner une lueur dorée jusque sur mon pupitre, comme une
invitation à la fête où je ne pourrais arriver avant trois heures, jusqu’au
moment où Françoise venait me chercher à la sortie, et où nous nous acheminions
vers les Champs-Élysées par les rues décorées de lumière, encombrées par la
foule, et où les balcons, descellés par le soleil et vaporeux, flottaient devant
les maisons comme des nuages d’or. Hélas! aux Champs-Élysées je ne trouvais pas
Gilberte, elle n’était pas encore arrivée. Immobile sur la pelouse nourrie par
le soleil invisible qui çà et là faisait flamboyer la pointe d’un brin d’herbe,
et sur laquelle les pigeons qui s’y étaient posés avaient l’air de sculptures
antiques que la pioche du jardinier a ramenées à la surface d’un sol auguste, je
restais les yeux fixés sur l’horizon, je m’attendais à tout moment à voir
apparaître l’image de Gilberte suivant son institutrice, derrière la statue qui
semblait tendre l’enfant qu’elle portait et qui ruisselait de rayons, à la
bénédiction du soleil. La vieille lectrice des Débats était assise sur son
fauteuil, toujours à la même place, elle interpellait un gardien à qui elle
faisait un geste amical de la main en lui criant: «Quel joli temps!» Et la
préposée s’étant approchée d’elle pour percevoir le prix du fauteuil, elle
faisait mille minauderies en mettant dans l’ouverture de son gant le ticket de
dix centimes comme si ç’avait été un bouquet, pour qui elle cherchait, par
amabilité pour le donateur, la place la plus flatteuse possible. Quand elle
l’avait trouvée, elle faisait exécuter une évolution circulaire à son cou,
redressait son boa, et plantait sur la chaisière, en lui montrant le bout de
papier jaune qui dépassait sur son poignet, le beau sourire dont une femme, en
indiquant son corsage à un jeune homme, lui dit: «Vous reconnaissez vos roses!»

J’emmenais Françoise au-devant de Gilberte jusqu’à l’Arc-de-Triomphe, nous ne la
rencontrions pas, et je revenais vers la pelouse persuadé qu’elle ne viendrait
plus, quand, devant les chevaux de bois, la fillette à la voix brève se jetait
sur moi: «Vite, vite, il y a déjà un quart d’heure que Gilberte est arrivée.
Elle va repartir bientôt. On vous attend pour faire une partie de barres.»
Pendant que je montais l’avenue des Champs-Élysées, Gilberte était venue par la
rue Boissy-d’Anglas, Mademoiselle ayant profité du beau temps pour faire des
courses pour elle; et M. Swann allait venir chercher sa fille. Aussi c’était ma
faute; je n’aurais pas dû m’éloigner de la pelouse; car on ne savait jamais
sûrement par quel côté Gilberte viendrait, si ce serait plus ou moins tard, et
cette attente finissait par me rendre plus émouvants, non seulement les
Champs-Élysées entiers et toute la durée de l’après-midi, comme une immense
étendue d’espace et de temps sur chacun des points et à chacun des moments de
laquelle il était possible qu’apparût l’image de Gilberte, mais encore cette
image, elle-même, parce que derrière cette image je sentais se cacher la raison
pour laquelle elle m’était décochée en plein cœur, à quatre heures au lieu de
deux heures et demie, surmontée d’un chapeau de visite à la place d’un béret de
jeu, devant les «Ambassadeurs» et non entre les deux guignols, je devinais
quelqu’une de ces occupations où je ne pouvais suivre Gilberte et qui la
forçaient à sortir ou à rester à la maison, j’étais en contact avec le mystère
de sa vie inconnue. C’était ce mystère aussi qui me troublait quand, courant sur
l’ordre de la fillette à la voix brève pour commencer tout de suite notre partie
de barres, j’apercevais Gilberte, si vive et brusque avec nous, faisant une
révérence à la dame aux Débats (qui lui disait: «Quel beau soleil, on dirait du
feu»), lui parlant avec un sourire timide, d’un air compassé qui m’évoquait la
jeune fille différente que Gilberte devait être chez ses parents, avec les amis
de ses parents, en visite, dans toute son autre existence qui m’échappait. Mais
de cette existence personne ne me donnait l’impression comme M. Swann qui venait
un peu après pour retrouver sa fille. C’est que lui et Mme Swann — parce que
leur fille habitait chez eux, parce que ses études, ses jeux, ses amitiés
dépendaient d’eux — contenaient pour moi, comme Gilberte, peut-être même plus
que Gilberte, comme il convenait à des lieux tout-puissants sur elle en qui il
aurait eu sa source, un inconnu inaccessible, un charme douloureux. Tout ce qui
les concernait était de ma part l’objet d’une préoccupation si constante que les
jours où, comme ceux-là, M. Swann (que j’avais vu si souvent autrefois sans
qu’il excitât ma curiosité, quand il était lié avec mes parents) venait chercher
Gilberte aux Champs-Élysées, une fois calmés les battements de cœur qu’avait
excités en moi l’apparition de son chapeau gris et de son manteau à pèlerine,
son aspect m’impressionnait encore comme celui d’un personnage historique sur
lequel nous venons de lire une série d’ouvrages et dont les moindres
particularités nous passionnent. Ses relations avec le comte de Paris qui, quand
j’en entendais parler à Combray, me semblaient indifférentes, prenaient
maintenant pour moi quelque chose de merveilleux, comme si personne d’autre
n’eût jamais connu les Orléans; elles le faisaient se détacher vivement sur le
fond vulgaire des promeneurs de différentes classes qui encombraient cette allée
des Champs-Elysées, et au milieu desquels j’admirais qu’il consentît à figurer
sans réclamer d’eux d’égards spéciaux, qu’aucun d’ailleurs ne songeait à lui
rendre, tant était profond l’incognito dont il était enveloppé.

Il répondait poliment aux saluts des camarades de Gilberte, même au mien
quoiqu’il fût brouillé avec ma famille, mais sans avoir l’air de me connaître.
(Cela me rappela qu’il m’avait pourtant vu bien souvent à la campagne; souvenir
que j’avais gardé mais dans l’ombre, parce que depuis que j’avais revu Gilberte,
pour moi Swann était surtout son père, et non plus le Swann de Combray; comme
les idées sur lesquelles j’embranchais maintenant son nom étaient différentes
des idées dans le réseau desquelles il était autrefois compris et que je
n’utilisais plus jamais quand j’avais à penser à lui, il était devenu un
personnage nouveau; je le rattachai pourtant par une ligne artificielle
secondaire et transversale à notre invité d’autrefois; et comme rien n’avait
plus pour moi de prix que dans la mesure où mon amour pouvait en profiter, ce
fut avec un mouvement de honte et le regret de ne pouvoir les effacer que je
retrouvai les années où, aux yeux de ce même Swann qui était en ce moment devant
moi aux Champs-Elysées et à qui heureusement Gilberte n’avait peut-être pas dit
mon nom, je m’étais si souvent le soir rendu ridicule en envoyant demander à
maman de monter dans ma chambre me dire bonsoir, pendant qu’elle prenait le café
avec lui, mon père et mes grands-parents à la table du jardin.) Il disait à
Gilberte qu’il lui permettait de faire une partie, qu’il pouvait attendre un
quart d’heure, et s’asseyant comme tout le monde sur une chaise de fer payait
son ticket de cette main que Philippe VII avait si souvent retenue dans la
sienne, tandis que nous commencions à jouer sur la pelouse, faisant envoler les
pigeons dont les beaux corps irisés qui ont la forme d’un cœur et sont comme les
lilas du règne des oiseaux, venaient se réfugier comme en des lieux d’asile, tel
sur le grand vase de pierre à qui son bec en y disparaissant faisait faire le
geste et assignait la destination d’offrir en abondance les fruits ou les
graines qu’il avait l’air d’y picorer, tel autre sur le front de la statue,
qu’il semblait surmonter d’un de ces objets en émail desquels la polychromie
varie dans certaines œuvres antiques la monotonie de la pierre et d’un attribut
qui, quand la déesse le porte, lui vaut une épithète particulière et en fait,
comme pour une mortelle un prénom différent, une divinité nouvelle.

Un de ces jours de soleil qui n’avait pas réalisé mes espérances, je n’eus pas
le courage de cacher ma déception à Gilberte.

— J’avais justement beaucoup de choses à vous demander, lui dis-je. Je croyais
que ce jour compterait beaucoup dans notre amitié. Et aussitôt arrivée, vous
allez partir! Tâchez de venir demain de bonne heure, que je puisse enfin vous
parler.

Sa figure resplendit et ce fut en sautant de joie qu’elle me répondit:

— Demain, comptez-y, mon bel ami, mais je ne viendrai pas! j’ai un grand goûter;
après-demain non plus, je vais chez une amie pour voir de ses fenêtres l’arrivée
du roi Théodose, ce sera superbe, et le lendemain encore à Michel Strogoff et
puis après, cela va être bientôt Noël et les vacances du jour de l’An. Peut-être
on va m’emmener dans le midi. Ce que ce serait chic! quoique cela me fera
manquer un arbre de Noël; en tous cas si je reste à Paris, je ne viendrai pas
ici car j’irai faire des visites avec maman. Adieu, voilà papa qui m’appelle.

Je revins avec Françoise par les rues qui étaient encore pavoisées de soleil,
comme au soir d’une fête qui est finie. Je ne pouvais pas traîner mes jambes.

—Ça n’est pas étonnant, dit Françoise, ce n’est pas un temps de saison, il fait
trop chaud. Hélas! mon Dieu, de partout il doit y avoir bien des pauvres
malades, c’est à croire que là-haut aussi tout se détraque.

Je me redisais en étouffant mes sanglots les mots où Gilberte avait laissé
éclater sa joie de ne pas venir de longtemps aux Champs-Élysées. Mais déjà le
charme dont, par son simple fonctionnement, se remplissait mon esprit dès qu’il
songeait à elle, la position particulière, unique — fût elle affligeante — où me
plaçait inévitablement par rapport à Gilberte, la contrainte interne d’un pli
mental, avaient commencé à ajouter, même à cette marque d’indifférence, quelque
chose de romanesque, et au milieu de mes larmes se formait un sourire qui
n’était que l’ébauche timide d’un baiser. Et quand vint l’heure du courrier, je
me dis ce soir-là comme tous les autres: Je vais recevoir une lettre de
Gilberte, elle va me dire enfin qu’elle n’a jamais cessé de m’aimer, et
m’expliquera la raison mystérieuse pour laquelle elle a été forcée de me le
cacher jusqu’ici, de faire semblant de pouvoir être heureuse sans me voir, la
raison pour laquelle elle a pris l’apparence de la Gilberte simple camarade.

Tous les soirs je me plaisais à imaginer cette lettre, je croyais la lire, je
m’en récitais chaque phrase. Tout d’un coup je m’arrêtais effrayé. Je comprenais
que si je devais recevoir une lettre de Gilberte, ce ne pourrait pas en tous cas
être celle-là puisque c’était moi qui venais de la composer. Et dès lors, je
m’efforçais de détourner ma pensée des mots que j’aurais aimé qu’elle m’écrivît,
par peur en les énonçant, d’exclure justement ceux-là — les plus chers, les plus
désirés — du champ des réalisations possibles. Même si par une invraisemblable
coïncidence, c’eût été justement la lettre que j’avais inventée que de son côté
m’eût adressée Gilberte, y reconnaissant mon œuvre je n’eusse pas eu
l’impression de recevoir quelque chose qui ne vînt pas de moi, quelque chose de
réel, de nouveau, un bonheur extérieur à mon esprit, indépendant de ma volonté,
vraiment donné par l’amour.

En attendant je relisais une page que ne m’avait pas écrite Gilberte, mais qui
du moins me venait d’elle, cette page de Bergotte sur la beauté des vieux mythes
dont s’est inspiré Racine, et que, à côté de la bille d’agathe, je gardais
toujours auprès de moi. J’étais attendri par la bonté de mon amie qui me l’avait
fait rechercher; et comme chacun a besoin de trouver des raisons à sa passion,
jusqu’à être heureux de reconnaître dans l’être qu’il aime des qualités que la
littérature ou la conversation lui ont appris être de celles qui sont dignes
d’exciter l’amour, jusqu’à les assimiler par imitation et en faire des raisons
nouvelles de son amour, ces qualités fussent-elles les plus oppressées à celles
que cet amour eût recherchées tant qu’il était spontané— comme Swann autrefois
le caractère esthétique de la beauté d’Odette — moi, qui avais d’abord aimé
Gilberte, dès Combray, à cause de tout l’inconnu de sa vie, dans lequel j’aurais
voulu me précipiter, m’incarner, en délaissant la mienne qui ne m’était plus
rien, je pensais maintenant comme à un inestimable avantage, que de cette mienne
vie trop connue, dédaignée, Gilberte pourrait devenir un jour l’humble servante,
la commode et confortable collaboratrice, qui le soir m’aidant dans mes travaux,
collationnerait pour moi des brochures. Quant à Bergotte, ce vieillard
infiniment sage et presque divin à cause de qui j’avais d’abord aimé Gilberte,
avant même de l’avoir vue, maintenant c’était surtout à cause de Gilberte que je
l’aimais. Avec autant de plaisir que les pages qu’il avait écrites sur Racine,
je regardais le papier fermé de grands cachets de cire blancs et noué d’un flot
de rubans mauves dans lequel elle me les avait apportées. Je baisais la bille
d’agate qui était la meilleure part du cœur de mon amie, la part qui n’était pas
frivole, mais fidèle, et qui bien que parée du charme mystérieux de la vie de
Gilberte demeurait près de moi, habitait ma chambre, couchait dans mon lit. Mais
la beauté de cette pierre, et la beauté aussi de ces pages de Bergotte, que
j’étais heureux d’associer à l’idée de mon amour pour Gilberte comme si dans les
moments où celui-ci ne m’apparaissait plus que comme un néant, elles lui
donnaient une sorte de consistance, je m’apercevais qu’elles étaient antérieures
à cet amour, qu’elles ne lui ressemblaient pas, que leurs éléments avaient été
fixés par le talent ou par les lois minéralogiques avant que Gilberte ne me
connût, que rien dans le livre ni dans la pierre n’eût été autre si Gilberte ne
m’avait pas aimé et que rien par conséquent ne m’autorisait à lire en eux un
message de bonheur. Et tandis que mon amour attendant sans cesse du lendemain
l’aveu de celui de Gilberte, annulait, défaisait chaque soir le travail mal fait
de la journée, dans l’ombre de moi-même une ouvrière inconnue ne laissait pas au
rebut les fils arrachés et les disposait, sans souci de me plaire et de
travailler à mon bonheur, dans un ordre différent qu’elle donnait à tous ses
ouvrages. Ne portant aucun intérêt particulier à mon amour, ne commençant pas
par décider que j’étais aimé, elle recueillait les actions de Gilberte qui
m’avaient semblé inexplicables et ses fautes que j’avais excusées. Alors les
unes et les autres prenaient un sens. Il semblait dire, cet ordre nouveau, qu’en
voyant Gilberte, au lieu qu’elle vînt aux Champs-Élysées, aller à une matinée,
faire des courses avec son institutrice et se préparer à une absence pour les
vacances du jour de l’an, j’avais tort de penser, me dire: «c’est qu’elle est
frivole ou docile.» Car elle eût cessé d’être l’un ou l’autre si elle m’avait
aimé, et si elle avait été forcée d’obéir c’eût été avec le même désespoir que
j’avais les jours où je ne la voyais pas. Il disait encore, cet ordre nouveau,
que je devais pourtant savoir ce que c’était qu’aimer puisque j’aimais Gilberte;
il me faisait remarquer le souci perpétuel que j’avais de me faire valoir à ses
yeux, à cause duquel j’essayais de persuader à ma mère d’acheter à Françoise un
caoutchouc et un chapeau avec un plumet bleu, ou plutôt de ne plus m’envoyer aux
Champs-Élysées avec cette bonne dont je rougissais (à quoi ma mère répondait que
j’étais injuste pour Françoise, que c’était une brave femme qui nous était
dévouée), et aussi ce besoin unique de voir Gilberte qui faisait que des mois
d’avance je ne pensais qu’à tâcher d’apprendre à quelle époque elle quitterait
Paris et où elle irait, trouvant le pays le plus agréable un lieu d’exil si elle
ne devait pas y être, et ne désirant que rester toujours à Paris tant que je
pourrais la voir aux Champs-Élysées; et il n’avait pas de peine à me montrer que
ce souci-là, ni ce besoin, je ne les trouverais sous les actions de Gilberte.
Elle au contraire appréciait son institutrice, sans s’inquiéter de ce que j’en
pensais. Elle trouvait naturel de ne pas venir aux Champs-Élysées, si c’était
pour aller faire des emplettes avec Mademoiselle, agréable si c’était pour
sortir avec sa mère. Et à supposer même qu’elle m’eût permis d’aller passer les
vacances au même endroit qu’elle, du moins pour choisir cet endroit elle
s’occupait du désir de ses parents, de mille amusements dont on lui avait parlé
et nullement que ce fût celui où ma famille avait l’intention de m’envoyer.
Quand elle m’assurait parfois qu’elle m’aimait moins qu’un de ses amis, moins
qu’elle ne m’aimait la veille parce que je lui avais fait perdre sa partie par
une négligence, je lui demandais pardon, je lui demandais ce qu’il fallait faire
pour qu’elle recommençât à m’aimer autant, pour qu’elle m’aimât plus que les
autres; je voulais qu’elle me dît que c’était déjà fait, je l’en suppliais comme
si elle avait pu modifier son affection pour moi à son gré, au mien, pour me
faire plaisir, rien que par les mots qu’elle dirait, selon ma bonne ou ma
mauvaise conduite. Ne savais-je donc pas que ce que j’éprouvais, moi, pour elle,
ne dépendait ni de ses actions, ni de ma volonté?

Il disait enfin, l’ordre nouveau dessiné par l’ouvrière invisible, que si nous
pouvons désirer que les actions d’une personne qui nous a peinés jusqu’ici
n’aient pas été sincères, il y a dans leur suite une clarté contre quoi notre
désir ne peut rien et à laquelle, plutôt qu’à lui, nous devons demander quelles
seront ses actions de demain.

Ces paroles nouvelles, mon amour les entendait; elles le persuadaient que le
lendemain ne serait pas différent de ce qu’avaient été tous les autres jours;
que le sentiment de Gilberte pour moi, trop ancien déjà pour pouvoir changer,
c’était l’indifférence; que dans mon amitié avec Gilberte, c’est moi seul qui
aimais. «C’est vrai, répondait mon amour, il n’y a plus rien à faire de cette
amitié-là, elle ne changera pas.» Alors dès le lendemain (ou attendant une fête
s’il y en avait une prochaine, un anniversaire, le nouvel an peut-être, un de
ces jours qui ne sont pas pareils aux autres, où le temps recommence sur de
nouveaux frais en rejetant l’héritage du passé, en n’acceptant pas le legs de
ses tristesses) je demandais à Gilberte de renoncer à notre amitié ancienne et
de jeter les bases d’une nouvelle amitié.

J’avais toujours à portée de ma main un plan de Paris qui, parce qu’on pouvait y
distinguer la rue où habitaient M. et Mme Swann, me semblait contenir un trésor.
Et par plaisir, par une sorte de fidélité chevaleresque aussi, à propos de
n’importe quoi, je disais le nom de cette rue, si bien que mon père me
demandait, n’étant pas comme ma mère et ma grand’mère au courant de mon amour:

— Mais pourquoi parles-tu tout le temps de cette rue, elle n’a rien
d’extraordinaire, elle est très agréable à habiter parce qu’elle est à deux pas
du Bois, mais il y en a dix autres dans le même cas.

Je m’arrangeais à tout propos à faire prononcer à mes parents le nom de Swann:
certes je me le répétais mentalement sans cesse: mais j’avais besoin aussi
d’entendre sa sonorité délicieuse et de me faire jouer cette musique dont la
lecture muette ne me suffisait pas. Ce nom de Swann d’ailleurs que je
connaissais depuis si longtemps, était maintenant pour moi, ainsi qu’il arrive à
certains aphasiques à l’égard des mots les plus usuels, un nom nouveau. Il était
toujours présent à ma pensée et pourtant elle ne pouvait pas s’habituer à lui.
Je le décomposais, je l’épelais, son orthographe était pour moi une surprise. Et
en même temps que d’être familier, il avait cessé de me paraître innocent. Les
joies que je prenais à l’entendre, je les croyais si coupables, qu’il me
semblait qu’on devinait ma pensée et qu’on changeait la conversation si je
cherchais à l’y amener. Je me rabattais sur les sujets qui touchaient encore à
Gilberte, je rabâchais sans fin les mêmes paroles, et j’avais beau savoir que ce
n’était que des paroles — des paroles prononcées loin d’elle, qu’elle
n’entendait pas, des paroles sans vertu qui répétaient ce qui était, mais ne le
pouvaient modifier — pourtant il me semblait qu’à force de manier, de brasser
ainsi tout ce qui avoisinait Gilberte j’en ferais peut-être sortir quelque chose
d’heureux. Je redisais à mes parents que Gilberte aimait bien son institutrice,
comme si cette proposition énoncée pour la centième fois allait avoir enfin pour
effet de faire brusquement entrer Gilberte venant à tout jamais vivre avec nous.
Je reprenais l’éloge de la vieille dame qui lisait les Débats (j’avais insinué à
mes parents que c’était une ambassadrice ou peut-être une altesse) et je
continuais à célébrer sa beauté, sa magnificence, sa noblesse, jusqu’au jour où
je dis que d’après le nom qu’avait prononcé Gilberte elle devait s’appeler Mme
Blatin.

— Oh! mais je vois ce que c’est, s’écria ma mère tandis que je me sentais rougir
de honte. A la garde! A la garde! comme aurait dit ton pauvre grand-père. Et
c’est elle que tu trouves belle! Mais elle est horrible et elle l’a toujours
été. C’est la veuve d’un huissier. Tu ne te rappelles pas quand tu étais enfant
les manèges que je faisais pour l’éviter à la leçon de gymnastique où, sans me
connaître, elle voulait venir me parler sous prétexte de me dire que tu étais
«trop beau pour un garçon». Elle a toujours eu la rage de connaître du monde et
il faut bien qu’elle soit une espèce de folle comme j’ai toujours pensé, si elle
connaît vraiment Mme Swann. Car si elle était d’un milieu fort commun, au moins
il n’y a jamais rien eu que je sache à dire sur elle. Mais il fallait toujours
qu’elle se fasse des relations. Elle est horrible, affreusement vulgaire, et
avec cela faiseuse d’embarras.»

Quant à Swann, pour tâcher de lui ressembler, je passais tout mon temps à table,
à me tirer sur le nez et à me frotter les yeux. Mon père disait: «cet enfant est
idiot, il deviendra affreux.» J’aurais surtout voulu être aussi chauve que
Swann. Il me semblait un être si extraordinaire que je trouvais merveilleux que
des personnes que je fréquentais le connussent aussi et que dans les hasards
d’une journée quelconque on pût être amené à le rencontrer. Et une fois, ma
mère, en train de nous raconter comme chaque soir à dîner, les courses qu’elle
avait faites dans l’après-midi, rien qu’en disant: «A ce propos, devinez qui
j’ai rencontré aux Trois Quartiers, au rayon des parapluies: Swann», fit éclore
au milieu de son récit, fort aride pour moi, une fleur mystérieuse. Quelle
mélancolique volupté, d’apprendre que cet après-midi-là, profilant dans la foule
sa forme surnaturelle, Swann avait été acheter un parapluie. Au milieu des
événements grands et minimes, également indifférents, celui-là éveillait en moi
ces vibrations particulières dont était perpétuellement ému mon amour pour
Gilberte. Mon père disait que je ne m’intéressais à rien parce que je n’écoutais
pas quand on parlait des conséquences politiques que pouvait avoir la visite du
roi Théodose, en ce moment l’hôte de la France et, prétendait-on, son allié.
Mais combien en revanche, j’avais envie de savoir si Swann avait son manteau à
pèlerine!

— Est-ce que vous vous êtes dit bonjour? demandai-je.

— Mais naturellement, répondit ma mère qui avait toujours l’air de craindre que
si elle eût avoué que nous étions en froid avec Swann, on eût cherché à les
réconcilier plus qu’elle ne souhaitait, à cause de Mme Swann qu’elle ne voulait
pas connaître. «C’est lui qui est venu me saluer, je ne le voyais pas.

— Mais alors, vous n’êtes pas brouillés?

— Brouillés? mais pourquoi veux-tu que nous soyons brouillés», répondit-elle
vivement comme si j’avais attenté à la fiction de ses bons rapports avec Swann
et essayé de travailler à un «rapprochement».

— Il pourrait t’en vouloir de ne plus l’inviter.

— On n’est pas obligé d’inviter tout le monde; est-ce qu’il m’invite? Je ne
connais pas sa femme.

— Mais il venait bien à Combray.

— Eh bien oui! il venait à Combray, et puis à Paris il a autre chose à faire et
moi aussi. Mais je t’assure que nous n’avions pas du tout l’air de deux
personnes brouillées. Nous sommes restés un moment ensemble parce qu’on ne lui
apportait pas son paquet. Il m’a demandé de tes nouvelles, il m’a dit que tu
jouais avec sa fille, ajouta ma mère, m’émerveillant du prodige que j’existasse
dans l’esprit de Swann, bien plus, que ce fût d’une façon assez complète, pour
que, quand je tremblais d’amour devant lui aux Champs-Élysées, il sût mon nom,
qui était ma mère, et pût amalgamer autour de ma qualité de camarade de sa fille
quelques renseignements sur mes grands-parents, leur famille, l’endroit que nous
habitions, certaines particularités de notre vie d’autrefois, peut-être même
inconnues de moi. Mais ma mère ne paraissait pas avoir trouvé un charme
particulier à ce rayon des Trois Quartiers où elle avait représenté pour Swann,
au moment où il l’avait vue, une personne définie avec qui il avait des
souvenirs communs qui avaient motivé chez lui le mouvement de s’approcher
d’elle, le geste de la saluer.

Ni elle d’ailleurs ni mon père ne semblaient non plus trouver à parler des
grands-parents de Swann, du titre d’agent de change honoraire, un plaisir qui
passât tous les autres. Mon imagination avait isolé et consacré dans le Paris
social une certaine famille comme elle avait fait dans le Paris de pierre pour
une certaine maison dont elle avait sculpté la porte cochère et rendu précieuses
les fenêtres. Mais ces ornements, j’étais seul à les voir. De même que mon père
et ma mère trouvaient la maison qu’habitait Swann pareille aux autres maisons
construites en même temps dans le quartier du Bois, de même la famille de Swann
leur semblait du même genre que beaucoup d’autres familles d’agents de change.
Ils la jugeaient plus ou moins favorablement selon le degré où elle avait
participé à des mérites communs au reste de l’univers et ne lui trouvaient rien
d’unique. Ce qu’au contraire ils y appréciaient, ils le rencontraient à un degré
égal, ou plus élevé, ailleurs. Aussi après avoir trouvé la maison bien située,
ils parlaient d’une autre qui l’était mieux, mais qui n’avait rien à voir avec
Gilberte, ou de financiers d’un cran supérieur à son grand-père; et s’ils
avaient eu l’air un moment d’être du même avis que moi, c’était par un
malentendu qui ne tardait pas à se dissiper. C’est que, pour percevoir dans tout
ce qui entourait Gilberte, une qualité inconnue analogue dans le monde des
émotions à ce que peut être dans celui des couleurs l’infra-rouge, mes parents
étaient dépourvus de ce sens supplémentaire et momentané dont m’avait doté
l’amour.

Les jours où Gilberte m’avait annoncé qu’elle ne devait pas venir aux
Champs-Elysées, je tâchais de faire des promenades qui me rapprochassent un peu
d’elle. Parfois j’emmenais Françoise en pèlerinage devant la maison
qu’habitaient les Swann. Je lui faisais répéter sans fin ce que, par
l’institutrice, elle avait appris relativement à Mme Swann. «Il paraît qu’elle a
bien confiance à des médailles. Jamais elle ne partira en voyage si elle a
entendu la chouette, ou bien comme un tic-tac d’horloge dans le mur, ou si elle
a vu un chat à minuit, ou si le bois d’un meuble, il a craqué. Ah! c’est une
personne très croyante!» J’étais si amoureux de Gilberte que si sur le chemin
j’apercevais leur vieux maître d’hôtel promenant un chien, l’émotion m’obligeait
à m’arrêter, j’attachais sur ses favoris blancs des regards pleins de passion.
Françoise me disait:

— Qu’est-ce que vous avez?

Puis, nous poursuivions notre route jusque devant leur porte cochère où un
concierge différent de tout concierge, et pénétré jusque dans les galons de sa
livrée du même charme douloureux que j’avais ressenti dans le nom de Gilberte,
avait l’air de savoir que j’étais de ceux à qui une indignité originelle
interdirait toujours de pénétrer dans la vie mystérieuse qu’il était chargé de
garder et sur laquelle les fenêtres de l’entre-sol paraissaient conscientes
d’être refermées, ressemblant beaucoup moins entre la noble retombée de leurs
rideaux de mousseline à n’importe quelles autres fenêtres, qu’aux regards de
Gilberte. D’autres fois nous allions sur les boulevards et je me postais à
l’entrée de la rue Duphot; on m’avait dit qu’on pouvait souvent y voir passer
Swann se rendant chez son dentiste; et mon imagination différenciait tellement
le père de Gilberte du reste de l’humanité, sa présence au milieu du monde réel
y introduisait tant de merveilleux, que, avant même d’arriver à la Madeleine,
j’étais ému à la pensée d’approcher d’une rue où pouvait se produire inopinément
l’apparition surnaturelle.

Mais le plus souvent — quand je ne devais pas voir Gilberte — comme j’avais
appris que Mme Swann se promenait presque chaque jour dans l’allée «des
Acacias», autour du grand Lac, et dans l’allée de la «Reine Marguerite», je
dirigeais Françoise du côté du bois de Boulogne. Il était pour moi comme ces
jardins zoologiques où l’on voit rassemblés des flores diverses et des paysages
opposés; où, après une colline on trouve une grotte, un pré, des rochers, une
rivière, une fosse, une colline, un marais, mais où l’on sait qu’ils ne sont là
que pour fournir aux ébats de l’hippopotame, des zèbres, des crocodiles, des
lapins russes, des ours et du héron, un milieu approprié ou un cadre
pittoresque; lui, le Bois, complexe aussi, réunissant des petits mondes divers
et clos — faisant succéder quelque ferme plantée d’arbres rouges, de chênes
d’Amérique, comme une exploitation agricole dans la Virginie, à une sapinière au
bord du lac, ou à une futaie d’où surgit tout à coup dans sa souple fourrure,
avec les beaux yeux d’une bête, quelque promeneuse rapide — il était le Jardin
des femmes; et — comme l’allée de Myrtes de l’Enéide — plantée pour elles
d’arbres d’une seule essence, l’allée des Acacias était fréquentée par les
Beautés célèbres. Comme, de loin, la culmination du rocher d’où elle se jette
dans l’eau, transporte de joie les enfants qui savent qu’ils vont voir l’otarie,
bien avant d’arriver à l’allée des Acacias, leur parfum qui, irradiant alentour,
faisait sentir de loin l’approche et la singularité d’une puissante et molle
individualité végétale; puis, quand je me rapprochais, le faîte aperçu de leur
frondaison légère et mièvre, d’une élégance facile, d’une coupe coquette et d’un
mince tissu, sur laquelle des centaines de fleurs s’étaient abattues comme des
colonies ailées et vibratiles de parasites précieux; enfin jusqu’à leur nom
féminin, désœuvré et doux, me faisaient battre le cœur mais d’un désir mondain,
comme ces valses qui ne nous évoquent plus que le nom des belles invitées que
l’huissier annonce à l’entrée d’un bal. On m’avait dit que je verrais dans
l’allée certaines élégantes que, bien qu’elles n’eussent pas toutes été
épousées, l’on citait habituellement à côté de Mme Swann, mais le plus souvent
sous leur nom de guerre; leur nouveau nom, quand il y en avait un, n’était
qu’une sorte d’incognito que ceux qui voulaient parler d’elles avaient soin de
lever pour se faire comprendre. Pensant que le Beau — dans l’ordre des élégances
féminines —était régi par des lois occultes à la connaissance desquelles elles
avaient été initiées, et qu’elles avaient le pouvoir de le réaliser, j’acceptais
d’avance comme une révélation l’apparition de leur toilette, de leur attelage,
de mille détails au sein desquels je mettais ma croyance comme une âme
intérieure qui donnait la cohésion d’un chef-d’œuvre à cet ensemble éphémère et
mouvant. Mais c’est Mme Swann que je voulais voir, et j’attendais qu’elle
passât, ému comme si ç’avait été Gilberte, dont les parents, imprégnés comme
tout ce qui l’entourait, de son charme, excitaient en moi autant d’amour
qu’elle, même un trouble plus douloureux (parce que leur point de contact avec
elle était cette partie intestine de sa vie qui m’était interdite), et enfin
(car je sus bientôt, comme on le verra, qu’ils n’aimaient pas que je jouasse
avec elle), ce sentiment de vénération que nous vouons toujours à ceux qui
exercent sans frein la puissance de nous faire du mal.

J’assignais la première place à la simplicité, dans l’ordre des mérites
esthétiques et des grandeurs mondaines quand j’apercevais Mme Swann à pied, dans
une polonaise de drap, sur la tête un petit toquet agrémenté d’une aile de
lophophore, un bouquet de violettes au corsage, pressée, traversant l’allée des
Acacias comme si ç’avait été seulement le chemin le plus court pour rentrer chez
elle et répondant d’un clin d’oeil aux messieurs en voiture qui, reconnaissant
de loin sa silhouette, la saluaient et se disaient que personne n’avait autant
de chic. Mais au lieu de la simplicité, c’est le faste que je mettais au plus
haut rang, si, après que j’avais forcé Françoise, qui n’en pouvait plus et
disait que les jambes «lui rentraient», à faire les cent pas pendant une heure,
je voyais enfin, débouchant de l’allée qui vient de la Porte Dauphine — image
pour moi d’un prestige royal, d’une arrivée souveraine telle qu’aucune reine
véritable n’a pu m’en donner l’impression dans la suite, parce que j’avais de
leur pouvoir une notion moins vague et plus expérimentale — emportée par le vol
de deux chevaux ardents, minces et contournés comme on en voit dans les dessins
de Constantin Guys, portant établi sur son siège un énorme cocher fourré comme
un cosaque, à côté d’un petit groom rappelant le «tigre» de «feu Baudenord», je
voyais — ou plutôt je sentais imprimer sa forme dans mon cœur par une nette et
épuisante blessure — une incomparable victoria, à dessein un peu haute et
laissant passer à travers son luxe «dernier cri» des allusions aux formes
anciennes, au fond de laquelle reposait avec abandon Mme Swann, ses cheveux
maintenant blonds avec une seule mèche grise ceints d’un mince bandeau de
fleurs, le plus souvent des violettes, d’où descendaient de longs voiles, à la
main une ombrelle mauve, aux lèvres un sourire ambigu où je ne voyais que la
bienveillance d’une Majesté et où il y avait surtout la provocation de la
cocotte, et qu’elle inclinait avec douceur sur les personnes qui la saluaient.
Ce sourire en réalité disait aux uns: «Je me rappelle très bien, c’était
exquis!»; à d’autres: «Comme j’aurais aimé! ç’a été la mauvaise chance!»; à
d’autres: «Mais si vous voulez! Je vais suivre encore un moment la file et dès
que je pourrai, je couperai.» Quand passaient des inconnus, elle laissait
cependant autour de ses lèvres un sourire oisif, comme tourné vers l’attente ou
le souvenir d’un ami et qui faisait dire: «Comme elle est belle!» Et pour
certains hommes seulement elle avait un sourire aigre, contraint, timide et
froid et qui signifiait: «Oui, rosse, je sais que vous avez une langue de
vipère, que vous ne pouvez pas vous tenir de parler! Est-ce que je m’occupe de
vous, moi!» Coquelin passait en discourant au milieu d’amis qui l’écoutaient et
faisait avec la main à des personnes en voiture, un large bonjour de théâtre.
Mais je ne pensais qu’à Mme Swann et je faisais semblant de ne pas l’avoir vue,
car je savais qu’arrivée à la hauteur du Tir aux pigeons elle dirait à son
cocher de couper la file et de l’arrêter pour qu’elle pût descendre l’allée à
pied. Et les jours où je me sentais le courage de passer à côté d’elle,
j’entraînais Françoise dans cette direction. A un moment en effet, c’est dans
l’allée des piétons, marchant vers nous que j’apercevais Mme Swann laissant
s’étaler derrière elle la longue traîne de sa robe mauve, vêtue, comme le peuple
imagine les reines, d’étoffes et de riches atours que les autres femmes ne
portaient pas, abaissant parfois son regard sur le manche de son ombrelle,
faisant peu attention aux personnes qui passaient, comme si sa grande affaire et
son but avaient été de prendre de l’exercice, sans penser qu’elle était vue et
que toutes les têtes étaient tournées vers elle. Parfois pourtant quand elle
s’était retournée pour appeler son lévrier, elle jetait imperceptiblement un
regard circulaire autour d’elle.

Ceux même qui ne la connaissaient pas étaient avertis par quelque chose de
singulier et d’excessif — ou peut-être par une radiation télépathique comme
celles qui déchaînaient des applaudissements dans la foule ignorante aux moments
où la Berma était sublime — que ce devait être quelque personne connue. Ils se
demandaient: «Qui est-ce?», interrogeaient quelquefois un passant, ou se
promettaient de se rappeler la toilette comme un point de repère pour des amis
plus instruits qui les renseigneraient aussitôt. D’autres promeneurs, s’arrêtant
à demi, disaient:

—«Vous savez qui c’est? Mme Swann! Cela ne vous dit rien? Odette de Crécy?»

—«Odette de Crécy? Mais je me disais aussi, ces yeux tristes . . . Mais
savez-vous qu’elle ne doit plus être de la première jeunesse! Je me rappelle que
j’ai couché avec elle le jour de la démission de Mac-Mahon.»

—«Je crois que vous ferez bien de ne pas le lui rappeler. Elle est maintenant
Mme Swann, la femme d’un monsieur du Jockey, ami du prince de Galles. Elle est
du reste encore superbe.»

—«Oui, mais si vous l’aviez connue à ce moment-là, ce qu’elle était jolie! Elle
habitait un petit hôtel très étrange avec des chinoiseries. Je me rappelle que
nous étions embêtés par le bruit des crieurs de journaux, elle a fini par me
faire lever.»

Sans entendre les réflexions, je percevais autour d’elle le murmure indistinct
de la célébrité. Mon cœur battait d’impatience quand je pensais qu’il allait se
passer un instant encore avant que tous ces gens, au milieu desquels je
remarquais avec désolation que n’était pas un banquier mulâtre par lequel je me
sentais méprisé, vissent le jeune homme inconnu auquel ils ne prêtaient aucune
attention, saluer (sans la connaître, à vrai dire, mais je m’y croyais autorisé
parce que mes parents connaissaient son mari et que j’étais le camarade de sa
fille), cette femme dont la réputation de beauté, d’inconduite et d’élégance
était universelle. Mais déjà j’étais tout près de Mme Swann, alors je lui tirais
un si grand coup de chapeau, si étendu, si prolongé, qu’elle ne pouvait
s’empêcher de sourire. Des gens riaient. Quant à elle, elle ne m’avait jamais vu
avec Gilberte, elle ne savait pas mon nom, mais j’étais pour elle — comme un des
gardes du Bois, ou le batelier ou les canards du lac à qui elle jetait du pain —
un des personnages secondaires, familiers, anonymes, aussi dénués de caractères
individuels qu’un «emploi de théâtre», de ses promenades au bois. Certains jours
où je ne l’avais pas vue allée des Acacias, il m’arrivait de la rencontrer dans
l’allée de la Reine-Marguerite où vont les femmes qui cherchent à être seules,
ou à avoir l’air de chercher à l’être; elle ne le restait pas longtemps, bientôt
rejointe par quelque ami, souvent coiffé d’un «tube» gris, que je ne connaissais
pas et qui causait longuement avec elle, tandis que leurs deux voitures
suivaient.

Cette complexité du bois de Boulogne qui en fait un lieu factice et, dans le
sens zoologique ou mythologique du mot, un Jardin, je l’ai retrouvée cette année
comme je le traversais pour aller à Trianon, un des premiers matins de ce mois
de novembre où, à Paris, dans les maisons, la proximité et la privation du
spectacle de l’automne qui s’achève si vite sans qu’on y assiste, donnent une
nostalgie, une véritable fièvre des feuilles mortes qui peut aller jusqu’à
empêcher de dormir. Dans ma chambre fermée, elles s’interposaient depuis un
mois, évoquées par mon désir de les voir, entre ma pensée et n’importe quel
objet auquel je m’appliquais, et tourbillonnaient comme ces taches jaunes qui
parfois, quoi que nous regardions, dansent devant nos yeux. Et ce matin-là,
n’entendant plus la pluie tomber comme les jours précédents, voyant le beau
temps sourire aux coins des rideaux fermés comme aux coins d’une bouche close
qui laisse échapper le secret de son bonheur, j’avais senti que ces feuilles
jaunes, je pourrais les regarder traversées par la lumière, dans leur suprême
beauté; et ne pouvant pas davantage me tenir d’aller voir des arbres
qu’autrefois, quand le vent soufflait trop fort dans ma cheminée, de partir pour
le bord de la mer, j’étais sorti pour aller à Trianon, en passant par le bois de
Boulogne. C’était l’heure et c’était la saison où le Bois semble peut-être le
plus multiple, non seulement parce qu’il est plus subdivisé, mais encore parce
qu’il l’est autrement. Même dans les parties découvertes où l’on embrasse un
grand espace, çà et là, en face des sombres masses lointaines des arbres qui
n’avaient pas de feuilles ou qui avaient encore leurs feuilles de l’été, un
double rang de marronniers orangés semblait, comme dans un tableau à peine
commencé, avoir seul encore été peint par le décorateur qui n’aurait pas mis de
couleur sur le reste, et tendait son allée en pleine lumière pour la promenade
épisodique de personnages qui ne seraient ajoutés que plus tard.

Plus loin, là où toutes leurs feuilles vertes couvraient les arbres, un seul,
petit, trapu, étêté et têtu, secouait au vent une vilaine chevelure rouge.
Ailleurs encore c’était le premier éveil de ce mois de mai des feuilles, et
celles d’un empelopsis merveilleux et souriant, comme une épine rose de l’hiver,
depuis le matin même étaient tout en fleur. Et le Bois avait l’aspect provisoire
et factice d’une pépinière ou d’un parc, où soit dans un intérêt botanique, soit
pour la préparation d’une fête, on vient d’installer, au milieu des arbres de
sorte commune qui n’ont pas encore été déplantés, deux ou trois espèces
précieuses aux feuillages fantastiques et qui semblent autour d’eux réserver du
vide, donner de l’air, faire de la clarté. Ainsi c’était la saison où le Bois de
Boulogne trahit le plus d’essences diverses et juxtapose le plus de parties
distinctes en un assemblage composite. Et c’était aussi l’heure. Dans les
endroits où les arbres gardaient encore leurs feuilles, ils semblaient subir une
altération de leur matière à partir du point où ils étaient touchés par la
lumière du soleil, presque horizontale le matin comme elle le redeviendrait
quelques heures plus tard au moment où dans le crépuscule commençant, elle
s’allume comme une lampe, projette à distance sur le feuillage un reflet
artificiel et chaud, et fait flamber les suprêmes feuilles d’un arbre qui reste
le candélabre incombustible et terne de son faîte incendié. Ici, elle
épaississait comme des briques, et, comme une jaune maçonnerie persane à dessins
bleus, cimentait grossièrement contre le ciel les feuilles des marronniers, là
au contraire les détachait de lui, vers qui elles crispaient leurs doigts d’or.
A mi-hauteur d’un arbre habillé de vigne vierge, elle greffait et faisait
épanouir, impossible à discerner nettement dans l’éblouissement, un immense
bouquet comme de fleurs rouges, peut-être une variété d’œillet. Les différentes
parties du Bois, mieux confondues l’été dans l’épaisseur et la monotonie des
verdures se trouvaient dégagées. Des espaces plus éclaircis laissaient voir
l’entrée de presque toutes, ou bien un feuillage somptueux la désignait comme
une oriflamme. On distinguait, comme sur une carte en couleur, Armenonville, le
Pré Catelan, Madrid, le Champ de courses, les bords du Lac. Par moments
apparaissait quelque construction inutile, une fausse grotte, un moulin à qui
les arbres en s’écartant faisaient place ou qu’une pelouse portait en avant sur
sa moelleuse plateforme. On sentait que le Bois n’était pas qu’un bois, qu’il
répondait à une destination étrangère à la vie de ses arbres, l’exaltation que
j’éprouvais n’était pas causée que par l’admiration de l’automne, mais par un
désir. Grande source d’une joie que l’âme ressent d’abord sans en reconnaître la
cause, sans comprendre que rien au dehors ne la motive. Ainsi regardais-je les
arbres avec une tendresse insatisfaite qui les dépassait et se portait à mon
insu vers ce chef-d’œuvre des belles promeneuses qu’ils enferment chaque jour
pendant quelques heures. J’allais vers l’allée des Acacias. Je traversais des
futaies où la lumière du matin qui leur imposait des divisions nouvelles,
émondait les arbres, mariait ensemble les tiges diverses et composait des
bouquets. Elle attirait adroitement à elle deux arbres; s’aidant du ciseau
puissant du rayon et de l’ombre, elle retranchait à chacun une moitié de son
tronc et de ses branches, et, tressant ensemble les deux moitiés qui restaient,
en faisait soit un seul pilier d’ombre, que délimitait l’ensoleillement
d’alentour, soit un seul fantôme de clarté dont un réseau d’ombre noire cernait
le factice et tremblant contour. Quand un rayon de soleil dorait les plus hautes
branches, elles semblaient, trempées d’une humidité étincelante, émerger seules
de l’atmosphère liquide et couleur d’émeraude où la futaie tout entière était
plongée comme sous la mer. Car les arbres continuaient à vivre de leur vie
propre et quand ils n’avaient plus de feuilles, elle brillait mieux sur le
fourreau de velours vert qui enveloppait leurs troncs ou dans l’émail blanc des
sphères de gui qui étaient semées au faîte des peupliers, rondes comme le soleil
et la lune dans la Création de Michel-Ange. Mais forcés depuis tant d’années par
une sorte de greffe à vivre en commun avec la femme, ils m’évoquaient la dryade,
la belle mondaine rapide et colorée qu’au passage ils couvrent de leurs branches
et obligent à ressentir comme eux la puissance de la saison; ils me rappelaient
le temps heureux de ma croyante jeunesse, quand je venais avidement aux lieux où
des chefs-d’œuvre d’élégance féminine se réaliseraient pour quelques instants
entre les feuillages inconscients et complices. Mais la beauté que faisaient
désirer les sapins et les acacias du bois de Boulogne, plus troublants en cela
que les marronniers et les lilas de Trianon que j’allais voir, n’était pas fixée
en dehors de moi dans les souvenirs d’une époque historique, dans des œuvres
d’art, dans un petit temple à l’amour au pied duquel s’amoncellent les feuilles
palmées d’or. Je rejoignis les bords du Lac, j’allai jusqu’au Tir aux pigeons.
L’idée de perfection que je portais en moi, je l’avais prêtée alors à la hauteur
d’une victoria, à la maigreur de ces chevaux furieux et légers comme des guêpes,
les yeux injectés de sang comme les cruels chevaux de Diomède, et que
maintenant, pris d’un désir de revoir ce que j’avais aimé, aussi ardent que
celui qui me poussait bien des années auparavant dans ces mêmes chemins, je
voulais avoir de nouveau sous les yeux au moment où l’énorme cocher de Mme
Swann, surveillé par un petit groom gros comme le poing et aussi enfantin que
saint Georges, essayait de maîtriser leurs ailes d’acier qui se débattaient
effarouchées et palpitantes. Hélas! il n’y avait plus que des automobiles
conduites par des mécaniciens moustachus qu’accompagnaient de grands valets de
pied. Je voulais tenir sous les yeux de mon corps pour savoir s’ils étaient
aussi charmants que les voyaient les yeux de ma mémoire, de petits chapeaux de
femmes si bas qu’ils semblaient une simple couronne. Tous maintenant étaient
immenses, couverts de fruits et de fleurs et d’oiseaux variés. Au lieu des
belles robes dans lesquelles Mme Swann avait l’air d’une reine, des tuniques
gréco-saxonnes relevaient avec les plis des Tanagra, et quelquefois dans le
style du Directoire, des chiffrons liberty semés de fleurs comme un papier
peint. Sur la tête des messieurs qui auraient pu se promener avec Mme Swann dans
l’allée de la Reine-Marguerite, je ne trouvais pas le chapeau gris d’autrefois,
ni même un autre. Ils sortaient nu-tête. Et toutes ces parties nouvelles du
spectacle, je n’avais plus de croyance à y introduire pour leur donner la
consistance, l’unité, l’existence; elles passaient éparses devant moi, au
hasard, sans vérité, ne contenant en elles aucune beauté que mes yeux eussent pu
essayer comme autrefois de composer. C’étaient des femmes quelconques, en
l’élégance desquelles je n’avais aucune foi et dont les toilettes me semblaient
sans importance. Mais quand disparaît une croyance, il lui survit — et de plus
en plus vivace pour masquer le manque de la puissance que nous avons perdue de
donner de la réalité à des choses nouvelles — un attachement fétichiste aux
anciennes qu’elle avait animées, comme si c’était en elles et non en nous que le
divin résidait et si notre incrédulité actuelle avait une cause contingente, la
mort des Dieux.

Quelle horreur! me disais-je: peut-on trouver ces automobiles élégantes comme
étaient les anciens attelages? je suis sans doute déjà trop vieux — mais je ne
suis pas fait pour un monde où les femmes s’entravent dans des robes qui ne sont
pas même en étoffe. A quoi bon venir sous ces arbres, si rien n’est plus de ce
qui s’assemblait sous ces délicats feuillages rougissants, si la vulgarité et la
folie ont remplacé ce qu’ils encadraient d’exquis. Quelle horreur! Ma
consolation c’est de penser aux femmes que j’ai connues, aujourd’hui qu’il n’y a
plus d’élégance. Mais comment des gens qui contemplent ces horribles créatures
sous leurs chapeaux couverts d’une volière ou d’un potager, pourraient-ils même
sentir ce qu’il y avait de charmant à voir Mme Swann coiffée d’une simple capote
mauve ou d’un petit chapeau que dépassait une seule fleur d’iris toute droite.
Aurais-je même pu leur faire comprendre l’émotion que j’éprouvais par les matins
d’hiver à rencontrer Mme Swann à pied, en paletot de loutre, coiffée d’un simple
béret que dépassaient deux couteaux de plumes de perdrix, mais autour de
laquelle la tiédeur factice de son appartement était évoquée, rien que par le
bouquet de violettes qui s’écrasait à son corsage et dont le fleurissement
vivant et bleu en face du ciel gris, de l’air glacé, des arbres aux branches
nues, avait le même charme de ne prendre la saison et le temps que comme un
cadre, et de vivre dans une atmosphère humaine, dans l’atmosphère de cette
femme, qu’avaient dans les vases et les jardinières de son salon, près du feu
allumé, devant le canapé de soie, les fleurs qui regardaient par la fenêtre
close la neige tomber? D’ailleurs il ne m’eût pas suffi que les toilettes
fussent les mêmes qu’en ces années-là. A cause de la solidarité qu’ont entre
elles les différentes parties d’un souvenir et que notre mémoire maintient
équilibrées dans un assemblage où il ne nous est pas permis de rien distraire,
ni refuser, j’aurais voulu pouvoir aller finir la journée chez une de ces
femmes, devant une tasse de thé, dans un appartement aux murs peints de couleurs
sombres, comme était encore celui de Mme Swann (l’année d’après celle où se
termine la première partie de ce récit) et où luiraient les feux orangés, la
rouge combustion, la flamme rose et blanche des chrysanthèmes dans le crépuscule
de novembre pendant des instants pareils à ceux où (comme on le verra plus tard)
je n’avais pas su découvrir les plaisirs que je désirais. Mais maintenant, même
ne me conduisant à rien, ces instants me semblaient avoir eu eux-mêmes assez de
charme. Je voudrais les retrouver tels que je me les rappelais. Hélas! il n’y
avait plus que des appartements Louis XVI tout blancs, émaillés d’hortensias
bleus. D’ailleurs, on ne revenait plus à Paris que très tard. Mme Swann m’eût
répondu d’un château qu’elle ne rentrerait qu’en février, bien après le temps
des chrysanthèmes, si je lui avais demandé de reconstituer pour moi les éléments
de ce souvenir que je sentais attaché à une année lointaine, à un millésime vers
lequel il ne m’était pas permis de remonter, les éléments de ce désir devenu
lui-même inaccessible comme le plaisir qu’il avait jadis vainement poursuivi. Et
il m’eût fallu aussi que ce fussent les mêmes femmes, celles dont la toilette
m’intéressait parce que, au temps où je croyais encore, mon imagination les
avait individualisées et les avait pourvues d’une légende. Hélas! dans l’avenue
des Acacias — l’allée de Myrtes — j’en revis quelques-unes, vieilles, et qui
n’étaient plus que les ombres terribles de ce qu’elles avaient été, errant,
cherchant désespérément on ne sait quoi dans les bosquets virgiliens. Elles
avaient fui depuis longtemps que j’étais encore à interroger vainement les
chemins désertés. Le soleil s’était caché. La nature recommençait à régner sur
le Bois d’où s’était envolée l’idée qu’il était le Jardin élyséen de la Femme;
au-dessus du moulin factice le vrai ciel était gris; le vent ridait le Grand Lac
de petites vaguelettes, comme un lac; de gros oiseaux parcouraient rapidement le
Bois, comme un bois, et poussant des cris aigus se posaient l’un après l’autre
sur les grands chênes qui sous leur couronne druidique et avec une majesté
dodonéenne semblaient proclamer le vide inhumain de la forêt désaffectée, et
m’aidaient à mieux comprendre la contradiction que c’est de chercher dans la
réalité les tableaux de la mémoire, auxquels manquerait toujours le charme qui
leur vient de la mémoire même et de n’être pas perçus par les sens. La réalité
que j’avais connue n’existait plus. Il suffisait que Mme Swann n’arrivât pas
toute pareille au même moment, pour que l’Avenue fût autre. Les lieux que nous
avons connus n’appartiennent pas qu’au monde de l’espace où nous les situons
pour plus de facilité. Ils n’étaient qu’une mince tranche au milieu
d’impressions contiguës qui formaient notre vie d’alors; le souvenir d’une
certaine image n’est que le regret d’un certain instant; et les maisons, les
routes, les avenues, sont fugitives, hélas, comme les années.





